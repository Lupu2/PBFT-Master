\section{Contributions}
REWRITE: CONTRIBUTIONS --> what is achieved in THIS PAPER, understanding is not an achievement! Try to combine contributions with the outline in some way. Example: Here is our contribution, this can be found here:... (structure).
The following contributions are achieved by this thesis:
\begin{itemize}
\item Understanding the functionality and general workflow of Practical Byzantine Fault Tolerance
\item Creating a Practical Byzantine Fault Tolerance implementation using async/await in combination with Cleipnirs reactive event handlers. This combination allowed for the our implementation to handle the main workflow of \ac{pbft} in a single function.
\item Learning how to use Cleipnir functionality to support asynchronous programming, reactive programming and persistent programming.
\item Providing relevant feedback for Cleipnir.
\end{itemize}

\section{Outline}
\begin{itemize}
\item In \textbf{\autoref{chapter:ProgrammingModels}} we briefly describe the background information in regards to this thesis. This includes information in regards to asynchronous programming and reactive programming

\item In \textbf{\autoref{chapter:Cleipnir}} we make an introduction to the Cleipnir framework. This includes the intended use-case for Cleipnir, and a summary of its functionality that are helpful for implementing the consensus algorithm. 

\item In \textbf{\autoref{chapter:PBFT}} we describe \ac{pbft} algorithm. This includes introducing the main goals and functionality of the consensus algorithm. Concepts used by or related to the algorithm. Finally, a detailed summary of all the operations taking place in the algorithm.

\item In \textbf{\autoref{chapter:RW}} we introduce previous work in regards to the Cleipnir framework and other related work that are similar in nature to this project.

\item \textbf{\autoref{chapter:Design}} introduces the overall design of the \ac{pbft} implementation.

\item \textbf{\autoref{chapter:Imp}} more detailed explanation of parts of the of the \ac{pbft} implementation workflow, important factors in the overall implementation that differentiate drastically from the a synchronous design. Explanation of how the benchmarking/results were obtained, describe the system framework, how did you run and test your system.

\item \textbf{\autoref{chapter:Dis}} introduces the overall results of (something I haven't done yet probably locally, docker and possibly servers in the university) afterwards we discuss how the results affect our judgement of the overall implementation. Afterwards based on the result we evaluate our current the \ac{pbft} implementation based on given criteria.

\item \textbf{\autoref{chapter:Con}} is the last chapter and it contains a conclusion for the given \ac{pbft} implementation based on criteria from \autoref{chapter:Dis}. Furthermore, we summarize the knowledge we accumulated during the thesis and finally give suggestions for potential future work.
\end{itemize}