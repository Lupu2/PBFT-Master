\chapter{Conclusion}
\label{chapter:Con}
\iffalse
This chapter concludes the thesis by first listing the lessons we learned while working on this thesis. Then we list the potential future work which can be applied to the \ac{pbft} implementation.
Finally a conclusion is drawn for the work performed for this thesis.

\section{Lessons Learned}
\iffalse
-PBFT
-Asynchronous programming with C#, Task architecture
-Reactive Programming basics
-Overall knowledge for Cleipnir
-Issues and advantages in regards to the topics listed over. For instance a lot of time was wasted due to not fully grasping how Cleipnir work internally when performing the reactive part and the CAwaitable emission --> resulting a month of frustration trying to figure out why collision errors occur.
-Lack of documentation can be quite fatal for continued support.
-The multitude of potential issues that could occur that aren't necessary dealt with in the theoretical consensus algorithm or pseudo code.
-Cleipnir and how it interacts with the other programming paradigms. Eks: A clear distinction has to made in regards to what code is run inside Cleipnir(the persistent part) and what is not called in Cleipnir (orthogonal part), mixing these will cause disastrous results, which we infact encountered several times during implementation.
%-Unit testing, simplicity of C# unit testing, issues in regards to unit testing networking as running tests in parallel causes inconsistent results and at worst case inf-loops(don't think this is really all that useful)
\fi
%first draft, probably be heavly changed after writing the other parts of the thesis

\subsection{Consensus algorithm}
At the start of this thesis our knowledge in regards to consensus algorithms were limited to practical analysis of the Paxos algorithm. We had never encountered any information in regards to the PBFT consensus algorithm, therefore some time needed to be spent on learning the inner workings of the PBFT algorithm. In addition, Cleipnir had already been used to implement the Paxos and Raft consensus algorithms, therefore some time was also spent on understanding the basics of the Raft consensus algorithm. The transition from one consensus algorithm when looking solely on the protocol descriptions is not all that complicated. This is mostly due to similarities found in their functionality. Components used to implement a functional consensus algorithm are usually shared by many consensus algorithms. This in turn makes it easier for someone familiar with one algorithm to understand another. An example of this being that all three previous mentioned algorithms uses an election model in order to make a decision over the network. Furthermore one party in the election is given the leader role and is therefore responsible for governing the election process. Hence understanding the basic principles behind the PBFT algorithm through the project description was not challenging. 

However, consensus algorithms are notoriously difficult to implement. This is because the protocol description are by design written to as simple as possible, otherwise developers would have issues fully understand the basics on how they operate. This can unfortunately lead to some information being omitted, which can cause issues when designing an implementation for the algorithm. This was especially apparent for our implementation, since the goal our goal was to make the protocol workflow as simplistic as possible. Several times during development new issues became apparent in regards specific scenarios or circumstances occurring during the protocol workflow. This was especially relevant when thinking of all the different issues that could potentially occur when a restarted replica with out of date persisted state attempted to collaborate with the other updated replicas. 

In most of these cases we had to decide whether or not it be worth to introduce additional complexity to the implementations in order to handle these issues, or to simply try to avoid them. In most realistic scenarios the obvious choice would be to fix the issue, even if it adds more complexity to your system. Unfortunately, since our goal was to attempt to implement a very simple implementation, in addition to upholding a time constraint, we had to prioritize differently. Which in turn made our implementation less desirable compare to other more complex implementations. In short our experience working on implementing the PBFT algorithm led us to believe that the largest difficulty in regards to the implementation of consensus algorithms does not necessarily lie lack of understanding the technicalities within the consensus algorithm, but rather in how to implement the protocol so that it can handle every possible errornous situation that may occur.

\subsection{Asynchronous programming}
Going into this thesis our experience using asynchronous programming were limited and were solely based on a few previous projects. In addition the asynchronous programming used in these previous projects were using the JavaScript asynchronous framework. Although the language barrier between the asynchronous tools were minimal, there were a few subtle differences. The naming conventions being different for similar operations were especially annoying. An example of this being how C\# \code{Task} not being far from JavaScript \code{Promise}. Overall since both the asynchronous frameworks supports the use of the async/await operators, programming asynchronous workflow were relatively similar. 

On the other hand there were issues encountered in our application due to lack of understanding behind the details for the async and await operators early in development. Originally our application used asynchronous programming for a lot tasks related to both networking and protocol handling which caused a lot of internal nested state machines being created. Not only was it a pain to attempt to debug issues regarding nested state machines, but it further escalated when nested async/await operators were used inside \code{CTask}'s for normal \code{Task}, which created additional threads. The result being a lot of race conditions, inconsistent states and generally a nightmare to debug. The simple solution was to make any unnecessary asynchronous task become synchronously operations, which in turn removed a lot of the nested state machines as well as removing the \code{CTask} threading issue.

In short, due to our over usage of async/await workflow for tasks that didn't necessarily needed to be asynchronous lead to issues for our application. Therefore, it is important when designing an asynchronous application to have a clear view over which computing tasks requires asynchronous workflow and which can be satisfied by synchronous workflow. Using asynchronous programming for tasks were it is not needed only causes extra complexity to the code and as a result is harder to debug and also unnecessarily slows down the system.

\subsection{Reactive programming}
At the beginning of this project, we had very little to no previous experience in regards to reactive programming. Therefore it became quite the challenge learning the basics for reactive programming. Specifically the main challenge became using the basics for reactive programming in order to understand Cleipnir's reactive functionality. Majority of the documentation and tutorials around the web in regards to reactive programming focused mostly on the basics and the corner stone used for implementing their own reactive operators. This did not quite translate well for our project as all of the reactive layer as already implemented in Cleipnir. Cleipnir reactive functionality in itself is very easy to use and is not all that hard to learn. However, making a direct comparison to the official reactive documentation~\cite{WEB:ReactiveXMainPage} and Cleipnir.Rx was not so simple. This mostly due to the cornerstones having different name schemes between the two. (add more stuff here later)

In terms of our experience using the Cleipnir reactive layer it exceptionally easy to use once you learned the basics. Although Cleipnir currently lacks support for the majority of the reactive operators listed in the documentation, the current support still has the majority of the most used reactive operators. During development only a single time did we encounter an issue in which we needed a reactive operator that was currently not supported. Thankfully Thomas added that missing reactive operator within a single day, essentially proving that Cleipnir's current design allows for developers to easily add missing reactive operators should the need ever arise. As for the usage of the reactive paradigm in the protocol workflow. The code operations performed over the reactive streams works well and easy to keep track of due to how simple it is to chain reactive operators. Not to mention quite easy to read. On the other hand, chaining reactive operations can be somewhat restricting in some circumstances. The most troublesome issue encountered in regards to working with reactive operators was to handle exceptions to the protocol workflow. In our case it was stopping the reactive operators in the case where a view-change occurred. When the program is required to wait for a reactive operator to finish it is required to wait until all of the reactive operators have finished in the stream, making it very easy to get stuck when the source doesn't get the desired items to the stream. There are two notable workarounds to this problem. The first is to simply ignore the problem since it source objects should only be listened to in \code{CTask} asynchronous functions, therefore it won't block the main execution thread even if it never finishes all of the reactive operators. If the reactive operators has strict \code{Where} clauses the old listeners won't be effected by new items, since the items are filtered out long before it can effect the program in any way. This means the workaround is essentially just letting the listener run stuck until it is eventually garbage collected while instead move on by creating new \code{CTask} instead. The second workaround uses the \code{Merge} operator to have the listener listen to changes on two different streams. By this method it is possible to effectively terminate the listener if it receives an item from the second source, as this is counted as irregular activity. This is the method used in our PBFT implementation to handle exiting existing instances of the PBFT workflow in order to change view for the system. This workaround also has its fair share of issues. In order to use the \code{Merge} operator it would require both the source objects to listen for the same type of object. This is not always easy to coordinate, especially when the other operators for the listener transforms the stream to work on another object type. In addition, the \code{Merge} operator also works like any other operator. If the \code{Merge} is triggered by the other source object and the operator is called early in the stream, than the item received is still required to pass the other operators in order to terminate the listener. Which puts it back to the state of the original problem. The item received by the other source must also be unique so that the rest of the workflow can terminate the process when it receives item from that other \code{Source} object. 

To summarize, the use of reactive handlers works well for segmenting operations to perform for the consensus algorithm when a new event is received in the network layer. In addition it is relatively easy for developers to use and is a lot easier to read the workflow in comparison to traditional programming. However, reactive handlers can be tricky to deal with when used in protocol workflow's that needs to handle execptions to the normal workflow. As consensus algorithms must handle situations where parties on the network stops responding, this can become a rather frequent issue. It therefore would be most beneficial if additional workarounds where discovered for handling this issue.  

\subsection{Cleipnir}


\section{Future Work}
%1 mention fixing the broken public key system, and give examples(if you can think of any).
%2 potential things: 1. fix it so that the protocol workflow can handle any message being received in any order or 2. Implement a timeout process for the commit section so that the process can not become stuck in any scenario. Both are somewhat needed, but I have no idea how I can handle waiting for prepares messages before receiving the pre-prepare.
%3 Fix persistency. We believe we layed most of the foundation in regards to getting the system to be persistent. However, as mention in imp 2 issues are mainly present. 1. figure out a way to get rid of the original source referanse/access the original source reference so there is no longer duplicate requests. 2. Something is wrong with the synchronization. Not sure what is the cause, assume its the synchronization process is not fully finished before a new request is added.
% 4 Generally make the application and client more interesting. Currently the application state is a simple list of commands written to the console. Make the application actually perform a set of commands, and redesign the client to accomodate for this change.

As mentioned in \autoref{chapter:Design} our current cryptographic signature architecture is susceptible to impersonation and sybil attacks. Clearly keeping public keys ephemeral and generating them uniquely before start up is not a smart design when the system supports persistency. Creating static private and public keys is also not recommended as this would make the system less secure. The simplest solution would be to generate a couple unique key pairs for each replica and have these stored securely or given to the system by a separate trusted system. This system could for instance be a database where the cryptographic keys are stored encrypted. During system startup or during certain scenarios, such as view-changes and or system restarts, the replica reassigns its current cryptographic key pairs and re-establishes its secret key with the other replicas in the system. The other replicas only accept the renewed connection if the separate system acknowledges that the public key given matches one of the unique public keys that replica can have.

Currently we are using a digital signatures scheme for all message types, with the exception of the session messages, which is unnecessary and only slows down the system. The desired alternative is to follow the original \ac{pbft} system model and use \ac{mac} for authentication instead, as this would be more efficient. Although, we still recommend to continue to use the digital signature structure for view-change and new-view messages. Otherwise the view-change workflow would need to be redesigned to follow the more advanced workflow described in Castro's and Liskov's updated paper for \ac{pbft}~\cite[p.~410-414]{PAPER:PBFTRecovery}.

The protocol workflow currently suffers from the inability to handle pre-prepare and prepare being received out of order. In addition, prepare messages can also be lost if the message is received before the prepare listener is initialized. As described in \autoref{sec:protocolwork} this issue can cause the workflow to become stuck if too many prepare messages are lost while the workflow waits for a pre-prepare message. This is obviously something that should be corrected if the application is to be used in the future. One workaround to this problem would be to have a timeout functionality active during the period where the workflow waits for the desired number of prepare and commit messages. The timeout is stopped if both the reactive listeners have successfully created both protocol certificates. Otherwise the timeout expires and the reactive listeners are terminated using the same functionality used for the pre-prepare listener. In order for this functionality to be possible, another \code{Source} object would need to be added as the reactive stream used for reactive listeners for the prepare and commit message is of type \code{Stream<CList<PhaseMessage>~>} due to the stream being transformed by the \code{Scan} operator.

Solving the actual message order issue is a lot more difficult. It is not as simple as to initialize the prepare listener earlier as the listener needs to filter away any phase message that has a different sequence number than its current iteration. Unfortunately, non-primary sets the current sequence number based on the received pre-prepare, creating quite the conundrum. One solution to this problem would be for the server to store copies of the phase messages received in the network layer. By having this logger store the list of phase messages within a dictionary, it would be possible for the workflow to easily search for missing phase messages. Obviously the phase message records would be garbage collected once the protocol has successfully created the two desired protocol certificates for a given sequence number. This would however cause additional complexity to the protocol workflow as functionality for looking up and re-emitting lost phase messages would need to be added.

Currently our application does not fully support persistency. In the future it would be beneficial for both Cleipnir and our application if the issues described in \autoref{chapter:Imp} can be fixed to allow for our application to fully test Cleipnir's capability in regards to persistency. All of the groundwork has been laid for the application to work with persistency. This includes assigning all protocol object types their proper serialization and deserialization for Cleipnir to use, which have been tested on a smaller scale and works as intended. In addition, the network functionality for replicas to reconnect to the system has already been implemented and tested. The only thing left is for the system to successfully read the data stored by Cleipnirs storage engine and successfully restore its old state.
There are at least two notable issues that must be fixed in order for the application to become persistent. The first issue is that the original \code{Source} objects are duplicated by having Cleipnir somehow restore the original \code{Source} while also creating the desired new copy which was supposed to replace the old. Currently both \code{Source} objects react whenever new items are emitted to them by the network layer, meaning that for the protocol workflow, two iterations are created for a single sequence number. This in turn creates issues for the logger when multiple records for the same sequence number is stored. The second issue is that the logger synchronization isn't working properly and as a result records in the logger disappear after the replica restarts. This issue is most likely due to the synchronization either not being fully finished before moving with other operations or the synchronization is not done properly and as a result, some records are skipped. We assume this issue is caused by incorrect usage of \code{Sync} points set for Cleipnir, resulting in the state not being persisted correctly. As for the duplicate \code{Source} objects we are frankly not quite sure how this issue occurs. We theorize that it may occur due to some records being persisted in multiple objects, which in turn when persisted are not treated as the same \code{Source} object, leading to the duplicate issue. If this is the case, the issue would lie in the relationship between the server and the protocol workflow.

Generally the application functionality could be a lot more advanced than it is now. Currently the only operation the application performs after a request is processed successfully though the \ac{pbft} algorithm is simply printing the message attached to the request to the console window. The message is then added to a \code{CList} representing the state of the system. In the future it would be beneficial if the application functionality was changed to be a bit more practical. For instance, changing the message content in the request to instead be an operation which is performed by the application. The state list would then instead store a record of the operation performed as well as whether or not the application was able to perform the requested operation. In order to change the application functionality, the client functionality for creating requests must also be adjusted.

\section{Conclusion}
\iffalse
%clearly state that you accomplish the goal of the thesis.
%Clearly state the answer to the main research question
%Summarize and reflect on the research done for the thesis. In our case discoveries you've made based on usage of Cleipnir + async
%future work + what you have learned --> seperate segments
In conclusion we achieved our goal of creating a \ac{pbft} implementation using Cleipnir with the intended focus of making it faithful to the protocol description which also takes advantage asynchronous and reactive programming paradigms.
Original goal: Our goal for this thesis is to use the Cleipnir framework to implement the Practical Byzantine Fault Tolerance (PBFT) consensus algorithm using functionality from both asynchronous programming and reactive programming. The desired PBFT implementation
\fi
In conclusion, we achieved our goal of creating a simplistic \ac{pbft} implementation using Cleipnir with the intended focus of making it faithful to the protocol description, which also is designed to take advantage of asynchronous and reactive programming paradigms. The result is \ac{pbft} implementation that can perform the \ac{pbft} protocol over several multiple clients and has functional checkpoint and view-change functionality. We managed to design a normal workflow that fit our original criteria, but unfortunately, the protocol struggles with handling out-of-order protocol messages. The checkpoint and view-change workflow became too complex for the processes to be handled within a single function. Persistency functionality was sadly not successful for our \ac{pbft} implementation. Asynchronous programming is shown to be helpful when designing consensus algorithms. Asynchronous programming was notably useful in regards to networking functionality and for designing multi-client protocol workflows.
Similarly, reactive programming turns out to be fairly helpful for handling the operations regarding protocol messages and other event-based processes. Reactive programming, however, did appear to struggle with protocol message ordering when using a synchronous design. These two programming paradigms showed quite clearly that they work well together. We believe implementation consensus algorithms can be further simplified using these tools in the future, despite the problems addressed in this thesis. In regards to the Cleipnir framework, we acknowledge that the overall workflow of the Cleipnir reactive framework is user-friendly and has, for the most part, the functionality desired for designing a proper event handler for a consensus algorithm. We were unsuccessful in evaluating Cleipnir’s persistency functionality on our application. However, based on our experience with using the hybrid persistency functionality on our implementation. In addition to testing the persistency functionality for smaller parts of the program, we deem Cleipnir`s persistency functionality to be excellent.
To conclude this thesis, we do believe that the tools we have tested and evaluated during our \ac{pbft} implementation do make it easier to design consensus algorithms. In the future, we believe that consensus algorithms can be implemented simpler and more accurately to the protocol description. However, we acknowledge that due to the complex nature of distributed systems, it will be challenging to create accurate consensus algorithm implementations due to the numerous problems that can occur.
\fi

This chapter concludes the thesis by first listing the lessons we learned while working on the thesis. Then we list the potential future work which can be applied to the \ac{pbft} implementation.
Finally, a conclusion is drawn for the work performed for this thesis.

\section{Lessons Learned}
\iffalse
-PBFT
-Asynchronous programming with C#, Task architecture
-Reactive Programming basics
-Overall knowledge for Cleipnir
-Issues and advantages in regards to the topics listed over. For instance a lot of time was wasted due to not fully grasping how Cleipnir work internally when performing the reactive part and the CAwaitable emission --> resulting a month of frustration trying to figure out why collision errors occur.
-Lack of documentation can be quite fatal for continued support.
-The multitude of potential issues that could occur that aren't necessary dealt with in the theoretical consensus algorithm or pseudo code.
-Cleipnir and how it interacts with the other programming paradigms. Eks: A clear distinction has to made in regards to what code is run inside Cleipnir(the persistent part) and what is not called in Cleipnir (orthogonal part), mixing these will cause disastrous results, which we infact encountered several times during implementation.
%-Unit testing, simplicity of C# unit testing, issues in regards to unit testing networking as running tests in parallel causes inconsistent results and at worst case inf-loops(don't think this is really all that useful)
\fi
%first draft, probably be heavly changed after writing the other parts of the thesis

\subsection{Consensus algorithm}
At the start of this thesis our knowledge in regards to consensus algorithms were limited to practical analysis of the Paxos algorithm. We had never encountered any information in regards to the PBFT consensus algorithm, therefore some time needed to be spent on learning the inner workings of the PBFT algorithm. In addition, Cleipnir had already been used to implement the Paxos and Raft consensus algorithms, therefore some time was also spent on understanding the basics of the Raft consensus algorithm. The transition from one consensus algorithm when looking solely on the protocol descriptions is not all that complicated. This is mostly due to similarities found in their functionality. Components used to implement a functional consensus algorithm are usually shared by many consensus algorithms. This in turn makes it easier for someone familiar with one algorithm to understand another. An example of this being that all three previous mentioned algorithms uses an election model in order to make a decision over the network. Furthermore one party in the election is given the leader role and is therefore responsible for governing the election process. Hence understanding the basic principles behind the PBFT algorithm through the project description was not challenging. 

However, consensus algorithms are notoriously difficult to implement. This is because the protocol description are by design written to as simple as possible, otherwise developers would have issues fully understand the basics on how they operate. This can unfortunately lead to some information being omitted, which can cause issues when designing an implementation for the algorithm. This was especially apparent for our implementation, since the goal our goal was to make the protocol workflow as simplistic as possible. Several times during development new issues became apparent in regards specific scenarios or circumstances occurring during the protocol workflow. This was especially relevant when thinking of all the different issues that could potentially occur when a restarted replica with out of date persisted state attempted to collaborate with the other updated replicas. 

In most of these cases we had to decide whether or not it be worth to introduce additional complexity to the implementations in order to handle these issues, or to simply try to avoid them. In most realistic scenarios the obvious choice would be to fix the issue, even if it adds more complexity to your system. Unfortunately, since our goal was to attempt to implement a very simple implementation, in addition to upholding a time constraint, we had to prioritize differently. Which in turn made our implementation less desirable compare to other more complex implementations. In short our experience working on implementing the PBFT algorithm led us to believe that the largest difficulty in regards to the implementation of consensus algorithms does not necessarily lie lack of understanding the technicalities within the consensus algorithm, but rather in how to implement the protocol so that it can handle every possible errornous situation that may occur.

\subsection{Asynchronous programming}
Going into this thesis our experience using asynchronous programming were limited and were solely based on a few previous projects. In addition the asynchronous programming used in these previous projects were using the JavaScript asynchronous framework. Although the language barrier between the asynchronous tools were minimal, there were a few subtle differences. The naming conventions being different for similar operations were especially annoying. An example of this being how C\# \code{Task} not being far from JavaScript \code{Promise}. Overall since both the asynchronous frameworks supports the use of the async/await operators, programming asynchronous workflow were relatively similar. 

On the other hand there were issues encountered in our application due to lack of understanding behind the details for the async and await operators early in development. Originally our application used asynchronous programming for a lot tasks related to both networking and protocol handling which caused a lot of internal nested state machines being created. Not only was it a pain to attempt to debug issues regarding nested state machines, but it further escalated when nested async/await operators were used inside \code{CTask}'s for normal \code{Task}, which created additional threads. The result being a lot of race conditions, inconsistent states and generally a nightmare to debug. The simple solution was to make any unnecessary asynchronous task become synchronously operations, which in turn removed a lot of the nested state machines as well as removing the \code{CTask} threading issue.

In short, due to our over usage of async/await workflow for tasks that didn't necessarily needed to be asynchronous lead to issues for our application. Therefore, it is important when designing an asynchronous application to have a clear view over which computing tasks requires asynchronous workflow and which can be satisfied by synchronous workflow. Using asynchronous programming for tasks were it is not needed only causes extra complexity to the code and as a result is harder to debug and also unnecessarily slows down the system.

\subsection{Reactive programming}
At the beginning of this project, we had very little to no previous experience in regards to reactive programming. Therefore it became quite the challenge learning the basics for reactive programming. Specifically the main challenge became using the basics for reactive programming in order to understand Cleipnir's reactive functionality. Majority of the documentation and tutorials around the web in regards to reactive programming focused mostly on the basics and the corner stone used for implementing their own reactive operators. This did not quite translate well for our project as all of the reactive layer as already implemented in Cleipnir. Cleipnir reactive functionality in itself is very easy to use and is not all that hard to learn. However, making a direct comparison to the official reactive documentation~\cite{WEB:ReactiveXMainPage} and Cleipnir.Rx was not so simple. This mostly due to the cornerstones having different name schemes between the two. (add more stuff here later)

In terms of our experience using the Cleipnir reactive layer it exceptionally easy to use once you learned the basics. Although Cleipnir currently lacks support for the majority of the reactive operators listed in the documentation, the current support still has the majority of the most used reactive operators. During development only a single time did we encounter an issue in which we needed a reactive operator that was currently not supported. Thankfully Thomas added that missing reactive operator within a single day, essentially proving that Cleipnir's current design allows for developers to easily add missing reactive operators should the need ever arise. As for the usage of the reactive paradigm in the protocol workflow. The code operations performed over the reactive streams works well and easy to keep track of due to how simple it is to chain reactive operators. Not to mention quite easy to read. On the other hand, chaining reactive operations can be somewhat restricting in some circumstances. The most troublesome issue encountered in regards to working with reactive operators was to handle exceptions to the protocol workflow. In our case it was stopping the reactive operators in the case where a view-change occurred. When the program is required to wait for a reactive operator to finish it is required to wait until all of the reactive operators have finished in the stream, making it very easy to get stuck when the source doesn't get the desired items to the stream. There are two notable workarounds to this problem. The first is to simply ignore the problem since it source objects should only be listened to in \code{CTask} asynchronous functions, therefore it won't block the main execution thread even if it never finishes all of the reactive operators. If the reactive operators has strict \code{Where} clauses the old listeners won't be effected by new items, since the items are filtered out long before it can effect the program in any way. This means the workaround is essentially just letting the listener run stuck until it is eventually garbage collected while instead move on by creating new \code{CTask} instead. The second workaround uses the \code{Merge} operator to have the listener listen to changes on two different streams. By this method it is possible to effectively terminate the listener if it receives an item from the second source, as this is counted as irregular activity. This is the method used in our PBFT implementation to handle exiting existing instances of the PBFT workflow in order to change view for the system. This workaround also has its fair share of issues. In order to use the \code{Merge} operator it would require both the source objects to listen for the same type of object. This is not always easy to coordinate, especially when the other operators for the listener transforms the stream to work on another object type. In addition, the \code{Merge} operator also works like any other operator. If the \code{Merge} is triggered by the other source object and the operator is called early in the stream, than the item received is still required to pass the other operators in order to terminate the listener. Which puts it back to the state of the original problem. The item received by the other source must also be unique so that the rest of the workflow can terminate the process when it receives item from that other \code{Source} object. 

To summarize, the use of reactive handlers works well for segmenting operations to perform for the consensus algorithm when a new event is received in the network layer. In addition it is relatively easy for developers to use and is a lot easier to read the workflow in comparison to traditional programming. However, reactive handlers can be tricky to deal with when used in protocol workflow's that needs to handle execptions to the normal workflow. As consensus algorithms must handle situations where parties on the network stops responding, this can become a rather frequent issue. It therefore would be most beneficial if additional workarounds where discovered for handling this issue.  

\subsection{Cleipnir}


\section{Future Work}
As mentioned in \autoref{chapter:Design} our current cryptographic signature architecture is susceptible to impersonation and spoofing attacks. Clearly, keeping public keys ephemeral and generating them uniquely before start-up was not a smart design when the system supports persistency. Creating static private and public keys is also not recommended since this design would make the system less secure. One solution would be to generate a couple of unique key pairs for each replica and have these stored securely or given to the system by a separate trusted system. This system could, for instance, be a database where the cryptographic keys are stored encrypted. During system start-up or during certain scenarios, such as view-changes and or system restarts, the replica reassigns its current cryptographic key pairs and re-establishes its secret key with the other replicas in the system. The other replicas only accept the renewed connection if the separate system acknowledges that the public key given matches one of the unique public keys listed for that replica.

We are currently using a digital signatures scheme for all message types, except for the session messages. This is frankly unnecessary and only slows down the system. The desired alternative is to follow the original \ac{pbft} system model and use \ac{mac} for authentication instead, as this would be more efficient. Although, we still recommend continuing to use the digital signature structure for view-change and new-view messages. Otherwise, the view-change workflow would need to be redesigned to follow the more advanced workflow described in Castro’s and Liskov’s updated paper for \ac{pbft}~\cite[p.~410-414]{PAPER:PBFTRecovery}.

The protocol workflow currently suffers from the inability to handle pre-prepare and prepare being received out of order. In addition, prepare messages can also be lost if the message is received before the prepare listener is initialized. As described in \autoref{sec:protocolwork} this issue can cause the workflow to become stuck if too many prepare messages are lost while the workflow waits for a pre-prepare message. This is something that should be corrected if the application is to be used in the future. A workaround to this problem would be to have a timeout functionality active when the workflow waits for the desired number of prepare and commit messages. The timeout is stopped if both the reactive listeners have successfully created both protocol certificates. Otherwise, the timeout expires, and the reactive listeners are terminated using the same functionality used for the pre-prepare listener. For this functionality to be possible, another \code{Source} object would need to be added to the workflow to work with the \code{Merge} operator. This is because the reactive stream for reactive listeners to the prepare and commit message is of type \code{Stream<CList<PhaseMessage>~>} due to the stream being transformed by the \code{Scan} operator.

Solving the actual message ordering issue is a lot more complicated. It is not as simple as initializing the prepare listener earlier, as the listener needs to filter away any phase message with a different sequence number than its current iteration. Unfortunately, non-primary replicas set the current sequence number based on the received pre-prepare message, creating quite the conundrum. A solution to this problem is making the server store copies of the phase messages received in the network layer. By having this logger store a list of phase messages received for a sequence number within a dictionary, it would be possible for the workflow to easily search for missing phase messages. The phase message records stored in this logger would have to be garbage collected once the protocol has successfully created the two desired protocol certificates for the given sequence number. However, this would cause additional complexity to the protocol workflow as functionality for looking up, and re-emitting lost phase messages would need to be added.

Currently, our application does not fully support persistency. In the future, it would be favorable for both Cleipnir and our application if the issues described in \autoref{chapter:Imp} can be fixed to allow for our application to thoroughly test Cleipnir’s capability in regards to persistency. The groundwork has been laid for the application to work with persistency. This includes assigning all protocol object types their proper serialization and deserialization functions for Cleipnir to use, which have been tested on a smaller scale and works as intended. In addition, the network functionality for replicas to reconnect to the system has already been implemented and tested. The only thing left is for the system to successfully read the data stored by Cleipnirs storage engine and successfully restore its old state.
There are at least two notable issues that must be fixed for the application to become persistent. The first issue is that the original \code{Source} objects are duplicated by having Cleipnir somehow restore the original \code{Source} while also creating the desired new copy, which was supposed to replace the old. Currently, both \code{Source} objects react whenever new items are emitted to them by the network layer, meaning that for the protocol workflow, two iterations are created for a single sequence number. This, in turn, creates issues for the logger when multiple records for the same sequence number are stored. The second issue is that the logger synchronization isn’t working properly and as a result, records in the logger disappear after the replica restarts. This issue likely due to the synchronization not being fully finished before moving with other operations, or the synchronization is not done correctly, and as a result, some records are skipped. We assume this issue is caused by incorrect usage of \code{Sync} points set for Cleipnir, resulting in the state not being persisted correctly. As for the duplicate \code{Source} objects, we are frankly not quite sure how this issue occurs. We theorize that it may occur due to some records being persisted in multiple objects, and because of this, when the objects are persisted, the objects are not treated as the same \code{Source} object, leading to the duplicate \code{Source} object. If this is the case, the issue lies in the relationship between the server and the protocol workflow.

Generally, the application functionality could be a lot more advanced than it is now. Currently, the only operation the application performs after a request is processed successfully through the \ac{pbft} algorithm is simply printing the message attached to the request to the console window. The message is then added to a \code{CList} representing the state of the system. In the future, it would be beneficial if the application functionality was changed to be a bit more practical. For instance, changing the message content in the request to be an operation that is to be performed by the application instead of a string. The state list would then rather store a record of the operation performed and whether or not the application successfully performed the requested operation. In order to change the application functionality, the client functionality for creating requests must also be adjusted.

\section{Conclusion}
In conclusion, we achieved our goal of creating a simplistic \ac{pbft} implementation using Cleipnir with the intended focus of making it faithful to the protocol description, which also is designed to take advantage of asynchronous and reactive programming paradigms. The result is \ac{pbft} implementation that can perform the \ac{pbft} protocol over several multiple clients and has functional checkpoint and view-change functionality. We managed to design a normal workflow that fit our original criteria, but unfortunately, the protocol struggles with handling out-of-order protocol messages. The checkpoint and view-change workflow became too complex for the processes to be handled within a single function. Persistency functionality was sadly not successful for our \ac{pbft} implementation. Asynchronous programming is shown to be helpful when designing consensus algorithms. Asynchronous programming was notably useful in regards to networking functionality and for designing multi-client protocol workflows.
Similarly, reactive programming turns out to be fairly helpful for handling the operations regarding protocol messages and other event-based processes. Reactive programming, however, did appear to struggle with protocol message ordering when using synchronous design. These two programming paradigms showed quite clearly that they work well together. We believe implementation consensus algorithms can be further simplified using these tools in the future, despite the problems addressed in this thesis. In regards to the Cleipnir framework, we acknowledge that the overall workflow of the Cleipnir reactive framework is user-friendly and has, for the most part, the functionality desired for designing a proper event handler for a consensus algorithm. We were unsuccessful in evaluating Cleipnir’s persistency functionality on our application. However, based on our experience with using the hybrid persistency functionality on our implementation. In addition to testing the persistency functionality for smaller parts of the program, we deem Cleipnir’s persistency functionality to be excellent.
To conclude this thesis, we do believe that the tools we have tested and evaluated during our \ac{pbft} implementation do make it easier to design consensus algorithms. In the future, we believe that consensus algorithms can be implemented simpler and more accurately to the protocol description. However, we acknowledge that due to the complex nature of distributed systems, it will be challenging to create accurate consensus algorithm implementations due to the numerous problems that can occur.

