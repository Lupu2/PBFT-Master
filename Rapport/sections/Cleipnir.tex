\chapter{Cleipnir}
\label{chapter:Cleipnir}
\iffalse
Cleipnir is a .NET framework primarily designed to be used for aiding in implementing consensus algorithms. Specifically, the framework's main contribution is aiding developers with creating persistent distributed systems. Prior to this thesis, Cleipnir and its predecessor Corums, have been used to implement two consensus algorithms, namely Paxos~\cite[p.~32-38]{PAPER:EivindPaper} and Raft~\cite[p.~13-15]{PAPER:PaxosCleipnir}.
Cleipnir is designed to support and work with the three following programming paradigms~\cite[p.~5]{PAPER:PaxosCleipnir}:
\begin{itemize}
\item {Reactive Programming}
\item {The Async/Await Model}
\item {Persistent Programming}
\end{itemize}

Two of these programming paradigms were already presented in \autoref{chapter:ProgrammingModels}, therefore only the Persistent Programming paradigm is introduced in this chapter. The async/await model used in Cleipnir is the official implementation from the .NET framework~\cite{DOC:AsyncAwait}. As mentioned in \autoref{section:reactive}, Cleipnir uses its own customized reactive framework and this framework is discussed in detail in this chapter.

The information that is presented in this chapter is based on the Cleipnir paper~\cite{PAPER:PaxosCleipnir}, its current documentation~\cite{DOC:Cleipnir} and from informative conversations with the frameworks creator Thomas Stidsborg Sylvest.

%Motivation
\section{Cleipnir Overview}
\label{section:CleipnirOv}
There are three main tools that Cleipnir provides developers to help design their application.
These three tools are:
\begin{itemize}
	\item{Persistent Synchronous Scheduler}
	\item{Storage Engine}
	\item{Object Store}
	\item{Reactive Programming Layer}
\end{itemize}
%Scheduler
Cleipnir uses an inbuilt event-driven \textbf{scheduler} which follows a single threaded structure similar to the JavaScript scheduler~\cites[p.~7]{PAPER:PaxosCleipnir}{WEB:CleipnirScheduler}. The scheduler schedules incoming tasks in a queue structure, meaning the ordering follows a \ac{fcfs}~\cite{WEB:FIFO} approach. Each task in the queue is executed sequentially using only a single thread, which in theory allows the program to avoid common threading issues~\cite[p.~7]{PAPER:PaxosCleipnir}. %find the actual phrase for this

The \textbf{storage engine} is responsible for the actual storage procedure. It is responsible for performing both the serialization and the deserialization process to each of the state objects that is to be persisted. The details in regards to setting up object information for the serialization and deserialization process that the storage engine uses is presented in \autoref{section:PersistentProgramming}. Cleipnir uses different storage engines that are correlated to which storage is used to store the data. Cleipnir currently supports these three storage engines:
\begin{itemize}
	\item{Memory Storage}
	\item{Simple File Storage}
	\item{Relational Database Storage}
\end{itemize}
\cites[p.~10,12]{PAPER:PaxosCleipnir}

The memory storage stores the persisted data directly into memory.
The simple file storage stores the persisted data in a single text file.
The relation database storage stores the persisted data into a Microsoft \acs{sql} Server~\cites{WEB:MSSQL}[p~10-12]{PAPER:PaxosCleipnir}.

The Cleipnir serialization process follows a graph-like structure. The original object graph that is to be serialized is called a \code{Roots} object. In order to accomplish this the \code{Roots} graph object to the persisted object is connected to the graph object through pathways leading to the references that are to also be persisted~\cite[p.~10]{PAPER:PaxosCleipnir}.

%If the object that is to be serialized has references to other objects that are also going to be serialized, then the graph object has pathways leading from the \code{Roots} object to the other objects graph object~\cite[p.~10]{PAPER:PaxosCleipnir}.

The \textbf{object store} is responsible for accessing the storage engine. The object store uses the storage engine whenever the application needs to restore some data that was previously persisted, using the storage engine to persist a new object, or to update existing object records in the persisted memory. The object store is also responsible for detecting changes done to any state variables that are set to be persisted.
The object store uses a statemap to keep track of records for each of the state variables that are to be persisted and stored by the storage engine~\cite[p.~11]{PAPER:PaxosCleipnir}.

\autoref{code:objectstore} shows a short example of how to use the object store to cache an object into the storage engine and then restore the object after the data is lost in the application. First both the storage engine and the object store are initialized, where the type of storage engine used for this example is the simple file storage. Then the object store uses the \code{Attach} function to register the request object to the object store. Object store now has a \code{Roots} entry for the request object. In this example the \code{Persist} function is used to serialize and store any objects currently registered in the object store that have either not been cached before or if any changes have affected the object. In this example, only the request object with its system references that the objectstore has in its registry. The object store is intentionally reset by assigning it the value null. The object store is then reloaded by attaching it with our previous storage engine that we know has the request object in its statemap. Therefore, using the \code{Resolve} function, we restore the request object and assign it to a new variable. Since the original request object was never tampered with, we know that the original request and the newly resolved request should be equal. Most of the functionality shown in \autoref{code:objectstore} is performed behind the scenes and a developer rarely has to attach an object to the object store directly. 
\fi

Cleipnir is a .NET framework primarily designed for aiding in implementing consensus algorithms. Specifically, the framework`s main contribution is assisting developers with creating persistent distributed systems. Prior to this thesis, Cleipnir and its predecessor Corums, have been used to implement two consensus algorithms, namely Paxos~\cite[p.~32-38]{PAPER:EivindPaper} and Raft~\cite[p.~13-15]{PAPER:PaxosCleipnir}.
Cleipnir is designed to support and work with the three following programming paradigms~\cite[p.~5]{PAPER:PaxosCleipnir}:
\begin{itemize}
\item {Reactive Programming}
\item {The Async/Await Model}
\item {Persistent Programming}
\end{itemize}

Two of these programming paradigms were already presented in \autoref{chapter:ProgrammingModels}. Therefore, only the Persistent Programming paradigm is introduced in this chapter. The async/await model used in Cleipnir is the official implementation from the .NET framework~\cite{DOC:AsyncAwait}. As mentioned in \autoref{section:reactive}, Cleipnir uses a custom-built reactive framework, and this framework is discussed in detail in this chapter.

The information presented in this chapter is based on the Cleipnir paper~\cite{PAPER:PaxosCleipnir}, its current documentation~\cite{DOC:Cleipnir} and from informative conversations with the creator of Cleipnir, Thomas Stidsborg Sylvest.

%Motivation
\section{Cleipnir Overview}
\label{section:CleipnirOv}
There are three main tools that Cleipnir provides developers to help design their application.
These three tools are:
\begin{itemize}
	\item{Persistent Synchronous Scheduler}
	\item{Storage Engine}
	\item{Object Store}
	\item{Reactive Programming Layer}
\end{itemize}
%Scheduler
Cleipnir uses an inbuilt event-driven \textbf{scheduler} that follows a single-threaded structure similar to the JavaScript scheduler~\cites{WEB:CleipnirScheduler}[p.~7]{PAPER:PaxosCleipnir}. The scheduler schedules incoming tasks in a queue structure, meaning the ordering follows a \ac{fcfs}~\cite{WEB:FIFO} approach. Each task in the queue is executed sequentially using only a single thread, which in theory allows the program to avoid common threading issues~\cite[p.~7]{PAPER:PaxosCleipnir}.

The \textbf{storage engine} is responsible for the actual storage procedure. It is responsible for performing both the serialization and the deserialization process to each state object that is to be persisted. The details regarding setting up object information for the serialization and deserialization for a state object is presented in \autoref{section:PersistentProgramming}. Cleipnir uses different storage engines correlated to the kind of storage that is used to store the data. Cleipnir currently supports these three storage engines:
\begin{itemize}
	\item{Memory Storage}
	\item{Simple File Storage}
	\item{Relational Database Storage}
\end{itemize}
\cites[p.~10,12]{PAPER:PaxosCleipnir}

The memory storage stores the persisted data directly into memory.
The simple file storage stores the persisted data in a single text file.
The relation database storage stores the persisted data into a Microsoft \acs{sql} Server~\cites{WEB:MSSQL}[p~10-12]{PAPER:PaxosCleipnir}.

The Cleipnir serialization process follows a graph-like structure. The original object graph that is to be serialized is called a \code{Roots} object. In order to accomplish this, the \code{Roots} graph object to the persisted object is connected to the graph object through pathways leading to the references that are also to be persisted~\cite[p.~10]{PAPER:PaxosCleipnir}.

%If the object that is to be serialized has references to other objects that are also going to be serialized, then the graph object has pathways leading from the \code{Roots} object to the other objects graph object~\cite[p.~10]{PAPER:PaxosCleipnir}.

The \textbf{object store} is responsible for accessing the storage engine. The object store uses the storage engine whenever the application needs to restore some previously persisted data, using the storage engine to persist a new object or updating existing object records in the persisted memory. The object store is also responsible for detecting changes done to any state variables that are registered to be persisted.
The object store uses a statemap to keep track of records for each of the state variables that are to be persisted and stored by the storage engine~\cite[p.~11]{PAPER:PaxosCleipnir}.

\autoref{code:objectstore} shows a short example of how to use the object store to cache an object into the storage engine and then restore that object after the data is lost in the application. First, both the storage engine and the object store are initialized, where the type of storage engine used for this example is the simple file storage. Then the object store uses the \code{Attach} function to register the request object to the object store.  The object store now has a \code{Roots} entry for the request object. In this example, the \code{Persist} function is used to serialize and store the objects currently registered in the object store. Object store only persists objects that have either not been cached before or if any changes have affected the object when \code{Persist} is called. In this example, only the request object with its system references are registered by the object store. Therefore only the request object is persisted. In the example, the object store is intentionally reset by assigning it the value null to test that the request object is properly persisted. The object store is then reloaded by attaching it with the previous storage engine that we know has the request object in its statemap. Therefore, using the \code{Resolve} function, we restore the request object and assign it to a new variable. Since the original request object was never tampered with, we know that the initial request and the newly resolved request should be equal. Most of the functionality shown in \autoref{code:objectstore} is performed behind the scenes, and a developer rarely has to attach an object to the object store directly. 

\begin{figure}[H]
	\centering
	%\lstset{style=sharpc}
	\begin{lstlisting}[label = code:objectstore, caption=Object Store example, captionpos=b, basicstyle=\scriptsize]
_storage = new SimpleFileStorageEngine("PersistentStorage.txt");
_objectStore = ObjectStore.New(_storage);
(_pri, _pub) = Crypto.InitializeKeyPairs();	
var currentTime = DateTime.Now.ToString();
Request req1 = new Request(1, "Hello World!", currentTime);
req.SignMessage(_pri);
           
_objectStore.Attach(req1);
_objectStore.Persist();
_objectStore = null;
_objectStore = ObjectStore.Load(_storage, false);
Request req2 = _objectStore.Resolve<Request>();
	\end{lstlisting}
\end{figure}

%Objectstore + Storage engine
%Execution engine
\iffalse
The scheduler and the object store operate independently from each other. In order for an application to take advantage of both of these tools, Cleipnir has an execution engine tool which uses both the scheduler and the object store to the best of their abilities.
The Cleipnir execution engine's overall architecture is constructed so that the scheduler and the object store can be used together and can collaborate within a single mechanism. Using the execution engine the developer can specify the task's that are to be executed by the scheduler and use the object store to persist the state of the application during certain parts of the execution. The execution engine uses what is known as \code{Sync} points in order to determine when to call the \code{Persist} function in the object store. These points have to be added manually by the developer in areas where the state can potentially become corrupt if not the state is persistent after a crash. This is important for implementing a consensus algorithm since interrupting a process mid-execution and not having a stable state ready for when the system reboots can most likely lead to major issues within the state for the distributed system. By default, if the scheduler does not have any tasks in its queue and is not working on any existing tasks, then it should also call the \code{Persist} function so that the state can be saved during a silent period. \autoref{code:executionEngine} shows an example of how to initialize the execution engine and how to schedule an operation\cite[p.~11]{PAPER:PaxosCleipnir}.
\fi

The scheduler and the object store operate independently from each other. In order for an application to take advantage of both of these tools, Cleipnir has an execution engine that utilizes both tools to the best of their abilities.
The Cleipnir execution engine's overall architecture is constructed so that the scheduler and the object store can be used together and collaborates within a single mechanism. Using the execution engine, the developer can specify the tasks that are to be executed by the scheduler and use the object store to persist the state of the application during certain parts of the execution. The execution engine uses what is known as \code{Sync} points to determine when to call the \code{Persist} function in the object store. The developer has to manually add these points in areas where the state can become corrupt if not crucial information is persisted when a system crash occurs. This is important for persisted consensus algorithms as it needs a stable state to reboot to even if an active process was cut mid-execution.  This would otherwise guarantee significant consequences for the overall state of the distributed system.
By default, if the scheduler does not have any tasks in its queue and is not working on any existing tasks, then it should also call the \code{Persist} function so that it can save changes in the state during a silent period. \autoref{code:executionEngine} shows an example of how to initialize the execution engine and how to schedule an operation~\cite[p.~11]{PAPER:PaxosCleipnir}.


\begin{figure}[H]
	\centering
	%\lstset{style=sharpc}
	\begin{lstlisting}[label = code:executionEngine, caption=Execution engine example, captionpos=b, basicstyle=\scriptsize]
var storageEngine = new SimpleFileStorageEngine(".PBFTStorage"+paramid+".txt", false);
scheduler = ExecutionEngineFactory.StartNew(storageEngine);
scheduler.Schedule(() =>
{
...
});
	\end{lstlisting}
\end{figure}

\iffalse
\section{Cleipnir Reactive Programming}
%TODO rewrite this segment, it is basically all wrong!
The Cleipnir framework has a custom-made reactive layer that follows most of the functionality provided by the Reactive X \ac{api}. However, the basic functionality introduced in \autoref{section:reactive} is mostly hidden and the overall workflow is simplified for use. This implementation uses a \code{Stream} object in order to replicate the data returned by the respective observers and operators. The \code{Stream} is similar to an observable object.
Cleipnir reactive layer support fewer reactive operators compared to most other current reactive implementations. However, the current operators that exist also support persistent programming, meaning the data stream and the scheduled operations are not lost if the system crashes during an operation. Traditional \ac{linq} commands do not work on the \code{Stream} object. Instead inbuilt \ac{linq} statements are available for the reactive \code{Stream} object to use. Cleipnir reactive operators can by design be chained together just like the majority of reactive operators. The main difference being that the Cleipnir’s reactive operators and \ac{linq} operators result in a new \code{Stream} object instead of a new observable or a new enumerable. Using reactive operator chains, it is possible to create a lot of the consensus algorithms workflow within a few lines of code. \autoref{code:operatorreq} shows an example of operator chaining using Cleipnir’s reactive framework. The objective here is to get the first valid Preprepare message emitted to the observable. In order for a message to be considered valid, it must pass all \code{Where} clauses. The \code{Next} operator at the end of the chain returns the resulting prepare message~\cites[p.~6,8,13]{PAPER:PaxosCleipnir, WEB:ReactiveOperator} The \code{Merge} operator is used to listen for incoming items from the \code{ShutdownBridgePhase} \code{Source} object. More information in regards to the \code{Merge} operator is discussed in \autoref{chapter:Imp}
\fi

\section{Cleipnir Reactive Programming}
The Cleipnir framework has a custom-made reactive layer that follows most of the functionality provided by the Reactive X \ac{api}. However, the basic functionality introduced in \autoref{section:reactive} is mostly hidden, and the overall workflow is simplified to make it easier for developers to use the reactive framework. This implementation uses a \code{Stream} object to replicate the data returned by the respective observers and operators. The \code{Stream} is similar to an observable object.
Cleipnir reactive layer supports less reactive operators compared to most other current reactive implementations. However, the current operators that exist also support persistent programming, meaning the data stream and the scheduled operations are not lost if the system crashes during an operation. Traditional \ac{linq} commands do not work on the \code{Stream} object. Instead inbuilt \ac{linq} statements are available for the reactive \code{Stream} object to use. Cleipnir reactive operators can by design be chained together just like the majority of reactive operators in other frameworks. The main difference being that Cleipnir`s reactive operators and \ac{linq} operators result in a new \code{Stream} object instead of a new observable or a new enumerable. It is possible to create and handle a lot of the consensus algorithm workflow within a few lines of code by simply chaining reactive operators together. \autoref{code:operatorreq} shows an example of chaining reactive operators using Cleipnir’s reactive framework. The objective here is to get the first valid pre-prepare message emitted to the observable. For a pre-prepare message to be considered valid, it must pass all \code{Where} clauses. The \code{Next} operator at the end of the chain returns the resulting prepare message~\cites[p.~6,8,13]{PAPER:PaxosCleipnir}{WEB:ReactiveOperator} The \code{Merge} operator is used to listen for incoming items from the \code{ShutdownBridgePhase} \code{Source} object. More information in regards to the \code{Merge} operator is discussed in \autoref{chapter:Imp}


\begin{figure}[H]
	\centering
	%\lstset{style=sharpc}
	\begin{lstlisting}[label = code:operatorreq, caption=Example of chaining Cleipnir reactive operators, captionpos=b, basicstyle=\scriptsize]
var preprepared = await MesBridge
                        .Where(pm => pm.PhaseType == PMessageType.PrePrepare)
                        .Where(pm => pm.Digest != null && pm.Digest.SequenceEqual(digest))
                        .Where(pm => pm.Validate(
                            Serv.ServPubKeyRegister[pm.ServID], 
                            Serv.CurView, 
                            Serv.CurSeqRange)
                         )
                        .Merge(ShutdownBridgePhase)
                        .Next();
\end{lstlisting}
\end{figure}

\iffalse
Cleipnir supports reactive subjects, however they are instead known as \code{Sources}. The user can emit items to the \code{Source} object, and any observer linked to it will receive the response item. The \code{Source} object needs to be used in order for the developer to access and interact with the reactive layer in Cleipnir. \autoref{code:sourcereq} shows an example on how to initialize, emit and wait for incoming events in regards to the \code{Source} object. The \code{await reqbridge.Next()} makes sure that the resulting variable \code{req} receives the \code{Request} emitted to the \code{Source} object~\cite[p.~8]{PAPER:PaxosCleipnir}.
\fi

Cleipnir supports reactive subject functionality. However, subject objects are instead referred to as \code{Source}. The user can emit items to the \code{Source} object, and any observer linked to the \code{Source} object receives the response item. The \code{Source} objects are required to be used for the developer to access and interact with the reactive layer in Cleipnir. \autoref{code:sourcereq} shows an example of how to initialize, emit and wait for incoming events in regards to the \code{Source} object. The \code{await reqbridge.Next()} makes sure that the resulting variable \code{req} receives the \code{Request} emitted to the \code{Source} object~\cite[p.~8]{PAPER:PaxosCleipnir}.

\begin{figure}[H]
	\centering
	%\lstset{style=sharpc}
	\begin{lstlisting}[label = code:sourcereq, caption=Source object example, captionpos=b, basicstyle=\scriptsize]
Source<Request> reqbridge = new Source<Request>();
reqbridge.Emit(new Request(1, "Hello World!", DateTime.Now.ToString());

Request req = await reqbridge.Next();
	\end{lstlisting}
\end{figure}

\iffalse
\section{Cleipnir Persistent Programming}
%Define Persistent Programming
\label{section:PersistentProgramming}
A system which follows the persistent programming paradigm will regularly save the information for the program state while the program is running. Persistent programming makes it possible to design systems which can easily restore their program state in the case of a system reboot~\cites{DOC:Cleipnir}[p.~6]{PAPER:PaxosCleipnir}. Consensus algorithms can take great advantage of this programming paradigm as systems in the network are likely to eventually crash. With persistent programming, it is theoretically simple for a system to recover its data and rejoin the distributed network. Unfortunately, the state of the system is likely to still be somewhat behind the other systems when compared directly to the other working systems, even if all of the previous data is recovered. 

Cleipnir supports easy to use hybrid persistent programming. Hybrid persistent programming allows the developer to freely choose which data is to be persistable. In this way it is possible to avoid storing unnecessary information that would slow down the process immensely~\cite[p.~9-10]{PAPER:PaxosCleipnir}. \autoref{code:interfaceexample} and \autoref{code:seriadeseria} combined shows an example of the workflow needed for an object to be serialized and deserialized to and from persistent memory. In order for an object to become visible to the storage engine, it needs to first inherit either the \code{IPersistable} or the \code{IPropertyPersistable} interface. \code{IPersistable} is usually the common choice as it can support hybrid persistency programming. The \code{IPersistable} allows the user to choose which data in the object is to be serialized and which constructor to use for the deserialization operation. The IPropertyPersistable can only use the default inbuilt constructor for a .NET object, which is why it does not support hybrid persistency and is therefore not the recommended interface.
When inheriting the \code{IPersistable} interface the program will inherit the \code{Serialize} function as shown in \autoref{code:seriadeseria}. In this function you use the object stores statemap to set the desirable object to a designated key, like a normal map workflow. The storage engine internally references to different graph objects for each object stored. It is therefore possible for a key in the statemap to have the same key for multiple objects as the objects are treated as different graph objects in the storage engine. Meaning a developer does not need to worry about duplicate keys over different objects.

However, the storage engine cannot store all types of data. The storage engine can handle the basic data types like int, string, boolean etc. Unfortunately, the storage engine does not support inbuilt data structures like arrays, dictionaries, etc. The storage engine also does not support any newly created objects or data types outside of the basic ones. In order to make a custom object to be serializable for Cleipnir, they need to inherit the IPersistable interface and have assigned their serializer and deserializer functions properly. This means data types like enum are not supported. However, Cleipnir supports inbuilt versions of common data structures like array, dictionary and list that can in fact be persisted by the storage engine. Therefore, an easy work around is to simply substitute normal data structures for the inbuilt Cleipnir versions of the data structure. For instance, a dictionary object can be substituted for Cleipnirs \code{CDictionary} object. For objects that have a data type which is not supported by Cleipnir, a common workaround is to type cast it into another format which Cleipnir can persist. An example of this can be seen in \autoref{code:seriadeseria} where the object \code{Phasetype} is of enum type and Cleipnir cannot persist enum type objects. Therefore, it is type casted to int while stored in memory. Then in the deserialize process, the correct enum type can be chosen based on the int value that was stored. For the deserialize process a private static function called \code{Deserialize} is needed which uses the state map as inparameter. Even if the content of the function must be unique for each object’s constructor, the format of the function follows the same format shown in \autoref{code:seriadeseria}. The deserialize function simply initializes the object through a constructor and then returns the new instance of the specified object based on the info which was currently stored in the statemap.
\fi

\section{Cleipnir Persistent Programming}
\label{section:PersistentProgramming}

A system that follows the persistent programming paradigm will regularly save the information for the program state while the program is running. Persistent programming makes it possible to design systems that can quickly restore their program state in the case of a system reboot~\cites{DOC:Cleipnir}[p.~6]{PAPER:PaxosCleipnir}. Consensus algorithms can take great advantage of this programming paradigm as systems in the network are likely to crash eventually. With persistent programming, it is theoretically simple for a system to recover its data and rejoin the distributed network. Unfortunately, the state of the system is likely to still be somewhat behind the other systems when compared directly to the other working systems, even if all of the previous data is recovered. 

Cleipnir supports easy to use hybrid persistent programming. Hybrid persistent programming allows the developer to freely choose which data is to be persistable. In this way, it is possible to avoid storing unnecessary information that would slow down the process immensely~\cite[p.~9-10]{PAPER:PaxosCleipnir}. \autoref{code:interfaceexample} and \autoref{code:seriadeseria} show an example of the workflow needed for an object to be serialized and deserialized to and from persistent memory. For an object to become visible to the storage engine, the object needs first to inherit either the \code{IPersistable} or the \code{IPropertyPersistable} interface. \code{IPersistable} is usually the common choice as it can support hybrid persistency programming. The \code{IPersistable} allows the user to choose which data in the object is to be serialized and which constructor to use for the deserialization operation. The \code{IPropertyPersistable} can only use the default inbuilt constructor for a .NET object, which is why it does not support hybrid persistency and is therefore not the recommended interface.
When inheriting the \code{IPersistable} interface the program will inherit the \code{Serialize} function as shown in \autoref{code:seriadeseria}. In this function,  the information that is desired to be persistable for the object is added to statemap from the object store. The object information added to the statemap is set to a designated key, like a normal map or dictionary workflow. The storage engine internally references different graph objects for each object stored. Therefore, a key in the statemap can have the same value for multiple objects because the objects are treated as different graph objects in the storage engine. Meaning a developer does not need to worry about duplicate keys over different objects.

However, the storage engine cannot store all types of data. The storage engine can handle the basic data types like int, string, boolean, etc. Unfortunately, the storage engine does not support inbuilt data structures like arrays, dictionaries, etc. The storage engine also not compatible with any newly created objects or data types outside of the basic ones. To make a custom object be serializable for Cleipnir, they need to inherit the \code{IPersistable} interface and have assigned their serializer and deserializer functions properly. This means data types like enum are not supported. However, Cleipnir supports inbuilt versions of common data structures like arrays, dictionaries, and lists that the storage engine can in fact persist. Therefore, an easy workaround is to substitute typical data structures for the inbuilt Cleipnir versions of the data structure. For instance, a dictionary object can be substituted for Cleipnirs \code{Cdictionary} object. For objects with a data type that Cleipnir does not support, a common workaround is to type cast it into another format that Cleipnir can persist. 
An example of this can be seen in \autoref{code:seriadeseria} where the object \code{Phasetype} is of enum type, and Cleipnir cannot persist enum type objects. Therefore, it is type cast to int while stored in memory. Then, in the deserialize process, the correct enum type can be chosen based on the stored int value. For the deserializing process, a private static function called \code{Deserialize} is needed, which uses the state map as a parameter. Even if the content of the function must be unique for each object’s constructor, the format of the function follows the same structure shown in \autoref{code:seriadeseria}. The deserialize function initializes the object through a constructor and then returns the new instance of the specified object based on the information currently stored in the statemap.

\begin{figure}[H]
	\centering
	%\lstset{style=sharpc}
	\begin{lstlisting}[label = code:interfaceexample, caption=Object persistentcy initializer, captionpos=b, basicstyle=\scriptsize]
public class PhaseMessage : IPersistable %inherit interface
		
%Construtor to Deserialize process
public PhaseMessage(int id, int seq, int view, byte[] dig, PMessageType phase, byte[] sign)
{
   ServID = id;
   SeqNr = seq;
   ViewNr = view;
   Digest = dig;
   PhaseType = phase;
   Signature = sign;
}
	\end{lstlisting}
\end{figure}

\begin{figure}[H]
	\centering
	%\lstset{style=sharpc}
	\begin{lstlisting}[label = code:seriadeseria, caption=Serialize/Deserialize code example, captionpos=b, basicstyle=\scriptsize]
public void Serialize(StateMap stateToSerialize, SerializationHelper helper)
{
   stateToSerialize.Set(nameof(ServID), ServID);
   stateToSerialize.Set(nameof(SeqNr), SeqNr);
   stateToSerialize.Set(nameof(ViewNr), ViewNr);
   stateToSerialize.Set(nameof(Digest), Serializer.SerializeHash(Digest));
   stateToSerialize.Set(nameof(PhaseType), (int)PhaseType);
   stateToSerialize.Set(nameof(Signature), Serializer.SerializeHash(Signature));
}

private static PhaseMessage Deserialize(IReadOnlyDictionary<string, object> sd)
{
   return new PhaseMessage(
       sd.Get<int>(nameof(ServID)),
       sd.Get<int>(nameof(SeqNr)),
       sd.Get<int>(nameof(ViewNr)),
       Deserializer.DeserializeHash(sd.Get<string>(nameof(Digest))),
       Enums.ToEnumPMessageType(sd.Get<int>(nameof(PhaseType))),
       Deserializer.DeserializeHash(sd.Get<string>(nameof(Signature)))
       );
}
	\end{lstlisting}
\end{figure}

\iffalse
Finally, the Cleipnir’s \emph{CTask} class needs to be introduced. As the name suggests \code{CTask} shares similar traits with the \code{Task} object mentioned in \autoref{section:AsyncProgramming}. An asynchronous function which returns a \code{CTask} is an asynchronous operation which is to be run by the Cleipnir execution engine. In a sense using \code{CTask} for an asynchronous function means the operation performed inside the asynchronous function is meant to be persistable. In order for an object to be persisted during execution it needs to be run synchronously or in an asynchronous \code{CTask} operation. An example of this would be if the user wanted to persist one of the reactive Cleipnir \code{Source} objects, than the function waiting for emitted items need to return a \code{CTask} rather than a \code{Task}. Otherwise Clepnir storage engine will crash upon attempting to persist it.

Keep in mind \code{CTask} should not be used when the asynchronous function has any asynchronous operations unless you intend to use Cleipnir to persist the data. Using asynchronous operations inside a \code{CTask} will cause Cleipnir to create a new thread to handle the asynchronous operation while continuing on with the rest of the operations inside the function. This also applies when the \code{await} operator is used, meaning the \code{await} becomes redundant and will not work as intended. This also applies when scheduling new operations for Cleipnir inside a \code{CTask} function, since the schedule function for the Cleipnir execution engine is treated as an asynchronous operation. It is important for a user of Cleipnir to avoid creating potential race conditions within their implementation due to running traditional asynchronous operations inside a \code{CTask} function. Normally it is best to try and avoid this situation entirely. Thereby restricting a \code{CTask} function to only operate with synchronous operations, while any asynchronous operations required should be performed inside other \code{Task} operators using the TAP workflow discussed in \autoref{section:AsyncProgramming}. In short, although it is possible to call asynchronous \code{Task} inside \code{CTask} functions the scheduler runs this operation separately in another thread. Therefore, it is recommended to keep asynchronous \code{Task} away from \code{CTask}. Instead \code{CTask} should only use the TAP workflow when working with other asynchronous operations from Cleipnir, which is essentially only other \code{CTask} or when listening for events in \code{Source} objects. An example of a \code{CTask} function can be seen in \autoref{code:CTaskEx}, where we have to use \code{CTask} to listen for new items in the \emph{shutemit} \code{Source} object which is to be persisted. The code here would result in an error in Cleipnir objectstore if the \code{CTask} assignment was instead a traditional \code{Task}.
\fi

Finally, Cleipnir`s \emph{CTask} class needs to be introduced. As the name suggests \code{CTask} shares similar traits with the \code{Task} object mentioned in \autoref{section:AsyncProgramming}. An asynchronous function that returns a \code{CTask} is an asynchronous operation that is to be run by the Cleipnir execution engine. In a sense using \code{CTask} for an asynchronous function means the operation performed inside the asynchronous function is meant to be persistable. For an object to be persisted during execution it needs to be run synchronously or in an asynchronous \code{CTask} operation. An example of this would be if the user wanted to persist one of the reactive Cleipnir \code{Source} objects, then the function waiting for emitted items need to return a \code{CTask} rather than a \code{Task}. Otherwise, the Cleipnir storage engine will crash upon attempting to persist it.

Keep in mind \code{CTask} should not be used when the asynchronous function has any asynchronous operations unless you intend to use Cleipnir to persist the data. Using asynchronous operations inside a \code{CTask} causes Cleipnir to create a new thread to handle the asynchronous operation while continuing with the rest of the operations inside the function. This also applies when the \code{await} operator is used for a traditional asynchronous operation, meaning the \code{await} becomes redundant and will not work as intended in this case. This also applies when scheduling new operations for Cleipnir inside a \code{CTask} function since the schedule function for the Cleipnir execution engine is treated as an asynchronous operation. A user of Cleipnir must avoid creating potential race conditions within their implementation due to running traditional asynchronous operations inside a \code{CTask} function. Normally it is best to try and avoid this situation entirely, thereby restricting a \code{CTask} function to only operate with synchronous operations. In contrast, any asynchronous operations required should be performed inside other \code{Task} operators using the TAP workflow discussed in \autoref{section:AsyncProgramming}. In short, although it is possible to call asynchronous \code{Task} inside \code{CTask} functions, the scheduler runs this operation separately in another thread. Therefore, it is recommended to keep asynchronous \code{Task} away from \code{CTask}. Instead, \code{CTask} should only use the TAP workflow when working with other asynchronous operations from Cleipnir, which is together with other \code{CTask} or when listening for events in \code{Source} objects. An example of a \code{CTask} function can be seen in \autoref{code:CTaskEx}, where we have to use \code{CTask} to listen for new items in the \emph{shutemit} \code{Source} object which is to be persisted. The code here would result in an error in the Cleipnir object store if the \code{CTask} assignment were instead a traditional \code{Task}.

\begin{figure}[h]
	\centering
	%\lstset{style=sharpc}
	\begin{lstlisting}[label = code:CTaskEx, caption= Example of a CTask function, captionpos = b, basicstyle=\scriptsize]
	public async CTask<bool> ListenForShutdown(Source<bool> shutemit)
       {
           var shut = await shutemit.Next();
           Console.WriteLine("View Change Received Shutdown");
           return shut;
       }
	\end{lstlisting}
\end{figure}

