\chapter{Related Work}
%1-2 papesa
%If you cannot demonstrate that you know, and understand, what others have done, you only demonstrate that you're clueless. 
%For an undergraduate thesis this, together with a thorough understanding of the problem, should be the result of the first session's work. 
%It is an unfortunate fact that many students do very little work during the first session of their thesis. 
%It usually shows here (and is usually reflected in their mark). 
%Don't think you can fool your thesis supervisor/assessor. And don't even dream about fooling the referee of a paper. 
%If you haven't done your homework here, it's probably not worth going any further.
%In this part you demonstrate that you are aware of what's going on in the field, and how it relates to your particular problem.
%If there is lots of related work, discuss related work early to differentiate your own work
%-Compare/contrast with your own work - don't just enumerate
%-Don't be dismissive
%-Refer to references by author or project names
\label{chapter:RW}
\iffalse
The University of Stavanger has previously supported the development of Cleipnir.
Therefore, there exists previous papers and thesis on Cleipnir usage for implementing consensus algorithms.
Two contributions in particular have been used as building blocks for this project. We now discuss these two works in detail and explain how they contributed to our work. 
 
\section{Cleipnir - Framework Support for Fault-tolerant Distributed Systems}
This paper is the original paper describing the Cleipnir framework and is written by its creator Thomas Stidsborg Sylvest with the help of two professors at the University of Stavanger, Leander Jehl and Hein Meling. The paper describes, in detail, the internal functionality and tools available in the Cleipnir framework. This includes their use cases, detailed explanations for how they work with practical demonstrations. The demonstrations are presented using an existing implementation of the Paxos consensus algorithm. The paper also presents a Raft implementation using the Cleipnir framework. This includes the overall architecture of the implementation with detailed examples of how Cleipnir is used to simplify tasks used in the Raft algorithm. Finally experiments are performed to evaluate the performance of the Raft implementation. The results of the experiments are compared directly to the Paxos implementation, both in terms of latency and code structure. Our thesis is a direct continuation of this paper with practically the same goals. The main difference between our thesis and this paper being the chosen consensus algorithm that is to be implemented using Cleipnir. Our contribution is to provide additional experiences on how well Cleipnir can be utilized in implementing complex consensus algorithms. This also implies discovering potential difficult problems that the current Cleipnir framework is not able to handle. Specifically, whether or not Cleipnir can handle all of the complex problems that can take place within the PBFT algorithm while still having a simple to read code structure~\cite{PAPER:PaxosCleipnir}.
 
\section{Implementing a Distributed Key-Value Store Using Corums}
In 2010, Eivind Bakkevig wrote a master thesis about Corums. In this thesis he used a Net framework called Corums to implement a dictionary based distributed system. This Corums based implementation had an implementation of the Paxos consensus algorithm to make decisions for the distributed system.
 
The Corums framework is the predecessor to the Cleipnir framework. It follows the same programming models as Cleipnir does. These models would be the ones described in \autoref{chapter:Cleipnir}. Namely Built-in Persistency, Reactive programming and Single-Threaded scheduler.
The main difference between Corums and Cleipnir is that Corums focus more on simplifying an abstraction for developers to handle communication using incoming/outgoing communication buses. Cleipnir has more focus on giving the developer the tools necessary to develop consensus algorithms which follow the persistent program paradigm in an easy to use and customizable manner. As an example, a major difference between Cleipnir and Corums frameworks lies in Corums support in reliable message delivery between distributed systems. Corums has support for bus abstraction that can simplify the handling of incoming/outgoing messages between the nodes in the system. Cleipnir does unfortunately not support this functionality. Instead Cleipnir has more focus on evolving the persistence functionality previously provided by Corums~\cites[p.~6-7]{PAPER:PaxosCleipnir}{DOC:Cleipnir}.
 
Corums is very similar to Gorums~\cites[p.~2]{WEB:Gorums}[p.~22]{PAPER:EivindPaper} which is intended based on how close the names are, the main difference being the supporting language.
Bakkevig succeeded in creating a distributed dictionary storage using the Corums framework. Additionally he built the client side for the implementation using ASP.NET Core Web \ac{api}~\cite{WEB:ASPNetCoreAPI}.
 
According to Bakkevig, he had no prior experience with the C\# before writing his thesis. Bakkevig did however have previous experience with the Paxos consensus algorithm. This made most of his work during the thesis about learning C\# and the Corums framework rather than having to extensively research Paxos. As for our thesis the exact opposite is true. We have some background knowledge regarding the C\# language but had little to no background knowledge of the PBFT algorithm. Therefore, a lot of work for this thesis revolved around learning and making our own PBFT algorithm based on its description. Although we had experience with the C\# language, we had no previous experience with the Cleipnir framework. Therefore, similarly to Bakkevig, our thesis also required us to study the Cleipnir framework. ~\cite[p.~8]{PAPER:EivindPaper}.
\fi

The University of Stavanger has previously supported the development of Cleipnir.
Therefore, there exist previous papers and thesis on Cleipnir usage for implementing consensus algorithms.
Two contributions, in particular, have been used as building blocks for this project. We now discuss these two works in detail and explain how they contributed to our work. 
 
\section{Cleipnir - Framework Support for Fault-tolerant Distributed Systems}
This paper is the original paper describing the Cleipnir framework. It was written by its creator Thomas Stidsborg Sylvest with the help of two professors at the University of Stavanger, Leander Jehl and Hein Meling. The paper describes, in detail, the internal functionality and tools available in the Cleipnir framework. The paper describes Cleipnir`s use cases and why Cleipnir priorities these functionalities. The paper has detailed explanations for how the tools work together with practical demonstrations. The demonstrations are presented using an existing implementation of the Paxos consensus algorithm. The paper also presents a Raft implementation using the Cleipnir framework. This includes the overall architecture of the implementation and detailed examples of how Cleipnir is used to simplify tasks performed in the Raft algorithm. Finally, experiments are performed to evaluate the performance of the Raft implementation. The results of the experiments are compared directly to an earlier Paxos implementation. The evaluation performed focuses both on latency and code structure. Our thesis is a direct continuation of this paper with relatively similar goals. The largest difference between our thesis and this paper is the chosen consensus algorithm to be implemented using Cleipnir.
Additionally, we do not evaluate our \ac{pbft} implementation in terms of latency. Our contribution is to provide additional experiences on how well Cleipnir can be utilized in implementing complex consensus algorithms. This also implies discovering potentially difficult problems that the current Cleipnir framework cannot handle, specifically, whether or not Cleipnir can handle all of the complex issues within the PBFT algorithm while still having a simple-to-read code structure~\cite{PAPER:PaxosCleipnir}.
 
\section{Implementing a Distributed Key-Value Store Using Corums}
In 2010, Eivind Bakkevig wrote a master thesis about Corums. In his thesis, Bakkevig used a .NET framework called Corums to implement a dictionary-based distributed system. This Corums based implementation implemented the Paxos consensus algorithm to make decisions for the dictionary-based distributed system.
 
The Corums framework is the predecessor to the Cleipnir framework. It follows the same programming models as Cleipnir does. These models would be the ones described in \autoref{chapter:Cleipnir}; built-in persistency, reactive programming, and a single-threaded scheduler.
The main difference between Corums and Cleipnir is that Corums focus more on simplifying abstraction for developers to handle communication using incoming/outgoing communication buses. Cleipnir instead focuses more on giving the developer the tools necessary to develop consensus algorithms that follow the persistent program paradigm in an easy-to-use and customizable manner. As an example, a major difference between Cleipnir and Corums frameworks lies in Corums support in reliable message delivery between distributed systems. Corums has support for bus abstraction that can simplify the process of handling incoming/outgoing messages between the nodes in the system. Cleipnir does unfortunately not support this functionality. Instead, Cleipnir prioritized evolving the persistence functionality previously provided by Corums~\cites[p.~6-7]{PAPER:PaxosCleipnir}{DOC:Cleipnir}.
 
Corums is very similar to Gorums~\cites{WEB:Gorums}[p.~22]{PAPER:EivindPaper}, which is intended based on how close the names are, the main difference being the supporting language.
Bakkevig succeeded in creating a distributed dictionary storage using the Corums framework. Additionally, Bakkevig built the client-side for the implementation using ASP.NET Core Web \ac{api}~\cite{WEB:ASPNetCoreAPI}.
 
According to Bakkevig, he had no prior experience with the C\# language before writing his thesis. Bakkevig did, however, have previous experience with the Paxos consensus algorithm. This made most of his work during the thesis about learning C\# and the Corums framework rather than extensively researching Paxos. As for our thesis, the exact opposite is true. We have some background knowledge regarding the C\# language but had little to no background knowledge of the PBFT algorithm. Therefore, much work for this thesis revolved around learning and making our own PBFT algorithm based on its description. Although we had experience with the C\# language, we had no previous experience with the Cleipnir framework. Therefore, similarly to Bakkevig, our thesis also required us to study the Cleipnir framework. ~\cite[p.~8]{PAPER:EivindPaper}.
