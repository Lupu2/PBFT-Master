\chapter{Related Work}
%1-2 papesa
%If you cannot demonstrate that you know, and understand, what others have done, you only demonstrate that you're clueless. 
%For an undergraduate thesis this, together with a thorough understanding of the problem, should be the result of the first session's work. 
%It is an unfortunate fact that many students do very little work during the first session of their thesis. 
%It usually shows here (and is usually reflected in their mark). 
%Don't think you can fool your thesis supervisor/assessor. And don't even dream about fooling the referee of a paper. 
%If you haven't done your homework here, it's probably not worth going any further.
%In this part you demonstrate that you are aware of what's going on in the field, and how it relates to your particular problem.
%If there is lots of related work, discuss related work early to differentiate your own work
%-Compare/contrast with your own work - don't just enumerate
%-Don't be dismissive
%-Refer to references by author or project names
\label{chapter:RW}

The University of Stavanger has previously supported the development of Cleipnir. As a result has published papers and thesis in regards to its usage in consensus algorithm implementation. In particular two contributions have been used as building blocks for this project. We now discuss these two works in detail and explain how they contributed to our work.  

\section{Cleipnir - Framework Support for Fault-tolerant Distributed Systems}
This paper is the original paper describing the Cleipnir framework and is written by its creator Thomas Stidsborg Sylvest and co-written with two professors at the University of Stavanger Leander Jehl and Hein Meling. The paper describes, in detail, the internal functionality and tools available in the Cleipnir framework. This includes their use case, detailed explanation of how they work and practical demonstration for how to use them. The demonstrations are presented using an existing implementation of the Paxos consensus algorithm. The paper also presents a Raft implementation using the Cleipnir framework. This includes overall architecture of the implementation, detailed examples of usuage of Cleipnirs tools, and evaluation of experiments performed on the Raft implementation to measure its performance. The results of the experiments are compared directly to the Paxos implementation, both in terms of latency and code structure. Our thesis is a direct continuation of this paper with practically the same goals. The main difference between our thesis and this paper being the chosen consensus algorithm that are to be implemented using Cleipnir. Our contribution is to provide additional experiences on how well Cleipnir can be utilized in implement complex consensus algorithms. This also implies discovering potential difficult problems that the current Cleipnir framework is currently not able to handle. Specifically whether or not Cleipnir can handle all of the complex problems that can take place, while still upholding the program paradigm in a easy to read code structure~\cite{PAPER:PaxosCleipnir}.

\section{Implementing a Distributed Key-Value Store Using Corums}
In 2010, Eivind Bakkevig wrote a master thesis where his goal was to implement a distributed system which made decisions based on the Paxos consensus algorithm. The consensus algorithm was implemented using the Corums .Net framework. 

The Corums framework is the predecessor to Cleipnir and includes the same programming model that Cleipnir offers. These models would be the closely related to the ones described in \autoref{chapter:Cleipnir}, namely Built-in Persistency, Reactive programming and Single-Threaded scheduler. 
The main difference between Corums and Cleipnir is that Corums focus more on simplifying an abstraction for handling communication using incoming/outgoing communication bus. Cleipnir has more focus on getting the persistent programming paradigm into a consensus algorithm in a easy to use, and customizable way as possible. As an example, a major difference between Cleipnir and Corums lies in Corums support in reliable message delivery between distributed systems. Corums uses a simple Bus abstraction to easily handle incoming/outgoing messages between the nodes in the system. Cleipnir does not support this functionality~\cites[p.~6-7]{PAPER:PaxosCleipnir}{DOC:Cleipnir}.

Corums is very similar to Gorums~\cites[p.~2]{WEB:Gorums}[p.~22]{PAPER:EivindPaper} which is intended based on how close the names are, the main difference being the supporting language. 
Bakkevig succeeded in creating a distributed dictionary storage using the Corums framework. Additionally he built the client side programming using the existing ASP.NET Core Web API~\cite{WEB:ASPNetCoreAPI}. 

According to Bakkevig, he had no prior experience with the C\# before this thesis. Bakkevig did however had previous experience with the Paxos consensus algorithm. This made most of his work during the thesis about learning C\# and the Corums framework rather than having to extensively research Paxos. In terms of background this thesis has the exact opposite background. We have some background knowledge regarding the C\# language but had little to no knowledge regarding the PBFT algorithm. Therefore, a lot of work needs to be put into fully understanding the PBFT consensus algorithm, rather than focus fully on the Cleipnir framework. It is important to note that our work do not have any background experience with the Cleipnir framework just like Bakkevig had no prior experience with the Corums framework~\cite[p.~8]{PAPER:EivindPaper}.
