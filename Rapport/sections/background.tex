\chapter{Programming Models}
\label{chapter:ProgrammingModels}
\iffalse
This chapter general information about the asynchronous programming and reactive programming models are introduced. This includes main use cases and general workflow. The asynchronous programming section includes an introduction to the async/await model~\cite{DOC:AsyncAwait}. The reactive programming section includes information about ReactiveX~\cite{WEB:ReactiveXMainPage} which is the corner stone for all Rx driven implementations.
\section{Asynchronous Programming}
\label{section:AsyncProgramming}
%\subsection{Introduction}
Asynchronous programming is a programming technique designed to handle a common problem that sometimes occurs in synchronous programming. Synchronous programming always blocks the execution until the previous line of code is handled. A synchronous program delegates the operative systems resources to finish a single operation in the program, before moving on to the next operation and so on. However, blocking the execution thread in general causes issues with scalability, latency as well as in general gives a very bad user experience. Meaning synchronous programming isn't optimal for operations which require long execution time. Especially if the operation itself spends most of its time waiting, such as database requests or I/O bound operations~\cite{VIDEO:AsyncConBack, WEB:AsyncAwaitTut}. Keep in mind that asynchronous programming for different programming languages usually follow relatively the same workflow, however the naming of operations may differ. In this thesis the terminology used for asynchronous programming follows the ones used in the .NET framework.

Asynchronous programming as the name implies is designed to run operations asynchronously. In the asynchronous programming model, operations are divided into a set of tasks that perform the operations whenever the scheduler has resources which it can freely delegate to it.
However, the task created does not block the main thread, instead the main thread continues on with the next operation~\cite{WEB:AsyncAwaitTut, VIDEO:AsyncConBack, DOC:AsyncAwait}.
The task has a reference to an awaiter that has information in regard to the task's current state. Eventually the asynchronous operation finishes, and the result is available in the awaiter for the main thread to collect. Not all tasks need to necessarily return a result. It is possible to run non returning asynchronous operations in tasks as well. Nevertheless a task must always return an awaiter so that the main thread has reference to all relevant information for the asynchronous task~\cite{WEB:AsyncAwaitTut}.

Normally the main thread needs to receive the result of the asynchronous operation before reaching specific parts of the program that requires the result in order to run properly. Asynchronous functionality supports this functionality by allowing the designer to specify to the awaiter that the program is to wait at this point until the asynchronous operation is finished. This still does not block the main thread as the other tasks can be performed in the background unlike synchronous programming. Additionally, asynchronous programming has the benefit that the operation can be initialized earlier and be worked on by the main thread while going through the operations to the point where the result is expected. This means asynchronous programming could avoid bottlenecks that occur in synchronous programming and thereby making asynchronous programming more responsive of the two programming models~\cite{DOC:TaskAsyncProgModel, WEB:AsyncAwaitTut}.
For this reason, asynchronous programming has become the preferred programming model when it comes to designing user-interfaces. As it is important to avoid potentially blocking user input while another task is performed~\cites{VIDEO:AsyncConBack}[p.~214]{BOOK:DotnetMultithreadCookBook}. Server design is another example where asynchronous design is preferred as it handles a large number of requests easier than a server with synchronous design~\cite{VIDEO:AsyncConBack, DOC:AsyncAwait}.

Asynchronous programming usually follows one or more of these three design patterns:
\begin{itemize}
	\item{\ac{apm}}
	\item{\ac{eap}}	
	\item{\ac{tap}}
\end{itemize}
\ac{tap} is the most used design pattern and is the model used by the async/await workflow~\cite{DOC:AsyncAwait, WEB:AsyncAwaitTut}.

Asynchronous programming should not be confused with parallel programming as asynchronous methods do not create new threads. It instead runs on the current thread whenever the scheduler has resources ready and the operation itself is ready to progress. Therefore, the work required to create new threads as well as a lot of the work to keep the threads consistent can be omitted~\cite{DOC:TaskAsyncProgModel}. %Potentially write more once you have more control over how consistency works between async functions and how they fail in my implementation.

\subsection{Async/Await}
Asynchronous programming is not a new concept and C\# has long had support for it~\cite{WEB:asyncNelsen}. However, before the async/await workflow became normalized programming asynchronously was quite difficult and even worse for others to read~\cite{DOC:TaskAsyncProgModel}. The workflow consisting of a lot of nested callback functions which is quite a struggle to manage properly. Today managing this kind of structure is referred to as \emph{callback hell}~\cites[p.~1-2]{PAPER:Callbackhell}[p~.2]{PAPER:PaxosCleipnir}.

As mentioned previously the async/await workflow follows the \ac{tap} abstraction~\cite{DOC:TaskAsyncProgModel} meaning the workflow initializes the asynchronous operations. Then in the workflow which calls the asynchronous operation there is a point chosen where the result of the asynchronous operation needs to be collected before the workflow can move on to other operations. The async/await workflow consists of three steps for the programmer. The first step is to assign the \code{async} modifier to a function to mark it as an asynchronous function. This allows asynchronous calls to be made inside the chosen function. The second step is to make an asynchronous call. Lastly specify the \code{await} operator for the awaiter for the asynchronous task~\cite{WEB:AsyncAwaitTut, DOC:AsyncAwait, VIDEO:AsyncConBack}.
It is important to remember that the \code{await} operator can only be used in a function marked with the \code{async} modifier. In order to use an asynchronous function call in a synchronous function, the traditional operators have to be used instead~\cite{DOC:AsyncAwait, DOC:TaskAsyncProgModel}.

In \autoref{code:asyncawaitex} we can see a practical example of the async/await workflow.
The code in \autoref{code:asyncawaitex} is the asynchronous function that is responsible for having a chosen \code{Socket} object connect to a designated \ac{ip} address. The \code{IPEndPoint} object being the reference to the chosen \ac{ip} address. In order for the \code{Connect} function to be marked as an asynchronous function it has a \code{async} modifier. \code{Connect} returns a .NET \code{Task} object of type boolean, meaning the function returns a reference to the active Connect \code{Task} which will return a boolean value once the \code{Task} is completed. In this case the \code{Connect} function returns true if the socket succeeds in connecting to the \ac{ip} address, otherwise it returns false. The asynchronous operation performed in the \code{Connect} function is when the \code{ConnectAsync} function is called for the socket. As we want to avoid the function returning the result before the asynchronous operation is finished, the \code{await} operator is used to have the \code{Task} wait for the asynchronous operation to finish.
\fi

Information about the asynchronous programming and reactive programming models are introduced in this chapter. This includes their intended use cases and general workflow. The asynchronous programming section mentions several design patterns used for asynchronous operations. We will mainly concentrate on the async/await model has ~\cite{DOC:AsyncAwait}. The reactive programming section covers information about ReactiveX~\cite{WEB:ReactiveXMainPage} which is the cornerstone for all Rx-driven implementations.
\section{Asynchronous Programming}
\label{section:AsyncProgramming}
%\subsection{Introduction}
Asynchronous programming is a programming technique designed to handle a common problem that sometimes occurs in synchronous programming. Synchronous programming always blocks the execution until the previous line of code is handled. A synchronous program forces the program to finish a single operation in the program before moving on to the next operation. However, blocking the execution thread usually leads to scalability issues, latency issues and generally results in an awful user experience. Meaning synchronous programming is not optimal for operations that require a long execution time. Especially if the operation itself spends most of its time waiting for a result, examples of such actions would be database requests or I/O bound operations~\cite{VIDEO:AsyncConBack, WEB:AsyncAwaitTut}. Keep in mind that asynchronous programming for different programming languages usually has similar workflows. However, the naming conventions for identical operations may differ. In this thesis, the terminology used for asynchronous programming follows the ones used in the .NET framework.

Asynchronous programming, as the name implies, is designed to run operations asynchronously. In the asynchronous programming model, operations are divided into a set of tasks. These tasks perform the assigned operations whenever the scheduler has resources it can delegate to them.
However, the task created does not block the main thread, instead, the main thread continues with the next operations~\cite{WEB:AsyncAwaitTut, VIDEO:AsyncConBack, DOC:AsyncAwait}.
The task has a reference to an awaiter that has information on the current state of the task. Eventually, the asynchronous operation finishes, and the result is available in the awaiter for the main thread to collect. Not all tasks need to return a result necessarily. It is possible to run non-returning asynchronous operations in tasks as well. Nevertheless, a task must always return an awaiter so that the main thread has reference to all relevant information for the asynchronous task~\cite{WEB:AsyncAwaitTut}.

Normally, the main thread needs to receive the result of the asynchronous operation before reaching specific parts of the program that requires the result to function correctly. Asynchronous programming supports this functionality by allowing the designer to specify to the awaiter that the program is to wait at this point until the asynchronous operation is finished. This still does not block the main thread, meaning other tasks can be performed in the background, unlike synchronous programming. Additionally, asynchronous programming has the benefit that the operation can be initialized earlier and be worked on by the main thread while going through the main thread operations to the point where the result is expected. This means asynchronous programming could avoid bottlenecks that occur in synchronous programming, thereby making asynchronous programming more responsive of the two programming models~\cite{DOC:TaskAsyncProgModel, WEB:AsyncAwaitTut}.
For this reason, asynchronous programming has become the preferred programming model for designing user interfaces since it is crucial to avoid potentially blocking user input when at the same time, other primary tasks are performed~\cites{VIDEO:AsyncConBack}[p.~214]{BOOK:DotnetMultithreadCookBook}. Server design is another example where asynchronous design is preferred as it handles many requests easier than a server with synchronous design~\cite{VIDEO:AsyncConBack, DOC:AsyncAwait}.

Asynchronous programming usually follows one or more of these three design patterns:
\begin{itemize}
	\item{\ac{apm}}
	\item{\ac{eap}}	
	\item{\ac{tap}}
\end{itemize}
\ac{tap} is the most used design pattern and is the model used by the async/await workflow~\cite{DOC:AsyncAwait, WEB:AsyncAwaitTut}.

Asynchronous programming should not be confused with parallel programming, as asynchronous methods do not create new threads. It instead runs on the current thread whenever the scheduler has resources ready, and the operation itself is ready to progress. Therefore, the work required to create new threads and a lot of the work to keep the threads consistent can be omitted~\cite{DOC:TaskAsyncProgModel}. 

\subsection{Async/Await}
.NET has long had support for asynchronous programming~\cite{WEB:asyncNelsen}. However, before the async/await workflow became normalized, programming asynchronously was quite difficult and even worse for others to read~\cites{DOC:TaskAsyncProgModel, WEB:asyncNelsen}. The old workflow consisted of a lot of nested callback functions, which is a struggle to manage properly. Today managing this kind of structure is referred to as \emph{callback hell}~\cites[p.~1-2]{PAPER:Callbackhell}[p~.2]{PAPER:PaxosCleipnir}.

As previously mentioned, the async/await workflow follows the \ac{tap} abstraction~\cite{DOC:TaskAsyncProgModel}.  The async/await workflow, therefore, consists of creating a task that performs the asynchronous operation. Then the original process that created the asynchronous task marks where the result of the task needs to be returned in the workflow. If the task is not finished when it reaches the marked area in the workflow, the process waits at this point until the result is ready. 
The async/await workflow consists of three steps for the programmer. The first step is to assign the \code{async} modifier to a function to mark it as an asynchronous function. This allows asynchronous calls to be made inside the chosen function. The second step is to make an asynchronous call. Lastly, specify the \code{await} operator for the awaiter for the asynchronous task to determine where in the workflow the result is obtained~\cite{WEB:AsyncAwaitTut, DOC:AsyncAwait, VIDEO:AsyncConBack}.
It is important to remember that the \code{await} operator can only be used in a function marked with the \code{async} modifier. The traditional asynchronous operators have to be used instead of the async/await workflow when making asynchronous calls inside synchronous functions~\cite{DOC:AsyncAwait, DOC:TaskAsyncProgModel}.

In \autoref{code:asyncawaitex} we can see a practical example of the async/await workflow.
The code in \autoref{code:asyncawaitex} is the asynchronous process that is responsible for having a chosen \code{Socket} object connect to a designated \ac{ip} address. The \code{IPEndPoint} object being the reference to the chosen \ac{ip} address. In order for the \code{Connect} function to be marked as an asynchronous function it has a \code{async} modifier. \code{Connect} returns a .NET \code{Task} object of type boolean, meaning the function returns a reference to the active Connect \code{Task} which returns a boolean value once the \code{Task} is completed. In this case the \code{Connect} function returns true if the socket succeeds in connecting to the \ac{ip} address, otherwise it returns false. The asynchronous operation performed inside the \code{Connect} function is the \code{ConnectAsync} function which is called by the socket object. As we want to avoid the function returning the result before the asynchronous operation is finished, the \code{await} operator is used to have the \code{Task} wait for the \code{ConnectAsync}  asynchronous operation to finish.

\begin{figure}[h]
	\centering
	%\lstset{style=sharpc}
	\begin{lstlisting}[label = code:asyncawaitex, caption=Example of async/await workflow, captionpos=b, basicstyle=\scriptsize]
public static async Task<bool> Connect(Socket sock, IPEndPoint endpoint)
{
    try
    {
        await sock.ConnectAsync(endpoint);
        return true;
    }
    catch (Exception e)
    {
        Console.WriteLine("Failed to connect to endpoint: " + endpoint.Address);
        Console.WriteLine(e);
        return false;
    }
}
	\end{lstlisting}
\end{figure}

\iffalse
%TODO find/write a better example.
\begin{figure}[h]
	\centering
	%\lstset{style=sharpc}
	\begin{lstlisting}[label = code:asyncawaitex, caption=Example of async/await workflow, captionpos=b, basicstyle=\scriptsize]
public async Task SendMessage(byte[] sermessage, 
                              Socket sock, 
                              MessageType type)
{
    Console.WriteLine($"Sending: {type} message");
    var mesidentbytes = Serializer
                        .AddTypeIdentifierToBytes(sermessage, type);
    var fullbuffmes = NetworkFunctionality
                      .AddEndDelimiter(mesidentbytes);
    await sock.SendAsync(fullbuffmes, SocketFlags.None);
}


	\end{lstlisting}
\end{figure}
\fi
%\cite{VIDEO:AsyncConBack}
%\cite{DOC:AsyncAwait}
%\cite{DOC:TaskAsyncProgModel}
%\cite{BOOK:DotnetMultithreadCookBook}
%\cite{WEB:AsyncAwaitTut}
\section{Reactive Programming}
\label{section:reactive}
Reactive programming is a programming paradigm that focuses on changing the state of the program in response to some outward changes~\cite{WEB:RxProgIntro, DOC:Cleipnir}.
Reactive programming follows an event driven workflow. An event can be triggered from one part of the system and when this event is received by the other part it starts altering the state of the system in response. Reactive programming works hand in hand with asynchronous event-based programming which was mentioned previously briefly in \autoref{section:AsyncProgramming}~\cite[p.~2-3]{BOOK:RxLinq}. Reactive programming is commonly used to handle a continuous stream of asynchronous data~\cite{VIDEO:dotnetsheffReactive}. 
Currently there exists a lot of support for Reactive programming. Specifically, the library Reactive X~\cite{WEB:ReactiveXMainPage} has presented a general \ac{api}~\cite{web:api} for implementing the core concepts of reactive programming. As a result, today there exist a lot of reactive extensions for multiple programming languages. Rx.Net~\cite{Github:ReactiveExtensions} is the official .Net reactive extension. Cleipnirs has implemented its own reactive extension that resembles Rx.Net very closely. The main difference between the two is that Cleipnirs reactive layer supports persistency, but lacks reactive operators that Rx.Net does support~\cite{DOC:Cleipnir}.
Although Cleipnir and Rx.Net vary somewhat from the general \ac{api}, the general workflow remains the same. Therefore we will introduce the main concepts of Reactive X in this section. Details specific for Cleipnir are instead presented in the upcoming \autoref{chapter:Cleipnir}.

\subsection{Reactive X}
ReactiveXs workflow can be easily summarized with the following tasks~\cite{WEB:ReactiveObservable}
\begin{enumerate}
	\item{Start an asynchronous operation that will perform some work and eventually return it}
	\item{Transform the asynchronous operation as an Observable object}
	\item{Use reactive operators to transform/filter the resulting data.}
	\item{Observers subscribe to the Observable and waits for the Observable to return the data}
\end{enumerate}

An observable object follows a similar structure to an enumerable object, where the main difference between an enumerable and an observable object is their method of accessibility. In an enumerable object will give the next object in storage whenever asked for it. In other words, the program will dictate when the next entry will be collected. In an Observable object the next result is instead pushed to its subscriber whenever the result is ready. The program has no control over when the next entry will be ready as it is waiting for an asynchronous operation to complete~\cites{WEB:ReactiveObservable, VIDEO:dotnetsheffReactive, VIDEO:MicroDev}[p.~15]{BOOK:RxLinq}. Observables, like enumerable, support the use of \ac{linq} queries on its resulting data. \ac{linq} add additional operators for filtering and transforming the resulting data into new enumerables~\cites{VIDEO:dotnetsheffReactive}[p.~3-4]{BOOK:RxLinq}[p.~208]{BOOK:DotnetMultithreadCookBook}.

Traditionally, the implementation is expected to incorporate the following functions for its observer object.
\begin{itemize}
	\item{OnNext}
	\item{OnError}
	\item{OnCompleted}
\end{itemize}

OnNext is the function that handles each new incoming event emitted by the Observable. OnError is the function that is called if an error occurs within handling one of the emitted events. OnCompleted is the function that is called when the observable is finished and will no longer emit any new events~\cite{WEB:ReactiveObservable}.

In some implementations the Observable and observer functionality are merged together into an object called subject. A subject object acts as a bridge of sorts between the observer and the observable where its main usage is to simplify the workflow for reactive programming. A subject has the ability to subscribe to an observable just like an observer. However, unlike an observer a subject can also re-emit events already processed in the observable, as well as be used for emitting new events to the observable. Eventually, all the items emitted by the subject will also be handled by the subject, making the programming workflow a lot simpler compared to its traditional style~\cite{WEB:ReactiveSubject}. Cleipnir supports subject in its implementation, however the objects are not referred to as subject, but rather source objects.

\iffalse
-a brief introduce reactive programming
-usecase
-the ReactiveX library(what it does, how it works and the mention Rx.Net)
-give brief through workflow, concepts with name and definitions and how it works.(Observable, stream of data, subjects, event driven programming)
-mention briefly Cleipnir support for reactive programming, how they differ, say detail information is given in Cleipnir chapter.
\fi

