\section{Lessons Learned}
\iffalse
-PBFT
-Asynchronous programming with C#, Task architecture
-Reactive Programming basics
-Overall knowledge for Cleipnir
-Issues and advantages in regards to the topics listed over. For instance a lot of time was wasted due to not fully grasping how Cleipnir work internally when performing the reactive part and the CAwaitable emission --> resulting a month of frustration trying to figure out why collision errors occur.
-Lack of documentation can be quite fatal for continued support.
-The multitude of potential issues that could occur that aren't necessary dealt with in the theoretical consensus algorithm or pseudo code.
-Cleipnir and how it interacts with the other programming paradigms. Eks: A clear distinction has to made in regards to what code is run inside Cleipnir(the persistent part) and what is not called in Cleipnir (orthogonal part), mixing these will cause disastrous results, which we infact encountered several times during implementation.
-Unit testing, simplicity of C# unit testing, issues in regards to unit testing networking as running tests in parallel causes inconsistent results and at worst case inf-loops
\fi
%first draft, probably be heavly changed after writing the other parts of the thesis

%The lessons I've learned for this project is that I'm useless
