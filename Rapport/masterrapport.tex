\documentclass[12pt, a4paper]{report}

% Legger inn pakker som skal brukes:	
\usepackage[utf8]{inputenc}
\usepackage{color}
\usepackage[T1]{fontenc}
\usepackage{titlesec}
\usepackage{amsmath}
\usepackage{float}
\usepackage[export]{adjustbox}
\usepackage[hidelinks]{hyperref}
\numberwithin{figure}{section}
\numberwithin{table}{section}	
\usepackage{graphicx,color,boxedminipage}
\usepackage{subcaption}
\usepackage{verbatim,amsmath}
\graphicspath{{bilder/}}
\usepackage{etoolbox}
\usepackage{tabularx}
\usepackage[margin=2.54cm]{geometry}
\usepackage[british]{babel}
\usepackage[backend=biber, style=ieee, dashed=false]{biblatex}
\bibliography{references.bib}
\usepackage{setspace}
\usepackage{csquotes}
\usepackage[titletoc]{appendix}
%\usepackage{color}
\usepackage{xcolor}
\definecolor{bluekeywords}{RGB} {0,0,255}%{0.13,0.13,1}
\definecolor{greencomments}{RGB} {0,128,0} %{0,0.5,0}
\definecolor{redstrings}{RGB} {204,0,0} %{0.9,0,0}
\usepackage{listings}
\lstset{language=[Sharp]C, %https://tex.stackexchange.com/questions/18376/beautiful-listing-for-csharp
  showspaces=false,
  showtabs=false,
  breaklines=true,
  showstringspaces=false,
  breakatwhitespace=true,
  escapeinside={(*@}{@*)},
  commentstyle=\color{greencomments},
  keywordstyle=\color{bluekeywords},
  stringstyle=\color{redstrings},
  numbers = left,
  %basicstyle=\ttfamily
}
\usepackage{enumerate}
\usepackage{enumitem}
\usepackage{environ}
\usepackage{fancyhdr,fancyvrb}
\usepackage{ragged2e,colortbl,appendix}
\usepackage{here,multirow,pdfpages}
\usepackage[font=small,labelfont=bf]{caption}
\usepackage{lipsum,todonotes} 
\usepackage{tikz}
\usetikzlibrary{trees}
\usepackage{wrapfig}
\usepackage{hyperref}
\usepackage[version = 2]{acro}
%\usepackage[toc,page]{appendix}
%\usepackage[acronym]{glossaries}
%\usepackage[printonlyused]{acronym} 
\usepackage{hyperref} % for autoref command
\hypersetup
{
	colorlinks = true,
	linkcolor = black,
	urlcolor = blue,
	citecolor = black,
}
\def\chapterautorefname{Chapter}
\def\sectionautorefname{Section}
\def\subsectionautorefname{Section}
\def\subsubsectionautorefname{Section}
\def\paragraphautorefname{Paragraph}
\def\tableautorefname{Table}
\def\equationautorefname{Equation}
\setcounter{secnumdepth}{5} %numbering of sections
\setcounter{tocdepth}{5}   %numbering of table of contents
%\setlength{\parskip}{1em}
%\titlespacing*{\paragraph}{1pt}
%Acronym list
%\newacronym{PBFT}{PBFT}{Practical Byzantine Fault Tolerance}
%\makeglossaries
%\newacronym{pbft}{PBFT}{Practical Byzantine Fault Tolerance}
\DeclareAcronym{pbft}
{
	short = PBFT,
	long = Practical Byzantine Fault Tolerance
}
\DeclareAcronym{apm}
{
	short = APM,
	long = Asynchronous Programming Model
}
\DeclareAcronym{eap}
{
	short = EAP,
	long = Event-based Asynchronous Pattern
}
\DeclareAcronym{tap}
{
	short = TAP,
	long = Task-based Asynchronous Pattern
}
\DeclareAcronym{mac}
{
	short = MAC,
	long = Message Authentication Code
}
\DeclareAcronym{tcp}
{
	short = TCP,
	long = Transmission Control Protocol
}
\DeclareAcronym{json}
{
	short = JSON,
	long = JavaScript Object Notation
}
\DeclareAcronym{fcfs}
{
	short = FCFS,
	long = First Come First Serve
}
\DeclareAcronym{ip}
{
	short = IP,
	long = Internet Protocol
}
\DeclareAcronym{linq}
{
	short = LINQ,
	long = Language Integrated Query
}
\DeclareAcronym{sql}
{
	short = SQL,
	long = Structured Query Language
}
\DeclareAcronym{api}
{
	short = API,
	long = Application Programming Interface
}

\lstdefinestyle{sharpc}{language=[Sharp]C, rulecolor=\color{blue!80!black}}
%C# listings https://tex.stackexchange.com/questions/30657/lkistings-package-for-c
%for å sette på ramme, frame=single

% Makro definisjoner
\newcommand{\code}[1]{\texttt{#1}}

%Fra rapport malen, tuller helt med centering
% kan definere bredere tekstbredde og -høyde 
%\textwidth125mm
%\textheight190mm
%\parindent0mm  % ingen innrykk ved begynnelsen av avsnitt

\setlength{\marginparwidth}{3cm}

%Usermanual setup found on this page: https://tex.stackexchange.com/a/130099 URL visited 10.05.2019
%code starts here
%\newlength\widest
%\makeatletter

%\NewEnviron{userManualItemlist}{%
%  \vbox{%
%    \global\setlength\widest{0pt}%
 %   \def\item[##1]{%
%     \settowidth\@tempdima{\textbf{##1}}%
%      \ifdim\@tempdima>\widest\global\setlength\widest{\@tempdima}\fi%
%    }%
%    \setbox0=\hbox{\BODY}%
%  }
%  \begin{description}[
%    leftmargin=\dimexpr\widest+0.5em\relax,
%    labelindent=0pt,
%    labelwidth=\widest]
%  \BODY
%  \end{description}%
%}
\makeatother
%code ends here

\begin{document}
	\setlength{\parskip}{0.5cm}   % denne lager 5mm avstand ved avsnitt
	
	\pagestyle{fancyplain}
	\renewcommand{\chaptermark}[1]{\markboth{#1}{#1}}
	\renewcommand{\sectionmark}[1]{\markright{\thesection\ #1}}
	\lhead[\fancyplain{}{\bfseries\thepage}]{\fancyplain{}{\bfseries\rightmark}}
	\rhead{}
	\chead{}
	\cfoot{\bfseries\thepage}
	\lfoot{}
	\rfoot{}	
	
	%FrontPage
	\iffalse	
	\begin{figure}[H]
		\centering
		\includegraphics[width=\textwidth]{figures/frontpageproto.pdf}
		\caption{FrontPage prototype!!!, write in info and write your signature before sending in master report.}
		\label{fig:multithread}
	\end{figure}
	\fi
	
	\hspace*{-7mm}\scalebox{0.80}{\includegraphics{figures/frontpageproto.pdf}}
	\vspace*{-15mm}
	\thispagestyle{empty}
	\newpage

	% romerske tall før kap.1
	\pagenumbering{roman}	
	
	
	\title{Implementing PBFT using Reactive programming and asynchronous workflows}
	\author{Jørgen Melstveit}
	\date{June 2021}
	\maketitle
	\tableofcontents
	\newpage

	%Content
	\abstract
	{
Consensus algorithms are notorious for being both difficult to understand in even harder to implement. Several frameworks and programming paradigms have been introduced to help make consensus algorithms easier to design and implement. One of these frameworks is the .NET Cleipnir framework which primarily focuses on making it simpler to develop a persistent consensus algorithm. In addition, Cleipnir supports functionality that makes both asynchronous and reactive programming paradigms easier for a developer to utilize in their implementation. We want to determine that the Cleipnir framework and the related programming paradigms can help design a simple and understandable consensus algorithm. To accomplish this task, we create a \acl{pbft} implementation that has its protocol workflow run as orderly and synchronous as possible using the Cleipnir framework and the aforementioned protocol paradigms.
Furthermore, we evaluate each of the previously mentioned tools to ascertain how they benefit and hinder our implementation. We discover that the benefits heavily outrank the disadvantages for both programming paradigms and works well together. We conclude that the Cleipnir framework does provide helpful tools for the implementation of consensus algorithms.  We further learn that the algorithm’s complexity can heavily affect the level of simplicity that can be provided to the algorithm workflow without the loss of functionality. 
	}
	%\abstract{}
	\newpage
	\section*{Acknowledgement}
I want to thank my supervisor Professor Leander Nikolaus Jehl, for providing consistent feedback and guidance throughout our thesis. I would also like to express my gratitude forwards Thomas Stidsborg Sylvest, who helped us learn the basics of the Cleipnir framework by sharing his expertise and answering any additional questions we had during our thesis.	
	
	\newpage		
	%\section*{Glossary}
	%\input{sections/glossary}
	%\glsaddall
	%\printglossary[type=\acronymtype,title=Acronyms]
	
	%\printglossaries
	%\newpage
	\printacronyms
	
	\setlength{\parindent}{0em} 
	\pagenumbering{arabic}
	\chapter{Introduction}
\section*{HUSK Å GJØRE OM ALLE CITATENE SOM HAR MED PAPER/BOK til format [ref, p.sidetall]!}
Systems today are required to be both efficient, secure, and reliable. Due to these factors most firmware and software today are organized over multiple systems in what we call a \textit{distributed system}~\cites{WEB:DistSys}[p.~16]{BOOK:MVstandver3}. In distributed systems, network nodes are required to share and collaborate so that the systems can agree on an overall state of the system. This state must remain consistent for the systems even in the event of failure, or in some cases malicious intent. A distributed system must be able to act as if it's a single system, even when in reality it is composed of multiple systems~\cite[p.~18]{BOOK:MVstandver3}. Advanced and technical consensus algorithms are currently being used to handle this functionality.
%The more commonly used consensus algorithms for distributed systems being Paxos, Raft and Practical Byzantine Fault Tolerance(PBFT)\cite{WEB:ConsesAlgo}.
However, most consensus algorithms are known for being difficult to fully understand and can be even more difficult to implement due to the unreliable nature of distributed networks~\cites[p.~459]{BOOK:MVstandver3}[p.~13]{PAPER:EivindPaper}. Because of this, alternative ways to describe and implement existing consensus algorithms are being discussed.

The University of Stavanger have previously published work that implements popular consensus algorithms, such as Paxos and Raft~\cite{WEB:ConsesAlgo}, in a simplified manner using frameworks that support reactive programming. In particular, "Cleipnir - Framework Support for Fault-tolerant Distributed Systems"~\cite{PAPER:PaxosCleipnir} and "Implementing a Distributed Key-Value Store Using Corums"~\cite{PAPER:EivindPaper} uses the .NET framework now known today as \textit{Cleipnir}~\cite{DOC:Cleipnir}.
Cleipnir is a .Net framework that is designed to help make implementations for consensus algorithms simpler for the developer.
These two previously mentioned works are predecessors for this thesis which intends to use the Cleipnir framework in order to implement another popular consensus algorithm to further analyze Cleipnir ability to simplify implementation of consensus algorithms.

Our goal for this thesis is to use the Cleipnir framework to implement the \ac{pbft} consensus algorithm using functionality from both asynchronous programming and reactive programming. The desired \ac{pbft} implementation should be devised using both async/await functionality existing in the .Net framework~\cite{DOC:AsyncAwait} and reactive event handling which Cleipnir has support for. Using these tools, the end goal is for the workflow of the \ac{pbft} implementation to be simple to understand as well as easier for others to recreate. In order to accomplish these goals we are looking into Cleipnir current support for reactive programming. We also look at the current workflow of modern asynchronous programming for .Net. A detailed summary of the \ac{pbft} algorithm and its operations are also given. Additionally, Cleipnir persistency functionality is also studied.
In the end the question is whether Cleipnir has the sufficient support required to accomplish these goals. We will also see how useful reactive programming paradigm and asynchronous programming are when designing a consensus algorithm.
TODO  IF there is missing info about what i'm evaluating add it so we can refer to the good code design based on the protocol description as protocol abstraction!

%\section{Introduction}
%Introduction Template
%1. Context/Motivation
%2. Problem
%	-Why this is a hard/open problem?
%	-State-of-the-Art
%3. Key idea/insight
%	-Solution overview/some detail (bigger picture)
%4. Summary of research
%	-Details of contribution
%5. Evidence of successful solution (evaluation results)
%6. Summary of contributions
%7.  Paper outline
\section{Contributions}
%REWRITE: CONTRIBUTIONS --> what is achieved in THIS PAPER, understanding is not an achievement! Try to combine contributions with the outline in some way. Example: Here is our contribution, this can be found here:... (structure).
%The following contributions are achieved by this thesis:
%\begin{itemize}
%\item Understanding the functionality and general workflow of Practical Byzantine Fault Tolerance
%\item Creating a Practical Byzantine Fault Tolerance implementation using async/await in combination with Cleipnirs reactive event handlers. This combination allowed for the our implementation to handle the main workflow of \ac{pbft} in a single function.
%\item Learning how to use Cleipnir functionality to support asynchronous programming, reactive programming and persistent programming.
%\item Providing relevant feedback for Cleipnir.
%\end{itemize}

\iffalse 
To tackle the problem we implemented a simple\ac{pbft} implementation that primarily uses async/await asynchronous programming together with reactive event handlers provided by the Cleipnir framework to design the normal workflow of \ac{pbft} inside a single function. This function is designed to follow the protocol description as closely as possible. To accomplish this goal the \ac{pbft} consensus algorithm was studied in great detail. In addition we had to implement the network layer for the \ac{pbft} implementation using .NET asynchronous socket programming~\cite{DOC:AsyncSocketProg, VIDEO:dotnetsocketprog}. In this thesis we have in addition looked at how both asynchronous programming and reactive programming has affected the simplicity of the protocol code and can conclude both positive and negative aspects for the programming model. To the best of our ability we designed the \ac{pbft} application to utilize Cleipnirs tools as much as possible. Although our current implementation does not fully support persistency, we have taken steps to at least ensure that the protocol objects and protocol-related functionality are designed with persistency in mind. We do believe that the thesis does present some helpful feedback for the future development of the Cleipnir framework.

\section{Outline}
%EDIT once the first draft is finished for each chapter!
\begin{itemize}
\item In \textbf{\autoref{chapter:ProgrammingModels}} we briefly describe the background information in regards to this thesis. This includes information in regards to asynchronous programming and reactive programming

\item In \textbf{\autoref{chapter:Cleipnir}} we make an introduction to the Cleipnir framework. This includes describing the intended use-case for Cleipnir, and summarizing its functionality that are helpful for implementing the consensus algorithm.

\item In \textbf{\autoref{chapter:PBFT}} we describe the \ac{pbft} algorithm. This includes introducing the main goals and functionality of the consensus algorithm. Concepts used by or related to the algorithm. Finally, a detailed summary of all the operations taking place in the algorithm.

\item In \textbf{\autoref{chapter:RW}} we introduce previous work in regards to the Cleipnir framework and other related work that are similar in nature to this project.

\item \textbf{\autoref{chapter:Design}} introduces an overview of our application. We first give a short summary of how the network is set up for the \ac{pbft} implementation. Then we go more in depth for how we’ve structured the code for the implementation. Finally the application is divided into separate segments based on whether or not the segment uses Cleipnir to persist its data.

\item \textbf{\autoref{chapter:Imp}} gives a detailed explanation of our \ac{pbft} implementation. Describing in detail how the normal workflow is implemented. In addition, we discuss how the implementation handles view-changes and checkpoints.  Important factors in the overall implementation are mentioned in greater detail. We describe how asynchronous programming and Cleipnir reactive programming has helped simplify the code for our implementation. Finally we discuss some drawbacks to our design.

\item \textbf{\autoref{chapter:Dis}} gives a summary of all of the benefits and disadvantages we encountered for each of the tools and designs we used in our \ac{pbft} implementation %NOT WRITTEN YET! Reintroduce debate topics in implementation and discuss their points once again. Give an opinion on the matters.

\item \textbf{\autoref{chapter:Con}} is the last chapter and it contains a conclusion for the given \ac{pbft} implementation based on the initial goals. Furthermore, we also summarize our results and discuss the knowledge we accumulated during the thesis and give suggestions for future work.
\end{itemize}
\fi

To tackle the problem, we implemented a simple \ac{pbft} implementation that primarily uses async/await asynchronous programming and reactive event handlers provided by the Cleipnir framework to design the normal workflow of \ac{pbft} inside a single function. This function is designed to follow the protocol description as closely as possible. To accomplish this goal, the \ac{pbft} consensus algorithm was studied in great detail. In addition, we had to implement the network layer for the \ac{pbft} implementation using .NET asynchronous socket programming~\cite{DOC:AsyncSocketProg, VIDEO:dotnetsocketprog}. In this thesis, we have also looked at how both asynchronous programming and reactive programming have affected the simplicity of the protocol code and can conclude both positive and negative aspects for the programming models. To the best of our ability, we designed the \ac{pbft} application to utilize Cleipnirs tools as much as possible. Although our current implementation does not fully support persistency, we have taken steps to ensure at least that the protocol objects and protocol-related functionality are designed with persistency in mind. We believe that the thesis does present some helpful feedback for future development of the Cleipnir framework.

\section{Outline}
%EDIT once the first draft is finished for each chapter!
\begin{itemize}
\item In \textbf{\autoref{chapter:ProgrammingModels}} we briefly describe the background information in regards to this thesis. This includes information in regards to asynchronous programming and reactive programming

\item In \textbf{\autoref{chapter:Cleipnir}} we make an introduction to the Cleipnir framework. This includes describing the intended use-case for Cleipnir and summarize its core functionalities that are potentially helpful for implementing consensus algorithms.

\item In \textbf{\autoref{chapter:PBFT}} we describe the \ac{pbft} algorithm. This includes introducing the main goals and processes of the consensus algorithm. We also briefly describe concepts used by or related to the algorithm. Finally, a detailed summary of all the operations taking place in the algorithm is presented.

\item In \textbf{\autoref{chapter:RW}} we introduce previous work in regards to the Cleipnir framework and other related work that are similar to this project.

\item \textbf{\autoref{chapter:Design}} introduces an overview of our application. We first give a short summary of how the network is set up for the \ac{pbft} implementation. Then we go more in-depth about how we’ve structured the code for the implementation. Finally, we describe how the application is divided into separate segments based on whether or not the segment uses Cleipnir to persist its data.

\item \textbf{\autoref{chapter:Imp}} gives a detailed explanation of our \ac{pbft} implementation. We start by first presenting our choices in design to accomplish our main objectives. Then the normal workflow implementation is described in detail. In addition, we discuss how the implementation handles view-changes and checkpoints. We describe for each workflow how asynchronous programming and Cleipnir reactive programming have helped or hindered simplifying the code for our implementation. Finally, we discuss some drawbacks to our design.

\item \textbf{\autoref{chapter:Dis}} gives a summary of all of the benefits and disadvantages we encountered for each of the tools and designs we used in our \ac{pbft} implementation.
%NOT WRITTEN YET! Reintroduce debate topics in implementation and discuss their points once again. Give an opinion on the matters.

\item \textbf{\autoref{chapter:Con}} is the last chapter and it contains a conclusion for the given \ac{pbft} implementation based on the initial goals. Furthermore, we also summarize our results, discuss the knowledge we accumulated during the thesis, and suggest future work.
\end{itemize}
	
	\chapter{Programming Models}
\label{chapter:ProgrammingModels}
\iffalse
This chapter general information about the asynchronous programming and reactive programming models are introduced. This includes main use cases and general workflow. The asynchronous programming section includes an introduction to the async/await model~\cite{DOC:AsyncAwait}. The reactive programming section includes information about ReactiveX~\cite{WEB:ReactiveXMainPage} which is the corner stone for all Rx driven implementations.
\section{Asynchronous Programming}
\label{section:AsyncProgramming}
%\subsection{Introduction}
Asynchronous programming is a programming technique designed to handle a common problem that sometimes occurs in synchronous programming. Synchronous programming always blocks the execution until the previous line of code is handled. A synchronous program delegates the operative systems resources to finish a single operation in the program, before moving on to the next operation and so on. However, blocking the execution thread in general causes issues with scalability, latency as well as in general gives a very bad user experience. Meaning synchronous programming isn't optimal for operations which require long execution time. Especially if the operation itself spends most of its time waiting, such as database requests or I/O bound operations~\cite{VIDEO:AsyncConBack, WEB:AsyncAwaitTut}. Keep in mind that asynchronous programming for different programming languages usually follow relatively the same workflow, however the naming of operations may differ. In this thesis the terminology used for asynchronous programming follows the ones used in the .NET framework.

Asynchronous programming as the name implies is designed to run operations asynchronously. In the asynchronous programming model, operations are divided into a set of tasks that perform the operations whenever the scheduler has resources which it can freely delegate to it.
However, the task created does not block the main thread, instead the main thread continues on with the next operation~\cite{WEB:AsyncAwaitTut, VIDEO:AsyncConBack, DOC:AsyncAwait}.
The task has a reference to an awaiter that has information in regard to the task's current state. Eventually the asynchronous operation finishes, and the result is available in the awaiter for the main thread to collect. Not all tasks need to necessarily return a result. It is possible to run non returning asynchronous operations in tasks as well. Nevertheless a task must always return an awaiter so that the main thread has reference to all relevant information for the asynchronous task~\cite{WEB:AsyncAwaitTut}.

Normally the main thread needs to receive the result of the asynchronous operation before reaching specific parts of the program that requires the result in order to run properly. Asynchronous functionality supports this functionality by allowing the designer to specify to the awaiter that the program is to wait at this point until the asynchronous operation is finished. This still does not block the main thread as the other tasks can be performed in the background unlike synchronous programming. Additionally, asynchronous programming has the benefit that the operation can be initialized earlier and be worked on by the main thread while going through the operations to the point where the result is expected. This means asynchronous programming could avoid bottlenecks that occur in synchronous programming and thereby making asynchronous programming more responsive of the two programming models~\cite{DOC:TaskAsyncProgModel, WEB:AsyncAwaitTut}.
For this reason, asynchronous programming has become the preferred programming model when it comes to designing user-interfaces. As it is important to avoid potentially blocking user input while another task is performed~\cites{VIDEO:AsyncConBack}[p.~214]{BOOK:DotnetMultithreadCookBook}. Server design is another example where asynchronous design is preferred as it handles a large number of requests easier than a server with synchronous design~\cite{VIDEO:AsyncConBack, DOC:AsyncAwait}.

Asynchronous programming usually follows one or more of these three design patterns:
\begin{itemize}
	\item{\ac{apm}}
	\item{\ac{eap}}	
	\item{\ac{tap}}
\end{itemize}
\ac{tap} is the most used design pattern and is the model used by the async/await workflow~\cite{DOC:AsyncAwait, WEB:AsyncAwaitTut}.

Asynchronous programming should not be confused with parallel programming as asynchronous methods do not create new threads. It instead runs on the current thread whenever the scheduler has resources ready and the operation itself is ready to progress. Therefore, the work required to create new threads as well as a lot of the work to keep the threads consistent can be omitted~\cite{DOC:TaskAsyncProgModel}. %Potentially write more once you have more control over how consistency works between async functions and how they fail in my implementation.

\subsection{Async/Await}
Asynchronous programming is not a new concept and C\# has long had support for it~\cite{WEB:asyncNelsen}. However, before the async/await workflow became normalized programming asynchronously was quite difficult and even worse for others to read~\cite{DOC:TaskAsyncProgModel}. The workflow consisting of a lot of nested callback functions which is quite a struggle to manage properly. Today managing this kind of structure is referred to as \emph{callback hell}~\cites[p.~1-2]{PAPER:Callbackhell}[p~.2]{PAPER:PaxosCleipnir}.

As mentioned previously the async/await workflow follows the \ac{tap} abstraction~\cite{DOC:TaskAsyncProgModel} meaning the workflow initializes the asynchronous operations. Then in the workflow which calls the asynchronous operation there is a point chosen where the result of the asynchronous operation needs to be collected before the workflow can move on to other operations. The async/await workflow consists of three steps for the programmer. The first step is to assign the \code{async} modifier to a function to mark it as an asynchronous function. This allows asynchronous calls to be made inside the chosen function. The second step is to make an asynchronous call. Lastly specify the \code{await} operator for the awaiter for the asynchronous task~\cite{WEB:AsyncAwaitTut, DOC:AsyncAwait, VIDEO:AsyncConBack}.
It is important to remember that the \code{await} operator can only be used in a function marked with the \code{async} modifier. In order to use an asynchronous function call in a synchronous function, the traditional operators have to be used instead~\cite{DOC:AsyncAwait, DOC:TaskAsyncProgModel}.

In \autoref{code:asyncawaitex} we can see a practical example of the async/await workflow.
The code in \autoref{code:asyncawaitex} is the asynchronous function that is responsible for having a chosen \code{Socket} object connect to a designated \ac{ip} address. The \code{IPEndPoint} object being the reference to the chosen \ac{ip} address. In order for the \code{Connect} function to be marked as an asynchronous function it has a \code{async} modifier. \code{Connect} returns a .NET \code{Task} object of type boolean, meaning the function returns a reference to the active Connect \code{Task} which will return a boolean value once the \code{Task} is completed. In this case the \code{Connect} function returns true if the socket succeeds in connecting to the \ac{ip} address, otherwise it returns false. The asynchronous operation performed in the \code{Connect} function is when the \code{ConnectAsync} function is called for the socket. As we want to avoid the function returning the result before the asynchronous operation is finished, the \code{await} operator is used to have the \code{Task} wait for the asynchronous operation to finish.
\fi

Information about the asynchronous programming and reactive programming models are introduced in this chapter. This includes their intended use cases and general workflow. The asynchronous programming section mentions several design patterns used for asynchronous operations. We will mainly concentrate on the async/await model has ~\cite{DOC:AsyncAwait}. The reactive programming section covers information about ReactiveX~\cite{WEB:ReactiveXMainPage} which is the cornerstone for all Rx-driven implementations.
\section{Asynchronous Programming}
\label{section:AsyncProgramming}
%\subsection{Introduction}
Asynchronous programming is a programming technique designed to handle a common problem that sometimes occurs in synchronous programming. Synchronous programming always blocks the execution until the previous line of code is handled. A synchronous program forces the program to finish a single operation in the program before moving on to the next operation. However, blocking the execution thread usually leads to scalability issues, latency issues and generally results in an awful user experience. Meaning synchronous programming is not optimal for operations that require a long execution time. Especially if the operation itself spends most of its time waiting for a result, examples of such actions would be database requests or I/O bound operations~\cite{VIDEO:AsyncConBack, WEB:AsyncAwaitTut}. Keep in mind that asynchronous programming for different programming languages usually has similar workflows. However, the naming conventions for identical operations may differ. In this thesis, the terminology used for asynchronous programming follows the ones used in the .NET framework.

Asynchronous programming, as the name implies, is designed to run operations asynchronously. In the asynchronous programming model, operations are divided into a set of tasks. These tasks perform the assigned operations whenever the scheduler has resources it can delegate to them.
However, the task created does not block the main thread, instead, the main thread continues with the next operations~\cite{WEB:AsyncAwaitTut, VIDEO:AsyncConBack, DOC:AsyncAwait}.
The task has a reference to an awaiter that has information on the current state of the task. Eventually, the asynchronous operation finishes, and the result is available in the awaiter for the main thread to collect. Not all tasks need to return a result necessarily. It is possible to run non-returning asynchronous operations in tasks as well. Nevertheless, a task must always return an awaiter so that the main thread has reference to all relevant information for the asynchronous task~\cite{WEB:AsyncAwaitTut}.

Normally, the main thread needs to receive the result of the asynchronous operation before reaching specific parts of the program that requires the result to function correctly. Asynchronous programming supports this functionality by allowing the designer to specify to the awaiter that the program is to wait at this point until the asynchronous operation is finished. This still does not block the main thread, meaning other tasks can be performed in the background, unlike synchronous programming. Additionally, asynchronous programming has the benefit that the operation can be initialized earlier and be worked on by the main thread while going through the main thread operations to the point where the result is expected. This means asynchronous programming could avoid bottlenecks that occur in synchronous programming, thereby making asynchronous programming more responsive of the two programming models~\cite{DOC:TaskAsyncProgModel, WEB:AsyncAwaitTut}.
For this reason, asynchronous programming has become the preferred programming model for designing user interfaces since it is crucial to avoid potentially blocking user input when at the same time, other primary tasks are performed~\cites{VIDEO:AsyncConBack}[p.~214]{BOOK:DotnetMultithreadCookBook}. Server design is another example where asynchronous design is preferred as it handles many requests easier than a server with synchronous design~\cite{VIDEO:AsyncConBack, DOC:AsyncAwait}.

Asynchronous programming usually follows one or more of these three design patterns:
\begin{itemize}
	\item{\ac{apm}}
	\item{\ac{eap}}	
	\item{\ac{tap}}
\end{itemize}
\ac{tap} is the most used design pattern and is the model used by the async/await workflow~\cite{DOC:AsyncAwait, WEB:AsyncAwaitTut}.

Asynchronous programming should not be confused with parallel programming, as asynchronous methods do not create new threads. It instead runs on the current thread whenever the scheduler has resources ready, and the operation itself is ready to progress. Therefore, the work required to create new threads and a lot of the work to keep the threads consistent can be omitted~\cite{DOC:TaskAsyncProgModel}. 

\subsection{Async/Await}
.NET has long had support for asynchronous programming~\cite{WEB:asyncNelsen}. However, before the async/await workflow became normalized, programming asynchronously was quite difficult and even worse for others to read~\cites{DOC:TaskAsyncProgModel, WEB:asyncNelsen}. The old workflow consisted of a lot of nested callback functions, which is a struggle to manage properly. Today managing this kind of structure is referred to as \emph{callback hell}~\cites[p.~1-2]{PAPER:Callbackhell}[p~.2]{PAPER:PaxosCleipnir}.

As previously mentioned, the async/await workflow follows the \ac{tap} abstraction~\cite{DOC:TaskAsyncProgModel}.  The async/await workflow, therefore, consists of creating a task that performs the asynchronous operation. Then the original process that created the asynchronous task marks where the result of the task needs to be returned in the workflow. If the task is not finished when it reaches the marked area in the workflow, the process waits at this point until the result is ready. 
The async/await workflow consists of three steps for the programmer. The first step is to assign the \code{async} modifier to a function to mark it as an asynchronous function. This allows asynchronous calls to be made inside the chosen function. The second step is to make an asynchronous call. Lastly, specify the \code{await} operator for the awaiter for the asynchronous task to determine where in the workflow the result is obtained~\cite{WEB:AsyncAwaitTut, DOC:AsyncAwait, VIDEO:AsyncConBack}.
It is important to remember that the \code{await} operator can only be used in a function marked with the \code{async} modifier. The traditional asynchronous operators have to be used instead of the async/await workflow when making asynchronous calls inside synchronous functions~\cite{DOC:AsyncAwait, DOC:TaskAsyncProgModel}.

In \autoref{code:asyncawaitex} we can see a practical example of the async/await workflow.
The code in \autoref{code:asyncawaitex} is the asynchronous process that is responsible for having a chosen \code{Socket} object connect to a designated \ac{ip} address. The \code{IPEndPoint} object being the reference to the chosen \ac{ip} address. In order for the \code{Connect} function to be marked as an asynchronous function it has a \code{async} modifier. \code{Connect} returns a .NET \code{Task} object of type boolean, meaning the function returns a reference to the active Connect \code{Task} which returns a boolean value once the \code{Task} is completed. In this case the \code{Connect} function returns true if the socket succeeds in connecting to the \ac{ip} address, otherwise it returns false. The asynchronous operation performed inside the \code{Connect} function is the \code{ConnectAsync} function which is called by the socket object. As we want to avoid the function returning the result before the asynchronous operation is finished, the \code{await} operator is used to have the \code{Task} wait for the \code{ConnectAsync}  asynchronous operation to finish.

\begin{figure}[h]
	\centering
	%\lstset{style=sharpc}
	\begin{lstlisting}[label = code:asyncawaitex, caption=Example of async/await workflow, captionpos=b, basicstyle=\scriptsize]
public static async Task<bool> Connect(Socket sock, IPEndPoint endpoint)
{
    try
    {
        await sock.ConnectAsync(endpoint);
        return true;
    }
    catch (Exception e)
    {
        Console.WriteLine("Failed to connect to endpoint: " + endpoint.Address);
        Console.WriteLine(e);
        return false;
    }
}
	\end{lstlisting}
\end{figure}

\iffalse
%TODO find/write a better example.
\begin{figure}[h]
	\centering
	%\lstset{style=sharpc}
	\begin{lstlisting}[label = code:asyncawaitex, caption=Example of async/await workflow, captionpos=b, basicstyle=\scriptsize]
public async Task SendMessage(byte[] sermessage, 
                              Socket sock, 
                              MessageType type)
{
    Console.WriteLine($"Sending: {type} message");
    var mesidentbytes = Serializer
                        .AddTypeIdentifierToBytes(sermessage, type);
    var fullbuffmes = NetworkFunctionality
                      .AddEndDelimiter(mesidentbytes);
    await sock.SendAsync(fullbuffmes, SocketFlags.None);
}


	\end{lstlisting}
\end{figure}
\fi
%\cite{VIDEO:AsyncConBack}
%\cite{DOC:AsyncAwait}
%\cite{DOC:TaskAsyncProgModel}
%\cite{BOOK:DotnetMultithreadCookBook}
%\cite{WEB:AsyncAwaitTut}
\section{Reactive Programming}
\label{section:reactive}
\iffalse

Reactive programming is a programming paradigm that focuses on changing the state of the program in response to some outward changes~\cite{WEB:RxProgIntro, DOC:Cleipnir}.
Reactive programming follows an event driven workflow. An event can be triggered from one part of the system and when this event is received by the other part it starts altering the state of the system in response. Reactive programming works hand in hand with asynchronous event-based programming which was mentioned previously briefly in \autoref{section:AsyncProgramming}~\cite[p.~2-3]{BOOK:RxLinq}. Reactive programming is commonly used to handle a continuous stream of asynchronous data~\cite{VIDEO:dotnetsheffReactive}. 
Currently there exists a lot of support for Reactive programming. Specifically, the library Reactive X~\cite{WEB:ReactiveXMainPage} has presented a general \ac{api}~\cite{WEB:api} for implementing the core concepts of reactive programming. As a result, today there exist a lot of reactive extensions for multiple programming languages. Rx.NET~\cite{Github:ReactiveExtensions} is the official .NET reactive extension. Cleipnirs has implemented its own reactive extension that resembles Rx.NET very closely. The main difference between the two is that Cleipnirs reactive layer supports persistency, but lacks reactive operators that Rx.NET does support~\cite{DOC:Cleipnir}.
Although Cleipnir and Rx.NET vary somewhat from the general \ac{api}, the general workflow remains the same. Therefore we will introduce the main concepts of Reactive X in this section. Details specific for Cleipnir are instead presented in the upcoming \autoref{chapter:Cleipnir}.

\subsection{Reactive X}
ReactiveXs workflow can be easily summarized with the following tasks~\cite{WEB:ReactiveObservable}
\begin{enumerate}
	\item{Start an asynchronous operation that will perform some work and eventually return it}
	\item{Transform the asynchronous operation as an Observable object}
	\item{Use reactive operators to transform/filter the resulting data.}
	\item{Observers subscribe to the Observable and waits for the Observable to return the data}
\end{enumerate}

An observable object follows a similar structure to an enumerable object, where the main difference between an enumerable and an observable object is their method of accessibility. In an enumerable object will give the next object in storage whenever asked for it. In other words, the program will dictate when the next entry will be collected. In an Observable object the next result is instead pushed to its subscriber whenever the result is ready. The program has no control over when the next entry will be ready as it is waiting for an asynchronous operation to complete~\cites{WEB:ReactiveObservable, VIDEO:dotnetsheffReactive, VIDEO:MicroDev}[p.~15]{BOOK:RxLinq}. Observables, like enumerable, support the use of \ac{linq} queries on its resulting data. \ac{linq} add additional operators for filtering and transforming the resulting data into new enumerables~\cites{VIDEO:dotnetsheffReactive}[p.~3-4]{BOOK:RxLinq}[p.~208]{BOOK:DotnetMultithreadCookBook}.

Traditionally, the implementation is expected to incorporate the following functions for its observer object.
\begin{itemize}
	\item{OnNext}
	\item{OnError}
	\item{OnCompleted}
\end{itemize}

OnNext is the function that handles each new incoming event emitted by the Observable. OnError is the function that is called if an error occurs within handling one of the emitted events. OnCompleted is the function that is called when the observable is finished and will no longer emit any new events~\cite{WEB:ReactiveObservable}.

In some implementations the Observable and observer functionality are merged together into an object called subject. A subject object acts as a bridge of sorts between the observer and the observable where its main usage is to simplify the workflow for reactive programming. A subject has the ability to subscribe to an observable just like an observer. However, unlike an observer a subject can also re-emit events already processed in the observable, as well as be used for emitting new events to the observable. Eventually, all the items emitted by the subject will also be handled by the subject, making the programming workflow a lot simpler compared to its traditional style~\cite{WEB:ReactiveSubject}. Cleipnir supports subject in its implementation, however the objects are not referred to as subject, but rather source objects.
\fi

\iffalse
-a brief introduce reactive programming
-usecase
-the ReactiveX library(what it does, how it works and the mention Rx.Net)
-give brief through workflow, concepts with name and definitions and how it works.(Observable, stream of data, subjects, event driven programming)
-mention briefly Cleipnir support for reactive programming, how they differ, say detail information is given in Cleipnir chapter.
\fi

Reactive programming is a programming paradigm whose main focus is to change the state of the program in response to some outward changes~\cite{WEB:RxProgIntro, DOC:Cleipnir}.
Reactive programming follows an event-driven workflow. An event can be triggered from one part of the system, and when the other part of the system receives this event, it alters the state of the system in response. Reactive programming works hand in hand with asynchronous event-based programming, which was previously mentioned briefly in \autoref{section:AsyncProgramming}~\cite[p.~2-3]{BOOK:RxLinq}. Reactive programming is commonly used to handle a continuous stream of asynchronous data~\cite{VIDEO:dotnetsheffReactive}. 
Currently, there exists a lot of support for Reactive programming. Specifically, the library Reactive X~\cite{WEB:ReactiveXMainPage} has presented a general \ac{api}~\cite{WEB:api} for implementing the core concepts of reactive programming. As a result, today, there exist a lot of reactive extensions for multiple programming languages. Rx.NET~\cite{Github:ReactiveExtensions} is the official .NET reactive extension. Cleipnirs has implemented its own reactive extension that closely resembles  Rx.NET. The main difference between the two is that Cleipnirs reactive layer supports persistency but lacks reactive operators that Rx.NET does support~\cite{DOC:Cleipnir}.
Although Cleipnir and Rx.NET vary somewhat from the general \ac{api}, the general workflow remains the same. Therefore we will introduce the main concepts of Reactive X in this section. Details specific for Cleipnir are instead presented in the upcoming \autoref{chapter:Cleipnir}.

\subsection{Reactive X}
ReactiveXs workflow can be easily summarized with the following tasks~\cite{WEB:ReactiveObservable}
\begin{enumerate}
	\item{Start an asynchronous operation that will perform some work and eventually return it}
	\item{Transform the asynchronous operation as an Observable object}
	\item{Use reactive operators to transform/filter the resulting data.}
	\item{Observers subscribe to the Observable and waits for the Observable to return the data}
\end{enumerate}

An observable object follows a similar structure to an enumerable object, where the main difference between an enumerable and an observable object is their method of accessibility. An enumerable object will give the next object in storage whenever asked for it. In other words, the program will dictate when the next entry is collected. In an Observable object, the next result is instead only pushed to its subscriber whenever the result is ready. The program has no control over when the next entry will be ready as it is waiting for an asynchronous operation to complete~\cites{WEB:ReactiveObservable, VIDEO:dotnetsheffReactive, VIDEO:MicroDev}[p.~15]{BOOK:RxLinq}. Observables, like enumerable, support the use of \ac{linq} queries on its resulting data. \ac{linq} add additional operators for filtering and transforming the resulting data into new enumerables~\cites{VIDEO:dotnetsheffReactive}[p.~3-4]{BOOK:RxLinq}[p.~208]{BOOK:DotnetMultithreadCookBook}.

Traditionally, the implementation is expected to incorporate the following functions for its observer object.
\begin{itemize}
	\item{OnNext}
	\item{OnError}
	\item{OnCompleted}
\end{itemize}

OnNext is the function that handles each new incoming event emitted by the Observable. OnError is the function that is called if an error occurs within handling one of the emitted events. OnCompleted is the function that is called when the observable is finished and will no longer emit any new events~\cite{WEB:ReactiveObservable}.

In some implementations, the Observable and observer functionality are merged together into an object referred to as a \emph{subject}. A subject object acts as a bridge of sorts between the observer and the observable, where its primary usage is to simplify the workflow for reactive programming. A subject has the ability to subscribe to an observable, just like an observer. However, unlike an observer, a subject can also re-emit events already processed in the observable, and be used for emitting new events to the observable. Eventually, all the items emitted by the subject is handled by the subject, making the programming workflow a lot simpler compared to its traditional style~\cite{WEB:ReactiveSubject}. Cleipnir supports subject object in its implementation, however, the objects are not called subject in Cleipnir`s implementation but are instead called \code{Source} objects.


	
	\chapter{Cleipnir}
\label{chapter:Cleipnir}

Cleipnir is a .Net framework primarily designed to be used for aiding in implementing consensus algorithms. Specifically, the framework main contribution is aiding developers with creating persistent distributed systems. Prior to this thesis, Cleipnir and its predecessor Corums, have been used to implement two consensus algorithms, namely Paxos~\cite[p.~32-38]{PAPER:EivindPaper} and Raft~\cite[p.~13-15]{PAPER:PaxosCleipnir}. 
Cleipnir is designed to support and work with the three following programming paradigms~\cite[p.~5]{PAPER:PaxosCleipnir}: 
\begin{itemize}
\item {Reactive Programming}
\item {The Async/Await Model}
\item {Persistent Programming}
\end{itemize}

Two of these programming paradigms were already presented in \autoref{chapter:ProgrammingModels}, therefore only the Persistent Programming paradigm will be introduced in this chapter. The async/await model used in Cleipnir is the official implementation from the .Net framework~\cite{DOC:AsyncAwait}. As mention in \autoref{section:reactive}, Cleipnir uses its own customized reactive framework and will be discussed in detail in this chapter.

The information that are presented in this chapter is based on the Cleipnir paper~\cite{PAPER:PaxosCleipnir}, its current documentation~\cite{DOC:Cleipnir} and from informative conversations with the frameworks creator Thomas Stidsborg Sylvest.

%Motivation
\section{Cleipnir Overview}

There are three main tools that Cleipnir provide developers to help design their application.
These three tools are:
\begin{itemize}
	\item{Persistent Synchronous Scheduler}
	\item{Storage Engine}
	\item{Object Store}
	\item{Reactive Programming Layer}
\end{itemize}
%Scheduler
Cleipnir uses an inbuilt event-driven scheduler which follows a single threaded structure similar to the JavaScript scheduler\cites[p.~7]{PAPER:PaxosCleipnir}{WEB:CleipnirScheduler}. The scheduler schedules incoming tasks in a queue structure, meaning the ordering follows a first-come first-serve(FCFS)\cite{WEB:FIFO} approach. Each task in the queue will be executed sequentially using only a single thread, which in theory will allow the program to avoid common threading issues\cite[p.~7]{PAPER:PaxosCleipnir}. %find the actual phrase for this

The storage engine is responsible for the actual storage procedure. It is responsible for performing both the serialization and the deserialization process to each of state object that is to be persisted. The details in regards to setting up object information for the serialization and deserialization process that the storage engine uses is presented in \autoref{section:PersistentProgramming}. Cleipnir uses different types of storage engines that are correlated to which storage is used to store the data. Cleipnir currently supports these three storage engines:
\begin{itemize}
	\item{Memory Storage}
	\item{Simple File Storage}
	\item{Relation Database Storage}
\end{itemize}
\cites[p.~10,12]{PAPER:PaxosCleipnir}
The memory storage stores the persisted data directly into memory.
The simple file storage stores the persisted data in a single text file(.txt).
The relation database storage stores the persisted data into a Microsoft SQL Server\cite{WEB:MSSQL}.

The Cleipnir serialization process follows a graph structure like structure. The original object graph that is to be serialized is called a \code{Roots} object. If the object that is to be serialized has references to other objects that are also going to be serialized, then the graph object has pathways leading from the \code{Roots} object to the other objects graph object\cite[p.~10]{PAPER:PaxosCleipnir}.

The object store is responsible for accessing the storage engine. The object store uses the storage engine whenever the application needs to restore some data that was previously persisted, using the storage engine to persist a new object, or to update existing object records in the persisted memory. The object store is also responsible for detecting changes done to any state variables that are set be persisted. 
The object store uses a statemap to keep track of records for each of the states variables that are to be persisted and stored by the storage engine\cite[p.~11]{PAPER:PaxosCleipnir}. 

In \autoref{code:objectstore} shows a short example of how to use the object store to cache an object into the storage engine and then restore the object after the data is lost in the application. First both the storage engine and the object store is first initialized, where the type of storage engine used for this example is the simple file storage. Then the object store uses the \code{Attach} function to register the request object to the object store. Object store now has a \code{Roots} entry for the request object. In this example the \code{Persist} function is used to serialize and store any objects currently registered in the object store that has either not been cached before or if any changes has affected the object. In this case it is only the request object. The object is then renewed and loads back the storage engine used earlier and calls the \code{Resolve} function to restore the request object and assign it to a new variable. This means that the first request and the second request will be equal. Most of the functionality shown in \autoref{code:objectstore} is performed behind the scenes and a developer rarely has to attach an object to the object store directly.  

\begin{figure}[H]
	\centering
	\lstset{style=sharpc}
	\begin{lstlisting}[label = code:objectstore, caption=Object Store example, captionpos=b, basicstyle=\scriptsize]
_storage = new SimpleFileStorageEngine("PersistentStorage.txt");
_objectStore = ObjectStore.New(_storage);
(_pri, _pub) = Crypto.InitializeKeyPairs();	
var currentTime = DateTime.Now.ToString();
Request req1 = new Request(1, "Hello World!", currentTime);
req.SignMessage(_pri);
           
_objectStore.Attach(req1);
_objectStore.Persist();
_objectStore = null;
_objectStore = ObjectStore.Load(_storage, false);
Request req2 = _objectStore.Resolve<Request>();
	\end{lstlisting}
\end{figure}

%Objectstore + Storage engine
%Execution engine
The scheduler and the object store operates independently from each other. In order for an application to take advantage of both of these tools, Cleipnir has an execution engine tool which uses both the scheduler and the object store to the best of their abilities.
The Cleipnir execution engines overall architecture is constructed so that the scheduler and the object store can be used together and can collaborate within a single mechanism. Using the execution engine the developer can specify the task's that are to be executed by the scheduler and use the object store to persist the state of the application during certain parts of the execution. The execution engine uses what are known as \code{Sync} points in order to determine when to call the \code{Persist} function in the object store. The these points have to be added manually by the developer in areas where the state can potentially become corrupt if not the state is persistent after a crash. This is important for implementing a consensus algorithm since interrupting a process mid execution while also lose vital information to the execution can cause major issues with the state of the distributed system. By default, if the scheduler does not have any tasks in its queue and is not working on any existing tasks, then it should also call the \code{Persist} function so that the state can be saved during a silent period. \autoref{code:executionEngine} shows an example of how to initialize the execution engine and how to schedule an operation\cite[p.~11]{PAPER:PaxosCleipnir}.

\begin{figure}[H]
	\centering
	\lstset{style=sharpc}
	\begin{lstlisting}[label = code:executionEngine, caption=Execution engine example, captionpos=b, basicstyle=\scriptsize]
var storageEngine = new SimpleFileStorageEngine(".PBFTStorage"+paramid+".txt", false);
scheduler = ExecutionEngineFactory.StartNew(storageEngine);
scheduler.Schedule(() => 
{
 ...
});
	\end{lstlisting}
\end{figure}

\section{Cleipnir Reactive Programming}
%TODO rewrite this segment, it is basically all wrong!
The Cleipnir framework has a custom-made reactive layer that follows most of the functionality provided by the Reactive X API. However, the basic functionality introduced in \autoref{section:reactive} is mostly hidden and the overall workflow is simplified for use. This implementation uses a \code{Stream} object in order to replicate the data returned by the respective observers and operators. The \code{Stream} is similar to an observable object.
Cleipnir reactive layer support less reactive operators compared to most other current reactive implementations. However, the current operators that exist also support persistent programming, meaning the data stream and the scheduled operations are not lost in if the system crashes during the operation. Traditional LINQ commands do not work on the \code{Stream} object. Instead inbuilt LINQ statements are available for the reactive \code{Stream} object to use. Cleipnir reactive operators can by design be chained together just like majority of reactive operators. The main difference being that the Cleipnir's reactive operators and Linq operators results in a new \code{Stream} object instead of a new observable or an new enumerable. Using reactive operator chains, it is possible to create a lot of the consensus algorithms workflow within a few lines of code. \autoref{code:operatorreq} shows an example of operator chaining using Cleipnir reactive framework. The objective here is to get the first valid Preprepare message emitted to the observable. In order for a message to be considered valid, it must pass the first two \code{Where} clause. The \code{Next} function returns the resulting prepare message~\cites[p.~6,8,13]{PAPER:PaxosCleipnir, WEB:ReactiveOperator}.


\begin{figure}[H]
	\centering
	\lstset{style=sharpc}
	\begin{lstlisting}[label = code:operatorreq, caption=Chaining Cleipnir Operators, captionpos=b, basicstyle=\scriptsize]
var preprepared = await MesBridge
                  .Where(pm => pm.PhaseType == PMessageType.PrePrepare) 
                  .Where(pm => pm.Validate(Serv.ServPubKeyRegister[pm.ServID], 
                                           Serv.CurView, 
                                           Serv.CurSeqRange)
                  )
                  .Next();
\end{lstlisting}
\end{figure}
Cleipnir supports reactive subject functionality, and it is called a \code{Source} object. The user can emit items to the \code{Source} object, and any observer linked to it will receive the response item. The \code{Source} object needs to be used in order for the developer to access and interact with the reactive layer in Cleipnir. \autoref{code:sourcereq} shows an example on how to initialize, emit and wait for incoming events in regards to the \code{Source} object. The \code{await reqbridge.Next()} makes sure that the resulting variable \code{req} receives the \code{Request} emitted to the \code{Source} object~\cite[p.~8]{PAPER:PaxosCleipnir}.

\begin{figure}[H]
	\centering
	\lstset{style=sharpc}
	\begin{lstlisting}[label = code:sourcereq, caption=Source object example, captionpos=b, basicstyle=\scriptsize]
Source<Request> reqbridge = new Source<Request>();
reqbridge.Emit(new Request(1, "Hello World!", DateTime.Now.ToString());

var req = await reqbridge.Next();
	\end{lstlisting}
\end{figure}

\section{Cleipnir Persistent Programming}
%Define Persistent Programming
\label{section:PersistentProgramming}
A system which follows the persistent programming paradigm will regularly save the information from the program state while the program is running. Persistent programming makes it possible to design systems which can easily restore its program state in the case of a system reboot~\cites[p.~6]{PAPER:PaxosCleipnir}{DOC:Cleipnir}. Consensus algorithm can take great advantage of this programming paradigm as systems in the network are likely to eventually crash. With persistent programming, it is simple for a system to recover its data and rejoin the distributed network. Unfortunately, the state of the system is likely to still be somewhat behind the other systems when compared directly to the other working systems, even if all the of the previous data is recovered.

Cleipnir supports easy to use hybrid persistent programming. Hybrid persistent programming allows the developer to freely choose which data is to be persistable. In this way it is possible to avoid storing unnecessary information that would slow down the process immensely. \autoref{code:interfaceexample} and \autoref{code:seriadeseria} shows an example of the workflow needed for an object to be serialized and deserialized to and from persistent memory. In order for an object to become visible to the storage engine, it needs to first inherit either the \code{IPersistable} or the \code{IPropertyPersistable} interface. \code{IPersistable} is usually the common choice as it can support hybrid persistency programming. The IPersistable allows the user to choose which data in the object is to be serialized and which constructor to use for the deserialize operation. The IPropertyPersistable can only use the default inbuilt constructor for a .Net object, which is why it does not support hybrid persistency and is therefore not the recommended interface. 
When inheriting the \code{IPersistable} interface the program will inherit the \code{Serialize} function as shown in \autoref{code:seriadeseria}. In this function you use the object stores statemap to set the desirable object to a designated key, like a normal map workflow. The storage engine internally references to different graph objects for each object stored. It is therefore possible for a key in the statemap to have the same key for multiple objects as the objects are treated as different graph objects in the storage engine. Meaning a developer does not need to worry about duplicate keys over different objects. 

However, the storage engine cannot store all types of data. The storage engine can handle the basic data types like int, string, boolean etc. Unfortunately, the storage engine does not support inbuilt data structures like arrays, dictionaries, etc. The storage engine does also not support any objects or data types outside of the basic ones or objects which inherit the IPersistable interface and has functional serializer and deserializer functions. This means data types like enum are not supported. However, Cleipnir supports inbuilt versions of common data structures like array, dictionary and list that can in fact be persisted by the storage engine. Therefore, a easy work around is to simply substitute normal data structures for the inbuilt Cleipnir versions of the data structure. For instance, a dictionary object can be substituted for Cleipnirs \code{Cdictionary} object. For objects that has a datatype which is not supported by Cleipnir, a common workaround is to type cast it into another format which Cleipnir can persist. An example of this can be seen in \autoref{code:seriadeseria} where the object \code{Phasetype} is of enum type and Cleipnir cannot persist enum type objects. Therefore, it is type casted to int while stored in memory. Then in the deserialize process, the correct enum type can be chosen based on the int value that was stored. For the deserialize process a private static function called \code{Deserialize} is needed which uses the state map as inparameter. Even if the content of the function must be unique for each object’s constructor, the format of the function follows the same format shown in \autoref{code:seriadeseria}. The deserialize function simply initializes the object through a constructor and then return the new instance of the specified object based on the info which was currently stored in the statemap.

\begin{figure}[H]
	\centering
	\lstset{style=sharpc}
	\begin{lstlisting}[label = code:interfaceexample, caption=Persistent initialize process, captionpos=b, basicstyle=\scriptsize]
public class PhaseMessage : IPersistable %inherit interface
		
%Construtor to Deserialize process
public PhaseMessage(int id, int seq, int view, byte[] dig, PMessageType phase, byte[] sign)
{
    ServID = id;
    SeqNr = seq;
    ViewNr = view;
    Digest = dig;
    PhaseType = phase;
    Signature = sign;
}
	\end{lstlisting}
\end{figure}

\begin{figure}[H]
	\centering
	\lstset{style=sharpc}
	\begin{lstlisting}[label = code:seriadeseria, caption=Serialize/Deserialize code example, captionpos=b, basicstyle=\scriptsize]
public void Serialize(StateMap stateToSerialize, SerializationHelper helper)
{
    stateToSerialize.Set(nameof(ServID), ServID);
    stateToSerialize.Set(nameof(SeqNr), SeqNr);
    stateToSerialize.Set(nameof(ViewNr), ViewNr);
    stateToSerialize.Set(nameof(Digest), Serializer.SerializeHash(Digest));
    stateToSerialize.Set(nameof(PhaseType), (int)PhaseType);
    stateToSerialize.Set(nameof(Signature), Serializer.SerializeHash(Signature));
}

private static PhaseMessage Deserialize(IReadOnlyDictionary<string, object> sd)
{
    return new PhaseMessage(
        sd.Get<int>(nameof(ServID)),
        sd.Get<int>(nameof(SeqNr)),
        sd.Get<int>(nameof(ViewNr)),
        Deserializer.DeserializeHash(sd.Get<string>(nameof(Digest))),
        Enums.ToEnumPMessageType(sd.Get<int>(nameof(PhaseType))),
        Deserializer.DeserializeHash(sd.Get<string>(nameof(Signature)))
        );
}
	\end{lstlisting}
\end{figure}

Finally, an introduction to the Cleipnir class \emph{CTask} needs to be introduced. As the name suggests \code{CTask} shares similar traits with the \code{Task} object. An asynchronous function which returns a \code{CTask} is an asynchronous operation which is to be run by the Cleipnir execution engine. In a sense using \code{CTask} for an asynchronous operation means the operation performed inside the asynchronous function is meant to be persistable. In order for an object to be persisted during execution it needs to be run synchronously or in an asynchronous \code{CTask} operation. An example of this would be if the user wanted to persist one of the reactive Cleipnir \code{Source} objects, than the function waiting for emitted items need to return a \code{CTask} rather than a \code{Task}. Otherwise Clepnir storage engine will crash upon attempting to persist it. 

Keep in mind \code{CTask} are not meant to be used on asynchronous functions unless you intend to use Cleipnir to persist the data. Using asynchronous operations inside a \code{CTask} will cause Cleipnir to create a new thread to handle the asynchronous operations while continuing on with the rest of the operations inside the function. This also applies when the \code{await} operator is used, meaning the \code{await} becomes redundant and will not work as intended. This also applies when scheduling new operations for Cleipnir inside a \code{CTask} function, since the schedule function for the Cleipnir execution engine is treated as an asynchronous operation. It is important for a user of Cleipnir to keep this in mind as to avoid creating potential race conditions within implementation. Normally it is best to try and avoid this situation entirely. Thereby restricting a \code{CTask} function to only operate with synchronous operations, while any asynchronous operations required should be performed instead using the TAP workflow with async/await operators discussed in \autoref{section:AsyncProgramming}
	
	\chapter{Practical Byzantine Fault Tolerance}
\label{chapter:PBFT}
%version 1
\iffalse
This chapter presents the \acl{pbft} consensus algorithm in detail.
Starting with introducing the system model commonly used for \ac{pbft}. Then a detailed explanation is given for how the protocol normally operates. This includes mechanisms such as checkpoint and leader changes.

\section{Introducing Practical Byzantine Fault Tolerance}
\acl{pbft} is a consensus algorithm specifically designed to handle Byzantine faults in an asynchronous distributed network. The algorithm was originally published in 1999 by Miguel Castro and Barbara Liskov~\cite{PAPER:OGPBFT}.
Notably the Linux foundation's open source blockchain by the name of Hyperledger~\cite{WEB:PBFTGeeks, SLIDES:PBFT} uses \ac{pbft}.

The problems derived from byzantine faults originally came to light through a well-known problem known as the Byzantine Generals Problem~\cites{WEB:BFTInfo}{ART:lamportByzGenProb}[p.~240-253]{BOOK:BuildDepDistSyst}.
The byzantine generals' problem can be summarized as a couple of army generals which are each leading their own armies and they need to together reach a decision. The most common scenario used is  that the armies try to coordinate an attack on a surrounded city. The armies can only survive if the majority of the generals agree to either attack the city together or majority agree to retreat to fight another day. There are also traitor generals that actively attempt to sabotage the order. The decision is also irreversible regardless of the action performed by the other armies. A byzantine fault tolerant system is a system that can handle the issue introduced by the byzantine generals problem and is the main goal for consensus algorithms to achieve this state. This includes the \ac{pbft} algorithm~\cite{WEB:BFTInfo, ART:lamportByzGenProb}.

The \ac{pbft} algorithm focuses on creating a state machine network that can withstand byzantine failures~\cite[p.~456]{BOOK:MVstandver3}. The protocol achieves this by providing the network with two main properties. These properties are referred to as safety and liveness.
To summarize these properties:\\
\textbf{Safety} is the property that ensures that the total ordering of requests is equal for all the non-faulty participating servers. In other words the system state should be similar to a synchronous system, operating one operation at the time, despite the fact that the system is operated over multiple remote machines.\\
\textbf{Liveness} is the property that ensures that the correct result is eventually agreed upon and returned by the system~\cites[p.~456]{BOOK:MVstandver3}{WEB:ConsesAlgo}[p.~2]{PAPER:OGPBFT}{SLIDES:PBFT}[p.~403]{PAPER:PBFTRecovery}[p.~257]{BOOK:BuildDepDistSyst}.
\fi

This chapter presents the \acl{pbft} consensus algorithm in detail.
We start by first introducing the system model commonly used for the \ac{pbft} algorithm. Then, a detailed explanation is given for how the protocol normally operates. This includes mechanisms such as checkpoint and leader changes.

\section{Introducing Practical Byzantine Fault Tolerance}
\acl{pbft} is a consensus algorithm specifically designed to handle Byzantine faults in an asynchronous distributed network. The algorithm was first published in 1999 by Miguel Castro and Barbara Liskov~\cite{PAPER:OGPBFT}.
Notably, the Linux foundation's open-source blockchain named Hyperledger~\cite{WEB:PBFTGeeks, SLIDES:PBFT} uses \ac{pbft}.

The problems derived from byzantine faults originally came to light through a well-known problem known as the Byzantine Generals Problem~\cites{WEB:BFTInfo}{ART:lamportByzGenProb}[p.~240-253]{BOOK:BuildDepDistSyst}.
The Byzantine Generals Problem can be summarized as a couple of army generals who are each leading their own armies, and they need to reach a decision together. The most common scenario used is that the armies try to coordinate an attack on a surrounded city. The armies can only survive if the majority of the generals agree to attack the city together or the majority agree to retreat to fight another day. There are also traitor generals that actively attempt to sabotage the order. The decision is also irreversible regardless of the action performed by the other armies. A Byzantine Fault Tolerant system is a system that can handle the issue introduced by the byzantine generals’ problem and is the main goal for consensus algorithms to achieve this state. This includes the \ac{pbft} algorithm~\cite{WEB:BFTInfo, ART:lamportByzGenProb}.

The \ac{pbft} algorithm focuses on creating a state machine network that can withstand Byzantine failures~\cite[p.~456]{BOOK:MVstandver3}. The protocol achieves this by providing the network with two main properties. These properties are referred to as safety and liveness.
To summarize these properties:\\

\textbf{Safety} is the property that ensures that the total ordering of requests is equal for all the non-faulty participating servers. In other words, the system state should be similar to a synchronous system, operating one operation at a time, even though the system is operated over multiple remote machines.\\
\textbf{Liveness} is the property that ensures that the correct result is eventually agreed upon and returned by the system~\cites[p.~456]{BOOK:MVstandver3}{WEB:ConsesAlgo}[p.~2]{PAPER:OGPBFT}{SLIDES:PBFT}[p.~403]{PAPER:PBFTRecovery}[p.~257]{BOOK:BuildDepDistSyst}.

\section{System Model}
\label{sec:systemModel}
The PBFT consensus algorithm is implemented using \emph{R} number of servers referred to as \emph{replicas}. When a replica is down or behaving maliciously then we say that the replica is faulty. The number of faulty replicas is symbolized as \emph{f}.
Quorum is a term used to refer to the limit of messages required to verify that the majority of replicas in the system agreed upon a decision\cites[p.~408-409]{PAPER:PBFTRecovery}. %Add another more detailed cite here!
A single replica is chosen as the leader called primary, and is represented as \emph{p}. The other replicas are referred to as backups. The responsibility of the primary is to order the request sent to the system by numerous clients~\cites[p.~456]{BOOK:MVstandver3}[p.~405]{PAPER:PBFTRecovery}. The replica that is chosen as the primary is based on the replica's identifier value~\cite[p.~258]{BOOK:BuildDepDistSyst}.

According to~\cites[p.~3]{PAPER:OGPBFT}[p.~405]{PAPER:PBFTRecovery}, replicas in the distributed network move through "successions of configurations known as views". A simpler definition for a view is the number that defines the set of non-faulty replicas which are participating in the current PBFT protocol round set up by the current primary. The current view number is symbolized by the letter \emph{v}.
As mentioned previously, the primary is chosen based on an identifier value \emph{i}. That identifier value is determined by the formula $p = v ~mod~ R$~\cites[p.~258]{BOOK:BuildDepDistSyst}[p.~3]{PAPER:OGPBFT}{SLIDES:PBFT}.
We decided to set the initial view number to zero, which results in the formula setting replica zero as the initial primary.

The protocol can only guarantee the safety and liveness properties of a system if the number of faulty replicas does not exceed a specified margin of the total replicas in the network. The total number of replicas required to be in the system should be derived by the formula $R > 3f + 1$.
The formula shows that for each new faulty replica that is to be handled in the network, three additional replicas are required. As an example, the lowest number of replicas a system can have is four. In this situation the system can only handle up to one faulty replica. In order to handle more faulty replicas the system has to scale up by adding three additional servers for each faulty server that exist in the system~\cites[p.~257]{BOOK:BuildDepDistSyst}[p.~403]{PAPER:PBFTRecovery}{SLIDES:PBFT}[p.~3]{PAPER:OGPBFT}.

All the messages sent between replicas are expected to be digitally signed by their sender. The signature process uses public-key cryptography~\cite[p.~257,p.267]{BOOK:BuildDepDistSyst}. A hidden private key is used to sign the messages while the other parties can use the replica's public key to verify this signature~\cite[p.~417]{PAPER:PBFTRecovery}. The signature procedure is used to verify that the sender is who they claim to be~\cite[p.~3]{PAPER:OGPBFT}. In some cases, the digital signatures are replaced with a Message Authentication Code (MAC). This is done for removing potential bottlenecks in performance as well as to detect tampering in messages\cites[p.~257]{BOOK:BuildDepDistSyst}[p.~3,8]{PAPER:OGPBFT}. In this PBFT implementation, digital signatures are used for all message types.


%According to~\cite{PAPER:OGPBFT,PAPER:PBFTRecovery} page 3 and 8 respectively, replicas move through "successions of configurations known as views". A simpler definition for a view is a number that defines the set of non-faulty replicas that are participating in the current PBFT protocol round set up by the current primary. , where p refers to the primary, v is the current view number and |R| is the number of replicas.

\iffalse
\section{Detailed Protocol Operations}
\label{sec:detailedProtocol}
The \ac{pbft} consensus protocol is divided into three phases. The Pre-Prepare, Prepare and the Commit phase. If the \ac{pbft} protocol operations are executed properly, consensus has been achieved for an operation once all three phases have transpired on $3f+1$ replicas~\cite[p.~257-259]{BOOK:BuildDepDistSyst}. Role of the pre-prepare phase and prepare phase is to propose an ordering for requests delivered to the system, while the combination of prepare phase and the commit phase establishes the execution order for the replicas in the system~\cite[p.~4]{PAPER:OGPBFT}. \autoref{fig:pbftnormalworkflow} shows an illustration of the \ac{pbft} workflow. The illustration shows the messages sent from the different replicas during the different protocol phases in \ac{pbft}.

The \ac{pbft} protocol starts once a client sends a request containing their desired operation to the primary~\cite[p.~4]{PAPER:OGPBFT}. Sometimes the client will also multicast their request to the other replicas in the system as well, which is the model that we followed in our implementation~\cites[p.~2]{PAPER:DPBFT}[p.~406]{PAPER:PBFTRecovery}[p.~258]{BOOK:BuildDepDistSyst}. Regardless of which of these models is used for the request message, the primary is the one responsible for starting the iteration of the \ac{pbft} algorithm to process the client's request. The primary will create a Pre-Prepare message and assign the request with a sequence number which is then multicasted to the other replicas in the network that have the same view number as the primary. Once a replica receives the Pre-Prepare message it will validate the Pre-Prepare message. The validation process consist of the following~\cites[p.~4]{PAPER:OGPBFT} {SLIDES:PBFT}[p.~259]{BOOK:BuildDepDistSyst}
\begin{itemize}
	\item[-]Validating the Signature in the message.
	\item[-]Checking that the view number in the message matches the current view number.
	\item[-]The message sequence number is not out of bounds with the current sequence number interval~\cites{SLIDES:PBFT}[p.~4]{PAPER:OGPBFT}.
	\item[-]Make sure the replica has not already received another Pre-Prepare message with the same sequence number, but with a different request.
\end{itemize}
Once the validation process is finished the replica officially starts the prepare phase by creating a prepare message and multicasting it over the network. The prepare phase ends for a replica once it's stored up to $2f+1$ validated pre-prepare/prepare messages from different replicas. After this condition is met, the replica enters the state known as \emph{prepared}. In this state it will log the message data thus far in what is called a \emph{prepare certificate}. A prepare certificate is essentially a record that shows that the prepared phase is finished and is properly executed for that given request. The proof provided in a prepare certificate is a list of the valid prepare messages, basically confirming that quorum has been reached for the certificate when the number of messages stored in the list is higher than the desired limit of $2f + 1$~\cites[p.~408]{PAPER:PBFTRecovery}[p.~457]{BOOK:MVstandver3}.
The last phase is the commit phase which functions very similar to the prepare phase. Each replica that is finished with the prepare phase will start the commit phase by multicasting commit messages to the other replicas in the system~\cite[p.~4]{PAPER:OGPBFT}. In this phase, the primary functions exactly the same as every other replica. The validation process is also the same as it was for prepare messages. The goal for the commit phase is also the same as in the prepare phase, which is for a replica to receive $2f+1$ commit messages, which includes the replica's own commit message~\cite[p.~5]{PAPER:OGPBFT}. Once a replica has received enough commit messages, then the protocol reaches the \emph{committed} phase for the replica. This essentially means that a commit certificate is created and is logged similar to a prepare certificate~\cites[p.~409]{PAPER:PBFTRecovery}[p.~457]{BOOK:MVstandver3}. When a replica has finished both a prepare certificate and a commit certificate, then consensus has been achieved and each replica will perform the operation requested by the client~\cites[p.~409]{PAPER:PBFTRecovery}[p.~5]{PAPER:OGPBFT}. After the operation is executed each replica sends back a reply message containing the appropriate identification values as well as the result of processing the given request. The last requests sent by the clients are also stored in memory, to account for the situation where the client does not receive the reply messages. In this case the client will re-transmit the same request to the system and the replicas will re-transmit their reply for that following request~\cite[p.~409]{PAPER:PBFTRecovery}. A client will accept the result if it gets $f+1$ replies back from the replicas.
 

The replicas can only handle a certain amount of requests before the system is required to save its state. As mentioned in the validation process, a replica can only process a protocol message that is within a given sequence number interval. This sequence interval length is always constant and will adjust based on the systems checkpoint period which will be discussed in the next section \autoref{sec:checkpoint}~\cites[p.~262]{BOOK:BuildDepDistSyst}[p.~4-5]{PAPER:OGPBFT}.
\fi

\section{Detailed Protocol Operations}
\label{sec:detailedProtocol}
The \ac{pbft} consensus protocol is divided into three phases. The Pre-Prepare, Prepare, and the Commit phase. If the \ac{pbft} protocol operations are properly executed, a consensus has been achieved for an operation once all three phases have transpired on $3f+1$ replicas~\cite[p.~257-259]{BOOK:BuildDepDistSyst}. The roles of the pre-prepare phase and prepare phase are to propose an ordering for requests delivered to the system. On the other hand, the combination of prepare phase and the commit phase establishes the execution order for the replicas in the system~\cite[p.~4]{PAPER:OGPBFT}. \autoref{fig:pbftnormalworkflow} shows an illustration of the \ac{pbft} workflow. The illustration shows the messages sent from the different replicas during the different protocol phases in \ac{pbft}.

The \ac{pbft} protocol starts once a client sends a request containing their desired operation to the primary~\cite[p.~4]{PAPER:OGPBFT}. Sometimes the client also multicasts their request to the other replicas in the system, which is the model that we followed in our implementation~\cites[p.~2]{PAPER:DPBFT}[p.~406]{PAPER:PBFTRecovery}[p.~258]{BOOK:BuildDepDistSyst}. Regardless of which of these models is used for the requested message, the primary is the replica responsible for starting the iteration of the \ac{pbft} algorithm to process the client’s request. The primary creates a Pre-Prepare message and assigns the request with a sequence number which is then multicasted to the other replicas in the network with the same view number as the primary. Once a replica receives the Pre-Prepare message, it validates the Pre-Prepare message. The validation process consists of the following~\cites{SLIDES:PBFT}[p.~4]{PAPER:OGPBFT}[p.~259]{BOOK:BuildDepDistSyst}
\begin{itemize}
	\item[-]Validating the Signature in the message.
	\item[-]Checking that the view number in the message matches the current view number.
	\item[-]The message sequence number is not out of bounds of the current sequence number interval~\cites{SLIDES:PBFT}[p.~4]{PAPER:OGPBFT}.
	\item[-]Make sure the replica has not already received another Pre-Prepare message with the same sequence number but with a different request.
\end{itemize}
Once the validation process is finished, the replica officially starts the prepare phase by creating a prepare message and multicasting it over the network. The prepare phase ends for a replica once it has stored up to $2f+1$ validated pre-prepare/prepare messages from different replicas. After this condition is met, the replica enters the state known as \emph{prepared}. In this state, the replica logs the message data thus far in what is called a \emph{prepare certificate}. A prepare certificate is essentially a record showing that the prepared phase is finished and properly executed for that given request. The proof provided in a prepare certificate is a list of the valid prepare messages, basically confirming that quorum has been reached for the certificate when the number of messages stored in the list is higher than the desired limit of $2f + 1$~\cites[p.~408]{PAPER:PBFTRecovery}[p.~457]{BOOK:MVstandver3}.
The last phase is the commit phase, which functions very similarly to the prepare phase. Each replica that is finished with the prepare phase starts the commit phase by multicasting commit messages to the other replicas in the system~\cite[p.~4]{PAPER:OGPBFT}. In this phase, the primary functions exactly the same as every other replica. The validation process is also the same as it was for prepare messages. The goal for the commit phase is also the same as in the prepare phase, which is for a replica to receive $2f+1$ commit messages, which includes the replica’s own commit message~\cite[p.~5]{PAPER:OGPBFT}. Once a replica has received enough commit messages, the protocol reaches the \emph{committed} phase for the replica. This essentially means that a commit certificate is created and logged similarly to a prepare certificate~\cites[p.~409]{PAPER:PBFTRecovery}[p.~457]{BOOK:MVstandver3}. When a replica has finished both a prepare certificate and a commit certificate, then consensus has been achieved, and each replica performs the operation requested by the client~\cites[p.~409]{PAPER:PBFTRecovery}[p.~5]{PAPER:OGPBFT}. After the operation is executed, each replica sends back a reply message containing the appropriate identification values and the result of processing the given request. The last requests sent by the clients are also stored in memory to account for the situation where the client does not receive the reply messages. In this case, the client will re-transmit the same request to the system, and the replicas will re-transmit their reply for that following request~\cite[p.~409]{PAPER:PBFTRecovery}. A client will accept the result if it gets $f+1$ replies back from the replicas.
 
The replicas can only handle a certain amount of requests before the system is required to save its state. As mentioned in the validation process, a replica can only process new protocol messages as long as the replica can exhaust a sequence number within a given sequence number interval. Once the replica no longer can exhaust any sequence number within the sequence number interval, the replica can no longer process incoming requests until the interval is updated. This sequence interval length is always constant and is adjusted based on the systems checkpoint period, which is discussed in the next section, \autoref{sec:checkpoint}~\cites[p.~262]{BOOK:BuildDepDistSyst}[p.~4-5]{PAPER:OGPBFT}.


\begin{figure}[!h]
	\centering
	\includegraphics[width=\linewidth]{figures/PBFTWorkflow}
	\caption{Practical Byzantine Fault Tolerance Normal Workflow}
	\label{fig:pbftnormalworkflow}
\end{figure}

\iffalse
\section{Checkpointing}
\label{sec:checkpoint}
\ac{pbft} also incorporates checkpointing, which is a mechanism used for garbage collecting the logs. Checkpointing is required so that the replica does not use up all of its memory for logging messages~\cite[p.~261]{BOOK:BuildDepDistSyst}. Therefore, the replicas must agree upon a point in which the system is stable for all the replicas. Afterwards the replicas can delete any records in the logs prior to the consented state~\cites[p.~5]{PAPER:OGPBFT}[p.~410]{PAPER:PBFTRecovery}.

Checkpoints are essentially the state records of the system after progressing a specific interval of requests. The checkpoint has information regarding the last sequence number that was performed for the system. This sequence number is used on the garbage collector to put an upper bound on the records that are to be removed. For instance, if the stable sequence number was set to 50, then the garbage collector would remove a set of logged data up to 50. The checkpoint also has a digest of the system for that stable sequence number. This digest is used to confirm that the replicas have the same system state for the given sequence number~\cites[p.~5]{PAPER:OGPBFT}[p.~410]{PAPER:PBFTRecovery}.

In order for replicas to be able to validate checkpoints, they each must multicast a checkpoint message over the network containing the information mentioned above with its own replica id. Like the rest of the \ac{pbft} protocol messages a checkpoint is considered to be valid for a replica if it has stored $2f+1$ checkpoint messages with different replica id`s with the same stable sequence number and system digest~\cites[p.~261-262]{BOOK:BuildDepDistSyst}[p.~5]{PAPER:OGPBFT}[p.~410]{PAPER:PBFTRecovery}. Once a checkpoint has been validated successfully it is referred to as a stable checkpoint~\cites[p.~3]{PAPER:DPBFT}[p.~261]{BOOK:BuildDepDistSyst}. The replical usually stores checkpoint messages for different sequence numbers in memory and has only a single record for a stable checkpoint. Once a new stable checkpoint is determined, any checkpoint records with lower sequence numbers are removed from memory and if there exists a previous stable checkpoint in memory with a lower sequence number, then it is replaced by  the new one~\cite[p.~261-262]{BOOK:BuildDepDistSyst}.

In \ac{pbft} checkpointing is usually performed periodically after a constant number of requests have been processed. This interval length is constant and is referred to as a checkpoint period~\cites[p.~261]{BOOK:BuildDepDistSyst}[p.~410]{PAPER:PBFTRecovery}. As mentioned earlier in~\autoref{sec:detailedProtocol} \ac{pbft} normally only processes a sequence number if it is in the set of currently available sequence numbers. The length of the sequence number interval is usually designed to follow the format $[checkpointinterval+1-2*checkpointinterval]$. Which means the system attempts to calculate two checkpoints during a single sequence number interval. Once a stable checkpoint is obtained, the system extends the sequence number interval where the new interval starts at the last stable sequence for the current stable checkpoint~\cites[p.~5]{PAPER:OGPBFT}[p.~410]{PAPER:PBFTRecovery}. Unless a replica has exceeded the upper bound of the sequence number interval, the replical usually performs the checkpoint functionality concurrently to the protocol workflow.

\section{View-change}
\label{sec:view-change}
In the scenario in which the primary is the faulty replica, a view-change eventually occurs. The purpose of the view-change is to reassign the responsibility for a primary away from the current primary replica that is deemed to be faulty. Which is then given to another replica that is not faulty~\cites[p.262]{BOOK:BuildDepDistSyst}. As mentioned in \autoref{sec:systemModel}, the replica that is chosen as the next primary is based on the replica id and the next view number. Therefore, the view-change updates the view number for the system in order to change the primary replica of the system. There are some operations that need to be performed for a view-change to be deemed successful. The first operation is to update the view number to set another replica as the primary~\cites[p.~6]{PAPER:OGPBFT}[p.~411]{PAPER:PBFTRecovery}{WEB:SawtoothPBFT}. This step includes multicasting view-change messages between replicas to start the new view session. The other more demanding operation is that the primary needs to make sure that the system is stable and that replicas start the new view with the exact same system state. Therefore, all requests that have been performed after the last stable sequence number, need to be reprocessed between the replicas. This is done so that the system can guarantee that the replicas are not missing any of the previous operations performed to the system.~\cites[p.~458]{BOOK:MVstandver3}[p.~263-265]{BOOK:BuildDepDistSyst}.

There are several ways for a replica to deem its primary to be faulty, the most common way is to have a timeout functionality for the protocol execution. It is most common to start a timeout once a replica has received a request from the client. If the replica does not receive any pre-prepare messages for that request before the timeout expires, than the replica goes into view-change mode and no longer participates in any of the protocol operations~\cites{SLIDES:PBFT}[p.~5-6]{PAPER:OGPBFT}[p.~263]{BOOK:BuildDepDistSyst}.  

The view-change process starts by having the replica increment its view number. Then the replica creates, signesand multicasts a view-change message over the network. The replica  then waits for $2f+1$ view-change messages~\cites{SLIDES:PBFT}[p.~6]{PAPER:OGPBFT}[p.~411]{PAPER:PBFTRecovery}{WEB:SawtoothPBFT}. A timeout is also used here, if the replica does not receive enough view-change messages in time, then the process repeats with the next incremental view number. In some cases a replica can also be designed to go into view-change mode if a replica has already received two view-change messages from other replicas, as it now only requires its own view-change message for the system to agree that a view-change is necessary\cite{BOOK:BuildDepDistSyst}. Once the appropriate number of view-change messages are received, then the new primary is responsible for creating, signing and multicasting a new-view message to the other replicas~\cite[p.~264]{BOOK:BuildDepDistSyst}. Before the new-view message can be multicast to the other replicas, a new primary must go through its log and create new pre-prepare messages for all sequence numbers that have occurred after the last stable sequence number. If the new primary lacks a record in the log for any of the sequence numbers, the new pre-prepare message has its request digest be set to be \code{null}. This information is included in the new-view message, which is then sent to the other backup replicas. The backup replicas then validates and re-process each of the sequence numbers that have a valid pre-prepare message. This essentially means that the other replicas have to multicast a new prepare message and then participate in a commit phase together with the new primary for each of the pre-prepare messages in the new-view message~\cites[p.~6]{PAPER:OGPBFT}[p.~458]{BOOK:MVstandver3}[p.~265]{BOOK:BuildDepDistSyst}. A timeout is once again being used to handle the situation where the reprocessing takes too long. This process can also fail if the pre-prepares in the new-view message fails the validation process. If either the timeout occurs or the validation fails, then it is back to the start for the view-change process. Once all pre-prepares have been reprocessed, the view-change procedure is over, and the replica returns to normal protocol operations with the new chosen primary. Keep in mind that any new requests received during the view-change process are ignored by the system~\cite[p.~263]{BOOK:BuildDepDistSyst}. 

\autoref{fig:pbftviewchange} shows an example of a view-change process. The figure shows the timeline for each of the processes needed for the view-change to be successfully completed. Starting with the timeout occurring on the backup replicas when the primary is no longer working properly. Then the replicas each multicast a view-change message to the other replicas in the system, including the faulty primary. After the new primary has received a sufficient number of view-change messages, it creates pre-prepares messages that need to be reprocessed in the network. Afterwards the new primary multicast new-view messages to the other replicas to start the re-processing phase. Finally the system multicasts both prepare and commit messages to validate pre-prepare messages. Once all that is done the system moves on to normal workflow again with the first backup replica now serving as the primary for the system.
\fi

\section{Checkpointing}
\label{sec:checkpoint}
\ac{pbft} also incorporates checkpointing, which is a mechanism used for garbage collecting the logs. Checkpointing is required so that the replica does not use up all of its memory for logging messages~\cite[p.~261]{BOOK:BuildDepDistSyst}. Therefore, the replicas must agree upon a point where the system is stable for all the replicas. Afterwards, the replicas can delete any records in the logs prior to the consented state~\cites[p.~5]{PAPER:OGPBFT}[p.~410]{PAPER:PBFTRecovery}.

Checkpoints are essentially the state records of the system after progressing a specific interval of requests. The checkpoint has information regarding the last sequence number that was performed for the system. This sequence number is used on the garbage collector to put an upper bound on the records that are to be removed. For instance, if the stable sequence number was set to 50, then the garbage collector would remove a set of logged data up to 50. The checkpoint also has a digest of the system for that stable sequence number. This digest is used to confirm that the replicas have the same system state for the given sequence number~\cites[p.~5]{PAPER:OGPBFT}[p.~410]{PAPER:PBFTRecovery}.

For replicas to validate checkpoints, they each must multicast a checkpoint message over the network containing the information mentioned above together with the replicas id. Like the rest of the \ac{pbft} protocol messages, a checkpoint is considered to be valid for a replica if it has stored $2f+1$ checkpoint messages with different replica ids with the same stable sequence number and system digest~\cites[p.~261-262]{BOOK:BuildDepDistSyst}[p.~5]{PAPER:OGPBFT}[p.~410]{PAPER:PBFTRecovery}. Once a checkpoint has been validated successfully, it is referred to as a stable checkpoint~\cites[p.~3]{PAPER:DPBFT}[p.~261]{BOOK:BuildDepDistSyst}. The replica usually stores checkpoint messages for different sequence numbers in memory and has only a single record for a stable checkpoint. Once a new stable checkpoint is determined, any checkpoint records with lower sequence numbers are removed from memory. If there exists a previous stable checkpoint in memory with a lower sequence number, then it is replaced by the new one~\cite[p.~261-262]{BOOK:BuildDepDistSyst}.

In \ac{pbft}, checkpointing is usually performed periodically after a constant number of requests have been processed. This interval length is constant and is referred to as a checkpoint period~\cites[p.~261]{BOOK:BuildDepDistSyst}[p.~410]{PAPER:PBFTRecovery}. As mentioned earlier, in~\autoref{sec:detailedProtocol}, \ac{pbft} normally only processes a sequence number in the set of currently available sequence numbers. The length of the sequence number interval is designed to follow the format $[checkpointinterval+1-2*checkpointinterval]$. This means the system attempts to calculate two checkpoints during a single sequence number interval. Once a stable checkpoint is obtained, the system extends the sequence number interval where the new interval starts at the last stable sequence for the current stable checkpoint~\cites[p.~5]{PAPER:OGPBFT}[p.~410]{PAPER:PBFTRecovery}. Unless a replica has exceeded the upper bound of the sequence number interval, the replica usually performs the checkpoint functionality concurrently with the protocol workflow.

\section{View-change}
\label{sec:view-change}
In the scenario in which the primary is the faulty replica, a view-change eventually occurs. The purpose of the view-change is to reassign the responsibility for a primary away from the current primary replica that is deemed faulty, which is then given to another replica that is not faulty~\cites[p.262]{BOOK:BuildDepDistSyst}. As mentioned in \autoref{sec:systemModel}, the replica that is chosen as the next primary is based on the replica id and the next view number. Therefore, the view-change updates the view number for the system to change the system’s primary replica. Some operations have to be performed for a view-change process to be deemed successful. The first operation is to update the view number to set another replica as the primary~\cites[p.~6]{PAPER:OGPBFT}[p.~411]{PAPER:PBFTRecovery}{WEB:SawtoothPBFT}. This step includes multicasting view-change messages between replicas to start the new view session. The other more demanding operation is that the primary needs to make sure that the system is stable and that replicas start the new view with the exact same system state. Therefore, all requests performed after the last stable sequence number need to be reprocessed between the replicas. This is done so that the system can guarantee that the replicas are not missing any of the previous operations performed to the system.~\cites[p.~458]{BOOK:MVstandver3}[p.~263-265]{BOOK:BuildDepDistSyst}.

There are several ways for a replica to deem its primary to be faulty. The most common way is to have a timeout functionality for the protocol execution. It is most common to start a timeout once a replica has received a request from the client. Suppose the replica does not accept any pre-prepare messages for that request before the timeout expires. In that case, the replica goes into view-change mode and no longer participates in any of the protocol operations~\cites{SLIDES:PBFT}[p.~5-6]{PAPER:OGPBFT}[p.~263]{BOOK:BuildDepDistSyst}.  

The view-change process starts by having the replica increment its view number. Then the replica creates, signs and multicasts a view-change message over the network. The replica  then waits for $2f+1$ view-change messages~\cites{SLIDES:PBFT}[p.~6]{PAPER:OGPBFT}[p.~411]{PAPER:PBFTRecovery}{WEB:SawtoothPBFT}. A timeout is also used here. If the replica does not receive enough view-change messages in time, the process repeats with the next incremental view number. 
In some cases, a replica can also be designed to go into view-change mode if a replica has already received two view-change messages from other replicas, as it now only requires its own view-change message for the system to agree that a view-change is necessary~\cite{BOOK:BuildDepDistSyst}. Once the appropriate number of view-change messages are received, the new primary is responsible for creating, signing, and multicasting a new-view message to the other replicas~\cite[p.~264]{BOOK:BuildDepDistSyst}. Before the new-view message can be multicast to the other replicas, a new primary must go through its log and each of the protocol certificates received from the view-change messages. This process is done so that the new primary can create new pre-prepare messages for all sequence numbers that have occurred after the last stable sequence number. If the new primary lacks information for any of the sequence numbers, the new pre-prepare message has its request digest set to the value null. This information is included in the new-view message, which is then sent to the other backup replicas. The backup replicas then validate and reprocess each of the sequence numbers that have a valid pre-prepare message. This essentially means that the other replicas have to multicast a new prepare message and then participate in a commit phase together with the new primary for each of the pre-prepare messages in the new-view message~\cites[p.~6]{PAPER:OGPBFT}[p.~458]{BOOK:MVstandver3}[p.~265]{BOOK:BuildDepDistSyst}. A timeout is once again being used to handle the situation where the reprocessing takes too long. This process can also fail if the pre-prepares in the new-view message fails the validation process. If either the timeout occurs or the validation fails, it is back to the start of the view-change process. Once all pre-prepares have been reprocessed, the view-change procedure is over, and the replica returns to normal protocol operations with the new chosen primary. Keep in mind that any new requests received during the view-change process are ignored by the system~\cite[p.~263]{BOOK:BuildDepDistSyst}. 

\autoref{fig:pbftviewchange} shows an example of a view-change process. The figure shows the timeline for each of the processes needed for the view-change to be successfully completed. Starting with the timeout occurring on the backup replicas when the primary is no longer working correctly. Then the replicas each multicast a view-change message to the other replicas in the system, including the faulty primary. After the new primary has received a sufficient number of view-change messages, it creates pre-prepares messages that need to be reprocessed in the network. Afterwards, the new primary multicast new-view messages to the other replicas to start the reprocessing phase. Finally, the system multicasts both prepare and commit messages to validate pre-prepare messages. The system then moves on to normal workflow, with the first backup replica now serving as the primary for the system.


\begin{figure}[!h]
	\centering
	\includegraphics[width=1.1\textwidth]{figures/PBFTViewChange}
	\caption{Practical Byzantine Fault Tolerance View-Change}
	\label{fig:pbftviewchange}
\end{figure}

%In the scenario in which the primary is the fault replica, a view-change occurs. The purpose of the view-change is to reassign the primary responsibility away from a replica that is deemed to be faulty. As mentioned in \autoref{sec:systemModel}, the replica which is chosen as the primary is based on the replica id and the current view number. Therefore, the view-change updates the view number to update the primary replica. There are 3 main operations that need to be performed for a view-change to be deemed successful. The first is to update the view number to set another replica as the primary. The second is to validate this new leader on whether it is a suitable replacement. Finally, all operations that are performed after the last stable checkpoint needs to be reprocessed between the replicas. There are several ways for a replica to deem its primary to be faulty, the most common way is to have a timeout functionality for the protocol execution. The one that is most common is to start a timeout once a replica has received a request from the client. If the replica does not receive any pre-prepare messages for that request before the timeout expires, the replica will go into the view-change mode and will no longer participate in any of the protocol operations.  

%The view-change process starts by having the replica increment its view number. Then the replica will create, sign and multicast a view-change message. The replica will then wait for $2f+1$ view-change messages. A timeout is also used here, if the replica does not receive enough view-change messages in time, then the process repeats with the next incremented view number. In some cases a replica can also be designed to go into view-change mode if a replica has already received two view-change messages from other replicas, as it now only requires its own view-change message for the system to agree upon the view-change. Once the appropriate number of view-change messages are received, then the new primary will be responsible for creating, signing and multicasting a new-view message to the other replicas. Before the new-view message can be multicast to the other replicas, a new primary must go through its log and create new pre-prepares for all sequence number that has occurred after the last stable sequence number. If the new primary lacks a record in the log for any of the sequence numbers, the new pre-prepare message will have its request digest be set to null. This information is included in the new-view message, which is then used by the other replicas to validate and re-process each of the sequence numbers that has a valid pre-prepare. This essentially means that the other replicas have to multicast a new prepare message and then participate in a commit phase together with the new primary for of the pre-prepare messages in the new-view message. A timeout is once again being used in the case where the reprocessing fails or takes too long. This process can also fail if the pre-prepares in the new-view message fails the validation process. If this happens, then its back to start and in the view-change process. Once all pre-prepares have been reprocessed, the view-change procedure is over, and the replica will return to normal protocol operations using the chosen new primary. Any requests received during the view-change process will be ignored by the system. 

%TODO rewrite after implementing the basics of view change for app
%In the scenario where the primary is the faulty replica a view change occurs. A view change increments the view number for all participating replicas, which in turn chooses as a new primary replica for the network following the formula mention in \autoref{section:systemModel}. 
%Normally each replica starts a timeout operation whenever it receives a request from a client~\cite[p.~415]{PAPER:PBFTRecovery}. 
%If the replica does not receive a pre-prepare message before the timeout expires, then the replica deems the primary to be faulty and desires a view change. The replica will multicast a view-change message to the other replicas in the view, including the potentially faulty primary, and will ignore any %new incoming messages with the exception of view-change, new view and checkpoint messages~\cites[p.~5-6]{PAPER:OGPBFT}[p.~263]{BOOK:BuildDepDistSyst}. 
%A replica receiving a view change will reply with a view-change-ack.  The new primary chosen will collect view-change messages and view-change-ack messages to create a view-change certificate. Once the certificate is valid and has reached $2f+1$ unique messages a new-view message is multicasted to %every replica in the network, officially starting the new view~\cites[p.~412-414]{PAPER:PBFTRecovery}[p.~263-p.265]{BOOK:BuildDepDistSyst}[p.~458]{BOOK:MVstandver3}.
%%PBFT also incorporates checkpointing, which is a mechanism used for garbage collecting the saved logs. Without checkpoints, the replica would eventually use up all of its memory for logging protocol messages~\cite{BOOK:BuildDepDistSyst}. 
%%Checkpoints are essentially a proof where the replicas have agreed on a stable state for the system after a specified interval of requests has been processed~\cite[p.~5]{PAPER:OGPBFT}. For instance, if the checkpoint interval was set to 50, then the system would only allow sequence numbers 0 to 50 to %be performed in the first interval. Once the checkpoint interval is reached the log data is hashed and saved as a checkpoint message. The message is then multicasted over the network. Consensus is reached when a replica receives $2f+1$ checkpoint messages with different identifiers, but they all have same checkpoint hash value. The checkpoint is then saved on the replica and all the protocol messages with lower or equal sequence number to the checkpoint interval will be deleted from the log. As an example, if the checkpoint interval was 50, then certificates saved up for requests 0 to 50 are deleted. Then the interval is updated to be $[checkpointinterval+1-2*checkpointinterval]$, so for the last example the new interval would be between 51 to 100~\cites[p.~3]{PAPER:DPBFT}[p.~5]{PAPER:OGPBFT}[p.~409-410]{PAPER:PBFTRecovery}[p.~p.261-262]{BOOK:BuildDepDistSyst}.

\iffalse
\cite[p.~415]{PAPER:PBFTRecovery}
\cite[p.~5-6]{PAPER:OGPBFT}
\cite[p.~263]{BOOK:BuildDepDistSyst}
\cites[p.~3]{PAPER:DPBFT}
\cite[p.~458]{BOOK:MVstandver3}
\fi

%\cite{WEB:PBFTGeeks}
%\cite{WEB:ImpPBFTBlock} %not used
%\cite{WEB:UnderpBFT} %not used
%\cite{WEB:PBFTConSeries} %not used
%\cite{SLIDES:PBFT}
%\cite{VIDEO:YPBFT}
%\cite{BOOK:MVstandver3}
%\cite{BOOK:BuildDepDistSyst}
%\cite{PAPER:OGPBFT}
%\cite{PAPER:DPBFT}
%\cite{PAPER:PBFTRecovery}
%\lstset{style=sharpc}
%\begin{lstlisting}
%Your c# code here
%class Request
%{
%	private m string;
%} 
%\end{lstlisting}	
	
	\chapter{Related Work}
%1-2 papesa
%If you cannot demonstrate that you know, and understand, what others have done, you only demonstrate that you're clueless. 
%For an undergraduate thesis this, together with a thorough understanding of the problem, should be the result of the first session's work. 
%It is an unfortunate fact that many students do very little work during the first session of their thesis. 
%It usually shows here (and is usually reflected in their mark). 
%Don't think you can fool your thesis supervisor/assessor. And don't even dream about fooling the referee of a paper. 
%If you haven't done your homework here, it's probably not worth going any further.
%In this part you demonstrate that you are aware of what's going on in the field, and how it relates to your particular problem.
%If there is lots of related work, discuss related work early to differentiate your own work
%-Compare/contrast with your own work - don't just enumerate
%-Don't be dismissive
%-Refer to references by author or project names
\label{chapter:RW}
\iffalse
The University of Stavanger has previously supported the development of Cleipnir.
Therefore, there exists previous papers and thesis on Cleipnir usage for implementing consensus algorithms.
Two contributions in particular have been used as building blocks for this project. We now discuss these two works in detail and explain how they contributed to our work. 
 
\section{Cleipnir - Framework Support for Fault-tolerant Distributed Systems}
This paper is the original paper describing the Cleipnir framework and is written by its creator Thomas Stidsborg Sylvest with the help of two professors at the University of Stavanger, Leander Jehl and Hein Meling. The paper describes, in detail, the internal functionality and tools available in the Cleipnir framework. This includes their use cases, detailed explanations for how they work with practical demonstrations. The demonstrations are presented using an existing implementation of the Paxos consensus algorithm. The paper also presents a Raft implementation using the Cleipnir framework. This includes the overall architecture of the implementation with detailed examples of how Cleipnir is used to simplify tasks used in the Raft algorithm. Finally experiments are performed to evaluate the performance of the Raft implementation. The results of the experiments are compared directly to the Paxos implementation, both in terms of latency and code structure. Our thesis is a direct continuation of this paper with practically the same goals. The main difference between our thesis and this paper being the chosen consensus algorithm that is to be implemented using Cleipnir. Our contribution is to provide additional experiences on how well Cleipnir can be utilized in implementing complex consensus algorithms. This also implies discovering potential difficult problems that the current Cleipnir framework is not able to handle. Specifically, whether or not Cleipnir can handle all of the complex problems that can take place within the PBFT algorithm while still having a simple to read code structure~\cite{PAPER:PaxosCleipnir}.
 
\section{Implementing a Distributed Key-Value Store Using Corums}
In 2010, Eivind Bakkevig wrote a master thesis about Corums. In this thesis he used a Net framework called Corums to implement a dictionary based distributed system. This Corums based implementation had an implementation of the Paxos consensus algorithm to make decisions for the distributed system.
 
The Corums framework is the predecessor to the Cleipnir framework. It follows the same programming models as Cleipnir does. These models would be the ones described in \autoref{chapter:Cleipnir}. Namely Built-in Persistency, Reactive programming and Single-Threaded scheduler.
The main difference between Corums and Cleipnir is that Corums focus more on simplifying an abstraction for developers to handle communication using incoming/outgoing communication buses. Cleipnir has more focus on giving the developer the tools necessary to develop consensus algorithms which follow the persistent program paradigm in an easy to use and customizable manner. As an example, a major difference between Cleipnir and Corums frameworks lies in Corums support in reliable message delivery between distributed systems. Corums has support for bus abstraction that can simplify the handling of incoming/outgoing messages between the nodes in the system. Cleipnir does unfortunately not support this functionality. Instead Cleipnir has more focus on evolving the persistence functionality previously provided by Corums~\cites[p.~6-7]{PAPER:PaxosCleipnir}{DOC:Cleipnir}.
 
Corums is very similar to Gorums~\cites[p.~2]{WEB:Gorums}[p.~22]{PAPER:EivindPaper} which is intended based on how close the names are, the main difference being the supporting language.
Bakkevig succeeded in creating a distributed dictionary storage using the Corums framework. Additionally he built the client side for the implementation using ASP.NET Core Web \ac{api}~\cite{WEB:ASPNetCoreAPI}.
 
According to Bakkevig, he had no prior experience with the C\# before writing his thesis. Bakkevig did however have previous experience with the Paxos consensus algorithm. This made most of his work during the thesis about learning C\# and the Corums framework rather than having to extensively research Paxos. As for our thesis the exact opposite is true. We have some background knowledge regarding the C\# language but had little to no background knowledge of the PBFT algorithm. Therefore, a lot of work for this thesis revolved around learning and making our own PBFT algorithm based on its description. Although we had experience with the C\# language, we had no previous experience with the Cleipnir framework. Therefore, similarly to Bakkevig, our thesis also required us to study the Cleipnir framework. ~\cite[p.~8]{PAPER:EivindPaper}.
\fi

The University of Stavanger has previously supported the development of Cleipnir.
Therefore, there exist previous papers and thesis on Cleipnir usage for implementing consensus algorithms.
Two contributions, in particular, have been used as building blocks for this project. We now discuss these two works in detail and explain how they contributed to our work. 
 
\section{Cleipnir - Framework Support for Fault-tolerant Distributed Systems}
This paper is the original paper describing the Cleipnir framework. It was written by its creator Thomas Stidsborg Sylvest with the help of two professors at the University of Stavanger, Leander Jehl and Hein Meling. The paper describes, in detail, the internal functionality and tools available in the Cleipnir framework. The paper describes Cleipnir`s use cases and why Cleipnir priorities these functionalities. The paper has detailed explanations for how the tools work together with practical demonstrations. The demonstrations are presented using an existing implementation of the Paxos consensus algorithm. The paper also presents a Raft implementation using the Cleipnir framework. This includes the overall architecture of the implementation and detailed examples of how Cleipnir is used to simplify tasks performed in the Raft algorithm. Finally, experiments are performed to evaluate the performance of the Raft implementation. The results of the experiments are compared directly to an earlier Paxos implementation. The evaluation performed focuses both on latency and code structure. Our thesis is a direct continuation of this paper with relatively similar goals. The largest difference between our thesis and this paper is the chosen consensus algorithm to be implemented using Cleipnir.
Additionally, we do not evaluate our \ac{pbft} implementation in terms of latency. Our contribution is to provide additional experiences on how well Cleipnir can be utilized in implementing complex consensus algorithms. This also implies discovering potentially difficult problems that the current Cleipnir framework cannot handle, specifically, whether or not Cleipnir can handle all of the complex issues within the PBFT algorithm while still having a simple-to-read code structure~\cite{PAPER:PaxosCleipnir}.
 
\section{Implementing a Distributed Key-Value Store Using Corums}
In 2010, Eivind Bakkevig wrote a master thesis about Corums. In his thesis, Bakkevig used a .NET framework called Corums to implement a dictionary-based distributed system. This Corums based implementation implemented the Paxos consensus algorithm to make decisions for the dictionary-based distributed system.
 
The Corums framework is the predecessor to the Cleipnir framework. It follows the same programming models as Cleipnir does. These models would be the ones described in \autoref{chapter:Cleipnir}; built-in persistency, reactive programming, and a single-threaded scheduler.
The main difference between Corums and Cleipnir is that Corums focus more on simplifying abstraction for developers to handle communication using incoming/outgoing communication buses. Cleipnir instead focuses more on giving the developer the tools necessary to develop consensus algorithms that follow the persistent program paradigm in an easy-to-use and customizable manner. As an example, a major difference between Cleipnir and Corums frameworks lies in Corums support in reliable message delivery between distributed systems. Corums has support for bus abstraction that can simplify the process of handling incoming/outgoing messages between the nodes in the system. Cleipnir does unfortunately not support this functionality. Instead, Cleipnir prioritized evolving the persistence functionality previously provided by Corums~\cites[p.~6-7]{PAPER:PaxosCleipnir}{DOC:Cleipnir}.
 
Corums is very similar to Gorums~\cites{WEB:Gorums}[p.~22]{PAPER:EivindPaper}, which is intended based on how close the names are, the main difference being the supporting language.
Bakkevig succeeded in creating a distributed dictionary storage using the Corums framework. Additionally, Bakkevig built the client-side for the implementation using ASP.NET Core Web \ac{api}~\cite{WEB:ASPNetCoreAPI}.
 
According to Bakkevig, he had no prior experience with the C\# language before writing his thesis. Bakkevig did, however, have previous experience with the Paxos consensus algorithm. This made most of his work during the thesis about learning C\# and the Corums framework rather than extensively researching Paxos. As for our thesis, the exact opposite is true. We have some background knowledge regarding the C\# language but had little to no background knowledge of the PBFT algorithm. Therefore, much work for this thesis revolved around learning and making our own PBFT algorithm based on its description. Although we had experience with the C\# language, we had no previous experience with the Cleipnir framework. Therefore, similarly to Bakkevig, our thesis also required us to study the Cleipnir framework. ~\cite[p.~8]{PAPER:EivindPaper}.
	
	
	\chapter{Design}
\label{chapter:Design}
%Overarching design in a top-down view: 
%-server/client design overall architecture, system model
%-describe file system, 
%-how protocol is peformed by the server
%-describe current functionality of the implementation. Handles protocol run, view-changes and checkpointing

%old version:-overall usage over different parameters, overaching stuff thats not the actual implementation of the algorithm, but the network layer to server to algorithm, how to run the algorithm/checkpointing/view changes.
%first draft!!!
This chapter present an overall summary of the PBFT application implemented. This includes a brief summary of the system model used for performing the PBFT algorithm. The structure of the application including a brief introduction to its file structure as well as describing the general design for performing the consensus algorithm. We will finally describe the current functionalities that are present in the current application. 

\section{System Model}
\begin{figure}
	\centering
	\includegraphics[width=\linewidth]{figures/meshnetwork}
	\caption{Overall architecture of the PBFT implementation networking}
	\label{fig:meshnetwork}
\end{figure}
The figure \autoref{fig:meshnetwork} shows the system model used for PBFT implementation. Generally our system model follows the same structure as the system model introduced in the PBFT chapter\autoref{sec:systemModel}. The system consist of four server implementations called \emph{replicas}, where the replica with the lowest identifier value is chosen as the primary. These four replicas are communicating over a mesh network using socket connections. This means each replica shares an unique network socket with each other replica in the PBFT network. In order to avoid creating multiple socket connections between replicas, the replica with the highest identifier is the one tasked with being the initiator when it comes to establishing a socket connection between other replicas. Meaning for instance that the primary replica, will not need to actively establish any connections to its fellow replicas. The primary will instead establish all of its socket connections by listening for any connections attempts on its local network address. The opposite scenario occurs for the replica with the highest identifier value, although the replica still listens for connections on its local network address, it is also responsible for establishing the connections with all the other replicas in the network with lower identifier values.

Even when the replicas have established connections, the replicas can not fully communicate with each other until they have exchanged public keys. This is required in order to verify messages sent by each replica using a digital signature. Public keys are exchanged in \emph{session messages}, which are messages that are automatically sent between parties once a socket connection has been established. If the public keys are for some reason not exchanged, than the replica will discard any message received from that host and thereafter terminate the connection. This also applies for clients. This current model is unfortunately not very secure, due to public keys being ephemeral and are intended to be replaced in the case of a crash occurring. Currently in this implementation, the private and public key pair for a replica are randomly created at the system start-up. Considering there is currently no way for the replicas to authenticate another replica after reboot, it will replace the key value pair currently in the register if a new session message with the same identifier value is received. This in turn means the system is susceptible to impersonation and spoofing attacks. Since the main goal of this thesis was more focused on the implementation of the PBFT workflow, this cryptographic system was deemed sufficient for simulating a network using digital signatures. However, it is important to be aware of this flaw in the system in order to avoid this issue in the future. %add how to fix this in future work.

The system performs the PBFT protocol by exchanging protocol messages over the mesh network until atleast three of the servers have finished all three protocol phases. In this implementation protocol messages are referred to as \emph{phase messages}. The PBFT protocol is trigger when the server receives a request from one of the connected client nodes. The primary is responsible for officially starting an instance of the PBFT protocol by multicasting a phase message of type pre-prepare. There are two important goals for the pre-prepare phase. The first is to make sure that the replicas have an agreement upon the ordering of the request. In other words, the replicas will perform the requests in the same order as the primary, which in turn means the request should have the same sequence numbers throughout the network. The second important goal is to determine whether or not the primary is fit to be leader. As mention in section \autoref{sec:view-change}, a view-change occurs when a leader no longer is eligible. In our application the view-changes are triggered by timeout which are set once a replica receives a client's request. If the primary takes too long in the pre-prepare phase, than the timeout will exceed and the other replicas will perform a view-change in order to change the primary replica. Although it would be useful to have timeouts in the commit phase in the instance where majority of servers are unable to properly finish the commit phase, it is currently not supported in our implementation due to how timeouts are handled inside the protocol workflow(should probably be moved elsewhere). The rest of the replicas will be the responsible party during the prepare phase by sending phase messages of type prepare, while every replica will participate in the commit phase using commit type phase messages. The last step of current PBFT implementation is to create a reply message and send it back to client responsible for the request. The details in regards to how the PBFT workflow is currently being handled will be discussed in the next chapter \autoref{chapter:Imp}.

\section{File structure}
In figure we \autoref{fig:filestruct} can see a short summary of the file system used in a PBFT replica system. The summary shows each of the root folders as well as the most important files. To start of the folders \emph{Messages} and \emph{Certificates} contains all the message types and certificate types available for this implementation. This includes files containing the interfaces. The \emph{Helper} folder contains all the static functions used in the PBFT implementation that are not linked to any specific object instance. This includes functionality for serializing and deserializing messages objects with JSON, cryptographic functions related to creating and validating digital signatures and files containing all the enum types used for this thesis. An enum is ... ~\cite{WEB:Enum}.

\newpage
\begin{wrapfigure}{r}{0.45\linewidth}
\centering
%\vspace{15pt}
%\rule{0.9\linewidth}{0.75\linewidth}
% ,scale=0.8, every node/.style={scale=0.8}
\tikzstyle{every node}=[draw=black,thick,anchor=west]
    \begin{tikzpicture}[%
      grow via three points={one child at (0.5,-0.7) and
      two children at (0.5,-0.7) and (0.5,-1.4)},
      edge from parent path={(\tikzparentnode.south) |- (\tikzchildnode.west)}]
      \node {PBFT}
        child { node {App.cs}}
        child { node {Certificates}
        	child {node {...}}
        	}	
        child [missing] {}
        child { node {Messages}
        	child {node {...}}
        	}
        child [missing] {}
        child { node {Helper}
        	child {node {...}}
        	}
        child [missing] {}
        child { node {JSONFiles}
        	child {node {serverInfo.json}}
        	child {node {testServerInfo.json}}
        	}
        child [missing] {}
        child [missing] {}
        child { node {Storage}
        	child {node {...}}
        }
        child [missing] {}
        child { node {Replica}
          child { node {Server.cs}}
          child { node {Network}
          	child {node {...}}
        	}
        }
        child [missing] {
          child { node {Protocol}
          child { node {Workflow.cs}}
          child { node {...}}
        	}
        };
\end{tikzpicture}
    \caption{Summary of the file architecture for the PBFT implementation}
    \label{fig:filestruct}
    \vspace{40pt}
\end{wrapfigure}

\section{Server and protocol interaction}

\begin{figure}
	\includegraphics[width=\linewidth]{figures/CleipnirStructurever1}
	\caption{Application divided into persistent parts and ephermeral parts and how they interact}
	\label{fig:PersistencyEphemeral}
\end{figure}	
	
	\chapter{Implementation}

%Implementation 
%describe code, but not to detailed
%motivation for using this design
%good parts/bad parts
\label{chapter:Imp}
%Implementation talks about the actual algorithm implementation that is run in protocol execution, includes process of checkpoints and view-changes. Should have figures to simplify explanation. Go over briefly the different phases, show some pseudocode. keeps to take note of, maybe include a model to show the persistency layers present. The importance of using the Cleipnir scheduler and CTask, where you have used them etc.

%outline aka what needs to be talked about, note not in any particular order
%detailed explanation of how the general workflow for the PBFT algorithm is performed, with code snippets.
%the simplicity of some of the necessary tools needed in PBFT is handled. Object oriented programming for Messages, Certificates. Static function for workflow, including Listeners, and handlers
%detailed explanation for how the checkpointing are handled
%detailed explanation for how the view-changes are handled
%Describe how you though of persistency during implementation(not much since its not general focus, and ofcourse doesn't work fully)
%small description on how clients are working, how they use the same reactive operators to count number of replies received.
%Add comment on how well they work, don't work do to design, this is an evaluation afterall.
In this chapter we introduce our \ac{pbft} implementation. We will introduce the implementation for the request handler, normal protocol workflow, view-changes, and finally checkpointing. We will also discuss how the Cleipnir framework is used to create the working \ac{pbft} implementation, as well as discuss some benefits and limitations within the current implementation design.

\section{Design Choices}
%Motivation is supposed to be short and summarize these points:
%why did we implement our application the way we did, what was the main focus
%-Usage/testing the usage of the tools
%	-Simplicity --> Protocol description as close to possible
%-Some thought went into persistency due to Cleipnir, CTask for protocol workflow, persisted objects gets assigned Sync/Desync funcs
%-async used for networking

With the goal of the thesis in mind, the \ac{pbft} protocol workflows were designed to be as close defined to the protocol description as possible. To accomplish this, we believed the best approach would be to design protocol-related workflow as orderly as possible, meaning we generally want to use synchronous programming workflow whenever it is possible. Several factors persuaded us to focus on keeping the majority of the protocol operations synchronous. The first reason was that it is generally easier for developers to keep track of the program’s progress, making it easier to debug. Modern async/await workflow can generally achieve a similar program workflow to that of synchronous workflow. However, unless you are worried about blocking the main thread, there are no added benefits in terms of complexity or efficiency in using asynchronous programming over synchronous programming. 
The second reason is in regards to using Cleipnir for our protocol workflows. We previously discussed in \autoref{sec:persvsephe} and \autoref{section:PersistentProgramming} that using normal asynchronous operations inside a function that uses Cleipnir is not a good idea. Because we wanted to take advantage of reactive programming to handle protocol-related messages in our protocol workflow, we needed to use Cleipnir reactive framework. In addition, since persistency is a core part of the Cleipnir framework, we decided it was also best to keep persistency in mind while designing our application. Therefore we decided to keep our protocol implementations inside \code{CTask} functions. Because of this, the only form of asynchronous operations that are performed inside any of the protocol workflows are restricted to other \code{CTask} asynchronous operations. The \code{await} operator still works the same as it does for traditional asynchronous operations. Allowing us to take advantage of the \code{await} to set waiting points for \code{CTask} operations as well as ongoing reactive streams; Thereby giving the protocol workflows an abstraction to that of synchronous workflows, even though, in reality, it is an asynchronous process. 
To enable Cleipnir to persist in our protocol workflow, we also had to make sure that objects we wanted to persist in the protocol were persistable. To accomplish this, we initialized both a serializer and deserializer functions for Cleipnir that follows the format specified in \autoref{section:PersistentProgramming} to use for each of our defined protocol objects.  
Finally, to take advantage of the Cleipnir persistency functionality, we have done our best to avoid creating circular dependencies. Circular dependencies would essentially cause the serialization process to fail, as it would lead to two references that depend on one another. We believe there are no circular dependencies in our current implementation because the Cleipnir serialization process does not crash during Cleipnir’s synchronization process. However, if there does exist a circular dependency in our application, the server’s relationship with the protocol workflows would be our primary suspect. This is because the server emits messages to the protocol workflow while the protocol workflow has an object reference to the server. We believe they do not have a circular dependency because the server interacts with the protocol workflow through the Cleipnir execution engine and does not directly reference the protocol workflow. However, if our assumption is wrong, then this design would have to be changed in the future.

To make it easier to understand the protocol workflow, we believed the best approach was to keep only the protocol processes described in the descriptions centred inside a single function or class whenever it was possible. We chose this design primarily because we wanted to make the code as readable as possible. It was deemed especially important when designing the standard \ac{pbft} workflow. This design was not entirely possible to replicate for checkpoint and view-change workflows. We also attempted to keep operations unrelated to the protocol outside the protocol workflow. Although in some cases, we cannot avoid this issue. In these cases, a simple function call with a good function name must suffice to avoid increasing the complexity of the overall workflow.
An example of this is using the server to send protocol messages to \ac{pbft}. It is a fundamental part of the \ac{pbft} consensus algorithm to interchange protocol messages. However, the operations performed in the sending operation itself do not affect the protocol workflow. Therefore, a simple call to the servers \code{Multicast} using the newly created protocol message as a parameter should be decisive enough for readers of the workflow. 

Another important topic discussed in \autoref{sec:persvsephe} was the need to use the Cleipnir execution engine to schedule operations when operations outside of Cleipnir are required to affect persistent systems. To simplify this design in our application, we made practically all of the scheduled operations that use the Cleipnir execution engine be performed within the server. This design is chosen to make it easier to keep track of where the items are emitted to the protocol workflows. The server has several emit functions ready for scheduling the given message type to its desired \code{Source} object. In addition, all functionality in regards to handling and sending incoming protocol messages from the \ac{pbft} network to their respective protocol workflow is centred in its own class called \code{MessageHandler}. In order for the protocol workflows to emit their protocol messages and take advantage of this design, they are required to have a reference to the server object; so they can easily call the correct emit functions. Alternatively, the workflows need to have access to and make a call to a given callback reference that calls the desired emit function in the server. The second alternative here is quite useful when the operations for the protocol workflow are initialized in the server, making it easy to add a callback reference as an initializer parameter. The reactive operations for the checkpoints and view-change can potentially be initialized in the server, making it simple to assign the correct callback function. An example of this is seen in \autoref{code:viewListener} and \autoref{code:CreateCheckpoint} and are discussed in more detail later in \autoref{sec:protocolwork}

Due to our goal of testing the Cleipnir reactive framework, we have deliberately chosen to use the reactive framework every time our protocol workflows needed to wait for and handle protocol messages. This means the functionality for listening in for desired protocol messages and the functionality used to make valid certificates are handled using Cleipnir reactive framework. In addition, in certain areas, we have used Cleipnir reactive framework to implement event-handlers that are set to activate certain processes once a signal or item is sent to the desired process \code{Source} object.

Our \ac{pbft} implementation takes advantage of traditional asynchronous programming for the network layer. We chose this design primarily due to asynchronous programming being generally preferred for multi-client server design~\cite{VIDEO:AsyncConBack, DOC:AsyncAwait}. Considering a replica needed to handle multiple client requests and protocol messages from the other replicas, this seemed like the best choice. The network layer does not take advantage of Cleipnir reactive programming and persistency functionality. Therefore we do not have to worry about \code{CTask} either, meaning the network functionality all uses traditional \code{Task}.


\iffalse
In this chapter we introduce our \ac{pbft} implementation. We will introduce the implementation for the request handler, normal protocol workflow, view-changes, and finally checkpointing. We will also discuss how the Cleipnir framework is used to create the working \ac{pbft} implementation, as well as discuss some benefits and limitations within the current implementation design.

\section{Design Choices}
%Motivation is supposed to be short and summarize these points:
%why did we implement our application the way we did, what was the main focus
%-Usage/testing the usage of the tools
%	-Simplicity --> Protocol description as close to possible
%-Some thought went into persistency due to Cleipnir, CTask for protocol workflow, persisted objects gets assigned Sync/Desync funcs
%-async used for networking
With the goal of the thesis in mind, the \ac{pbft} protocol workflows were designed to be as close defined to the protocol description as possible. To accomplish this, we believed the best approach would be to design protocol-related workflow as synchronous as possible. Several factors persuaded us to focus on keeping the protocol operations mostly synchronous. The first reason was that it is generally easier for developers to keep track of the program progress, making it easier to debug. Modern async/await can generally achieve a similar program workflow. Still, unless you are worried about blocking the main thread, there are no added benefits in terms of complexity or efficiency in using asynchronous programming. The second reason is in regards to Cleipnir. We previously discussed in \autoref{sec:persvsephe} and \autoref{section:PersistentProgramming} using normal asynchronous operations inside a function that uses Cleipnir is not a good idea. Because we wanted to take advantage of reactive programming to handle protocol-related messages in our protocol workflow, we needed to use Cleipnir reactive framework. Since persistency is a core part of the Cleipnir framework, we decided it was best to keep persistency in mind while designing our application. Therefore we decided to keep our protocol implementations inside \code{CTask} functions. Because of this, the only form of asynchronous operations that are performed inside any of the protocol workflows are restricted to other \code{CTask} asynchronous operations. The \code{await} operator still works the same as it does for traditional asynchronous operations. This allows us to take advantage of the \code{await} to set waiting points for \code{CTask} operations as well as ongoing reactive streams; Thereby giving the protocol the abstraction of a synchronous process, even though, in reality, it is an asynchronous process. 
To enable Cleipnir to persist in our protocol workflow, we also had to make sure that objects we wanted to persist in the protocol were persistable. To accomplish this we initialized both a serializer and deserializer functions for Cleipnir that follows the format specified in \autoref{section:PersistentProgramming} to use for each of our defined protocol objects.  
Finally, we have done our best to avoid creating circular dependencies. Circular dependencies would essentially cause the serialization process to fail, as it would lead to two references that depend on one another. We believe there are no circular dependencies in our implementation because the Cleipnir serialization process does not crash during Cleipnir's synchronization process. However, in the case where it does exist a circular dependency in our application, the server and protocol workflow would be our primary suspect. This is because the server emits messages to the protocol workflow while the protocol workflow has an object reference to the server. We believe they do not have a circular dependency because the server interacts with the protocol workflow through the Cleipnir execution engine and does not directly reference the protocol workflow.
%The \ac{pbft} protocol description is orderly constructed. By this, we mean the tasks described in the protocol description can easily be made into a list and can be further split operations in the description based on which protocol phase it is performed. (come back when you have any idea where to go from here)

To make it easier to understand the protocol workflow, we believed the best approach was to keep only the protocol processes described in the descriptions centred inside a single function or class whenever it was possible. We chose this design primarily because we wanted to make the code as readable as possible. It was deemed especially important when designing the standard \ac{pbft} workflow. This design was not entirely possible to replicate for checkpoint and view-change workflows. We also attempted to keep operations unrelated to the protocol outside the protocol workflow. Although in some cases, we cannot avoid this issue. In these cases, a simple function call with a good function name must suffice to avoid increasing the complexity of the overall workflow.
An example of this is using the server to send protocol messages to \ac{pbft}. It is a fundamental part of the \ac{pbft} consensus algorithm to interchange protocol messages. However, the operations performed in the sending operation itself do not affect the protocol workflow. Therefore, a simple call to the servers \code{Multicast} using the newly created protocol message as a parameter should be decisive enough for readers of the workflow. 

Another important topic discussed in \autoref{sec:persvsephe} was the need to use the Cleipnir execution engine to schedule operations when operations outside of Cleipnir are required to affect persistent systems. To simplify this design, practically all of the scheduled operations using the Cleipnir execution engine are performed within the server. This design is chosen to make it easier to keep track of where the items are emitted to the \code{Source} objects. The server has several emit functions ready for scheduling the given message type to its desired \code{Source} object. In addition, all functionality in regards to sending incoming protocol messages from the \ac{pbft} to their respective protocol workflow is centred in its own class called \code{MessageHandler}. In order for the protocol workflows to emit their protocol messages and take advantage of this design, they are required to have a reference to the server object, so they can easily call the correct emit functions, or the workflow needs to make a call to a given callback reference that calls the desired emit function in the server. The second alternative here is quite useful when the operations for the protocol workflow are initialized in the server, making it easy to add a callback reference as an initializer parameter. The reactive operations for the checkpoints and view-change can potentially be initialized in the server, making it simple to assign the correct callback function. An example of this is seen in \autoref{code:viewListener} and \autoref{code:CreateCheckpoint} and are discussed in more detail later in \autoref{sec:protocolwork}

Our \ac{pbft} implementation takes advantage of traditional asynchronous programming for the network layer. We chose this design primarily due to asynchronous programming being generally preferred for multi-client server design~\cite{VIDEO:AsyncConBack, DOC:AsyncAwait}. Considering a replica needed to handle multiple client requests and protocol messages from the other replicas, this seemed like the best choice. The network layer does not take advantage of Cleipnir reactive programming and persistency functionality. Therefore we do not have to worry about \code{CTask} either, meaning the network functionality all uses traditional \code{Task}.
\fi

\iffalse
\section{Motivation}
With the goal of the thesis in mind, the \ac{pbft} protocol workflows were designed to be as close to the protocol description as possible. In order to accomplish this we believed the best approach would be to design protocol related workflow as synchronous as possible. In addition, in order to make it easier to understand the protocol workflow, we believe the best approach is to keep the protocol related code centred around single function or class when possible. This is deemed especially important when constructing the normal protocol workflow. This is because the normal protocol description is orderly constructed, going through different phases until consensus as been reached over the \ac{pbft} network. This could in theory also apply to the view-change description as it is divided into several detailed steps. However, there are several factors which leads to the view-change functionality being split into three separate, but nested, functions. The first reason being that view-changes requires the ability to restart the processes in the case where the process is stationary for too long. To handle this functionality we currently use a mix of timeout operations and goto statements in order to reroute the program flow back to the beginning of the view-change process~\cite{WEB:goto}. The second reason, which also applies to checkpoints, is that the view-change process can be initialized early by receiving view-change messages from other replicas. This may seem similar to protocol operations since its initialized by client requests, however the difference lie in the amount of messages required to initialize the process. Currently, the server needs to support the functionality of starting reactive listeners for view-changes if it ever receives a view-change, however the view-change process itself doesn't start until either the timeout occurs or the replica as received $2f$ messages. Because of this functionality, keeping the code completely synchronous and centered around a single is not possible. As for checkpoints, because the checkpoint processes can be initialized whenever, and majority of the timespent for checkpoints is simply waiting for a replica to receive $2f+1$ unique checkpoints with identical sequence numbers, the source code cannot be centered around a single function.

As we previously mention in \autoref{sec:persvsephe} and \autoref{section:PersistentProgramming} using normal asynchronous operations inside Cleipnir is not a good idea. Therefore, the only form of asynchronous operations performed inside any of the protocol workflows are restricted to other \code{CTask} operations. The \code{await} operator still works well with creating statemachines for asynchronous \code{CTask} operations and waiting for \code{Source} objects to finish all of its operators. Thereby giving the protocol an abstraction as an synchronous process, when in reality it is not. Other than that, traditional \code{Task} based asynchronous operations are well used in the network layer for our implementation. 

Another important topic discussed in \autoref{sec:persvsephe} was the need to use the Cleipnir execution engine to schedule operations when operations outside of Cleipnir required to affect the system within Cleipnir. To simply this design, practically almost all of the scheduled operations using the Cleipnir execution engine are performed within the server. This design is chosen to make it easier to keep track of the origin of the message emit. The server has several emit functions ready to schedule the given message type to its desired \code{Source} object. In order for the protocols workflow to take advantage of this design, they are either required to have a reference to the server object to call the function, or the workflow gets a callback referring to the emit function in the server. The second functionality is quite useful when the operations are initialized by the server. Checkpoints and view-change listeners tend to be initialized by the server, making it simple to assign the correct callback function.

There are currently exists two different implementations for checkpointing and view-changes. Both versions are still documented in the source code, where the second implementations have the letter 2 at the end of its name. The workflow for the both of them remain practically the same for both implementations. The main difference lies with how the implementations handle creating and handling certificates. Essentially the certificate is not deemed valid until it has received $2f+1$ unique and valid message for said message type. This processes is handled differently between the implementation. The second implementation performs the message validation, adding message to proof list and proof list validation over a \code{Source} object by using reactive operators. Meanwhile the first implementation performs these same operations for certificates inside the checkpoint certificate itself inside an append function. Once the certificates are deemed valid, another emit to \code{Source} object has to be made, so that the view-change and checkpoint operations are given the signal to continue with the next operations. The first implementation is required to have the callback function to the server emit message inside the certificate, while the second implementation simply has this function callback right after the reactive listener. The design was changed in order to accommodate the need for more reactive operations in the application. From our experience the second implementation generally performed better than the first implementation and was also for the most part more consistent. The first implementation sometimes encountered issues with the callback function, especially the view-change implementation.

As persistency is a core part of the Cleipnir it was decided to design the application to handle some form of persistency. This meant the protocol related operations had to be run in Cleipnir, which in turn meant the protocol had to be either synchronous or asynchronous using \code{CTask}'s. In addition objects run in the protocols are required to be persistable, meaning the objects needed to get a serializer function and deserializer function connected to Cleipnir. This also includes using Cleipnir inbuilt data structures when the persisted objects needed to also persist data structures. This is especially present in certificates since they need to persist their list of proofs, meaning the proof list uses the inbuilt Cleipnir \code{CList} to substitute normal list. Unfortunately there is an unfortunate oversight in our part when we designed the application. The application currently uses \ac{json}~\cite{WEB:NewJSON} to serialize and deserialize messages when sent over the \ac{pbft} network. \ac{json} formatting does not support inbuilt Cleipnir data structures, which is not surprising yet was not directly discovered until very late into the \ac{pbft} implementation. Currently this issue is solved by converting between traditional data structures and inbuilt Cleipnir data structures whenever a message with said data structure needs to be serialized by \ac{json}. The conversation itself is rather simple, a copy of the data storage in the traditional data structure format is created, then the content is copied from the source data storage to the newly created data storage. The process is then reversed on the receiver end after the message as been properly deserialized. Although in retrospect, it might have been more beneficial if the serialization and deserialization for the networking were the same used by Cleipnir as to avoid this issue in the future. Finally, we have done our best to avoid creating circular dependencies. Circular dependencies would essentially cause the serialization process to fail, as both references depend on each other. We do believe there currently are no circular dependencies in our implementation, because the Cleipnir serialization process does not crash during Cleipnir's synchronization process. The server and protocol workflow relationship can potentially be too close to being one. The server emits messages to the protocol workflow while the protocol workflow has a reference to the server. The only reason this does not create a circular dependency is because the server interacts with the protocol workflow through the Cleipnir execution engine and does not have a direct reference to the protocol workflow. Currently the \ac{pbft} implementation does not currently support functional persistency. The reason for as to why the persistency  does not work completely is uncertain. There are two main issues as of now. The first issue is that protocol logger for some reason does not persist all the entries in the logger. As of now, no distinguishable pattern as been found with the data which is lost. Either way, losing certificates in the logger does create big problems. The other reason is that some of the \code{Source} objects linked to the server gets duplicated, meaning that in the system there exist two \code{Source} objects with the exact same reference. This becomes a problem, because each time the server emits a message to the \code{Source} object, it will emit the message to both the original and duplicate \code{Source}. This in turn can essentially cause two identical iterations of the protocol to occur. This even includes even storing the resulting protocol certificates to the logger with the same sequence. As it is not ever intended to store four certificates to a single sequence number, it is very clearly not working properly. 
%As our main goal for this thesis was to evaluate the tools given and not focus on making the implementation persistable, we decided on prioritising other factors of our implementation. We do believe however, that if it were possible to solve the previously mention issues, that the current implementation has a decent (word for solid background for getting it to work). 
\fi

\section{Workflow Details}
\label{sec:protocolwork}
\subsection{Protocol Workflow Implementation}

\subsubsection{Starting protocol instance}
A normal sequence for the \ac{pbft} implementation begins once the request handler receives a request message from the server. The source code for the request handler can be seen in \autoref{code:StartProtocol}. The request handler listens for new requests messages emitted to the \code{Source} object \emph{requestMessage} as seen on line 3. As mentioned in \autoref{sec:persvsephe}, the server is tasked with emitting messages received in the network layer to the appropriate \code{Source} object for the protocol to access the message. The request handler is responsible for making sure that the request received is valid. In addition, the request handler only starts a new iteration of the \ac{pbft} protocol when the next sequence number is within the current sequence number interval. This condition is handled by the if condition spanning lines 4-10. Finally, a new protocol instance is not initialized when the system is performing a view-change. This is determined by the boolean value \code{active} which is tied to the protocol execution object. Once all checks are passed, the request handler updates and collects the current sequence number. Then it calls the asynchronous \code{CTask} function \emph{PerformProtocol} which initializes and starts the \ac{pbft} protocol for the given request. It is important that the request handler is not forced to wait for the \code{PerformProtocol} function to finish because the application must have access to the \emph{requestMessage} \code{Source<Request>} object. This is because we desire an application that can process multiple requests from clients at the same time. If the application does not have access to the \emph{requestMessage} for a long period, then it is likely that a request message emitted by the server gets lost.

%\paragraph{Testsec1}
%\paragraph{Testsec2}
\begin{figure}[H]
	\centering
	%\lstset{style=sharpc}
	\begin{lstlisting}[label = code:StartProtocol, caption=Code section from the request handler, captionpos = b, basicstyle=\scriptsize]
while (true)
{
    var req = await requestMessage.Next();
    if (Crypto.VerifySignature(
            req.Signature, 
            req.CreateCopyTemplate().SerializeToBuffer(), 
            serv.ClientPubKeyRegister[req.ClientID]
        ) 
        && serv.CurSeqNr < serv.CurSeqRange.End.Value
    )
    {
        if (execute.Active)
        {
            int seq = ++serv.CurSeqNr;
            Console.WriteLine("Curseq: " + seq + " for request: " + req);
            _ = PerformProtocol(execute, serv, scheduler, shutdownPhaseSource, req, seq);
        }
    }
}
	\end{lstlisting}
\end{figure}

\subsubsection{Pre-Prepare phase}
\label{sec:prepare}
\iffalse
The pre-prepare phase is the only part of the normal operation workflow that has a different structure depending on whether or not the replica is the primary replica. The source code for the primary replica`s pre-prepare phase can be seen in \autoref{code:Pre-PreparePrimary}. If the replica is the primary, it uses the sequence number that the protocol instance was initialized with and creates a pre-prepare message for this sequence number. The pre-prepare message also contains information regarding the primary’s server id, current view, and the request digest. The pre-prepare message dictates the other replicas sequence number for the processing of the given request. The primary then initializes the protocol certificate used for storing the proof of the prepare phase. Since the first received phase message in the prepare phase is always supposed to be the pre-prepare message, the protocol certificate used for the prepare phase always has the pre-prepare message as its first entry in its proof list. The protocol instance then uses the server reference to multicast the pre-prepare message to the other replicas in the network. 

The source code for the pre-prepare phase for the non-primary replicas is shown in \autoref{code:Pre-PrepareNonPrimary}. The non-primary replica starts its protocol instance by subscribing to the \code{Source<PhaseMessage>} \emph{MesBridge} and listens for incoming phase messages. The subscribe, listening, and handling process of the incoming items to the \emph{MesBridge} is performed on lines 3-12. Considering the replicas only want a pre-prepare message in this reactive listener, it uses a \code{WHERE} clause to ignore any other phase message other than ones that use the pre-prepare messages enum type. In addition, another \code{WHERE} clause is assigned to avoid any pre-prepare messages designated for other requests by comparing request digests. Therefore an incoming phase message can only pass the \code{WHERE} clause if it involves the same request which the protocol instance is processing. The final \code{WHERE} clause validates the phase message where the validation criteria are the same as the ones mentioned in \autoref{sec:detailedProtocol} for pre-prepare messages. Once the replica receives a pre-prepare phase message which passes all the \code{WHERE} clauses, it creates a protocol certificate that uses the same sequence number as the primary`s pre-prepare phase message. Each protocol certificates for the prepare phase now have a matching sequence number for each replica. The non-primary replica finally ends the pre-prepare phase and starts the prepare phase by creating a prepare message and multicasting this message using the same method the primary used for multicasting its pre-prepare phase.

The \code{MERGE} operator is used to ensure that the protocol execution is terminated if a view-change occurs. If the timeout occurs, a unique phase message is emitted to the \code{Source<PhaseMessage>} \emph{ShutdownBridgePhase}. The \code{MERGE} operator binds \emph{ShutdownBridgePhase} reactive stream together with the \emph{MesBridge} stream. This means it is now possible for the \emph{MesBridge} to be unsubscribed without having to pass the previous operators in the reactive chain. This scenario only applies whenever a phase message is detected in the \emph{ShutdownBridgePhase}. As the \code{Merge} operator is the last reactive operator in the chain, the stream returns the phase message received from the \emph{ShutdownBridgePhase} as the resulting phase message. Because this phase message is intentionally faulty, it is not allowed to be used in the prepare phase of the protocol. Therefore a timeout exception is instead called, which closes the instance of the protocol execution.

The design chosen for the source code to the pre-prepare phase is simple and follows a synchronous workflow as we desired, making it easier for developers to write. Unfortunately, there are two severe issues with our current implementation of the pre-prepare phase. These issues are caused by a combination of splitting the code based on primary versus non-primary and the importance of initializing instances of the reactive listeners early. Both problems are theoretically very similar as they both are caused by improper initialization of the reactive listeners used in the \ac{pbft} implementation. The first issue occurs when the primary sends out its pre-prepare phase message before the non-primary replicas have initialized the pre-prepare reactive listener. This results in the pre-prepare phase message not being received by the non-replica, which means it fails the pre-prepare phase. As the pre-prepare phase fails, the timeout will occur, which puts the replica into view-change mode as it believes the primary replica is faulty. The second issue is that a non-primary can receive a prepare message before it has received the initial pre-prepare message from the primary. When this situation occurs, the prepare message gets filtered out by the pre-prepare reactive listener and is therefore not available once this non-primary reaches the prepare phase. In the worst-case scenario, the replica loses all of the prepare phase messages from the other replicas, meaning the protocol instance is stuck in the prepare phase once it finally receives its pre-prepare message.

These issues just discussed are caused primarily because the application struggles with handling phase messages that are received out of intended order. There are several workarounds to handle messages that arrive out of order. However, most of the workarounds available would require adding a lot more complexity to the implementation. As our goal for this thesis is to create an \ac{pbft} implementation that is very simple and accurate to the protocol description, we decided not to redesign the protocol workflow to handle issues regarding pre-prepare messages out of order. As it is meant for the pre-prepare message to get the other non-primary to start processing the request by providing the correct sequence number, we feel it would not be faithful to the original algorithm to change this design. One workaround to this issue would be to initialize the prepare phase reactive listeners at the start of the workflow. Once the pre-prepare message is received, the reactive listener for the prepare messages that did not have the same sequence number that matches the received pre-prepare message would be filtered out. Currently, to somewhat mitigate this issue, the primary is forced to wait for at least a second before starting to multicast its pre-prepare message. Performing this waiting period allows the other replicas to catch up. Which makes it less likely that a replica is far enough behind to lose out on prepare messages before completing handling their pre-prepare message. With this workaround, the issues discussed are, for the most part, stable. As an estimate, an average of 15 operations can be progressed without incident before a user encounters these issues.
\fi

The pre-prepare phase is the only part of the normal operation workflow that has a different structure depending on whether or not the replica is the primary replica. The source code for the primary replica’s pre-prepare phase can be seen in \autoref{code:Pre-PreparePrimary}. If the replica is the primary, it uses the sequence number that the protocol instance was initialized with and creates a pre-prepare message for this sequence number. The pre-prepare message also contains information regarding the primary’s server id, current view, and the request digest. The pre-prepare message dictates the other replicas sequence number for the processing of the given request. The primary then initializes the protocol certificate used for storing the proof of the prepare phase. Since the first received phase message in the prepare phase is always supposed to be the pre-prepare message, the protocol certificate used for the prepare phase always has the pre-prepare message as its first entry in its proof list. The protocol instance then uses the server reference to multicast the pre-prepare message to the other replicas in the network. 

The source code for the pre-prepare phase for the non-primary replicas is shown in \autoref{code:Pre-PrepareNonPrimary}. The non-primary replica starts its protocol instance by subscribing to the \code{Source<PhaseMessage>} \emph{MesBridge} and listens for incoming phase messages. The subscribe, listening, and handling process of the incoming items to the \emph{MesBridge} is performed on lines 3-12. Considering the replicas only want a pre-prepare message in this reactive listener, it uses a \code{WHERE} clause to ignore any other phase message other than ones that use the pre-prepare messages enum type. In addition, another \code{WHERE} clause is assigned to avoid any pre-prepare messages designated for other requests by comparing request digests. Therefore an incoming phase message can only pass the \code{WHERE} clause if it involves the same request which the protocol instance is processing. The final \code{WHERE} clause validates the phase message where the validation criteria are the same as the ones mentioned in \autoref{sec:detailedProtocol} for pre-prepare messages. Once the replica receives a pre-prepare phase message which passes all the \code{WHERE} clauses, it creates a protocol certificate that uses the same sequence number as the primary’s pre-prepare phase message. The protocol certificate for the prepare phase now has a matching sequence number for each replica. The non-primary replica finally ends the pre-prepare phase. The normal workflow implementation starts the prepare phase by creating a prepare message and multicasting this message using the same method the primary used for multicasting its pre-prepare phase.

The \code{MERGE} operator is used to ensure that the protocol execution is terminated if a view-change occurs. If the timeout occurs, a unique phase message is emitted to the \code{Source<PhaseMessage>} \emph{ShutdownBridgePhase}. The \code{MERGE} operator binds \emph{ShutdownBridgePhase} reactive stream together with the \emph{MesBridge} stream. This means it is now possible for the \emph{MesBridge} to be unsubscribed without having to pass the previous operators in the reactive chain. This scenario only applies whenever a phase message is detected in the \emph{ShutdownBridgePhase}. As the \code{Merge} operator is the last reactive operator in the chain, the stream returns the phase message received from the \emph{ShutdownBridgePhase} as the resulting phase message. This phase message is intentionally faulty and is not allowed to be used in the prepare phase of the protocol. Therefore once this faulty phase message is received, a timeout exception is instead called, which closes the instance of the protocol execution.

The design chosen for the source code to the pre-prepare phase is simple and follows a synchronous workflow as we desired, making it easier for developers to write. Unfortunately, there are two severe issues with our current implementation of the pre-prepare phase. These issues are caused by a combination of splitting the code based on primary versus non-primary and the importance of initializing instances of the reactive listeners early. Both problems are theoretically very similar as they both are caused by improper initialization of the reactive listeners used in the \ac{pbft} implementation. The first issue occurs when the primary sends out its pre-prepare phase message before the non-primary replicas have initialized the pre-prepare reactive listener. This results in the pre-prepare phase message not being received by the non-replica, which means it fails the pre-prepare phase. As the pre-prepare phase fails, the timeout will eventually occur, which puts the replica into view-change mode as it believes that the primary replica is faulty. The second issue is that a non-primary can receive a prepare message before it has received the initial pre-prepare message from the primary. When this situation occurs, the prepare message gets filtered out by the pre-prepare reactive listener and is therefore not available once this non-primary reaches the prepare phase. In the worst-case scenario, the replica loses all of the prepare phase messages from the other replicas, meaning the protocol instance is stuck in the prepare phase once it finally receives its pre-prepare message.

These issues just discussed are caused primarily because the application struggles with handling phase messages that are received out of intended order. There exist several workarounds to handle messages that arrive out of order. However, most of the workarounds available would require adding a lot more complexity to the implementation. As our goal for this thesis is to create an \ac{pbft} implementation that is very simple and accurate to the protocol description, we decided not to redesign the protocol workflow to handle issues regarding pre-prepare messages out of order. As it is meant for the pre-prepare message to get the other non-primary to start processing the request by providing the correct sequence number, we feel it would not be faithful to the original algorithm to change this design. Once the pre-prepare message is received, the reactive listener for the prepare messages that did not have the same sequence number that matches the received pre-prepare message would be filtered out. Currently, to somewhat mitigate this issue, the primary is forced to wait for at least a second before starting to multicast its pre-prepare message. Performing this waiting period allows the other replicas to catch up. Which makes it less likely that a replica is far enough behind to lose out on prepare messages before completing handling their pre-prepare message. With this workaround, the issues discussed here are, for the most part, stable. As an estimate, an average of 15 operations can be progressed without incident before a user encounters these issues.


\begin{figure}[H]
	\centering
	%\lstset{style=sharpc}
	\begin{lstlisting}[label = code:Pre-PreparePrimary, caption= Source code for pre-prepare phase for primary replica, captionpos = b, basicstyle=\scriptsize]
ProtocolCertificate qcertpre;
byte[] digest = Crypto.CreateDigest(clireq);
int curSeq; 
if (Serv.IsPrimary()) //Primary
{
    curSeq = leaderseq;
    Console.WriteLine("CurSeq:" + curSeq);
    Serv.InitializeLog(curSeq);
    PhaseMessage preprepare = new PhaseMessage(
        Serv.ServID, 
    	curSeq, 
        Serv.CurView, 
        digest, 
        PMessageType.PrePrepare
    );
    Serv.SignMessage(preprepare, MessageType.PhaseMessage);
    qcertpre = new ProtocolCertificate(
        preprepare.SeqNr, 
        preprepare.ViewNr, 
        digest, 
        CertType.Prepared, 
        preprepare
    );
    await Sleep.Until(1000);
    Serv.Multicast(preprepare.SerializeToBuffer(), MessageType.PhaseMessage);
}
	\end{lstlisting}
\end{figure}

\begin{figure}[H]
	\centering
	%\lstset{style=sharpc}
	\begin{lstlisting}[label = code:Pre-PrepareNonPrimary, caption= Pre-prepare phase for non-primary replica, captionpos = b, basicstyle=\scriptsize]
else	//Not Primary
{ 
    var preprepared = await MesBridge
    	              .Where(pm => pm.PhaseType == PMessageType.PrePrepare)
                      .Where(pm => pm.Digest != null && pm.Digest.SequenceEqual(digest))
                      .Where(pm => pm.Validate(
                        Serv.ServPubKeyRegister[pm.ServID],
                        Serv.CurView, 
                        Serv.CurSeqRange)
                       )
                       .Merge(ShutdownBridgePhase)
                       .Next();
                
    if (preprepared.ServID == -1 && preprepared.PhaseType == PMessageType.End) 
        throw new TimeoutException("Timeout Occurred! System is no longer active!");
    qcertpre = new ProtocolCertificate(
        preprepared.SeqNr, 
        Serv.CurView, 
        digest, 
        CertType.Prepared, 
        preprepared
    );
    curSeq = qcertpre.SeqNr; 
    Serv.InitializeLog(curSeq);
    PhaseMessage prepare = new PhaseMessage(
        Serv.ServID, 
        curSeq, 
        Serv.CurView, 
        digest, 
        PMessageType.Prepare
    );
    Serv.SignMessage(prepare, MessageType.PhaseMessage);
    qcertpre.ProofList.Add(prepare);
    Serv.Multicast(prepare.SerializeToBuffer(), MessageType.PhaseMessage);
}
	\end{lstlisting}
\end{figure}		

\iffalse
\subsubsection{Prepare phase}
In comparison to the Pre-prepare phase and the start of the prepare phase, the rest of the workflow in the implementation is relatively stable and straightforward. The prepare and commit phase source code can be seen in \autoref{code:PrepareAndCommit}. The first step of the prepare phase is to initialize the reactive listeners for prepare and commit phase messages. Due to the listeners having several reactive operators connected to their stream, the code must span several code lines to make it more readable. The prepare listener is initialized on lines two to 18, and the commit listener is initialized on lines 25 to 42 in \autoref{code:PrepareAndCommit}. There are two reasons why the reactive listeners for prepare and commit messages are initialized early. The first reason is to reduce the time it takes for the workflow to move from the pre-prepare listener to the following reactive listeners. This time needs to be small to avoid losing potential incoming phase messages to the reactive streams. 
The other reason is to avoid ordering issues between prepare and commit messages. Since the sequence number for the workflow has already been determined during the pre-prepare phase, the prepare and commit phase can initialize their reactive streams early and be active simultaneously. Because of this, the prepare and commit phase does not suffer issues in regards to phase messages being out of order. If the pre-prepare message did not dictate the sequence number for non-primary replicas, this would have also been the ideal design for handling phase messages during the pre-prepare phase.

The reactive listeners used for the prepare phase and the commit phase are almost practically identical. The only significant difference between the two is that they only accept phase messages in the stream with their respective protocol phase. Meaning the reactive listener for the prepare phase filters away phase messages that do not have protocol phase-type prepare. This operation is performed by the first \code{WHERE} clause. In addition, the certificates for both protocol phases are also initialized early. This is because the certificates are now actively updated through the operations in the reactive listeners’ stream instead of returning the emitted phase message.

During the prepare phase, the workflow waits until the prepare certificate has added $2f+1$ unique prepare phase messages to its proof list. In order for a phase message to be added to the prepare certificate, it must pass all of the \code{WHERE} clauses assigned for the reactive listener. In actuality, the workflow only waits for $2f$ prepare phase messages due to the pre-prepare message already been added to the protocol certificate in the pre-prepare phase. Once a valid phase message passes all of the first \code{WHERE} operators, it is added to the designated protocol certificate using the \code{SCAN} operator. The \code{SCAN} operator transforms the certificate’s proof list to include the incoming phase message.  The final \code{WHERE} clause determines whether or not the certificate has reached a sufficient number of valid phase messages in its proof list.
The \code{ValidateCertificate} function essentially calculates the number of phase messages inside the proof list when it excludes duplicates. It also makes sure that the phase messages in the list are indeed valid. The asynchronous \code{await} operator on line 45 is used to wait for the \emph{CAwaitable} in the prepare phase reactive listener to finish all of the linked operators for the listener before moving on with the protocol. Once the validation process has succeeded for the protocol certificate, the workflow can move past the \code{await} operator. The prepare phase finishes after the prepare protocol certificate is added to the protocol log in the server on line 47.

\subsubsection{Commit Phase}
As for the commit phase, like the other protocol phases, the first step is to have each replica create a commit phase message and use the server to multicast the commit phase over the \ac{pbft} network. Afterwards, the commit phase performs practically the same operations as the prepare reactive listener. The commit reactive listener waits for the proof list for the commit certificate to have at least $2f+1$ commit phase messages. The reactive listener for the commit phase has an additional \code{WHERE} clause that makes sure that the prepare phase has already finished which is visible on line 41 in \autoref{code:PrepareAndCommit}. This extra \code{WHERE} clause is used to avoid the commit certificate from being finished before the prepare phase is complete.  After the commit certificate is successfully validated, the protocol workflow is almost finished processing the given request. The protocol workflow first adds the commit certificate to the protocol logger as done prior to the prepare certificate before starting the remaining operations in the protocol workflow. The server now has two valid certificates for the given sequence number assigned to the client request, meaning the replica has the necessary proof that the replicas in the \ac{pbft} network agree to have the application perform the operation in the request. The application finally performs the operation within the request. The last remaining process is to create a reply message, digitally sign this reply message and send the reply message to the client who initially sent the processed request. The reply message includes information to the client in regards to whether the operation given in the original request was completed successfully or not. In our \ac{pbft} implementation, the only  operation the application can do is to write the 'operation' received from the request to the console window and add the operation to a persistent list. The persistent list representing the application state is discussed more in \autoref{section:ImpCheckpointing}.


\subsubsection{Protocol Workflow Evaluation}
%Insert whether you believe the code for each section is defined as good code, explain why. How did usage async, reactive operation help/hinder the protocol workflow
%Around 135 lines of code excluding extra lines for initializing objects. Where 30\% is from the reactive operators.
We managed to achieve our goal of creating an implementation that performs the standard processes of the \ac{pbft} algorithm into a single function. We believe our resulting implementation is relatively faithful to the \ac{pbft} protocol when based only on the protocol description. The operations listed in the protocol workflow follow a synchronous workflow, excluding the operations performed by the reactive operators, making it easier to read the code. The reactive operators can still work well together with the synchronous workflow by completing the required checks and operations on the incoming phase messages independently from the rest of the protocol workflow. By taking advantage of the \code{await} command, we can easily mark the areas in the protocol workflow where we know the protocol workflow cannot function without the result from the reactive operators. We believe the most significant benefit for our design in the \ac{pbft} protocol workflow within a single function is that it became a lot easier to keep track of the protocol operations. For example, due to how the implementation handles the workflow that creates the protocol certificates, it is considerably straightforward to differentiate between the \ac{pbft} protocol phases by looking at the source code. Basically, by looking for the \code{await} points in the protocol workflow, we can approximately determine where one of the three protocol phases finishes. It is an approximation since there are still a couple of operations required to be performed, such as adding the certificate to the protocol log for the prepare and commit phases.  We would argue that it is a significant challenge to simplify the code to prepare and commit phases further without causing a severe issue for protocol workflow. Most of the complexity we currently have in our implementation comes from the fact that the primary and non-primary replicas have different operations in the pre-prepare phase. In addition, our current stop functionality is not exactly straightforward, which further hurts the simplicity of our implementation. We also believe that despite the functionality of the reactive operators being convenient for handling protocol messages, they may be difficult for inexperienced programmers to read. The programmers should at the very least have some fundamental knowledge in regards to chaining operators using query languages such as \ac{sql}, preferably knowing the fundamentals of \ac{linq}~\cite{WEB:sql} statements, to fully grasp most of the reactive operations available in the Cleipnir framework. The normal workflow implementation is around 135 lines of code when we exclude any additional spaces used to make object initialization easier for others to read. In addition, approximately 30\% of the lines are used for the reactive operators. All in all, based on these results, we would argue that the implementation is relatively short to be able to handle all three protocol phases inside a single function. However, an apparent problem with keeping the functionality in this format was the difficulty of handling protocol messages out of order, which forces the developer to deploy workarounds to avoid this problem. We choose to initialize the \code{Source} objects as soon as possible to reduce the number of phase messages dropped. Unfortunately, we cannot deal with the pre-prepare phase messages due to only being used by non-primary replicas, which is a big downside to using the desired format with reactive operators.

%asynchronous operations
Due to us performing the protocol workflow inside \code{CTASK} functions, we are not able to use traditional asynchronous operations inside the protocol workflow. This typically means tasks regarding reading data from files, server requests, networking, or any other job that is preferred to be performed asynchronously should do so outside the protocol workflow. A developer would therefore need to keep this in mind when designing the protocol-related workflows. Still, regardless of whether or not the protocol workflow is performed inside a \code{Task} or a \code{CTask}, the workflow is run asynchronously, allowing us to effortless run instances of the protocol workflow separately. Although, like when using threads, we need to ensure that the separate asynchronous functions do not alter the same properties if the execution order matters. Otherwise, the result of the application state would become unpredictable. Although our \ac{pbft} implementation does add the two resulting protocol certificates to a shared log, the sequence number assigned to the protocol workflow is unique for each iteration of the \ac{pbft} workflow, allowing us to avoid corrupting the protocol state. This is because all of the iterations will have a unique key to store their protocol information. Despite being run in a \code{CTask}, the \code{await} operator is still valuable for the protocol workflow as it is used to wait for the certificates to finish. Without the ability to use the \code{await} operator, we would not have been able to create the desired protocol workflow.
All in all, the asynchronous workflow may hinder the developer from performing certain operations directly inside the workflow. It is still beneficial when looking at the benefits of running the protocol workflow asynchronously. The most significant advantage of running the workflow asynchronous is how simple it is to start multiple iterations of the protocol workflow. In addition, running the protocol workflow asynchronous should scale better for numerous clients in comparison to creating separate threads.

%reactive:
Cleipnir reactive framework was handy for handling protocol workflow received from the server. Although we did have quite the big issue with the usage of \code{Source} objects to create the protocol certificate before using the Cleipnir execution engine to schedule the emits. Once we moved on to using universal formatting for emitting items to the workflows in Cleipnir, it has worked as intended. It was initially challenging to use a couple of Cleipnir reactive operators such as \code{Merge} and \code{Scan}, but relatively simple to learn the general approach of chaining reactive operations. Using \code{Source} objects for sending the protocol messages to their appropriate code section is simple once Cleipnir execution engine was used to schedule the emits in order. From our experience keeping the emit functionality centred to a designated object or class that has reference to the execution engine is the recommended structure. Although, we must consider one aspect when we use reactive programming to handle protocol messages. Each protocol message, regardless of the owner, must all be sent to the protocol workflow in the same way. In our case, all phase messages had to be emitted to the \code{Source} object for the phase message to be validated correctly. This includes its own phase message, meaning a functionality must be available for the protocol workflow to emit phase messages created during protocol workflow. The main advantage of using Cleipnir reactive framework to handle protocol messages for the protocol workflow is that we can structure all of the code related to the phase message inside a single block of code. In addition, due to the nature of the reactive operations chain, it is easy to control the order of the operations that need to be on the protocol message. Finally, combined with the \code{await} operator, we can also easily dictate the waiting for the condition to be met scenario of the consensus algorithm. In our case, the creation of valid protocol certificates by submitting validated proofs until reaching the quota.
To summarize the main benefits, the Cleipnir reactive framework provided our normal \ac{pbft} workflow implementation was a simple way to validate and collect phase messages to create valid protocol certificates.  In addition, to making the collaboration between the network layer and the protocol layer easier to develop. The downsides being that they struggle with handling phase messages that are received out of order. Therefore, as a consequence must be initialized as soon as possible to counteract this issue. 
\fi
%\subsubsection{Protocol Workflow Evaluation}



\iffalse
\subsubsection{Prepare phase and Commit phase}
In comparison to the Pre-prepare phase and starting the prepare phase, the rest of the workflow in the implementation is relatively stable and straightforward. The prepare and commit phase source code can be seen in \autoref{code:PrepareAndCommit}. The first step is to initialize the reactive listeners for prepare and commit phase messages. This is done early for two reasons! The first is to mitigate the time between waiting for pre-prepare messages and prepare messages as two avoid potentially losing prepare messages. The other reason is so that the protocol can listen for both prepare messages and commit messages, which means there aren't any ordering issues between messages during the prepare and commit phase. If the pre-prepare message did not dictate the sequence number for non-primary replicas, this would have also been the ideal design for handling pre-prepare message. 

The reactive listener used for the prepare phase are pretty much the same for both phases. The only major difference between the two is that they only accept phase messages for their respective protocol phase. Additionally a commit certificate is initialized early to be used together with the commit reactive listener. Since the prepare messages and prepare certificates are already been initialized in the pre-prepare phase, there is only one more thing to do in the prepare phase. The prepare phase will wait until the prepare certificate as received $2f+1$ unique prepare phase messages which passes all of the \code{WHERE} clauses assigned. In actuality it is to wait for $2f$ prepares and one pre-prepare message. To add the valid phase messages to the designated certificates, we use the \code{SCAN} operator to transform the original proof list for the certificate to include the messages received in the reactive listener. The final \code{WHERE} clause determines whether or not the certificate has received the desired number of unique valid phase messages. Essentially calculating the number of phase messages inside the proof list excluding duplicates, and making sure the phase messages in the list are valid. The asynchronous \code{await} operator is used to wait for the \emph{CAwaitable} to finish this all of the operators linked to the prepare reactive listener before moving on with the protocol. Once the prepare certificate has succeeded its validation process, the prepare certificate is added to protocol log in the server and the commit phase officially starts. 

Like the other phases, the first step will be for each replica to create their commit phase messages and use the server to multicast their commit phase over the \ac{pbft} network. Afterwards the protocol will wait for the proof list for the commit certificate to reach $2f+1$. This rule applies to each of the replica as there are no difference in operations between primaries and non-primaries in the commit phase. The reactive listener for the commit phase will additionally check that prepare phase as already finished validating as to avoid finishing the commit certificate before the protocol certificate. However, this extra check does not effect the protocol workflow in either way, since the protocol certificate is awaited earlier in the process. After the commit certificate is successfully validated, the protocol workflow is essentially completed. The remaining operations performed in the protocol execution is to add the commit certificate to the logger similar to the prepare certificate. The server will now have two valid certificates for the given sequence number to request, meaning the replica now has proof that the protocol was successful for the given request. Finally a reply message will be created, signed and sent to the client that sent the processed request. Additionally the operation within the request will be performed by the application. In our \ac{pbft} implementation the 'operation' will simply be to write the operation to the console window and add the operation to a persistent list. The persistent list representing the application state will be more discussed in \autoref{section:ImpCheckpointing}.
\fi

\subsubsection{Prepare phase}
In comparison to the Pre-prepare phase and the start of the prepare phase, the rest of the workflow in the implementation is relatively stable and straightforward. The prepare and commit phase source code can be seen in \autoref{code:PrepareAndCommit}. The first step of the prepare phase is to initialize the reactive listeners for prepare and commit phase messages. Due to the listeners having several reactive operators connected to their stream, the code must span several code lines to make it more readable. The prepare listener is initialized on lines 2 -18, and the commit listener is initialized on lines 25-42 in \autoref{code:PrepareAndCommit}. There are two reasons why the reactive listeners for prepare and commit messages are initialized early. The first reason is to reduce the time it takes for the workflow to move from the pre-prepare listener to the following reactive listeners. This time needs to be small to avoid losing potential incoming phase messages to the reactive streams. 
The other reason is to avoid ordering issues between prepare and commit messages. Since the sequence number for the workflow has already been determined during the pre-prepare phase, the prepare and commit phase can initialize their reactive streams early and be active simultaneously. Because of this, the prepare and commit phase does not suffer issues in regards to phase messages being out of order. If the pre-prepare message did not dictate the sequence number for non-primary replicas, this would have also been the ideal design for handling phase messages during the pre-prepare phase.

The reactive listeners used for the prepare phase and the commit phase are almost practically identical. The only significant difference between the two reactive listeners is that they only accept phase messages in the stream with their respective protocol phase. For example, the reactive listener for the prepare phase filters away phase messages that do not have protocol phase-type prepare. This operation is performed by the first \code{WHERE} clause. In addition, the certificates for both protocol phases are also initialized early. This is because the certificates are now actively updated through the operations in the reactive listeners’ stream instead of returning the emitted phase message.

During the prepare phase, the workflow waits until the prepare certificate has added $2f+1$ unique prepare phase messages to its proof list. For a phase message to be added to the prepare certificate, it must pass all of the \code{WHERE} clauses assigned for the reactive listener. In actuality, the workflow only waits for $2f$ prepare phase messages due to the pre-prepare message already been added to the protocol certificate during the pre-prepare phase. Once a valid phase message passes all of the first \code{WHERE} operators, it is added to the designated protocol certificate using the \code{SCAN} operator. The \code{SCAN} operator transforms the certificate’s proof list to include the incoming phase message.  The final \code{WHERE} clause determines whether or not the certificate has reached a sufficient number of valid phase messages in its proof list.
The \code{ValidateCertificate} function essentially calculates the number of phase messages inside the proof list when it excludes duplicates. It also makes sure that the phase messages in the list are indeed valid. The asynchronous \code{await} operator on line 45 is used to wait for the \emph{CAwaitable} in the prepare phase reactive listener to finish all of the linked operators for the listener before moving on with the protocol. Once the validation process has succeeded for the protocol certificate, the workflow can move past the \code{await} operator. The prepare phase finishes after the prepare protocol certificate is added to the protocol log in the server on line 47.

\subsubsection{Commit Phase}
As for the commit phase, like the other protocol phases, the first step is to have each replica create a commit phase message and use the server to multicast the commit phase over the \ac{pbft} network. Afterwards, the commit phase performs practically the same operations as the prepare reactive listener. The commit reactive listener waits for the proof list for the commit certificate to have at least $2f+1$ commit phase messages. The reactive listener for the commit phase has an additional \code{WHERE} clause that makes sure that the prepare phase has already finished before exiting the commit reactive listener, which is visible on line 41 in \autoref{code:PrepareAndCommit}. This extra \code{WHERE} clause is used to avoid the commit certificate from being finished before the prepare phase is complete.  After the commit certificate is successfully validated, the protocol workflow is almost finished processing the given request. The protocol workflow first adds the commit certificate to the protocol logger as done prior to the prepare certificate before starting the remaining operations in the protocol workflow. The server now has two valid certificates for the given sequence number assigned to the client request, meaning the replica has the necessary proof that the replicas in the \ac{pbft} network agree to have the application perform the operation for the given request. The application finally performs the operation within the request. The last remaining process is to create a reply message, digitally sign this reply message and send the reply message to the client who initially sent the processed request. The reply message includes information to the client in regards to whether the operation given in the original request was completed successfully or not. In our \ac{pbft} implementation, the only operation the application can do is to write the ‘operation’ received from the request to the console window and add the operation to a persistent list. The persistent list representing the application state is discussed more in \autoref{section:ImpCheckpointing}.

\subsubsection{Protocol Workflow Evaluation}
%Insert whether you believe the code for each section is defined as good code, explain why. How did usage async, reactive operation help/hinder the protocol workflow
%Around 135 lines of code excluding extra lines for initializing objects.Where 30\% is from the reactive operators.
We succeeded in our objective of creating an implementation that performs the standard processes of the \ac{pbft} algorithm into a single function. We believe our resulting implementation is relatively faithful to the \ac{pbft} protocol when based only on the protocol description. The majority of the operations performed in the normal protocol workflow are synchronous, making it easier to read the code. The reactive \code{Source} objects and their chain of operators are the only operations that are performed asynchronously in the normal protocol workflow. The reactive operators still work well together with the synchronous workflow by completing the required checks and operations on the incoming phase messages independently from the rest of the protocol workflow. By taking advantage of the \code{await} command, we easily mark the areas in the protocol workflow where we know the protocol workflow cannot function without the result from the reactive operators. We believe the most significant benefit for our design in the \ac{pbft} protocol workflow within a single function is that it became a lot easier to keep track of the protocol operations. For example, due to how the implementation handles the workflow that creates the protocol certificates, it is considerably straightforward to differentiate between the \ac{pbft} protocol phases by looking at the source code. Basically, by looking for the \code{await} points in the protocol workflow, we can approximately determine where one of the three protocol phases finishes. It is an approximation since there are still a couple of operations required to be performed, such as adding the certificate to the protocol log for the prepare and commit phases.  We would argue that it is a significant challenge to simplify the code to prepare and commit phases further without causing a severe issue for protocol workflow. Most of the complexity we currently have in our implementation comes from the fact that the primary and non-primary replicas have different operations in the pre-prepare phase. In addition, our current stop functionality is not exactly straightforward, which further hurts the simplicity of our implementation. We also believe that despite the functionality of the reactive operators being convenient for handling protocol messages, they may be difficult for inexperienced programmers to read. The programmers should at the very least have some fundamental knowledge in regards to chaining operators using query languages such as \ac{sql}, preferably knowing the fundamentals of \ac{linq}~\cite{WEB:sql} statements, to fully grasp most of the reactive operations available in the Cleipnir framework. The normal workflow implementation is around 135 lines of code when we exclude any additional spaces used to make object initialization easier for others to read. In addition, approximately 30\% of the lines are used for the reactive operators. All in all, based on these results, we would argue that the implementation is relatively short to be able to handle all three protocol phases inside a single function. However, an apparent problem with keeping the functionality in this format was the difficulty of handling protocol messages out of order, which forces the developer to deploy workarounds to avoid this problem. We choose to initialize the \code{Source} objects as soon as possible to reduce the number of phase messages dropped. Unfortunately, we cannot deal with the pre-prepare phase messages due to only being used by non-primary replicas, which is a big downside to using the desired format with reactive operators.

%asynchronous operations
Due to us performing the protocol workflow inside \code{CTASK} functions, we are not able to use traditional asynchronous operations inside the protocol workflow. This typically means tasks regarding reading data from files, server requests, networking, or any other job that is preferred to be performed asynchronously should do so outside the protocol workflow. A developer would therefore need to keep this in mind when designing the protocol-related workflows. Still, regardless of whether or not the protocol workflow is performed inside a \code{Task} or a \code{CTask}, the workflow is run asynchronously, allowing us to effortless run instances of the protocol workflow separately. Although, like when using threads, we need to ensure that the separate asynchronous functions do not alter the same properties if the execution order matters. Otherwise, the result of the application state would become unpredictable. Although our \ac{pbft} implementation does add the two resulting protocol certificates to a shared log, the sequence number assigned to the protocol workflow is unique for each iteration of the \ac{pbft} workflow, allowing us to avoid corrupting the protocol state. This is because all of the iterations will have a unique key to store their protocol information. Despite being run in a \code{CTask}, the \code{await} operator is still valuable for the protocol workflow as it is used to wait for the certificates to finish. Without the ability to use the \code{await} operator, we would not have been able to create the desired protocol workflow.
All in all, the asynchronous workflow may hinder the developer from performing certain operations directly inside the workflow. It is still beneficial when looking at the benefits of running the protocol workflow asynchronously. The most significant advantage of running the workflow asynchronous is how simple it is to start multiple iterations of the protocol workflow. In addition, running the protocol workflow asynchronous should scale better for various clients in comparison to creating separate threads.

%reactive:
Cleipnir reactive framework was handy for handling protocol workflow received from the server. Although we did have quite the big issue with the usage of \code{Source} objects to create the protocol certificate before using the Cleipnir execution engine to schedule the emits. Once we moved on to using universal formatting for emitting items to the workflows in Cleipnir, it has worked as intended. It was initially challenging to use a couple of Cleipnir reactive operators such as \code{Merge} and \code{Scan}, but relatively simple to learn the general approach of chaining reactive operations. Using \code{Source} objects for sending the protocol messages to their appropriate code section is simple once the Cleipnir execution engine was used to schedule the emits in order. From our experience keeping the emit functionality centred to a designated object or class that has reference to the execution engine is the recommended structure. Although, we must consider one aspect when we use reactive programming to handle protocol messages. Each protocol message, regardless of the owner, must all be sent to the protocol workflow in the same way. In our case, all phase messages had to be emitted to the \code{Source} object for the phase message to be validated correctly. This includes its own phase message, meaning a functionality must be available for the protocol workflow to emit phase messages created during protocol workflow. The main advantage of using Cleipnir reactive framework to handle protocol messages for the protocol workflow is that we can structure all of the code related to the phase message inside a single block of code. In addition, due to the nature of chaining operators together in a reactive chain, it is easy to control the order of the operations that need to be on the protocol message. Finally, combined with the \code{await} operator, we can also easily dictate wherein the workflow we want to wait for the sufficient number of protocol messages to be received until the condition required by the consensus algorithm is met. In our case, this would substitute for the process of creating valid protocol certificates by submitting validated proofs until reaching the quota.
To summarize the main benefits, the Cleipnir reactive framework provided our normal \ac{pbft} workflow implementation was a simple way to validate and collect phase messages to create valid protocol certificates.  In addition, to making the collaboration between the network layer and the protocol layer easier to develop. The downsides being that they struggle with handling phase messages that are received out of order. Therefore, as a consequence must be initialized as soon as possible to counteract this issue. 

\begin{figure}[H]
	\centering
	%\lstset{style=sharpc}
	\begin{lstlisting}[label = code:PrepareAndCommit, caption= Prepare and Commit phase, captionpos = b, basicstyle=\scriptsize]
var prepared = MesBridge
               .Where(pm => pm.PhaseType == PMessageType.Prepare)
               .Where(pm => pm.SeqNr == qcertpre.SeqNr)
               .Where(pm => pm.Validate(
                    Serv.ServPubKeyRegister[pm.ServID], 
                    Serv.CurView, 
                    Serv.CurSeqRange, 
                    qcertpre)
                )
                .Where(pm => pm.Digest.SequenceEqual(qcertpre.CurReqDigest))
                .Scan(qcertpre.ProofList, (prooflist, message) =>
                {
                    prooflist.Add(message);
                    return prooflist;
                })
                .Where(_ => qcertpre.ValidateCertificate(FailureNr))
                .Next();
ProtocolCertificate qcertcom = new ProtocolCertificate(
    qcertpre.SeqNr, 
    Serv.CurView, 
    digest, 
    CertType.Committed
);   
var committed = MesBridge
                .Where(pm => pm.PhaseType == PMessageType.Commit)
                .Where(pm => pm.SeqNr == qcertcom.SeqNr)
                .Where(pm => pm.Validate(
                    Serv.ServPubKeyRegister[pm.ServID], 
                    Serv.CurView, 
                    Serv.CurSeqRange, 
                    qcertcom)
                )
                .Where(pm => pm.Digest.SequenceEqual(qcertcom.CurReqDigest))
                .Scan(qcertcom.ProofList, (prooflist, message) =>
                {
                    prooflist.Add(message);
                    return prooflist;
                })
                .Where(_ => qcertcom.ValidateCertificate(FailureNr))
                .Where(_ => qcertpre.ValidateCertificate(FailureNr))
                .Next();
                
Console.WriteLine("Waiting for prepares");
if (Active) await prepared;
else throw new ConstraintException("System is no longer active!");
Serv.AddProtocolCertificate(qcertpre.SeqNr, qcertpre); //add first certificate to Log

//Commit phase
PhaseMessage commitmes = new PhaseMessage(
    Serv.ServID, 
    curSeq, 
    Serv.CurView, 
   	digest, 
    PMessageType.Commit
);
Serv.SignMessage(commitmes, MessageType.PhaseMessage);
Serv.Multicast(commitmes.SerializeToBuffer(), MessageType.PhaseMessage);
Serv.EmitPhaseMessageLocally(commitmes);
Console.WriteLine("Waiting for commits");
if (Active) await committed;
else throw new ConstraintException("System is no longer active!");
Serv.AddProtocolCertificate(qcertcom.SeqNr, qcertcom); //add second certificate to Log
	\end{lstlisting}
\end{figure}

\iffalse
\subsection{Checkpoint Implementation}
\label{section:ImpCheckpointing}
The checkpointing process only occurs after a certain number of requests have been processed by the \ac{pbft} implementation. The \emph{checkpoint interval} determines the number of requests.
For our implementation, the \emph{checkpoint interval} is set to five, meaning after processing five requests, a new checkpoint is created for the system. Our implementation of the checkpoint workflow is divided into three sections. The first section revolves around creating a checkpoint certificate and starting an instance of the reactive checkpoint workflow. The reactive checkpoint workflow performs the second part of the checkpoint workflow. In this part, a reactive \code{Source} object listens for incoming checkpoint messages, which are then validated and added to the certificate’s proof list. This reactive process ends once a certificate has received sufficient checkpoint messages that are deemed valid. The final part consists of emitting the finished stable checkpoint to the server to replace the last stable checkpoint in memory and start the garbage collection process. We will now discuss each of these parts in more detail.

\subsubsection{Initialize Checkpoint Certificate}
The checkpoint certificate is initialized using the last sequence number used by the protocol workflow. The checkpoint certificate also needs to create and store a digest of the current state of the application. Our implementation makes the system digest based on the persistent list that represents the application state. The persistent list contains the operation messages from each of the fully processed requests by the \ac{pbft} protocol. Therefore assuming no errors occur, then the checkpoint for sequence number five has the digest of the list containing the operation from requests one to five. 

The checkpoint workflow starts by first initializing the checkpoint certificate. The certificate includes the information just described, such as the stable sequence number and the digest of the application state. Once the initialization of the checkpoint certificate is done, the checkpoint workflow starts an instance of the checkpoint reactive workflow for the newly created checkpoint certificate. 
We refer to an instance of checkpoint reactive workflow process as an \emph{Checkpoint Listener}. Additionally, the checkpoint certificate is added to the checkpoint logger using the stable sequence number as the key. The process just described can be started in two separate ways. The first method is when the replica itself actively starts the checkpoint process. This is when the replica has processed enough requests in the \ac{pbft} workflow to reach the checkpoint interval. The other approach is when the replica receives a checkpoint message with a sequence number currently not in the checkpoint logger. The checkpoint logger also needs to verify that the checkpoint message has a higher sequence number than the last stable checkpoint stored on the replica. Both methods perform the initialization of the checkpoint certificate and checkpoint listener. However,  the sequence number used for the initialization process differ. The first method uses the last sequence number the protocol processed that initially triggered the checkpoint process. The other way uses the sequence number from the received checkpoint message.  One thing to note is that the replica only performs this process once for a sequence number. Meaning the protocol logger is checked to determine whether or not the checkpoint certificate has already been initialized or not. If the checkpoint certificate is already stored in the logger, the initialization process is not performed again.

Regardless of the way the checkpoint certificate and listener are initialized, the replica is still required to create and multicast its checkpoint message to the \ac{pbft} network once the sequence number matches a checkpoint interval. The checkpoint message created in the replica is also emitted to the checkpoint listener to allow it to be handled the same way as the other checkpoint messages received from the \ac{pbft} network. The checkpoint certificates may be initially stored in the checkpoint logger; we are still not entirely done with them, as we want a checkpoint certificate to become stable. However, for a checkpoint to be deemed stable, it needs to pass the certificate validation processes in the checkpoint listener, which follow the same guidelines as the protocol certificate. A replica can only store one stable checkpoint, meaning the previous stable checkpoint is overwritten whenever a new stable checkpoint with a higher sequence is available. 
The stable checkpoint certificate is used as definitive proof that the \ac{pbft} network agrees on the state of the application up to the stable sequence, meaning the replicas in the \ac{pbft} network now can garbage collect the protocol data from the logger up to the stable sequence number. The garbage collection includes removing any stored checkpoint certificates in the checkpoint logger with lower or equal sequence numbers to the stable checkpoint certificate.
\fi

\subsection{Checkpoint Implementation}
\label{section:ImpCheckpointing}
The checkpointing process only occurs after a certain number of requests have been processed by the \ac{pbft} implementation. The \emph{checkpoint interval} determines the number of requests.
For our implementation, the \emph{checkpoint interval} is set to five, meaning after processing five requests, a new checkpoint is created for the system. Our implementation of the checkpoint workflow is divided into three sections. The first section revolves around creating a checkpoint certificate and starting an instance of the reactive checkpoint workflow. The reactive checkpoint workflow performs the second part of the checkpoint workflow. In this part, a reactive \code{Source} object listens for incoming checkpoint messages, which are then validated and added to the checkpoint certificate’s proof list. This reactive process ends once a certificate has received sufficient checkpoint messages that are deemed valid. The final part consists of emitting the finished stable checkpoint to the server to replace the last stable checkpoint in memory and start the garbage collection process. We are now going to discuss each of these parts in more detail.

\subsubsection{Initialize Checkpoint Certificate}
The checkpoint certificate is initialized using the last sequence number used by the protocol workflow. The checkpoint certificate also needs to create and store a digest of the current state of the application. Our implementation makes the system digest based on the persistent list that represents the application state. The persistent list contains the operation messages from each of the fully processed requests by the \ac{pbft} protocol. Therefore assuming no errors occur, then the checkpoint for sequence number five has the digest of the list containing the operation from requests one to five. 

The checkpoint workflow starts by first initializing the checkpoint certificate. The certificate includes the information just described, such as the stable sequence number and the digest of the application state. Once the initialization of the checkpoint certificate is done, the checkpoint workflow starts an instance of the checkpoint reactive workflow for the newly created checkpoint certificate. 
We refer to an instance of checkpoint reactive workflow process as an \emph{Checkpoint Listener}. Additionally, the checkpoint certificate is added to the checkpoint logger using the stable sequence number as the key. The process just described can be started in two separate ways. The first method is when the replica itself actively starts the checkpoint process. This is when the replica has processed enough requests in the \ac{pbft} workflow to reach the checkpoint interval. The other approach is when the replica receives a checkpoint message with a sequence number currently not in the checkpoint logger. The checkpoint logger also needs to verify that the checkpoint message has a higher sequence number than the last stable checkpoint stored on the replica. Both methods perform the initialization of the checkpoint certificate and checkpoint listener. However,  the sequence number used for the initialization process differ. The first method uses the last sequence number the protocol processed that initially triggered the checkpoint process. The other way uses the sequence number from the received checkpoint message.  One thing to remember is that the replica only performs this process only once for a sequence number. Meaning the protocol logger is checked to determine whether or not the checkpoint certificate has already been initialized or not. If the checkpoint certificate is already stored in the logger, the initialization process is not performed again.

Regardless of the way the checkpoint certificate and listener are initialized, the replica is still required to create and multicast its checkpoint message to the \ac{pbft} network once the sequence number matches a checkpoint interval. The checkpoint message created in the replica is also emitted to the checkpoint listener to allow it to be handled the same way as the other checkpoint messages received from the \ac{pbft} network. The checkpoint certificates initially stored in the checkpoint logger are not stable checkpoint certificates, and our goal is to make at least one of these certificates stable. However, for a checkpoint to be deemed stable, it needs to pass the certificate validation processes in the checkpoint listener, which follow the same guidelines as the protocol certificate. A replica can only store one stable checkpoint, meaning the previous stable checkpoint is overwritten whenever a new stable checkpoint with a higher sequence is available. 
The stable checkpoint certificate is used as definitive proof that the \ac{pbft} network agrees on the state of the application up to the stable sequence, meaning the replicas in the \ac{pbft} network now can garbage collect the protocol data from the logger up to the stable sequence number. The garbage collection includes removing any stored checkpoint certificates in the checkpoint logger with lower or equal sequence numbers to the stable checkpoint certificate.

\subsubsection{Checkpoint Listener Workflow}
\iffalse
The source code for an instance of a checkpoint listener can be seen in \autoref{code:CreateCheckpoint}. The checkpoint listener works similarly to how reactive \code{Source} objects were used in the protocol workflow. The server once it receives a checkpoint message from the network emits the checkpoint message to the \code{Source<Checkpoint>} shared by the server and the checkpoint listener. The checkpoint listener listens for any item emitted by the server to the \code{Source<Checkpoint>} object. The reactive operations performed on the \code{Source<Checkpoint>} object can be seen on lines eight to 17. The checkpoint message received on the stream is first validated before the checkpoint certificate proof list is transformed to have the checkpoint message in its proof list. Unlike the protocol workflow and view-change workflow, each iteration of the checkpoint workflow is not required to finish its execution. In addition, the checkpoint functionality is performed separately to the protocol workflow, meaning it is possible to still process new requests while the checkpoint is created, assuming the protocol workflow has not exceeded the sequence number interval.
This means if the protocol processes enough requests, a new checkpoint listener is created for a checkpoint with a higher sequence number than the previous one. This means it is possible to have multiple checkpoint listeners active at the same time. However, it then becomes a race for the checkpoint listeners to see which one creates the next stable checkpoint certificate. Although it is important to remember that the system does not process any new requests after it has exceeded the current sequence number interval. The reactive listener is finished when all of the reactive operators have been successfully finished, which requires the checkpoint certificate to be deemed stable. A checkpoint certificate is deemed stable once it has $2f+1$ unique and valid checkpoint messages in its proof list. The checkpoint messages must obviously match the checkpoint certificate sequence number and digest.
\fi
\iffalse
\label{sec:checkpointList}
The source code for an instance of a checkpoint listener is presented in \autoref{code:CreateCheckpoint}. The checkpoint listener uses \code{Source<Checkpoint>} similar to how \code{Source<PhaseMessage>} objects were used in the protocol workflow. The server, once it receives a checkpoint message from the network, emits the checkpoint message to the \code{Source<Checkpoint>} shared by the server and the checkpoint listeners. The checkpoint listener listens for checkpoint messages emitted by the server to the \code{Source<Checkpoint>} object. The reactive operations performed on the \code{Source<Checkpoint>} object can be seen on lines 8-17. The checkpoint message received on the stream is first validated before the checkpoint certificate proof list is transformed to have the checkpoint message in its proof list. The \code{WHERE} clauses on line 9 and 10 performs the validation for incoming checkpoint messages. The \code{SCAN} operator is once again used to add the checkpoint to the certificate proof list. 

Unlike the protocol workflow, the iterations of the checkpoint listeners do not need to finish their execution. In addition, the checkpoint functionality is performed separately from the protocol workflow, meaning the protocol workflow can process new requests while the checkpoint workflow tries to create a stable checkpoint certificate. Assuming the protocol workflow has not exceeded the sequence number interval otherwise, no additional requests are processed by the protocol workflow.
If the protocol workflow processes enough requests, a new checkpoint listener is created for another checkpoint with a higher sequence number than the previous one. This means it is possible to have multiple checkpoint listeners active at the same time. However, it then becomes a race for the checkpoint listeners to see which one creates the next stable checkpoint certificate. Although it is important to remember that the system does not process any new requests after it has exceeded the current sequence number interval. 

The reactive listener is finished when all of the reactive operators for the \code{Source<Checkpoint>} have ended, which requires the checkpoint certificate to be stable. A checkpoint certificate is deemed stable once it has $2f+1$ unique and valid checkpoint messages in its proof list. The checkpoint messages in the proof list must obviously match the checkpoint certificate sequence number and digest, which are checked during the certificate validation in the \code{WHERE} clause on line 16.
\fi

\label{sec:checkpointList}
The source code for an instance of a checkpoint listener is presented in \autoref{code:CreateCheckpoint}. The checkpoint listener uses \code{Source<Checkpoint>} similar to how \code{Source<PhaseMessage>} objects were used in the protocol workflow. The server, once it receives a checkpoint message from the network, emits the checkpoint message to the \code{Source<Checkpoint>} shared by the server and the checkpoint listeners. The checkpoint listener listens for checkpoint messages emitted by the server to the \code{Source<Checkpoint>} object. The reactive operations performed on the \code{Source<Checkpoint>} object can be seen on lines 8-17. The checkpoint message received on the stream is first validated before the checkpoint certificate proof list is transformed to have the checkpoint message in its proof list. The \code{WHERE} clauses on line 9 and 10 performs the validation for incoming checkpoint messages. The \code{SCAN} operator is once again used to add the checkpoint to the certificate proof list. 

Unlike the protocol workflow, the iterations of the checkpoint listeners do not need to finish their execution. In addition, the checkpoint functionality is performed separately from the protocol workflow, meaning the protocol workflow can process new requests while the checkpoint workflow tries to create a stable checkpoint certificate. Assuming the protocol workflow has not exceeded the sequence number interval otherwise, no additional requests are processed by the protocol workflow.
If the protocol workflow processes enough requests, a new checkpoint listener is created for another checkpoint with a higher sequence number than the preceding one. This means it is possible to have multiple checkpoint listeners active at the same time. However, it then becomes a race for the checkpoint listeners to see which one creates the next stable checkpoint certificate. Although it is important to remember that the system does not process any new requests after it has exceeded the current sequence number interval. 

The reactive listener is finished when all of the reactive operators for the \code{Source<Checkpoint>} have ended, which requires the checkpoint certificate to be stable. A checkpoint certificate is deemed stable once it has $2f+1$ unique and valid checkpoint messages in its proof list. The checkpoint messages in the proof list must match the checkpoint certificate sequence number and digest, which are checked during the certificate validation in the \code{WHERE} clause on line 16.

\begin{figure}[H]
	\centering
	%\lstset{style=sharpc}
	\begin{lstlisting}[label = code:CreateCheckpoint, caption=Source code for the Checkpoint Listener, captionpos = b, basicstyle=\scriptsize]
public async CTask Listen(
CheckpointCertificate cpc, 
Dictionary<int, RSAParameters> keys, 
Action<CheckpointCertificate> finCallback
)
{
    Console.WriteLine("Checkpoint Listener: " + StableSeqNr);
    await CheckpointBridge
    .Where(check => check.StableSeqNr == StableSeqNr)
    .Where(check => check.Validate(keys[check.ServID]))
    .Scan(cpc.ProofList, (prooflist, message) =>
    {
        prooflist.Add(message);
        return prooflist;
    })
    .Where(_ => cpc.ValidateCertificate(FailureNr))
    .Next();
    finCallback(cpc);
}
    \end{lstlisting}
\end{figure}

\iffalse
\subsubsection{Initiate Garbage Collection}
The third part of the checkpoint functionality is rather simplistic. During startup, the replica initializes its server functionality, including an asynchronous function that listens on a reactive \code{Source<CheckpointCertificate>}. This reactive listener listens for a new stable checkpoint certificate. Once the \code{Source<CheckpointCertificate>} object receives a stable checkpoint certificate, the current stable checkpoint is overwritten by the one it received. Afterwards, the operations in regards to garbage collection are performed. The source code for listening for stable checkpoint certificate can be seen in \autoref{code:ListenForCheckpoint}.
The \code{Source<CheckpointCertificate>} object connected to this function is persisted on the server. The server has a predefined function that uses the Cleipnir scheduler to schedule an emit to this \code{Source} object. Each checkpoint listener is initialized with the callback reference to this function, which allows the checkpoint listener to immediately call the callback address with the finished stable checkpoint certificate whenever all reactive operations are done. The call on the callback reference can be seen in \autoref{code:CreateCheckpoint} on line 18. The Cleipnir execution engine will then schedule the stable checkpoint to be emitted to the reactive listener for stable checkpoint certificates. Once the \code{Source} object receives, the old stable checkpoint certificate is replaced by the new one, even in the case where the replica does not have any existing stable checkpoint certificates. After the new stable checkpoint certificate is assigned to the replica, the garbage collection process begins. The garbage collection process consists of removing records with a lower or equal sequence number to the new stable checkpoint certificate for the protocol, reply, and checkpoint logger.
After the garbage collection is completed, the sequence number interval is extended, allowing the protocol workflow to process more requests.
\fi

\subsubsection{Initiate Garbage Collection}
The third part of the checkpoint functionality is rather simplistic. During startup, the replica initializes its server functionality, including an asynchronous function that listens on a reactive \code{Source<CheckpointCertificate>}. This reactive listener listens for a new stable checkpoint certificate. Once the \code{Source<CheckpointCertificate>} object receives a stable checkpoint certificate, the current stable checkpoint is overwritten by the one it received. Afterwards, the operations in regards to garbage collection are performed. The source code for listening for stable checkpoint certificate can be seen in \autoref{code:ListenForCheckpoint}
The \code{Source<CheckpointCertificate>} object connected to this function is persisted on the server. The server has a predefined function that uses the Cleipnir scheduler to schedule an emit to this \code{Source} object. Each checkpoint listener is initialized with the callback reference to this function, which allows the checkpoint listener to immediately call the callback address with the finished stable checkpoint certificate whenever all reactive operations are done. The call on the callback reference can be seen in \autoref{code:CreateCheckpoint} on line 18. The Cleipnir execution engine then schedules the stable checkpoint to be emitted to the reactive listener for stable checkpoint certificates. Once the \code{Source} object receives, the old stable checkpoint certificate is replaced by the new one, even in the case where the replica does not have any existing stable checkpoint certificates. After the new stable checkpoint certificate is assigned to the replica, the garbage collection process begins. The garbage collection process consists of removing records with a lower or equal sequence number to the new stable checkpoint certificate for the protocol, reply, and checkpoint logger.
After the garbage collection is completed, the sequence number interval is extended, allowing the protocol workflow to process more requests.

\begin{figure}[H]
	\centering
	%\lstset{style=sharpc}
	\begin{lstlisting}[label = code:ListenForCheckpoint, caption=Reactive handler for new stable checkpoints, captionpos = b, basicstyle=\scriptsize]
public async CTask ListenForStableCheckpoint()
{
    Console.WriteLine("Listen for stable checkpoints");
    while (true)
    {
    	var stablecheck = await Subjects.CheckpointFinSubject.Next();
        Console.WriteLine("Update Checkpoint State");
        Console.WriteLine(stablecheck);
        StableCheckpointsCertificate = stablecheck;
        GarbageCollectLog(StableCheckpointsCertificate.LastSeqNr);
        GarbageCollectReplyLog(StableCheckpointsCertificate.LastSeqNr);
        GarbageCollectCheckpointLog(StableCheckpointsCertificate.LastSeqNr);
        UpdateRange(stablecheck.LastSeqNr);
     }
}
    \end{lstlisting}
\end{figure}

\iffalse
\subsubsection{Checkpoint Workflow Evaluation}
%General Protocol workflow 
\label{sec:checkpointEval}
Unfortunately, unlike the normal protocol workflow, we could not keep the checkpoint workflow centered around a single function or file. 
The main challenge design-wise for the checkpoint workflow was the fact that the checkpoint process could be initialized by any replica in the \ac{pbft} network. The checkpoint workflow needed to handle both initialization methods in addition to having the same workflow regardless of whether or not the process was initialized by a received checkpoint message or by the checkpoint interval. Ultimately since the server network layer received the checkpoint messages, we decided it was best to divide up the workflow and instead initialize only if no record existed for the stable sequence number. Regardless we would argue that design-wise, we managed to divide up the program in such a way that the process itself remains simple. It's not more complicated than summarizing the checkpoint workflow as initialization, listening, and initiate garbage collection. Of course, it is a lot more difficult when looking at individual operations in more detail. 

%asynchronous
Due to the checkpoint processes being performed wholly separate from the rest of the protocol workflow, running its operations asynchronously was important. The checkpoint initialization process is only part of the checkpoint workflow that was performed synchronously. Both the checkpoint listener and the listen for stable checkpoint certificate take advantage of asynchronous programming and reactive programming. Both parts of the workflow are required to wait until the desired criteria are met. The checkpoint listener is ideal for asynchronous workflow because it is performed independently from the protocol workflows. In addition, it is unclear when or if an instance of the checkpoint listener would ever finish. Therefore making sure the checkpoint listener doesn't block the thread or steal unnecessary resources important.
The last part of the checkpoint functionality is needed to be practically active all the time as it is unclear when a new stable checkpoint is ready. However, since the garbage collection process rarely occurs and most of the time, the checkpoint functionality is simply waiting, thereby making it desirable to use asynchronous workflow here.

%reactive
Initially, we did not take much advantage of reactive programming when we designed the process of making a checkpoint certificate stable by adding proofs to it. However, changes were made to the design to accommodate for more reactive programming. The result being the checkpoint listener workflow we shown in \autoref{code:CreateCheckpoint} which we described in \autoref{sec:checkpointList}. The original implementation performed all checkpoint message validations and certificate checks whenever a checkpoint message was being added to the proof list. We managed this functionality by using a designated append function, which performed the same operations that are now performed in the reactive chain in the checkpoint listener. The original implementation was functional; however, it was also relatively unstable, meaning we had many additional conditions to check based on where the append function was performed. The primary benefactor to the issues came with the usage of the callback functionality. Not only did we have to assign a callback reference as part of the checkpoint certificate, but the call process also became somewhat unpredictable. A significant contributor was that it was common to call the append function more times than necessary due to receiving more checkpoint messages than was needed. Combine this with the short time intervals between each checkpoint message, and you will get unpredictable results. Not to mention, the Cleipnir execution engine had to schedule the append function calls to avoid the state of the checkpoint certificate becoming unpredictable. Safe to say, we generally preferred the second implementation, which why it was the workflow presented. Generally, the design for the second implementation was a lot more readable and easier to keep track of the workflow. The second implementation also was a lot more stable due to splitting up the processes in the append function into separate reactive operators with more focus on completing a single task. It had better performance due to only having to use the Cleipnir execution engine to schedule the emit to the checkpoint listener rather than having to schedule all of the operations for each checkpoint message as we were forced to with our original design. Finally, the checkpoint certificate no longer needed to have a record of the callback address to the desired emit function in the server. Instead, it was added as a parameter when the checkpoint listener started listening.
The garbage collector functionality has remained the same for both implementations and uses Cleipnir reactive \code{Source} object similar to how channels are used in Golang programming. Generally, this structure works well for the garbage collector as emits only occurs whenever a  new stable checkpoint certificate is ready.
\fi

\subsubsection{Checkpoint Workflow Evaluation}
%General Protocol workflow 
\label{sec:checkpointEval}
Unfortunately, unlike the normal protocol workflow, we could not keep the checkpoint workflow centred around a single function or class. 
The main challenge design-wise for the checkpoint workflow was the fact that the checkpoint process could be initialized by any replica in the \ac{pbft} network. The checkpoint workflow needed to handle both initialization methods in addition to having the same workflow regardless of whether or not the process was initialized by a received checkpoint message or by the checkpoint interval. Ultimately since the server network layer received the checkpoint messages, we decided it was best to divide up the workflow and instead initialize the certificate and listener process only if no record existed for the stable sequence number. Regardless we would argue that design-wise, we managed to divide up the program in such a way that the process itself remains simple. It is not more complicated than summarizing the checkpoint workflow as initialization, listening, and initiate garbage collection. Of course, it is a lot more difficult when looking at individual operations in more detail. 

%asynchronous
Due to the checkpoint processes being performed wholly separate from the rest of the protocol workflow, running most of the checkpoint workflow asynchronously was important. The checkpoint initialization process is only part of the checkpoint workflow that was performed synchronously. Both the checkpoint listener and the listen for stable checkpoint certificate take advantage of both asynchronous programming and reactive programming. Both parts of the workflow are required to wait until the desired criteria are met. The checkpoint listener is ideal for asynchronous workflow because it is performed independently from the protocol workflows. In addition, it is unclear when or if an instance of the checkpoint listener would ever finish. Therefore, it was crucial to make sure the checkpoint listener does not block the thread or steal unnecessary resources.
Regarding the last part of the checkpoint functionality, it is required to be practically active all the time as it is unclear when a new stable checkpoint is ready. However, since the garbage collection process rarely occurs and most of the time, the checkpoint functionality is simply waiting, thereby making it desirable to use asynchronous workflow here.

%reactive
Initially, we did not take much advantage of reactive programming when we designed the process of making a checkpoint certificate stable by adding proofs to it. However, changes were made to the design to accommodate for more reactive programming. The result being the checkpoint listener workflow we shown in \autoref{code:CreateCheckpoint}, which we described in \autoref{sec:checkpointList}. The original implementation performed all checkpoint message validations and certificate checks whenever a checkpoint message was being added to the proof list. We managed this functionality by using a designated append function, which performed the same operations that are now performed in the reactive chain in the checkpoint listener. The original implementation was functional; however, it was also relatively unstable, meaning we had many additional conditions to check based on where the append function was called. The primary benefactor to the issues came with the usage of the callback functionality. Not only did we have to assign a callback reference as part of the checkpoint certificate, but the call process also became somewhat unpredictable. A significant contributor was that it was common to call the append function more times than necessary due to receiving more checkpoint messages than was needed. Combine this with the short time intervals between each checkpoint message, and you will get unpredictable results. Not to mention, the Cleipnir execution engine had to schedule the append function calls to avoid the state of the checkpoint certificate becoming unpredictable. Safe to say, we generally preferred the second implementation, which is why it is the presented workflow. Generally, the design for the second implementation was a lot more readable and easier to keep track of the workflow. The second implementation also was a lot more stable due to splitting up the processes in the append function into separate reactive operators with more focus on completing a single task. It had better performance due to only having to use the Cleipnir execution engine to schedule the emit to the checkpoint listener rather than having to schedule all of the operations for each checkpoint message as we were forced to with our original design. Finally, the checkpoint certificate no longer needed to have a record of the callback address to the desired emit function in the server. Instead, it was added as a parameter when the checkpoint listener started listening.
The garbage collector functionality has remained the same for both implementations and uses Cleipnir reactive \code{Source} object similar to how channels are used in Golang programming. Generally, this structure works well for the garbage collector because emits only occurs whenever a  new stable checkpoint certificate is ready.

\subsection{View-change Implementation}
As previously mentioned in \autoref{sec:view-change} the goal of a view-change is to replace a faulty primary replica with another non-faulty replica successfully. In order for a view-change to be successful, the replicas in the \ac{pbft} network must agree upon a  protocol state that each replica can move on from after the leader change has occurred. Furthermore, the view-change must ensure that the newly selected primary replica is not also faulty. The implementation of the view-change functionality is a lot more complex in comparison to the normal protocol workflow. Several aspects make the view-change functionality challenging to handle appropriately. The view-change must first have some functionality to stop the regular protocol workflow, even when the protocol is still processing a request. Afterwards the view-change messages are exchanged over the \ac{pbft} network until $2f+1$ replicas agree that the system needs to change view. Finally, the replicas have to reprocess any protocol certificates saved in the protocol logger. Our implementation of the view-change can be better described by dividing the workflow into three segments. The first part consists of starting the view-change process. This includes the functionality for stopping active protocol instances. In this section, the application is also set to ignore future protocol messages received during the view-change process. The second part consists of updating the replica's view information and creating and multicasting a view-change message to the \ac{pbft} network. The second part is also responsible for creating the view-change certificate. The last segment is the functionality in regards to setting up the correct protocol state of the \ac{pbft} network for the new view.
We will in the following sections describe the different parts of our view-change implementation in the order in which they are performed.

\iffalse
As previously mentioned in \autoref{sec:view-change} the goal of a view-change is to successfully replace a faulty primary replica with another non-faulty replica. In order for a primary change to be successful, the replicas in the \ac{pbft} network must agree upon a  protocol state that each replica can move on from after the leader change has occurred. Furthermore, the view-change must ensure that the newly selected primary replica is not also faulty. The implementation of the view-change functionality is a lot more complex in comparison to the normal protocol workflow. There are several aspects that make the view-change functionality challenging to handle properly. The view-change must first have some functionality to stop the regular protocol workflow, even in the case where the protocol is still processing a request. Afterwards the view-change messages are exchanged over the \ac{pbft} network until $2f+1$ replicas agree that the system needs to change view. Finally the replicas have to reprocess any protocol certificates saved in the protocol logger. Our implementation of the view-change can be divided into three segments. The first part consists of starting the view-change process and how we set the application to stop and ignore future protocol messages received during the view-change process. The second part consists of updating the replica's view information and creating and multicasting a view-change message to the \ac{pbft} network. The second part is also responsible for creating the view-change certificate. The last segment is the functionality in regards to setting up the correct protocol state of the \ac{pbft} network for the new view.
We will in the following sections describe the different sections of our view-change implementation in the order in which they are performed.
\fi

\subsubsection{Starting a View-Change}
%insert how to start a view-change. Including timeout, setting application to inactive mode and how view-changes can be started by timeout vs protocol messages.
A View-change is started whenever a replica deems the current primary to be faulty. In our implementation, a replica can determine that a primary is defective in two separate ways. The first is the more common approach. We use a timeout functionality to detect irregular activity for the primary replica. The other condition that can start a view-change for the replica is when the replica has received a total of $2f$ view-changes messages from the other replicas in the \ac{pbft} network. In this situation, the replica knows that the view-change exchange only needs its own view-change message for it to be successful. 

In our case, we only support timeout functionality in the protocol workflow during the period where a replica is waiting for a pre-prepare phase message from the primary for a request the replica previously has received. \autoref{code:timeout} shows the source code for where the timeout functionality is initialized. \autoref{code:timeout} also shows how we initialize and how we stop the overall protocol workflow. On line 9-12 we can initialize the \code{AppOperation} within a \code{WhenAny} asynchronous function. The \code{WhenAny} creates a \code{CTask} for the two asynchronous \code{CTask} operations \code{AppOperation} and \code{ListenForShutdown}. The \code{CTask} created for \code{WhenAny} finishes whenever either of the \code{CTask} has finished its operation. In our case the \code{ListenForShutdown} is simply waiting for the given \code{Source} object \code{ShutdownSubject} to receive an item which constitutes as a shutdown signal. When a timeout occurs for the protocol workflow, the timeout emits an item to the \code{ShutdownSubject}, which in turn results in \code{ListenForShutdown} finishes first. Each iteration of the protocol workflow is given a \code{CancellationToken} to stop the timeout functionality after it has received a pre-prepare message from the primary. The \code{CTask<bool>} result provided by the \code{WhenAny} is used to tell the workflow whether or not the \code{AppOperation} managed to finish or if the timeout occurred first. If the AppOperation finishes first, then the return value is true. Otherwise, a timeout has occurred, and the boolean value is false.  If the boolean has false value, we set the application to be in what we refer to as inactive mode. In inactive mode, all requests and protocol-related messages such as phase messages and checkpoints are ignored. The application remains in inactive mode until all of the segments of view-change have been successfully completed. 

After the application is set to be inactive, the application must also stop any active normal protocol workflows. Theoretically, it is possible to keep existing protocol iterations alive during and after a view-change occurs. However, it would be rather wasteful because the \code{CTask}s are never finished. The \code{CTask}s are never stopped because the reactive streams never finish all of their reactive operators. Which in turn would unnecessarily drain the system of resources due to each time an item is emitted to the protocol \code{Source} objects, the old iterations would receive these items as well and would perform the checks as usual. The old protocol workflows would drop them quite quickly because the view number of the received message never matches the old protocols, but they are still unnecessary processes. Therefore, we decided it would be best to terminate any active protocol process when a view-change occurred. 

To accomplish this the application emits a clearly faulty phase message to secondary \code{Source<Phasemessage>} called \code{shutdownPhaseSource}. This \code{Source} object corresponds to the \code{Source<Phasemessage>} used in the \code{Merge} operator shown in \autoref{code:Pre-PrepareNonPrimary}. As we mentioned earlier in the \autoref{sec:prepare}, once the protocol workflow iterations receive the faulty pre-prepare message, it exits the function by throwing as well as catching a \code{TimeoutException}. In the case where the system as already received $2f$ view-change messages, the system emits the shutdown signal to the same \code{ShutdownSubject} \code{Source} object we use in the \code{ListenForShutdown} \code{CTask} function. Meaning the initialization for the view-change functionality does remain the same regardless of the method used to initiate it. The details in regards to handling view-change messages are described in detail in  \autoref{sec:viewchangeListener}. Line 29 in \autoref{code:timeout} is where the view-change functionality begins, and the \code{await} operator is used to make sure the view-change is completed before the protocol can go back to being active. 

\iffalse
A View-change is started whenever a replica deems the current primary to be faulty. In our implementation a replica can determine that a primary is faulty in two seperate ways. The first is the more common approach. We use a timeout functionality to detect irregular activity for the primary replica. The other condition that can start a view-change for the replica is when the replica has received a total of $2f$ view-changes messages from the other replicas in the \ac{pbft} network. In this situation the replica knows that the view-change exchange only needs its own view-change to be successful. 

In our case, we only support timeout functionality in the protocol workflow during the period where a replica is waiting for a pre-prepare phase message from the primary for a request the replica previously has received. \autoref{code:timeout} shows the source code for where the timeout functionality is initialized. \autoref{code:timeout} also shows how we initialize and how we stop the overall protocol workflow. On line nine-12 we can initialize the \code{AppOperation} within a \code{WhenAny} asynchronous function. The \code{WhenAny} creates a \code{CTask} for the two asynchronous \code{CTask} operations \code{AppOperation} and \code{ListenForShutdown}. The \code{CTask} created for \code{WhenAny} finishes whenever either of the \code{CTask} has finished its operation. In our case the \code{ListenForShutdown} is simply waiting for the given \code{Source} object \code{ShutdownSubject} to receive an item which constitutes a shutdown signal. When a timeout occurs for the protocol workflow, the timeout emits an item to the \code{ShutdownSubject}, which in turn results in \code{ListenForShutdown} finishes first. Each iteration of the protocol workflow is given a \code{CancellationToken} in order to stop the timeout functionality after it has received a pre-prepare message from the primary. To make sure we know whether or not the \code{AppOperation} finishes or not is determined by the boolean value returned from the \code{CTask} result from the \code{WhenAny} process. If the AppOperation finishes first then the return value is true, otherwise a timeout has occurred and the boolean value has false value.  If the boolean has false value, we set the application to be in what we refer to as inactive mode. In inactive mode all requests and protocol related messages such as phase messages and checkpoints are ignored. The application remains in inactive mode until all of the segments of view-change have been successfully completed. 

After the application has been set to inactive mode, any existing protocol workflows have to be stopped. Theoretically it is possible to keep existing protocol iterations alive during and after a view-change occurs. However, it would be rather wasteful as the \code{CTask} could never finish as the reactive streams never finish all of its reactive operators.  This in turn would unnecessarily drain the system of resources since each time an item is emitted to the protocol \code{Source} the old iterations would receive these items as well. However, the protocol would just end up dropping them fairly quickly due to the fact that the view number of the received message never matches the old protocols. Therefore, we decided it would be best to terminate any active protocol process when a view-change occured. To accomplish this the application emits a faulty phase message to secondary \code{Source<Phasemessage>} called \code{shutdownPhaseSource}. This \code{Source} object corresponds to the \code{Source<Phasemessage>} used in the \code{Merge} operator shown in \autoref{code:Pre-PrepareNonPrimary}. As we mentioned earlier in the \autoref{sec:prepare}, once the protocol workflow iterations receives the faulty pre-prepare message it exits the function by throwing as well as catching a \code{TimeoutException}. In the case where the system as already received $2f$ view-change messages, the system emits the shutdown signal to the same \code{ShutdownSubject} \code{Source} object we use in the \code{ListenForShutdown} \code{CTask} function. This means that the initialization for the view-change functionality does remain the same regardless of method used to initiate it. The details in regards to handling view-change messages are described in detail in  \autoref{sec:viewchangeListener}. Line 29 in \autoref{code:timeout} is where the view-change functionality begins and the \code{await} operator is used to make sure the view-change is completed before the protocol can go back to being active. 

(mention the inactive/active mode, mention timeout func, how the view-change functionality is repeated to stop faulty replicas from ruining everything)
(Action: )
(DONT FORGET ME!)
In inactive mode all protocol related messages and requests are denied by the main protocol execution. This mode is active until all segments of the view-change functionality is completed successfully. The view-change functionality differs from the other functionality due to the handling of timeout. It has already been mentioned that the view-change functionality starts once a replica exceeds its timeout before receiving a pre-prepare message. However, there are two additional timeouts present in the view-change functionality. These timeouts exist in order for the system to be absolutely sure that the new primary chosen by the $p = v ~mod~ R$ formula does not result in a faulty replica. If the formula does result in a faulty replica, then either the view-change process or the redo protocol process most likely is going to fail. Setting a timeout for these two functionalities, the protocol can recover from a potential frozen state and restart the view-change process by now selecting the next replica on the list. Specifically, the view number is incremented every time the view-change protocol starts, meaning a new primary is going to continue being swapped until a non-faulty primary is finally chosen. Unfortunately the current implementation only handles timeout at the start of the normal protocol workflow, which also gets stopped once the replica receives a pre-prepare message. This means the protocol workflow effectively gets stuck in the situation where the protocol fails during either handling prepare and commit messages. 

(Problems with timeout in our protocol workflow, how we currently stop the protocol execution when a view-change occur)
(Action: )
%(might be useful for the view-change evaluation)There were two main reasons for why this issue was not resolved in our implementation. The first reason was that timeout functionality relies on the \code{WhenAny} asynchronous function~\cite{WEB:whenany}. This function creates a \code{CTask} that is set to finish once either of the attached \code{Task}'s are completed. In our implementation this effectively is set to either the timeout is exceeded or the intended workflow completes. This was unfortunately not very well integrated with reactive \code{Source} objects. Even if the \code{WhenAny} moves on with the program once the timeout is reached, the protocol workflow cannot be exited due to the reactive \code{Source} is forced to finish all the operators before it is deemed completed. To solve this issue we use the \code{Merge} operator in order to enforce \code{Source} object to stop and dispose of the active reactive stream when a timeout occurs. The \code{Merge} operator requires that both the reactive streams that were to merge have the same format. This means that the stop signal to the \code{Merge} operator needed to also be a phase message for the pre-prepare reactive stream. The current workflow for handling the timeout functionality can be seen in \autoref{code:timeout}. The timeout is initialized with a cancellationtoken which can be used to stop the timeout process. This cancellationtoken is brought into the main protocol workflow so that the replica can cancel the timeout operation after it has received  a pre-prepare message. The timeout used in the current implementation is set to 10 seconds. The timeout operation has a reference to an active \code{Source} object which is the same \code{Source} object which is listened to at the function ListenForShutdown. When the timeout exceeds, the timeout function emits an item to the shutdown \code{Source} which in turn makes the \code{CTask} in ListenForShutdown to return before the AppOperation, which lets the program flow to continue. The AppOperation is still active as an asynchronous function, meaning we want to forcefully shut it down so as to avoid creating conflicts with the future emissions to the protocol \code{Source} object. To solve this we emit an obviously faulty phase message with an unique phase message type called \emph{End}. Thanks to the \code{Merge} operator, the pre-prepare reactive listener will finish and return the faulty pre-prepare message. As seen in \autoref{code:Pre-PrepareNonPrimary}, the protocol calls a timeout exception if the pre-prepare reactive listeners return the faulty phase message, meaning the protocol effectively shuts down as intended. 
\fi

\begin{figure}[H]
	\centering
	%\lstset{style=sharpc}
	\begin{lstlisting}[label = code:timeout, caption=Handling timeout for the normal protocol workflow and initiate the View-Change process, captionpos = b, basicstyle=\scriptsize]
CancellationTokenSource cancel = new CancellationTokenSource();
_ = TimeoutOps.AbortableProtocolTimeoutOperation( //starts timeout
  serv.Subjects.ShutdownSubject,
  10000,
  cancel.Token,
  scheduler
);
execute.Serv.ChangeClientStatus(req.ClientID);
bool res = await WhenAny<bool>.Of(
               AppOperation(req, execute, seq, cancel),
               ListenForShutdown(serv.Subjects.ShutdownSubject)
);
Console.WriteLine("Result: " + res);
if (res)
{
  Console.WriteLine($"APP OPERATION {seq} FINISHED");
  ...
}
else
{
  if (execute.Active)
  {
     Console.WriteLine("View-Change starting");
     execute.Active = false;
     serv.ProtocolActive = false;
     await scheduler.Schedule(() =>
        shutdownPhaseSource.Emit(new PhaseMessage(-1, -1, -1, null, PMessageType.End)
     ));
     await execute.HandlePrimaryChange2();
     Console.WriteLine("View-Change completed");
     serv.UpdateSeqNr();
     if (serv.CurSeqNr % serv.CheckpointConstant == 0 && serv.CurSeqNr != 0
        || serv.StableCheckpointsCertificate == null && serv.CurSeqNr > serv.CheckpointConstant
        || serv.StableCheckpointsCertificate != null &&
          (serv.StableCheckpointsCertificate.LastSeqNr + serv.CheckpointConstant) < serv.CurSeqNr)
        serv.CreateCheckpoint2(execute.Serv.CurSeqNr, PseudoApp);
     execute.Active = true;
     serv.ProtocolActive = true;
     serv.GarbageViewChangeRegistry(serv.CurView);
     serv.ResetClientStatus();
   \end{lstlisting}
\end{figure}


\subsubsection{View-Change functionality}
\autoref{code:viewchangefunc} shows the overall workflow for our implementation of the view-change functionality.  The workflow shown in \autoref{code:viewchangefunc} is responsible for initializing and keeping in order the two last segments of our view-change implementation.  Meaning it is responsible for updating the view information for the replica. Then the replica starts participating in the view-change process, which includes both creating and multicasting the replica`s view-change message, in addition to listening and handling any view-change messages received from the \ac{pbft} network. Once a valid view-change certificate is made, the replica starts the new-view phase of the view-change workflow. 

A view-change can pick a new non-faulty primary as its leader due to the next primary being solely dependent on the $p = v ~mod~ R$ formula. Therefore we needed to implement a functionality that could restart the view-change process indefinitely if the view-change exchange or new-view process were to fail or take too long. To handle this functionality, we currently use a mix of timeout operations and \code{goto} statements to reroute the program flow back to the beginning of the view-change process~\cite{WEB:goto}. Specifically, by restarting the view-change workflow whenever a timeout occurs, we force the view-change functionality to keep updating its view information. Therefore, the view number is incremented each time the view-change protocol restarts, meaning the new primary chosen for the new view is continually being swapped until a non-faulty primary is finally chosen. The timeout operations are initialized and used the same as they were for stopping the normal protocol workflow. This includes initializing them with cancellation tokens so that they can be stopped once the workflow has succeeded in performing the desired operation. Both the view-change exchange and the new-view process have their respective timeout operation. The view-change exchange has a timeout set to 10 seconds, just like the normal workflow, while the new-view process has a timeout set to 15 seconds. Extra time is added for the new-view process since it needs to reprocess at worst-case five requests for our implementation. The worst-case scenario number is determined by the checkpoint interval, which is set to five requests for our case. Therefore, the protocol logger could only have up to four finished requests where the last request is never fully processed by the protocol workflow. The \code{WhenAny} asynchronous function is once again used together with the timeout operations. If the view-change process is successful, the program moves on as intended. In the case where the timeout occurs first, then the \code{goto} statement moves the program back to the \emph{ViewChange} label we initialized at the very first line of the view-change workflow. Just like in the checkpoint workflow, we refer to the functionality that uses \code{Source} object to listen for view-change messages emitted by the server to create a valid view-change certificate as an iteration of a \emph{View-change Listener}. Depending on whether or not the server has received any view-change for the current next view number or not, the view-change workflow may need to initialize the view-change certificate and view-change listener. This initialization process occurs on lines 8-17. After the view information is updated and the view-change certificate and view-change listener is initialized, the replica creates a view-change message and multicasts this over the \ac{pbft} network. Afterwards the view-change workflow needs to wait for the view-change listener to keep adding view-change messages until the view-change certificate becomes valid by having $2f+1$ unique and valid view-change messages in its proof list. If this process takes too long, then the timeout occurs, and the view-change workflow starts anew. This functionality is visible on lines 45-47, where the \code{listener} refers to a function that listens on a \code{Source<bool>} that only returns true whenever it receives an item on its reactive stream. The \code{Source<bool>} is only emitted to by the server when the view-change listener is finished making a valid view-change certificate. Once the view-change exchange is complete, the view-change workflow moves on to the new-view phase. This functionality is performed in the \code{ViewChangeProtocol} referred to in the next \code{WhenAny} function at line 56-58. If this process takes too long, then the timeout once again is triggered, and the program starts at the top of the view-change workflow. The \code{ViewChangeProtocol} is responsible for having the new primary create a valid new-view message and multicast this message to the other replicas. The other replicas are responsible for validating that the information in the new-view is valid. Finally, the workflow will reprocess the request that needs to be processed again. Once the view-change workflow shown in \autoref{code:viewchangefunc} is finished, then the view-change is completed, and the application can once again start processing new client requests.

\iffalse
\autoref{code:viewchangefunc} shows the overall workflow for our implementation of the view-change functionality.  The workflow shown in \autoref{code:viewchangefunc} is responsible for initializing and keeping in order the two last segments of our view-change implementation.  Meaning it is responsible for updating the view information for the replica. Then the replica starts participating in the view-change process, which includes both creating and multicasting the replica's view-change message, in addition to listening and handling any view-change messages received from the \ac{pbft} network. Once a valid view-change certificate is made, the replica starts the new-view phase of the view-change workflow. 

It is possible for a view-change to pick a new non-faulty primary as its leader due to the next primary being solely dependent on the $p = v ~mod~ R$ formula. Therefore we needed to ensure that the view-change process could be indefinitely restarted in the case the view-change exchange or new-view process were to fail or take too long. To handle this functionality we currently use a mix of timeout operations and goto statements in order to reroute the program flow back to the beginning of the view-change process~\cite{WEB:goto}. Specifically, by restarting the view-change workflow whenever a timeout occurs, we force the view-change functionality to keep updating its view information. The view number is therefore incremented each time the view-change protocol restarts, meaning the new primary chosen for the new view is continually being swapped until a non-faulty primary is finally chosen. The timeout operations are initialized and used the same as they were for stopping the normal protocol workflow. This includes initializing them with cancellation tokens so that they can be stopped once the workflow has succeeded in performing the desired operation. Both the view-change exchange and the new-view process have their respective timeout operation. The view-change exchange has a timeout set to 10 seconds just like the normal workflow, while the new-view process has a timeout set to 15 seconds. Extra time is added for the new-view process since it needs to reprocess at worst case five requests for our implementation. This is because the checkpoint interval is set to five requests, meaning the protocol logger could potentially only have four finished protocol certificates where the last request is never fully processed by the protocol workflow.  The \code{WhenAny} asynchronous function is once again used together with the timeout operations. If the view-change process is successful the program moves on as intended. In the case where the timeout occurs first, then the \code{goto} statement moves the program back to the \emph{ViewChange} label we initialized at the very first line of the view-change workflow. Just like in the checkpoint workflow, we refer to the functionality that uses a \code{Source} object to listen for view-change messages emitted by the server in order to create a valid view-change certificate as an iteration of a view-change listener. Depending on whether or not the server has received any view-change for the current next view number or not, the view-change workflow may need to initialize the view-change certificate and view-change listener. This initialization process occurs from line eight-17. After the view information is updated and the view-change certificate and view-change listener is initialized the replica creates a view-change message and multicasts this over the \ac{pbft} network. Afterwards the view-change workflow needs to wait for the view-change listener to keep adding view-change messages until the view-change certificate becomes valid by having $2f+1$ unique and valid view-change messages in its proof list. If this process takes too long, then the timeout occurs and the view-change workflow starts anew. This functionality is visible on lines 45-47, where the \code{listener} refers to a function which listens on a \code{Source<bool>} that only returns true whenever it receives an item on its reactive stream. The \code{Source<bool>} is only emitted to by the server when the view-change listener is finished making a valid view-change certificate. Once the view-change exchange is complete the view-change workflow moves on to the new-view phase. This functionality is performed in the \code{ViewChangeProtocol} referred to in the next \code{WhenAny} function at line 56-58. If this process takes too long, then the timeout once again is triggered and the program starts at the top of the view-change workflow. The  \code{ViewChangeProtocol} is responsible for having the new primary create a valid new-view message and multicast this message to the other replicas. The other replicas are responsible for validating that the information in the new-view is valid. Finally the requests that need to be reprocessed are reprocessed. Once the view-change workflow is finished with the workflow shown in \autoref{code:viewchangefunc} then the view-change is completed and the application can once again start processing new client requests.
\fi

\begin{figure}[H]
	\centering
	%\lstset{style=sharpc}
	\begin{lstlisting}[label = code:viewchangefunc, caption=Overall source code for handling view-changes., captionpos = b, basicstyle=\scriptsize]
ViewChange:
//Initialize
Serv.CurPrimary.NextPrimary();
Serv.CurView++;
ViewChangeCertificate vcc;
if (!Serv.ViewMessageRegister.ContainsKey(Serv.CurView))
{
    vcc = new ViewChangeCertificate(Serv.CurPrimary, Serv.StableCheckpointsCertificate, null, null);
    Serv.ViewMessageRegister[Serv.CurView] = vcc;
    ViewChangeListener vclListener = new ViewChangeListener(
        Serv.CurView, 
        Quorum.CalculateFailureLimit(Serv.TotalReplicas), 
        Serv.CurPrimary, 
        Serv.Subjects.ViewChangeSubject, 
        false
    );
    _ = vclListener.Listen(vcc, Serv.ServPubKeyRegister, Serv.EmitViewChange, null);
}
else
{   
    vcc = Serv.ViewMessageRegister[Serv.CurView];
}
var listener = ListenForViewChange();
var shutdownsource = new Source<bool>();
ViewChange vc;
CDictionary<int, ProtocolCertificate> preps;
if (Serv.StableCheckpointsCertificate == null)
{
    preps = Serv.CollectPrepareCertificates(-1);
    vc = new ViewChange(0,Serv.ServID, Serv.CurView, null, preps);
}
else
{
    int stableseq = Serv.StableCheckpointsCertificate.LastSeqNr;
    preps = Serv.CollectPrepareCertificates(stableseq);
    vc = new ViewChange(stableseq,Serv.ServID, Serv.CurView, Serv.StableCheckpointsCertificate, preps);
} 

//View-change
Serv.SignMessage(vc, MessageType.ViewChange);
Serv.EmitViewChangeLocally(vc);
Serv.Multicast(vc.SerializeToBuffer(), MessageType.ViewChange);
CancellationTokenSource cancel = new CancellationTokenSource();
_= TimeoutOps.AbortableProtocolTimeoutOperationCTask(shutdownsource, 10000, cancel.Token);
bool vcs = await WhenAny<bool>.Of(
    listener, 
    ListenForShutdown(shutdownsource)
);
if (!vcs) goto ViewChange;
cancel.Cancel();
            
//New-view.
Source<bool> shutdownsource2 = new Source<bool>();
CancellationTokenSource cancel2 = new CancellationTokenSource();
_= TimeoutOps.AbortableProtocolTimeoutOperationCTask(shutdownsource2, 15000, cancel2.Token);
bool val = await WhenAny<bool>.Of(
    ViewChangeProtocol(preps, vcc), 
    ListenForShutdown(shutdownsource2)
);
if (!val) goto ViewChange;
cancel2.Cancel();
   \end{lstlisting}
\end{figure}

\paragraph{View-Change Listener Workflow}
\label{sec:viewchangeListener}
%\vspace{1cm}
%\parskip
\autoref{code:viewListener} shows the source code for our implementation of a view-change listener. Similar to the checkpoint listener, there are two separate ways to initialize an instance of the view-change listener. The first and most common approach is for a timeout to occur in the protocol workflow due to not receiving the pre-prepare message. This in turn, starts the view-change workflow for the replica, which initializes the view-change listener and view-change certificate for the following view number. The alternative way to start a view-change listener is for the server to receive a view-change message with a view number that it currently does not have in its view-change log. The server has a view-change log for each view-change certificate that the application is currently working on. This is for the case when the \ac{pbft} network is very large and the replicas may disagree upon the next view number. Therefore, in the situation where the replica has been set to inactive mode or is participating for another next view number, it needs to collect any view-change messages it can for other next view numbers.

The view-change listener deviates a bit from the protocol workflow and checkpoint listener. The main difference is that it requires the ability to call upon a shutdown emit in the case where the system already has gotten $2f$ view-change messages. The reason for this functionality is due to making the system more efficient. The replica does not need to wait for a timeout to occur if it already has received $2f$ view-change messages since the \ac{pbft} network only requires that replica`s view-change message to instantiate the new view. Therefore the process is sped up by calling a shutdown emit if it already has $2f$. Of course, this functionality is only helpful if the replica is still in active mode. This is the reason why we added the option toggle on whether or not to use the shutdown emit functionality. To utilize the shutdown functionality, the boolean parameter \code{Shutdown} must be set to true, and the view-change listener must have a callback address to the server shutdown emit function. The server function schedules an item to be sent to the same \code{Source<bool>} \emph{ShutdownSubject} that is used for the timeout functionality in the protocol workflow. This allows us to effectively stop the protocol workflow and set the application to be in inactive mode whenever the replica knows its vote can start a new view. 

The reactive listener performs relatively the same operators for the reactive stream as the normal protocol workflow did for its reactive handlers waiting for prepare and commit messages, just like the checkpoint listener. Firstly, we want only to accept view-change messages that belong to the same view number used for the view-change certificate that the view-change listener is handling. Secondly, the view-change messages received are validated to make sure that it is a valid view-change message. Assuming the validation process is successful, the view-change message is added to the proof list of the view-change certificate. The final reactive operator validates the view-change certificate to see if it has received a sufficient number of valid view-change messages in its proof list. After the view-change reactive listener is finished and a valid view-change certificate is ready, the callback function \emph{finCallback} calls the server to emit a signal to the view-change workflow that the view-change certificate is finished. Once it receives the signal, view-change workflow moves on to the next-view process of the view-change workflow. 

As can be seen in \autoref{code:viewListener}, there are two reactive chains used in the view-change listener. They both listen on the same \code{Source} object, but only one of them is active at a time. The first reactive chain belongs to the shutdown functionality and can be seen on lines 3-15. This one is used when the shutdown functionality is active for the view-change listener. The other reactive chain is seen on lines 17-26. They both perform the same initial \code{Where} clauses and add a valid view-change message to the view-change certificate. The only thing differentiating between the two is the last \code{Where} clause. The shutdown reactive chain determines whether or not the view-change certificate has received $2f$ unique view-change messages in its proof list, while the other has the traditional check of $2f+1$ unique view-change messages. Regardless of whether or not the shutdown functionality is used or not, the last reactive chain determines whether or not the view-change certificate is valid or not. Therefore, whenever the shutdown functionality has finished all of its operations, the program workflow naturally must wait in the second reactive chain for the view-change certificate to receive its last view-change message, which in likelihood should be replicas own view-change message.

\iffalse
\autoref{code:viewListener} shows the source code for our implementation of a view-change listener. Similar to the checkpoint listener, an instance of the view-change listener can be initialized in two seperate ways. The first and most common way is for a timeout to occur in the protocol workflow due to not receiving the pre-prepare message. This in turn starts the view-change workflow for the replica, which as a result initializes the view-change listener and view-change certificate for the next view number. The alternative way to start a view-change listener is for the server to receive a view-change message with a view number that it currently does not have in its view-change register. The server has a view-change registry for each of the view-change certificates that the application is currently working on. This is for the case when the \ac{pbft} network is very large and the replicas may disagree upon the next view number to move on to. Therefore, in the situation where the replica has been set to inactive mode or is participating another for another next view number, it needs to collect any view-change messages it can for other next view numbers.

The view-change listener deviates a bit from the protocol workflow and checkpoint listener. The main difference is that it requires the ability to call upon a shutdown emit in the case where the system already has gotten $2f$ view-change messages. The reason for this functionality is due to making the system more efficient. The replica does not need to wait for a timeout to occur if it already has received $2f$ view-change messages, since the \ac{pbft} network only requires that replica's view-change message in order for the new view to be instantiated. Therefore the process is sped up by calling a shutdown emit if it already has $2f$. Of course this functionality is only useful if the replica is still in active mode. This is the reason as to why we added the option toggle on whether or not to use the shutdown emit functionality. In order to use the shutdown functionality the boolean parameter \code{Shutdown} must be set to true, and the view-change listener must have a callback address to the server shutdown emit function. This function schedules an item to be sent to the same \code{Source<bool>} \emph{ShutdownSubject} that is used for the timeout functionality in the protocol workflow. This allows us to effectively stop the protocol workflow and set the application to be in inactive mode whenever the replica knows its vote can start a new view. 

The reactive listener performs relatively the same operators for the reactive stream as the protocol workflow did for its reactive handlers for prepare and commit messages, just like the checkpoint listener. Firstly, we want to only accept view-change messages that belong to the same view number used for the view-change certificate that the view-change listener is responsible for. Secondly the view-change messages received are validated to make sure that it is a valid view-change message. Assuming the validation process is successful, the view-change message is added to the proof list of the view-change certificate. The final reactive operator validates the view-change certificate to see if it has received the sufficient number of valid view-change messages in its proof list. After the view-change reactive listener is finished and a valid view-change certificate is ready, the callback function \emph{finCallback} calls the server to emit a signal to the view-change workflow that the view-change certificate is finished and can then move on to the next-view process of the view-change workflow. As can be seen in \autoref{code:viewListener}, there are two reactive chains used in the view-change listener. They both listen on the same \code{Source} object but only one of them is active at a time. The first reactive chain belongs to the shutdown functionality and can be seen on lines three-15. This one is used when the shutdown functionality is active for the view-change listener. The other reactive chain is seen on lines 17-26. They both perform the same initial \code{Where} clauses and add a valid view-change message to the view-change certificate. The only thing differentiating between the two is the last \code{Where} clause. The shutdown reactive chain determines whether or not the view-change certificate has received $2f$ unique view-change messages in its proof list, while the other has the traditional check of $2f+1$ unique view-change messages. Regardless of whether or not the shutdown functionality is used or not, the last reactive chain determines whether or not the view-change certificate is valid or not. Therefore, whenever the shutdown functionality has finished all of its operations, the program workflow naturally must wait in the second reactive chain for the view-change certificate to receive its last view-change message, which in likelihood should be its own view-change message. 
\fi

\iffalse
As previously mention in \autoref{sec:view-change} the goal of a view-change is to successfully replace a faulty primary replica with another non-faulty replica. In order for a primary change to be successful, the replicas in the \ac{pbft} network needs to agree upon the state the program continues on after the primary change has occurred. Furthermore the view-change must ensure that the new replica selected for primary responsibility is not faulty. 

The operations to ensure these criteria were briefly mentioned in \autoref{sec:view-change}. Although in total there are a quite lot of operations needed for a successful view-change to take place. However, it is possible to divide up the operations into two segments based on which goal the operations attempt to fulfill. Excluding the processes of shutting down the protocol execution, the first part of the view-change process is for the replicas in the network to agree on that a view-change is necessary. This goal is achieved by having the replicas multicast and listen for view-change messages. Since the next primary is determined by the formula $p = v ~mod~ R$, \ac{pbft} doesn't require any election process. The view-change messages instead contains information of the replicas current checkpoint information as well as current state of the logged certificates. This is so that the new primary can have all the relevant information to create the new state for the \ac{pbft} system. The goal of the second segment is to initialize the \ac{pbft} system state after the view-change is finished. This goal is fulfilled by looking at the current stable checkpoint and the current protocol certificates stored in memory. In order to make sure that requests were not fully processed before the view-change occured. The \ac{pbft} needs to redo each of the requests stored in the logger up to the highest sequence number seen in the \ac{pbft} system. Thankfully, due to stable checkpoints, the process does not need to take into account every single request ever processed. The new primary is responsible for starting this process by creating and multicast a new-view message. This message acts as an introduction letter, telling the other replicas in the network that it is the new primary and additionally provide a view-change certificate proving this fact. The new-view message also contain a list of pre-preprepares which are created from the information stored within the protocol certificates in the view-messages received. This new-view message is validated by the other replicas. If the replicas deem the information in new-view message as valid, the replicas will use these pre-prepares message to create prepare and commit messages and redo the \ac{pbft} algorithm for these pre-prepares. The system is finally finished with the view-change once all the pre-prepares have stored their respective two protocol certificates. 

As for implementing this functionality, our implementation can be divided into four segments. The first consist of the timeout functionality that when triggers puts the application into non-active mode. The second part consist of updating the view data, creating view-change messages, multicast these view-changes over the \ac{pbft} network and finally store the collection of view-changes until quorum has been reached. The third consist of creating and validating functionality for new-view message. Finally the last segment consist of the redo protocol functionality. 

In non-active mode all protocol related messages and requests are denied by the main protocol execution. This mode is active until all four segments of the view-change functionality as been completed successfully. The view-change functionality differs from the other functionality due to the handling of timeout. It has already been mention that the view-change functionality start once a replica exceeds its timeout before receiving a pre-preprepare message. However, there are two additional timeout present in the view-change functionality. These timeout exists in order for the system to be absolute sure that the new primary chosen by the $p = v ~mod~ R$ formula does not result in a faulty replica. If the formula does result in a faulty formula, then either view-change process or redo protocol process will most likely fail. Setting a timeout for these two functionalities, the protocol can recover from a potential frozen state and restart the view-change process by now selecting the next replica on the list. Essentially, the view number is incremented every time the view-change protocol changes, meaning a new primary is selected until a non-faulty primary is chosen. Unfortunately the current implementation only handles timeout at the start of the normal protocol workflow, which also gets stopped once the replica receives a pre-prepare message. This means the protocol gets effectively stuck in the case where the protocol fails at handling prepare and commit messages. There were to main reason for why this issue was not resolved in our implementation. The first reason was that timeout functionality relies on the \code{WhenAny} asynchronous function~\cite{WEB:whenany}. This function creates a \code{Task} that is set to finish once either of the attached \code{Task}'s completes. In our implementation this effectively is set to either the timeout is exceeded or the process that is waited for completes. This was unfortunately not very well integrated with reactive listeners, as it is forced to finish all the operators before it is deemed completed. It required the \code{Merge} operator in order to enforce reactive listeners to stop and dispose of the active reactive stream when a timeout occured. The \code{Merge} operator required that both the reactive streams that were to merge had the same format. This means that the stop signal to the \code{Merge} operator needed to also be a phase message for the pre-prepare reactive stream. The current workflow for handling the timeout functionality can be seen in \autoref{code:timeout}. The time is first initialized with a cancellationtoken which is brought into the main protocol workflow so that the protocol can cancel the timeout when it receives a pre-prepare message. The timeout used in the current implementation is set to ten seconds. The timeout gets a reference to an active \code{Source} object which is the same \code{Source} which is listened to at the function ListenForShutdown. When the timeout exceeds, the timeout function will emit a message to the shutdown \code{Source} which in turn makes the \code{CTask} in ListenForShutdown to return before the AppOperation, which lets the program flow to continue. The AppOperation is still active as an asynchronous function, meaning we want to forcefully shut it down so as to avoid creating conflicts with the future emits to the protocol \code{Source} object. To solve this we emit an obviously faulty phase message with an unique phase message type called \emph{End}. Thanks to the \code{Merge} operator, the pre-prepare reactive listener will finish and returns the faulty pre-prepare message. As seen in \autoref{code:Pre-PrepareNonPrimary}, the protocol calls a timeout exception if the pre-prepare reactive listeners returns the faulty phase message, meaning the protocol effectively shuts down as intended.   
\begin{figure}[H]
	\centering
	%\lstset{style=sharpc}
	\begin{lstlisting}[label = code:timeout, caption=Handling timeout for the normal protocol workflow and initiate View-Change, captionpos = b, basicstyle=\scriptsize]
CancellationTokenSource cancel = new CancellationTokenSource();
_ = TimeoutOps.AbortableProtocolTimeoutOperation( //starts timeout
   serv.Subjects.ShutdownSubject,
   10000,
   cancel.Token,
   scheduler
);
execute.Serv.ChangeClientStatus(req.ClientID);
bool res = await WhenAny<bool>.Of(
                AppOperation(req, execute, seq, cancel),
                ListenForShutdown(serv.Subjects.ShutdownSubject)
);
Console.WriteLine("Result: " + res);
if (res)
{
   Console.WriteLine($"APP OPERATION {seq} FINISHED");
   ...
}
else
{
   if (execute.Active)
   {
      Console.WriteLine("View-Change starting");
      execute.Active = false;
      serv.ProtocolActive = false;
      await scheduler.Schedule(() =>
         shutdownPhaseSource.Emit(new PhaseMessage(-1, -1, -1, null, PMessageType.End)
      ));
      await execute.HandlePrimaryChange2(); 
      Console.WriteLine("View-Change completed");
      serv.UpdateSeqNr();
      ...
    \end{lstlisting}
\end{figure} 
\fi
\iffalse
The view-change exchange segment of the code starts by first setting the replica into the next view by incrementing its view number. The next operation sets the replicas view-change certificate. This step is dependent on whether or not the replica has received previously received view-change messages from another replica. If the replica has not received any view-messages from other replicas than it initializes the view-change certificate and initializes the view-change reactive listener. Info about the rest of the view-change process in the main workflow...

The view-change listener deviates a bit from the other reactive listeners. The main difference is that it also requires the ability to call upon a shutdown emit in the case where the system already has gotten $2f$ view-change messages. The reason for this functionality is mostly due to making the system more efficient. The replica does not need to wait for a timeout to occur if it already has received $2f$ view-change messages since the \ac{pbft} network only requires that replica's view-change in order for the new view to be initialized. Therefore the process is speed up by calling for a shutdown emit if already has $2f$. Ofcourse this functionality is only useful if the replica is still in active mode. This is the reason as to why the option to not trigger the shutdown emit is an option. Other than that the reactive listener performs relatively the same operators for the reactive stream. Firstly, we want to only accept view-change message that belongs to the same next view number as the replica. Secondly the view-change messages received are validated to make sure that it is a valid view-change message. Assuming the validation process is successful, the view-change message is added to the view-change certificate proof list. The final reactive operator validates that the view-change certificate to see if it has received the sufficient number of valid view-change messages in its proof list. After the view-change reactive listener is finished and a valid view-change certificate is ready, the callback function \emph{finCallback} calls the servers to emit a signal to the view-change workflow that the view-change certificate is finished and can move on with the next step of the view-change process. 
 
segment from motivation, rewrite and get it in the main text somehow!
This could in theory also apply to the view-change description as it is divided into several detailed steps. However, there are several factors which lead to the view-change functionality being split into three separate, but nested, functions. The first reason being that view-changes require the ability to restart the processes in the case where the request processing remains stationary for too long. To handle this functionality we currently use a mix of timeout operations and goto statements in order to reroute the program flow back to the beginning of the view-change process~\cite{WEB:goto}. The second reason, which also applies to checkpoints, is that the view-change process can be initialized early by receiving view-change messages from other replicas. This may seem similar to protocol operations since it is initialized by client requests, however the difference lies in the amount of messages required to initialize the process. Currently, the server needs to support the functionality of starting reactive listeners for view-changes if it ever receives a view-change, however the view-change process itself doesn't start until either the timeout occurs or the replica has received $2f$ messages. Because of this functionality, keeping the code completely synchronous and centered around a single function is not possible. As for checkpoints, because the checkpoint processes can be initialized whenever, and the majority of the time spent for checkpoints is used waiting for a replica to receive $2f+1$ unique checkpoints with identical sequence numbers, the source code cannot be centered around a single function.
\fi

\begin{figure}[H]
	\centering
	%\lstset{style=sharpc}
	\begin{lstlisting}[label = code:viewListener, caption=Source code for View-Change Listener, captionpos = b, basicstyle=\scriptsize]
if (Shutdown && shutdownCallback != null)
{
   Console.WriteLine("With shutdown");
   await ViewBridge
      .Where(vc => vc.NextViewNr == NewViewNr)
      .Where(vc => vc.Validate(keys[vc.ServID], ServerViewInfo.ViewNr))
      .Scan(vcc.ProofList, (prooflist, message) =>
      {
        prooflist.Add(message);
        return prooflist;
      })
      .Where(_ => vcc.ShutdownReached(FailureNr))
      .Next();
   Console.WriteLine("Calling shutdown");
   shutdownCallback();
}
await ViewBridge
   .Where(vc => vc.NextViewNr == NewViewNr)
   .Where(vc => vc.Validate(keys[vc.ServID], ServerViewInfo.ViewNr))
   .Scan(vcc.ProofList, (prooflist, message) =>
   {
     prooflist.Add(message);
     return prooflist;
   })
   .Where(_ => vcc.ValidateCertificate(FailureNr))
   .Next();
Console.WriteLine("Finished Listen view changes");
finCallback();
    \end{lstlisting}
\end{figure}

\paragraph{New-View Workflow}
\vspace{1cm}

The goal of this segment is to initialize the new common \ac{pbft} protocol state of the system after the view-change. The protocol state is determined by looking at the current stable checkpoint and the different protocol certificates obtained from the view-change messages. Thankfully the stable checkpoint can choose the last sequence number in which the majority of the replicas have agreed upon the application state. On the other hand, requests that are processed with higher sequence numbers do not have that guarantee. Therefore, the system is unsure whether or not the majority of replicas have actually managed to process these requests appropriately. The only way to be sure none of the protocol certificates are corrupted or incomplete is to redo their processing. Since each replica sends copies of their protocol certificates stored and their last checkpoint proof in their view-change message, it is possible to determine the requests that have to be reprocessed, including vital information needed to reprocess the request. The new primary is given the task to ready pre-prepare messages for each request that has to be reprocessed. In the unfortunate situation where there does not exist a protocol certificate record for a request that does need to be reprocessed, the digest of the request is set to \code{null}, indicating a missing operation. The list of pre-prepare phase messages created by the new primary is added to a new view message and multicasted to the other replicas in the \ac{pbft}. The new-view message also contains the view-change certificate created from the view-change message exchange. In this way, the new-view message is essentially a message to the other replicas in the network, relaying that it is the new primary of the \ac{pbft} system, and here is the proof to show it. The other replicas validate the new-view message they receive from the new primary. If the information in the new-view message is incorrect, the replica treats this as a failure of the new-view phase of the workflow and restarts the view-change process with the following view number. Otherwise, the replicas join the new primary in reprocessing the requests anew. 

The source code for the reprocessing functionality can be seen in \autoref{code:redoprotocol}. The code here is clearly very similar to the normal protocol workflow shown earlier in \autoref{code:PrepareAndCommit} as we are literally attempting to do the same operations. The most obvious difference between the two workflows is the lack of pre-prepare phase for the redo processing since all pre-prepare phase messages have already been made and multicasted to the replicas. Since we know that the reprocess functionality will be repeated until all of the pre-prepare phase messages for the protocol state are reprocessed, we decided it was best to iterate the reprocess functionality over the list of pre-prepare messages. As the pre-prepare phase messages have already been exchanged with the other replicas, the new primary only needs to listen and wait for the prepare phase messages from the other replicas during the prepare phase. Just like in the normal workflow, every replica, including the new primary, has to participate in the commit phase by creating a commit message and multicasting it over the \ac{pbft} network. The reactive operators used for the reactive chain are the same as those used in the prepare and commit phase in the normal protocol workflow. Although to avoid potentially cause issues with the Cleipnir execution engine, we decided to use a different \code{Source<PhaseMessage>} object for the reprocessing functionality. This \code{Source} object is known as \emph{ReMesBridge} as can be seen on line 16 and 27. The reason why we decided to use different \code{Source} objects to differentiate between the normal workflow and the reprocess workflow is because we wanted to avoid potentially scheduling phase messages for the wrong workflow. From our experience scheduling for the same \code{Source} object for two separate workflows can, in the best case, be easily filtered out by the \code{Where} clauses. However, in the worst-case scenario, it can freeze the program due to scheduling an emit without any listeners for said \code{Source} object, when the program still needs to schedule other additional operations before the listeners are initialized. Therefore, to avoid this issue, we instead use two \code{Source<PhaseMessage>}, where the server schedules the phase message to be emitted to the appropriate \code{Source} object based on whether the application is active or inactive. 

Unfortunately, just like the protocol workflow, the redo functionality can fail and get stuck. On the other hand, this is not a significant issue as the view-change workflow can move on to the next view number and restart the view-change workflow. This means the application never gets thoroughly stuck, even in the case the redo functionality fails. The redo functionality can fail if too many messages are received and emitted before the reactive listeners are ready. As the redo functionality immediately moves on to reprocess the next pre-prepare message whenever it is finished with another, it is possible to lose the phase messages for the next one if the other replicas are a lot faster. To mitigate the loss of phase messages, we decided to add wait periods where we know it is possible to miss phase messages. The first wait period is set to half a second and is performed right after the reactive listeners are defined on line 38. The other wait period is set to three-quarters of a second and is performed after the prepare phase is finished on line 53. Once the redo functionality has successfully reprocessed all of the pre-prepare phase messages for previous requests, the protocol logger should now have the two valid prepare certificates for each sequence number up to the point when the view-change was initiated. The view-change is then completed, and the application is once again set to active mode. We also perform garbage collection for the view-change log once the view-change is deemed successful, as seen in \autoref{code:timeout}.

\iffalse
The goal of this segment is to initialize the new common \ac{pbft} protocol state of the system after the view-change. The protocol state is determined by looking at the current stable checkpoint and the different protocol certificates obtained from the view-change messages. Thankfully the stable checkpoint can determine the last sequence number in which the majority of the replicas have agreed upon the application state. On the other hand, requests that are processed with higher sequence numbers do not have that guarantee. The system is therefore unsure whether or not the majority of replicas has actually managed to process these requests properly. The only way to be sure none of the protocol certificates are corrupted or incomplete is to redo their processing. Since each replica sends copies of their protocol certificates stored after their last checkpoint in their view-change message,  it is possible to determine both the requests that have to be reprocessed including intergral information needed to reprocess the request. The new primary is given the task to ready pre-prepare messages for each of the requests that has to be reprocessed. In the unfortunate situation where there does not exist a protocol certificate record for a request that does need to be reprocessed, the digest of the request is set to \code{null}, indicating a missing operation. The list of pre-prepare phase messages created by the new primary are added to a new view message and multicasted to the other replicas in the \ac{pbft}. The new-view message also contains the view-change certificate created from the view-change message exchange. In this way the new-view message is essentially a message to the other replicas in the network relaying that it is the new primary of the \ac{pbft} system and here is the proof to show it. The other replicas validate the new-view message they receive from the new primary. In the case where the information in the new-view message is incorrect, the replica treats this as a failure of the new-view phase of the program and restarts the view-change process with the next view number available. Otherwise the replicas join the new primary in reprocessing the requests anew. 

The source code for the reprocessing functionality can be seen in \autoref{code:redoprotocol}. The code here is obviously very similar to the normal protocol workflow shown earlier in \autoref{code:PrepareAndCommit} as we are literally attempting to do the same operations. The most obvious difference between the two workflows is the lack of pre-prepare phase for the redo processing, since all of the pre-prepare phase messages have already been made and multicasted to the replicas. Since we know that the reprocess functionality is going to be repeated until all of the pre-prepare phase messages for the protocol state are reprocessed, we decided it was best to simply iterate the reprocess functionality over the list of reprocess functionality. As the pre-prepare phase messages have already been exchanged with the other replicas, the new primary only needs to listen and wait for the prepare phase messages from the other replicas during the prepare phase. Just like in the normal workflow, every replica, including the new primary, has to participate in the commit phase by creating a commit message and multicasting it over the \ac{pbft} network. The reactive operators used for the reactive chain are the same as the one used in the prepare and commit phase in the normal protocol workflow. Although, to avoid potentially cause issues with the Cleipnir execution engine, we decided to use a different \code{Source<PhaseMessage>} object for the reprocessing functionality. This \code{Source} object is known as \emph{ReMesBridge} as can be seen on line 16 and 27. The reason why we decided to use different \code{Source} objects to differentiate between the normal workflow and the reprocess workflow is because we wanted to avoid potentially scheduling phase messages for the wrong workflow. From our experience scheduling for the same \code{Source} object for two separate workflows can at the best case be easily filtered out by the \code{Where} clauses. However, in the worst case scenario the program can be freezed due to scheduling an emit without any listeners for said \code{Source} object, when the program still needs to schedule other additional operations before the listeners are initialized. Therefore, in order to avoid this issue we instead use two \code{Source<PhaseMessage>}, where the server schedules the phase message to be emitted to the appropriate \code{Source} object based on whether the application is active or inactive. Unfortunately, just like the protocol workflow it is possible for the redo functionality to fail and get stuck. On the other hand, this is not a major issue as the view-change workflow simply moves on to the next view number and restarts the processes in the view-change workflow. This means the application never gets fully stuck even in the case the redo functionality fails. The redo functionality can fail if too many messages are received and emitted before the reactive listeners are ready. As the redo functionality immediately moves on to reprocess the next pre-prepare message whenever it is finished with another, it is possible to lose the phase messages for the next one if the other replicas are a lot faster. To mitigate the loss of phase messages, we decided to add wait periods where we know it is possible to miss phase messages. The first wait period is set to half a second and is performed right after the reactive listeners are defined on line 38. The other wait period is set to three-quarters of a second and is performed after the prepare phase is finished on line 53. Once the redo functionality has successfully reprocessed all of the pre-prepare phase messages for previous requests, the protocol logger should now have the two vald prepare certificates for each of the sequence numbers up to the point when the view-change was initiated. The view-change is thereby completed and the application is once again set to active mode. We also perform garbage collection for the view-change logger once the view-change is deemed successful as seen in \autoref{code:timeout}.
\fi

\begin{figure}[H]
	\centering
	\begin{lstlisting}[label = code:redoprotocol, caption=Redo Protocol Functionality, captionpos = b, basicstyle=\scriptsize]
foreach (var prepre in oldpreList)
{
     var precert = new ProtocolCertificate(
	    prepre.SeqNr, 
	    prepre.ViewNr, 
	    prepre.Digest, 
	    CertType.Prepared, prepre
     );
     var comcert = new ProtocolCertificate(
        prepre.SeqNr, 
        prepre.ViewNr, 
        prepre.Digest, 
        CertType.Committed
     );
     Serv.InitializeLog(prepre.SeqNr);
     var preps = ReMesBridge
     	         .Where(pm => pm.PhaseType == PMessageType.Prepare)
                 .Where(pm => pm.SeqNr == prepre.SeqNr)
                 .Where(pm => pm.ValidateRedo(Serv.ServPubKeyRegister[pm.ServID], prepre.ViewNr))
                 .Scan(precert.ProofList, (prooflist, message) =>
                 {
                    prooflist.Add(message);
                    return prooflist;
                 })
                 .Where(_ => precert.ValidateCertificate(FailureNr))
                 .Next();
     var coms = ReMesBridge
                 .Where(pm => pm.PhaseType == PMessageType.Commit)
                 .Where(pm => pm.SeqNr == comcert.SeqNr)
                 .Where(pm => pm.ValidateRedo(Serv.ServPubKeyRegister[pm.ServID], prepre.ViewNr))
                 .Scan(comcert.ProofList, (prooflist, message) =>
                 {
                    prooflist.Add(message);
                    return prooflist;
                 })
                 .Where(_ => comcert.ValidateCertificate(FailureNr))
                 .Next();
     await Sleep.Until(500);              
     if (!Serv.IsPrimary())
     {
        var prepare = new PhaseMessage(
            Serv.ServID, 
            prepre.SeqNr, 
            prepre.ViewNr, 
            prepre.Digest, 
            PMessageType.Prepare
        );
        Serv.SignMessage(prepare, MessageType.PhaseMessage);
        Serv.Multicast(prepare.SerializeToBuffer(), MessageType.PhaseMessage);
        Serv.EmitRedistPhaseMessageLocally(prepare);
     }
     await preps;
     await Sleep.Until(750);
     Console.WriteLine("Prepare certificate: " + precert.SeqNr + " is finished");
     Serv.AddProtocolCertificate(prepre.SeqNr, precert);

     var commes = new PhaseMessage(
        Serv.ServID, 
        prepre.SeqNr, 
        prepre.ViewNr, 
        prepre.Digest, 
        PMessageType.Commit
     );
     Serv.SignMessage(commes, MessageType.PhaseMessage);
     Serv.Multicast(commes.SerializeToBuffer(), MessageType.PhaseMessage);
     Serv.EmitRedistPhaseMessageLocally(commes);
     await coms;
     
     Console.WriteLine("Commit certificate: " + comcert.SeqNr + " is finished");
     Serv.AddProtocolCertificate(prepre.SeqNr, comcert);
}
	\end{lstlisting}
\end{figure}

\subsubsection{View-Change Evaluation}

\iffalse
\subsection{View-change Implementation}
\subsection{Starting a View-Change}
%insert how to start a view-change. Including timeout, view-change listener, started by timeout/protocol messages.
\subsection{View-Change functionality}
\subsubsection{Initialize View-Change}
\subsubsection{View-Change Listener Workflow}
\subsubsection{New-View Workflow}

As previously mention in \autoref{sec:view-change} the goal of a view-change is to successfully replace a faulty primary replica with another non-faulty replica. In order for a primary change to be successful, the replicas in the \ac{pbft} network needs to agree upon the state the program continues on after the primary change has occurred. Furthermore the view-change must ensure that the new replica selected for primary responsibility is not faulty. 

The operations to ensure these criteria were briefly mentioned in \autoref{sec:view-change}. Although in total there are a quite lot of operations needed for a successful view-change to take place. However, it is possible to divide up the operations into two segments based on which goal the operations attempt to fulfill. Excluding the processes of shutting down the protocol execution, the first part of the view-change process is for the replicas in the network to agree on that a view-change is necessary. This goal is achieved by having the replicas multicast and listen for view-change messages. Since the next primary is determined by the formula $p = v ~mod~ R$, \ac{pbft} doesn't require any election process. The view-change messages instead contains information of the replicas current checkpoint information as well as current state of the logged certificates. This is so that the new primary can have all the relevant information to create the new state for the \ac{pbft} system. The goal of the second segment is to initialize the \ac{pbft} system state after the view-change is finished. This goal is fulfilled by looking at the current stable checkpoint and the current protocol certificates stored in memory. In order to make sure that requests were not fully processed before the view-change occured. The \ac{pbft} needs to redo each of the requests stored in the logger up to the highest sequence number seen in the \ac{pbft} system. Thankfully, due to stable checkpoints, the process does not need to take into account every single request ever processed. The new primary is responsible for starting this process by creating and multicast a new-view message. This message acts as an introduction letter, telling the other replicas in the network that it is the new primary and additionally provide a view-change certificate proving this fact. The new-view message also contain a list of pre-preprepares which are created from the information stored within the protocol certificates in the view-messages received. This new-view message is validated by the other replicas. If the replicas deem the information in new-view message as valid, the replicas will use these pre-prepares message to create prepare and commit messages and redo the \ac{pbft} algorithm for these pre-prepares. The system is finally finished with the view-change once all the pre-prepares have stored their respective two protocol certificates. 

As for implementing this functionality, our implementation can be divided into four segments. The first consist of the timeout functionality that when triggers puts the application into non-active mode. The second part consist of updating the view data, creating view-change messages, multicast these view-changes over the \ac{pbft} network and finally store the collection of view-changes until quorum has been reached. The third consist of creating and validating functionality for new-view message. Finally the last segment consist of the redo protocol functionality. 

In non-active mode all protocol related messages and requests are denied by the main protocol execution. This mode is active until all four segments of the view-change functionality as been completed successfully. The view-change functionality differs from the other functionality due to the handling of timeout. It has already been mention that the view-change functionality start once a replica exceeds its timeout before receiving a pre-preprepare message. However, there are two additional timeout present in the view-change functionality. These timeout exists in order for the system to be absolute sure that the new primary chosen by the $p = v ~mod~ R$ formula does not result in a faulty replica. If the formula does result in a faulty formula, then either view-change process or redo protocol process will most likely fail. Setting a timeout for these two functionalities, the protocol can recover from a potential frozen state and restart the view-change process by now selecting the next replica on the list. Essentially, the view number is incremented every time the view-change protocol changes, meaning a new primary is selected until a non-faulty primary is chosen. Unfortunately the current implementation only handles timeout at the start of the normal protocol workflow, which also gets stopped once the replica receives a pre-prepare message. This means the protocol gets effectively stuck in the case where the protocol fails at handling prepare and commit messages. There were to main reason for why this issue was not resolved in our implementation. The first reason was that timeout functionality relies on the \code{WhenAny} asynchronous function~\cite{WEB:whenany}. This function creates a \code{Task} that is set to finish once either of the attached \code{Task}'s completes. In our implementation this effectively is set to either the timeout is exceeded or the process that is waited for completes. This was unfortunately not very well integrated with reactive listeners, as it is forced to finish all the operators before it is deemed completed. It required the \code{MERGE} operator in order to enforce reactive listeners to stop and dispose of the active reactive stream when a timeout occured. The \code{MERGE} operator required that both the reactive streams that were to merge had the same format. This means that the stop signal to the \code{MERGE} operator needed to also be a phase message for the pre-prepare reactive stream. The current workflow for handling the timeout functionality can be seen in \autoref{code:timeout}. The time is first initialized with a cancellationtoken which is brought into the main protocol workflow so that the protocol can cancel the timeout when it receives a pre-prepare message. The timeout used in the current implementation is set to ten seconds. The timeout gets a reference to an active \code{Source} object which is the same \code{Source} which is listened to at the function \code{ListenForShutdown}. When the timeout exceeds, the timeout function will emit a message to the shutdown \code{Source} which in turn makes the \code{CTask} in \code{ListenForShutdown} to return before the AppOperation, which lets the program flow to continue. The AppOperation is still active as an asynchronous function, meaning we want to forcefully shut it down so as to avoid creating conflicts with the future emits to the protocol \code{Source} object. To solve this we emit an obviously faulty phase message with an unique phase message type called \emph{End}. Thanks to the \code{MERGE} operator, the pre-prepare reactive listener will finish and returns the faulty pre-prepare message. As seen in \autoref{code:Pre-PrepareNonPrimary}, the protocol calls a timeout exception if the pre-prepare reactive listeners returns the faulty phase message, meaning the protocol effectively shuts down as intended.   

\begin{figure}[H]
	\centering
	%\lstset{style=sharpc}
	\begin{lstlisting}[label = code:timeout, caption=Handling timeout for the normal protocol workflow and initiate View-Change, captionpos = b, basicstyle=\scriptsize]
CancellationTokenSource cancel = new CancellationTokenSource();
_ = TimeoutOps.AbortableProtocolTimeoutOperation( //starts timeout
   serv.Subjects.ShutdownSubject,
   10000,
   cancel.Token,
   scheduler
);
execute.Serv.ChangeClientStatus(req.ClientID);
bool res = await WhenAny<bool>.Of(
                AppOperation(req, execute, seq, cancel),
                ListenForShutdown(serv.Subjects.ShutdownSubject)
);
Console.WriteLine("Result: " + res);
if (res)
{
   Console.WriteLine($"APP OPERATION {seq} FINISHED");
   ...
}
else
{
   if (execute.Active)
   {
      Console.WriteLine("View-Change starting");
      execute.Active = false;
      serv.ProtocolActive = false;
      await scheduler.Schedule(() =>
         shutdownPhaseSource.Emit(new PhaseMessage(-1, -1, -1, null, PMessageType.End)
      ));
      await execute.HandlePrimaryChange2(); 
      Console.WriteLine("View-Change completed");
      serv.UpdateSeqNr();
      ...
    \end{lstlisting}
\end{figure} 
 
The view-change exchange segment of the code starts by first setting the replica into the next view by incrementing its view number. The next operation sets the replicas view-change certificate. This step is dependent on whether or not the replica has received previously received view-change messages from another replica. If the replica has not received any view-messages from other replicas than it initializes the view-change certificate and initializes the view-change reactive listener. Info about the rest of the view-change process in the main workflow...

The view-change listener deviates a bit from the other reactive listeners. The main difference is that it also requires the ability to call upon a shutdown emit in the case where the system already has gotten $2f$ view-change messages. The reason for this functionality is mostly due to making the system more efficient. The replica does not need to wait for a timeout to occur if it already has received $2f$ view-change messages since the \ac{pbft} network only requires that replica's view-change in order for the new view to be initialized. Therefore the process is speed up by calling for a shutdown emit if already has $2f$. Ofcourse this functionality is only useful if the replica is still in active mode. This is the reason as to why the option to not trigger the shutdown emit is an option. Other than that the reactive listener performs relatively the same operators for the reactive stream. Firstly, we want to only accept view-change message that belongs to the same next view nr as the replica. Secondly the view-change messages received are validated to make sure that it is a valid view-change message. Assuming the validation process is successful, the view-change message is added to the view-change certificate proof list. The final reactive operator validates that the view-change certificate to see if it has received the sufficient number of valid view-change messages in its proof list. After the view-change reactive listener is finished and a valid view-change certificate is ready, the callback function \emph{finCallback} calls the servers to emit a signal to the view-change workflow that the view-change certificate is finished and can move on with the next step of the view-change process. 

\begin{figure}[H]
	\centering
	%\lstset{style=sharpc}
	\begin{lstlisting}[label = code:viewListener, caption=Source code for View-Change Listener, captionpos = b, basicstyle=\scriptsize]
if (Shutdown && shutdownCallback != null)
{
   Console.WriteLine("With shutdown");
   await ViewBridge
      .Where(vc => vc.NextViewNr == NewViewNr)
      .Where(vc => vc.Validate(keys[vc.ServID], ServerViewInfo.ViewNr))
      .Scan(vcc.ProofList, (prooflist, message) =>
      {
        prooflist.Add(message);
        return prooflist;
      })
      .Where(_ => vcc.ShutdownReached(FailureNr))
      .Next();
   Console.WriteLine("Calling shutdown");
   shutdownCallback();
}
await ViewBridge
   .Where(vc => vc.NextViewNr == NewViewNr)
   .Where(vc => vc.Validate(keys[vc.ServID], ServerViewInfo.ViewNr))
   .Scan(vcc.ProofList, (prooflist, message) =>
   {
     prooflist.Add(message);
     return prooflist;
   })
   .Where(_ => vcc.ValidateCertificate(FailureNr))
   .Next();
Console.WriteLine("Finished Listen view changes");
finCallback();
    \end{lstlisting}
\end{figure} 
segment from motivation, rewrite and get it in the main text somehow!
This could in theory also apply to the view-change description as it is divided into several detailed steps. However, there are several factors which lead to the view-change functionality being split into three separate, but nested, functions. The first reason being that view-changes require the ability to restart the processes in the case where the request processing remains stationary for too long. To handle this functionality we currently use a mix of timeout operations and goto statements in order to reroute the program flow back to the beginning of the view-change process~\cite{WEB:goto}. The second reason, which also applies to checkpoints, is that the view-change process can be initialized early by receiving view-change messages from other replicas. This may seem similar to protocol operations since it is initialized by client requests, however the difference lies in the amount of messages required to initialize the process. Currently, the server needs to support the functionality of starting reactive listeners for view-changes if it ever receives a view-change, however the view-change process itself doesn't start until either the timeout occurs or the replica has received $2f$ messages. Because of this functionality, keeping the code completely synchronous and centered around a single function is not possible. As for checkpoints, because the checkpoint processes can be initialized whenever, and the majority of the time spent for checkpoints is used waiting for a replica to receive $2f+1$ unique checkpoints with identical sequence numbers, the source code cannot be centered around a single function.
\fi


\iffalse 
The checkpointing process follows the \emph{checkpoint interval}. This means it only gets used once the system has processed a certain number of requests equal to the checkpoint interval. In our implementation the current checkpoint interval is set to five, meaning after processing five requests a new checkpoint is created. In our implementation we divided the workflow of the checkpointing into two parts.
The first part is the creation part, which is essentially initializing the checkpoint certificate to the last sequence number using the current application state as digest. In our implementation, we create the system digest from a persistent list which represent the current state of the system. The list contains the operation messages from each of the requests that has been fully processed by the \ac{pbft} protocol. So assuming no errors occurs, than the checkpoint for sequence number five will be the digest of a list containing the operation from requests one to five. After creating the checkpoint certificate and the checkpoint message, a checkpoint reactive listener is initialized. This reactive listener works similar to how reactive listeners worked in the protocol workflow. The server once it receives a checkpoint message from network will emit the checkpoint message to the \code{Source<Checkpoint>} registered in the server. The checkpoint listener will listen for any item emitted by the server and the given checkpoint message will be first validated before transforming the proof list of the checkpoint  certificate to be a proof list which has the checkpoint message. Unlike the protocol workflow, checkpoints can theoretically not be completed during execution and runs separate to the normal protocol workflow. This means if the protocol processes enough requests, a checkpoint checkpoint will be created with higher sequence number the previous one. This means it can be possible to have multiple checkpoint listeners active at the time. However, it becomes a race for the checkpoint certificates to see which one becomes next stable one. After the checkpoint listener has been created, than the replica will also emit its own local checkpoint to the checkpoint listener, meaning it will have to pass all of the same checks as the networked checkpoint messages has to. Since we're never sure which of the replica in the network is the fastest when it comes to setting up the checkpoint certificate, it means the server is also prepared to initialize the checkpoint processes if it receives a checkpoint with higher sequence number than the current stable checkpoint. Unlike the protocol certificates, the checkpoint certificates are added to the checkpoint logger once its been created, no validation is required. However, in order for a checkpoint to be deemed stable it will need to pass the certificate validation processes which follows the same guidelines as the protocol certificate. A replica can only have one stable checkpoint. The goal of the checkpoint process is to attempt to try replace this stable checkpoint, so we can garbage collect the protocol data from the logger. The garbage collection also includes active checkpoints in the checkpoint logger with lower or equal sequence number to the stable checkpoint certificate.
The second part of the checkpoint functionality is rather simplistic. Originally when the server side of a replica was initialized it also initialized another reactive listener, which is set to await for new stable checkpoint certificate. Once it receives a stable checkpoint certificate in the reactive \code{Source} object, it will overwrite the stable checkpoint registered on the system and then perform the garbage collection process. The \code{Source} is linked to the server and it schedules the new checkpoint certificate similar to schedules any message emit to the persistent layer. This is important because each checkpoint listener will have a reference to the callback function which schedules the emit to the \code{Source} object. This means that once the checkpoint certificate passes all of the reactive operators and the checkpoint is deemed valid, the callback function will be called with the resulting checkpoint certificate, which in turn will overwrite the stable checkpoint. Checkpoint process is then deemed successful and the garbage collection processes is started. The source code for the an instance of a checkpoint listener can be seen in \autoref{code:CreateCheckpoint}. The source code for listening for stable checkpoint certificate can be seen in \autoref{code:ListenForCheckpoint}
\fi

\section{Client}
\iffalse
The client implementation created for the \ac{pbft} implementation is a primitive console application that is interactable by the user. The client uses interactivity to create unique operations that are to be handled by the \ac{pbft} algorithm. In our current \ac{pbft} implementation, we treat operations as simple string objects, meaning mostly any assigned string value can be used as an operation value. However, an exception to this rule is that the operation cannot contain a pipeline symbol. This is because the pipeline symbol is used as an end delimiter for serialized messages in order to resolve a \ac{tcp} issue that can occur, which links two messages together. An operation is created by prompting the user for a value representing the value of the operation in the request message. 
Just like the replicas in the system, the client takes the network addresses stored in a \ac{json} file and then establishes a socket connection to each of the network addresses. Unfortunately, this means the client can not be initialized before the replicas, since it expects all replicas to be up and running.

In principle, the workflow for the client implementation is straightforward. The client starts by first initializing its connection to each of the replicas in the system based on the addresses found in the chosen \ac{json} file. Then the user is prompted for the value to be used in the operation. Once the operation is deemed valid, the client creates a new request message using the operation provided by the user. The request is signed by the client’s private key and then multicast to the replicas in the \ac{pbft} network. After the request is sent, the client waits for replicas to reply to the request the same way the normal protocol workflow does for a phase shift. A reply certificate is created, and the client uses a \code{Source<Reply>} to listen for reply messages reactively. When the \code{Source<Reply>} receives a new valid reply message, it is added to the reply certificate until the certificate has received at least $f+1$ valid replies from different replicas. The $f+1$ criteria is referred to as a weak certificate, which is a certificate that can guarantee that at least \emph{f} non-faulty replica stored the request in its protocol log~\cites[p.~9]{PAPER:PBFTRecovery}[p.~2]{PAPER:DPBFT}. Because the client is not part of the \ac{pbft} system, it only requires \emph{f} number of replies to guarantee that the \ac{pbft} system properly processed the original request~\cites[p.~3]{PAPER:OGPBFT}[p.~9]{PAPER:PBFTRecovery}.

If the reply certificate receives $f+1$ replies from different replicas, the certificate is stored in the client’s log. The client application restarts its workflow by again prompting the user for the next operation for the subsequent request. However, if the reply certificate does not become valid within a specific time duration, a timeout will occur, and the request is once again multicasted to the \ac{pbft} network. This process is repeated until the $f+1$ criteria is met. Unfortunately, if the \ac{pbft} application gets stuck on one of the client operations, the server does not accept the resent request as it believes it is already working on another request from the same client.  Unfortunately, this usually leads to an endless loop. A way to get out of this loop would be for a view-change to occur on the \ac{pbft} since the client status information on the replica is reset after the view-change is finished. The resent request is now treated as a new one, and the entire request processing starts anew.

The client shares a lot of the network-related code with the \ac{pbft} replicas. The main difference between the two lies in the client always being responsible for initiating the socket connection. The client also tries to reconnect to replicas it has previously been connected to but now is lost. The reconnection attempt is made whenever the client is about to multicast a request to the \ac{pbft} network. In the case where the reconnection fails, the client moves onto the other replicas. The client does, however, retries to reconnect to the lost replica whenever a new
request is sent to the \ac{pbft} network. 

We decided not to include persistency for the client implementation. Despite this, the network portion of the client still uses the Cleipnir execution engine when it sends incoming replies from the network layer to its reactive listener. The reason for this is because scheduling the emit using the Cleipnir execution engine enforces synchrony. Enforcing synchrony helps the client avoid a potential race condition that can potentially occur in this section of the code. We are currently not sure what is causing this issue. We are running the reactive listener completely outside of Cleipnir’s influence, which means additional threads are not supposed to be created. Despite this, we still have encountered race conditions in this section without using the Cleipnir execution engine.
\fi

The client implementation created for the \ac{pbft} implementation is a primitive console application that is interactable by the user. The client uses interactivity to create unique operations that are to be handled by the \ac{pbft} algorithm. In our current \ac{pbft} implementation, we treat operations as simple string objects, meaning mostly any assigned string value can be used as an operation value. However, an exception to this rule is that the operation cannot contain a pipeline symbol. This is because the pipeline symbol is used as an end delimiter for serialized messages to resolve a \ac{tcp} issue that can occur, which links two messages together. An operation is created by prompting the user for a value representing the value of the operation in the request message. 
Just like the replicas in the system, the client takes the network addresses stored in a \ac{json} file and then establishes a socket connection to each of the network addresses. Unfortunately, this means the client can not be initialized before the replicas since it expects all replicas to be up and running when it attempts to establish socket connections.

In principle, the workflow for the client implementation is straightforward. The client starts by first initializing its connection to each of the replicas in the system based on the addresses found in the chosen \ac{json} file. Then the user is prompted for the value to be used in the operation. Once the operation is deemed valid, the client creates a new request message using the operation provided by the user. The request is signed by the client’s private key and then multicast to the replicas in the \ac{pbft} network. After the client sends the request, it waits for the replicas to reply to the request the same way the normal protocol workflow does for a phase shift. A reply certificate is created, and the client uses a \code{Source<Reply>} to listen for reply messages reactively. When the \code{Source<Reply>} receives a new valid reply message, it is added to the reply certificate until the certificate has received at least $f+1$ valid replies from different replicas. The $f+1$ criteria is referred to as a weak certificate, which is a certificate that can guarantee that at least \emph{f} non-faulty replica stored the request in its protocol log~\cites[p.~9]{PAPER:PBFTRecovery}[p.~2]{PAPER:DPBFT}. Because the client is not part of the \ac{pbft} system, it only requires \emph{f} number of replies to guarantee that the \ac{pbft} system properly processed the original request~\cites[p.~3]{PAPER:OGPBFT}[p.~9]{PAPER:PBFTRecovery}.

If the reply certificate receives $f+1$ replies from different replicas, the certificate is stored in the client’s log. The client application restarts its workflow by again prompting the user for the next operation for the subsequent request. However, if the reply certificate does not become valid within a specific time duration, a timeout will occur, and the request is once again multicasted to the \ac{pbft} network. This process is repeated until the $f+1$ criteria is met. Unfortunately, if the \ac{pbft} application gets stuck on one of the client operations, the server does not accept the resent request as it believes it is already working on another request from the same client.  Unfortunately, this usually leads to an endless loop. A way to get out of this loop would be for a view-change to occur on the \ac{pbft}. This is because the client status information on the replica is reset after the view-change is finished. The resent request is then treated as a new one, and the entire request processing starts anew.

The client shares a lot of the network-related code with the \ac{pbft} replicas. The main difference between the two lies in the client always being responsible for initiating the socket connection. The client also tries to reconnect to replicas it has previously been connected to but now is lost. The reconnection attempt is made whenever the client is about to multicast a request to the \ac{pbft} network. In the case where the reconnection fails, the client moves onto the other replicas. The client does, however, retry to reconnect to the lost replica whenever a new
request is sent to the \ac{pbft} network. 

We decided not to include persistency for the client implementation. Despite this, the network portion of the client still uses the Cleipnir execution engine when it emits incoming replies from the network layer to its reactive listener. The reason for this is because scheduling the emit using the Cleipnir execution engine enforces synchrony. Enforcing synchrony helps the client avoid a potential race condition that can potentially occur in this section of the code. We are currently uncertain in regards to what is causing this issue. We are running the reactive listener completely outside of Cleipnir’s influence, which means additional threads are not supposed to be created. Despite this, we still have encountered race conditions in this section without using the Cleipnir execution engine.
	
	
	\chapter{Discussion}
\label{chapter:Dis}
Summaries your arguments for why these tools helped/hindered the implementation of the PBFT algorithm. And perhaps give your thoughts on the matter.
	
	
	\chapter{Conclusion}
\label{chapter:Con}
\iffalse
This chapter concludes the thesis by first listing the lessons we learned while working on this thesis. Then we list the potential future work which can be applied to the \ac{pbft} implementation.
Finally a conclusion is drawn for the work performed for this thesis.

\iffalse
\section{Lessons Learned}
\iffalse
-PBFT
-Asynchronous programming with C#, Task architecture
-Reactive Programming basics
-Overall knowledge for Cleipnir
-Issues and advantages in regards to the topics listed over. For instance a lot of time was wasted due to not fully grasping how Cleipnir work internally when performing the reactive part and the CAwaitable emission --> resulting a month of frustration trying to figure out why collision errors occur.
-Lack of documentation can be quite fatal for continued support.
-The multitude of potential issues that could occur that aren't necessary dealt with in the theoretical consensus algorithm or pseudo code.
-Cleipnir and how it interacts with the other programming paradigms. Eks: A clear distinction has to made in regards to what code is run inside Cleipnir(the persistent part) and what is not called in Cleipnir (orthogonal part), mixing these will cause disastrous results, which we infact encountered several times during implementation.
%-Unit testing, simplicity of C# unit testing, issues in regards to unit testing networking as running tests in parallel causes inconsistent results and at worst case inf-loops(don't think this is really all that useful)
\fi
%first draft, probably be heavly changed after writing the other parts of the thesis
%REWRITE THIS SECTION!!!! MINDRE historie lesing, bryr oss ikke om det du har gjort noe særlig med mindre det er godt. Hvis du har noe viktig/revosulerende så beskriv det, men ellers ikke nødvendig.
\subsection{Consensus algorithm}
At the start of this thesis our knowledge in regards to consensus algorithms were limited to having previously implemented the Paxos algorithm using Golang language~\cite{WEB:golangmainpage}. We had never encountered any information in regards to the \ac{pbft} consensus algorithm, therefore some time needed to be spent on learning the inner workings of the \ac{pbft} algorithm. In addition, Cleipnir had already been used to implement the Paxos and Raft consensus algorithms. Therefore some time was also spent on understanding the basics of the Raft consensus algorithm to fully understand the source code used for the Raft implementation. The transition from one consensus algorithm when looking solely on the protocol descriptions is not all that complicated. This is mostly due to similarities found in their functionality. Components used to implement a functional consensus algorithm are shared by many consensus algorithms. This in turn makes it easier for someone familiar with one algorithm to understand another. An example of this being that all three previously mentioned algorithms use an election model in order to make a decision over the network. Furthermore one party in the election is given the leader role and is therefore responsible for governing the election process. Hence understanding the basic principles behind the \ac{pbft} algorithm through the project description was not challenging.

However, consensus algorithms are notoriously difficult to implement. This is because the protocol descriptions are by design written to be as simple as possible, otherwise developers would have issues fully understanding the basics on how they operate. This can unfortunately lead to some information being omitted, which can cause issues when designing an implementation for the algorithm. This was especially apparent for our implementation, since our goal was to make the protocol workflow as simplistic as possible using the tools at hand. Several times during development new issues became apparent in our design when certain scenarios or circumstances occurred during the protocol workflow. This was especially noticeable when thinking of all the different issues that could potentially occur when a restarted replica with an out of date persisted state attempted to collaborate with the other replicas.

In most of these cases we had to decide whether or not it was worth it to introduce additional complexity to the implementations in order to handle these issues, or to simply try to avoid them altogether. In most realistic scenarios the obvious choice would be to fix the issue, even if it adds more complexity to your system. Unfortunately, since our goal was to attempt to implement a very simple implementation, in addition to upholding a time constraint, we had to prioritize differently. Which in turn made our implementation less desirable compared to other more complex implementations. In short, our experience working on implementing the \ac{pbft} algorithm led us to believe that the largest difficulty in regards to the implementation of a consensus algorithm does not necessarily lie in lack of understanding the technicalities within the consensus algorithm. Instead we believe it lies in having to make an implementation that follows the simple protocol description, while still needing to make sure that the state of the system is not affected by any potential issue that can occur in any of the units in the systems. 

\subsection{Asynchronous programming}
Going into this thesis our experience using asynchronous programming were limited and were solely based on a few previous code projects. In addition the asynchronous programming used in these projects used the JavaScript asynchronous framework. Although the language barrier between the asynchronous tools were minimal, there were a few subtle differences. An example of this being how C\# \code{Task} objects function very similarly to \code{Promise} in JavaScript. Overall since both the asynchronous frameworks support the use of the async/await operators, programming asynchronous workflows were relatively similar.

On the other hand there were issues encountered in our application due to lack of understanding behind the details for the async and await operators early in development. Originally our application used asynchronous programming for a lot of tasks related to both networking and protocol handling which caused a lot of internal nested state machines to be created due to using async/await inside other async/await operators. Not only was it a pain to attempt to debug issues regarding nested state machines, but it further escalated when nested async/await operators were used inside \code{CTask}'s for normal \code{Task} functions, which created additional threads to appear as well. The result being a lot of race conditions, inconsistent states and just generally a nightmare to debug. The simple solution was to make any unnecessary asynchronous task become synchronous operations, which in turn removed a lot of the nested state machines. The second change being to separate \code{Task} functions and \code{CTask} functions as much as possible, which further helped since it removed the race conditions.

In short, due to our over usage of async/await workflow for tasks that didn't necessarily need to be asynchronous lead to a lot of issues for our application. Therefore, it is important when designing an asynchronous application to have a clear view over which computing tasks require asynchronous workflow and which can be satisfied by synchronous workflow. Using asynchronous programming for tasks where it is not needed only causes extra complexity to the code and as a result is not only harder to debug but also unnecessarily slows down the system.

\subsection{Reactive programming}
At the beginning of this project, we had very little to no previous experience in regards to reactive programming. Therefore it became quite the challenge learning the basics for reactive programming. Specifically the main challenge became using the basics for reactive programming in order to understand Cleipnir's reactive functionality. Majority of the documentation and tutorials around the web in regards to reactive programming focused mostly on the basics and the cornerstones used for implementing their own reactive operators. This did not quite translate well for our project as all of the reactive layers were already implemented in Cleipnir. Cleipnir reactive functionality in itself is very easy to use and is not all that hard to learn. However, making a direct comparison to the official reactive documentation~\cite{WEB:ReactiveXMainPage} and Cleipnir.Rx was not so simple. This mostly due to the cornerstones having different name schemes between the two.  (add more stuff here later)

\iffalse
In terms of our experience using the Cleipnir reactive layer it is exceptionally easy to use once the basics is learned. Although Cleipnir currently lacks support for the majority of the reactive operators listed in the documentation. The current support does however cover most used reactive operators. During development only a single instance did we encounter an issue in which we required the use of a reactive operator that the reactive layer did not support. Thankfully Thomas added that missing reactive operators within a few hours, essentially proving that Cleipnir's current design allows for developers to easily add missing reactive operators should the need ever arise. As for the usage of the reactive paradigm in the protocol workflow. The code operations performed over the reactive streams works well and are easy to keep track of due to how simple it is to chain reactive operators. On the other hand, chaining reactive operations can be somewhat restricting in some circumstances. The most troublesome issue encountered in regards to working with reactive operators was to handle exceptions to the protocol workflow. In our case stopping the reactive operators when a view-change occurred was quite troublesome to implement. When the program is required to wait for a reactive \code{Source} to finish its operations, the \code{Source} must receive an item in a stream which manages to pass an perform each and every reactive operator that is chained to the \code{Source} object. This functionality can get very easily stuck when the source doesn't get the desired items to the stream. There are two notable workarounds to this problem. The first is to simply ignore the problem since its \code{Source} objects should only be listened to in \code{CTask} asynchronous functions, therefore it won't block the main execution thread even if it never finishes all of the reactive operators. Meaning the program simply creates new iterations for the workflow whenever the protocol starts and never stops any existing asynchronous operations that are stuck. This can be achieved if the reactive stream has strict \code{Where} clauses as it allows for the old listeners to not be affected by any new items received in the reactive stream. The \code{Where} clause filters out the items long before it can affect the program in any way. This means the workaround is essentially just letting the listener run stuck until it is eventually garbage collected. This method does slow the system down somewhat since the old \code{Source} objects are still actively listening, receiving and filtering out items emitted to the stream even if it can never proceed past the \code{Where} clause. The second workaround uses the \code{Merge} operator to have the listener listen to changes on two different streams. By this method it is possible to effectively terminate the listener if it receives an item from the second source, as this is counted as irregular activity. This is the method used in our \ac{pbft} implementation to handle exiting active instances of the \ac{pbft} workflow in order to change view for the system. This workaround also has its fair share of issues. In order to use the \code{Merge} operator it requires both the \code{Source} objects to listen for the same type of object. This is not always easy to coordinate, especially when the other operators for the listener transform the stream to work on another object type. In addition, the \code{Merge} operator also works like any other operator. If the \code{Merge} is triggered by the other source object and the operator is called early in the stream, then the item received is still required to pass the other operators in order to terminate the listener. Which puts it back to the state of the original problem. The item received by the other source must also be unique so that the rest of the workflow can terminate the process when it receives the item from that other \code{Source} object.
\fi

To summarize, the use of reactive handlers works well for segmenting operations to perform for the consensus algorithm when a new event is received in the network layer. In addition it is relatively easy for developers to use and is a lot easier to read the workflow in comparison to traditional programming. However, reactive handlers can be tricky to deal with when used in protocol workflow that needs to handle exceptions to the normal workflow. As consensus algorithms must handle situations where parties on the network stop responding, this can become a rather frequent issue. It therefore would be most beneficial if additional workarounds were discovered for handling this issue.  

\subsection{Cleipnir}
%-Overall knowledge for Cleipnir
%-Issues and advantages in regards to the topics listed over. For instance a lot of time was wasted due to not fully grasping how Cleipnir work
%internally when performing the reactive part and the CAwaitable emission --> resulting a month of frustration trying to figure out why collision errors occur.
%-Lack of documentation can be quite fatal for continued support.
%-Cleipnir and how it interacts with the other programming paradigms. Eks: A clear distinction has to made in regards to what code is run inside Cleipnir(the persistent part) and what is not(ephemeral)

As for our experience with using Cleipnir during this thesis it has been a mixed experience. The most challenging part of Cleipnir was learning about its functionality when we only had access to its source code and a few practical examples. Bakkevig previously also seemed to struggle in this department in his thesis~\cite[p.~43-44]{PAPER:EivindPaper}. This is somewhat our fault as we were not accustomed to learning about frameworks by reading its source code. During our study practically all tools and frameworks used had a form of written documentation. Although not all frameworks have well written documentations, frameworks usually have some form of community that uses which you can discuss unexpected issues when the need arises. As Cleipnir is still in development it obviously does not have a community. We also understand that since Cleinir is constantly being updated, writing a detailed documentation could be seen as wasteful. This is because the functionality changes frequently which means that the documentation would also need to be constantly updated, leading to a lot of extra work for each update. However, we do share Bakkevig's opinion that if  Cleipnir is to become well liked by developers, time must be invested into writing at least a small description for its unique components as well as a guideline for how each tool available in the framework should be used and what users should actively avoid. 

Cleipnir does have a lot of different practical examples that are very efficient in teaching users the basics of Cleipnir. It is possible to learn a lot through the practical examples, however without a written documentation it is very likely that the user needs to make assumptions on the tools used. In the case where the assumption is wrong it leads to the user experiencing both confusion and frustration when things are not working as intended. We were lucky enough that Stidsborg was available to answer any questions we had in regards to Cleipnir, but we still made some misconceptions which lead to problems for the design of our application. The biggest misconception being the usage of \code{CTask} which lead to our project being delayed for about a month due to confusion as to where the additional threads came that lead to many race conditions.   In short we learned that making assumptions can be quite detrimental and we should have perhaps queried Stidsborg earlier in the development about the difference between Cleipnir implementation of commonly used classes compared to their traditional use. 
\fi

\section{Lessons Learned}
\subsection{Consensus algorithms}
At the start of this thesis, our knowledge in regards to consensus algorithms was limited to having previously implemented the Paxos algorithm using Golang language~\cite{WEB:golangmainpage}. We had never encountered any information in regards to the \ac{pbft} consensus algorithm; therefore, some time was spent on learning the inner workings of the \ac{pbft} algorithm. In addition, Cleipnir had already been used to implement the Paxos and Raft consensus algorithms. Therefore some time was also spent on understanding the basics of the Raft consensus algorithm to help understand the source code used for the Raft implementation. We realize that transition from one consensus algorithm to another when looking solely at the protocol descriptions is not all that complicated. Many components for dealing with specific issues regarding consensus algorithms are shared for many consensus algorithms. As a result, it became simpler for someone familiar with one consensus algorithm to understand another.

\subsection{Asynchronous Programming}
Asynchronous programming proved to be suitable for the network layer of the application. In addition, asynchronous programming became an excellent boon for designing a multi-client application when used on the protocol workflows. Since the \code{CTask} also took advantage of the async/await workflow, implementing \code{CTask} functions was just as simple as the .NET traditional asynchronous workflow.

However, we acknowledge that our inexperience with the background operations occurring in the async/await workflow hurt our initial design for our application. This, combined with our wrong assumption regarding \code{CTask}, delayed our thesis considerably. 
\subsection{Reactive Programming}
At the beginning of this project, we had very little to no previous experience in regards to reactive programming. Therefore it was challenging to learn the basics of reactive programming. The most significant complication became understanding Cleipnir’s reactive functionality knowing only the basics of reactive programming. Using the Cleipnir reactive functionality is straightforward once you learn the basics. However, making a direct comparison to the official reactive documentation~\cite{WEB:ReactiveXMainPage} to Cleipnir.Rx was not simple. This primarily due to the cornerstones having different name schemes between the two.

Regardless, we have demonstrated in our application that we ascertained the knowledge to use the reactive framework to handle the \ac{pbft} protocol messages and their resulting operation. It was showed that the framework was appropriate for handling event-driven scenarios in consensus algorithms. In addition, simply enough to have our reactive workflow be reused for several parts of our reactive implementations. 
\subsection{Cleipnir}
Starting the thesis, we had little to no prior experience with working with the Cleipnir framework. The most challenging part of learning how to use Cleipnir functionality was the lack of a detailed documentation. We only had access to its source code and a couple of well written practical examples. Bakkevig previously also seemed to struggle in this department in his thesis~\cite[p.~43-44]{PAPER:EivindPaper}. This is somewhat our fault as we were not accustomed to learning about frameworks by reading their source code. During our study, practically all tools and frameworks used had a form of written documentation. However, not all frameworks have well-written documentation. Although, most commonly used frameworks usually have some form of community that uses which you can discuss unexpected issues when the need arises. As Cleipnir is still in development, it does not have a large community. We also understand that since Cleinir is constantly updated, writing a detailed documentation could be seen as wasteful because the functionality changes frequently. This means that the documentation would also need to be continuously updated, leading to a lot of extra work for each update. However, we do share Bakkevig’s opinion that if Cleipnir is to become well liked by developers, time must be invested into writing at least a small description for its unique components as well as a guideline for how each tool available in the framework should be used and what users should actively avoid. Stidsborg was available to answer any questions we had regarding Cleipnir.  Despite this, flawed assumptions were made, leading to problems for the development of our implementation. Safe to say, we learned that making assumptions can be quite detrimental, and we should have perhaps queried Stidsborg earlier in the development about the difference between Cleipnir implementation of commonly used classes compared to their traditional counterparts.


\iffalse
%REWRITE THIS SECTION!!!! MINDRE historie lesing, bryr oss ikke om det du har gjort noe særlig med mindre det er godt. Hvis du har noe viktig/revosulerende så beskriv det, men ellers ikke nødvendig.
\subsection{Consensus algorithm}

At the start of this thesis, our knowledge in regards to consensus algorithms was limited to having previously implemented the Paxos algorithm using Golang language~\cite{WEB:golangmainpage}. We had never encountered any information in regards to the \ac{pbft} consensus algorithm; therefore, some time was spent on learning the inner workings of the \ac{pbft} algorithm. In addition, Cleipnir had already been used to implement the Paxos and Raft consensus algorithms. Therefore some time was also spent on understanding the basics of the Raft consensus algorithm to help understand the source code used for the Raft implementation. The transition from one consensus algorithm when looking solely at the protocol descriptions is not all that complicated. This is primarily due to similarities found in their functionality. Components used to implement a functional consensus algorithm are shared by many consensus algorithms. Which consequently made it easier for someone familiar with one algorithm to understand another.
An example of this is that two previously mentioned algorithms use an election model to make a decision over the network. Furthermore, one party in the election is given the leader role and is therefore responsible for governing the election process. Hence understanding the basic principles behind the \ac{pbft} algorithm through the project description was not challenging.

However, consensus algorithms are notoriously difficult to implement. This is because the protocol descriptions are by design written to be as simple as possible. Otherwise, developers would have issues fully understanding the basics of how they operate. This can, unfortunately, lead to some information being omitted, which can cause problems when designing an implementation for the algorithm. This was especially apparent for our implementation since our goal was to make the protocol workflow as simplistic as possible using the tools at hand. During development, new issues became apparent in our design when certain scenarios or circumstances occurred during the protocol workflow. This was especially noticeable when considering all the different issues that could occur when a restarted replica with an out-of-date persisted state attempted to collaborate with the other replicas.

In most cases, we had to decide whether or not it was worth introducing additional complexity to the implementations to handle these issues or try to avoid them altogether. The obvious choice would be to fix the issue in the most realistic scenarios, even if it adds more complexity to your system. Unfortunately, since our goal was to attempt to implement a very simple implementation, in addition to upholding a time constraint, we had to prioritize differently. Which, in turn, made our implementation less desirable compared to other more complex implementations. In short, our experience working on implementing the \ac{pbft} algorithm led us to believe that the largest difficulty in regards to the implementation of a consensus algorithm does not necessarily lie in a lack of understanding of the technicalities within the consensus algorithm. Instead, we believe it lies in having to make an implementation that follows the simple protocol description while still needing to make sure that the state of the system is not affected by any potential issue that can occur in any of the units in the distributed system. 

\subsection{Asynchronous programming}
Going into this thesis, our experience using asynchronous programming was limited and was solely based on a few previous code projects. In addition, the asynchronous programming used in these projects used the JavaScript asynchronous framework. Although the language barrier between the asynchronous tools was minimal, there were a few subtle differences. An example of this being how C\# \code{Task} objects function very similarly to \code{Promise} in JavaScript. Overall since both the asynchronous frameworks support the use of the async/await operators, asynchronous programming workflows were relatively similar.

On the other hand, our application had issues due to a lack of understanding behind the details for the async and await operators early in development. Originally, our application used asynchronous programming for many tasks related to networking and protocol handling, which caused many internal nested state machines to be created due to using async/await inside other async/await operators. Not only was it a pain to attempt to debug issues regarding nested state machines, but it further escalated when nested async/await operators were used inside \code{CTask} ’s for normal \code{Task} functions, which created additional threads to be added into the mix. The result was many race conditions, inconsistent states, and just generally a nightmare to debug. The simple solution was to make any unnecessary asynchronous task become synchronous operations, which removed a lot of the nested state machines. The second change was to separate \code{Task} functions and \code{CTask} functions as much as possible, which further helped since it removed the race conditions.

In short, due to our over usage of async/await workflow for tasks that didn’t necessarily need to be asynchronous led to a lot of issues for our application. Therefore, it is important when designing an asynchronous application to have a clear view of which computing tasks require asynchronous workflow and which can be satisfied by synchronous workflow. Using asynchronous programming for tasks where it is not needed only caused extra complexity to the code. As a result, it became harder to debug and unnecessarily slowed down our system.

\subsection{Reactive programming}
At the beginning of this project, we had very little to no previous experience in regards to reactive programming. Therefore it became quite a challenge to learn the basics of reactive programming. Specifically, the main challenge was using the basics for reactive programming to understand Cleipnir’s reactive functionality. The majority of the documentation and tutorials around the web regarding reactive programming focused mainly on the basics and the cornerstones used for implementing their own reactive operators. This did not quite translate well for our project as all of the reactive layers were already implemented in Cleipnir. Cleipnir reactive functionality in itself is very easy to use and is not all that hard to learn. However, making a direct comparison to the official reactive documentation~\cite{WEB:ReactiveXMainPage} and Cleipnir.Rx was not so simple. This mostly due to the cornerstones having different name schemes between the two.  (add more stuff here later)

Our experience using the Cleipnir reactive layer is that it is exceptionally easy to use once the basics are learned. Although Cleipnir currently lacks support for the majority of the reactive operators listed in the documentation. The current support does, however, cover the most used reactive operators. During development, only a single instance did we encounter an issue in which we required the use of a reactive operator that the reactive layer did not support. Thankfully Thomas added that missing reactive operators within a few hours, essentially proving that Cleipnir’s current design allows for developers to easily add missing reactive operators should the need ever arise. As for the usage of the reactive paradigm in the protocol workflow. The code operations performed over the reactive streams works well and are easy to keep track of due to how simple it is to chain reactive operators. On the other hand, chaining reactive operations can be somewhat restricted in some circumstances. The most troublesome issue encountered in working with reactive operators was handling exceptions to the protocol workflow. In our case, stopping the reactive operators when a view-change occurred was quite troublesome to implement. When the program is required to wait for a reactive \code{Source} to finish its operations, the \code{Source} must receive an item in a stream that manages to pass and perform each and every reactive operator that is chained to the \code{Source} object. This functionality can get very easily stuck when the \code{Source} object does not get the desired items to the stream. There are two notable workarounds to this problem. The first is to ignore the problem since its \code{Source} objects should only be listened to in \code{CTask} asynchronous functions; therefore it won’t block the main execution thread even if it never finishes all of the reactive operators. Meaning the program creates new iterations for the workflow whenever the protocol starts and never stops any existing asynchronous operations that are stuck. This can be achieved if the reactive stream has strict \code{Where} clauses as it allows for the old listeners not to be affected by any new items received in the reactive stream. The \code{Where} clause filters out the items long before they can affect the program in any way. This means the workaround is essentially just letting the listener run stuck until it is eventually garbage collected. This method does slow the system down somewhat since the old \code{Source} objects are still actively listening, receiving, and filtering out items emitted to the stream even if it can never proceed past the \code{Where} clause. The second workaround uses the \code{Merge} operator to have the listener listen to changes on two different streams. This method can effectively terminate the listener if it receives an item from the second code{Source} object, as this is counted as an irregular activity. This is the method used in our \ac{pbft} implementation to handle exiting active instances of the \ac{pbft} workflow to change view for the system. This workaround also has its fair share of issues. To use the \code{Merge} operator, it requires both the \code{Source} objects to listen for the same type of object, which is not always easy to coordinate. Especially when the other operators for the listener transform the stream to work on another object type. In addition, the \code{Merge} operator also works like any other operator. If the \code{Merge} is triggered by the other source object and the operator is called early in the stream, then the item received is still required to pass the other operators to terminate the listener. Which puts it back to the state of the original problem. The item received by the other \code{Source} object must also be unique so that the rest of the workflow can terminate the process when it receives the item from that other \code{Source} object.

To summarize, reactive handlers work well for segmenting operations to perform for the consensus algorithm when a new event is received in the network layer. In addition, it is relatively easy for developers to use and is a lot easier to read the workflow in comparison to traditional programming. However, reactive handlers can be tricky to deal with when used in protocol workflow that needs to handle exceptions to the normal workflow. As consensus algorithms must handle situations where parties on the network stops responding, this can become a rather frequent issue. Therefore would be most beneficial if additional workarounds were discovered for handling this issue.  

\subsection{Cleipnir}
%-Overall knowledge for Cleipnir
%-Issues and advantages in regards to the topics listed over. For instance a lot of time was wasted due to not fully grasping how Cleipnir work
%internally when performing the reactive part and the CAwaitable emission --> resulting a month of frustration trying to figure out why collision errors occur.
%-Lack of documentation can be quite fatal for continued support.
%-Cleipnir and how it interacts with the other programming paradigms. Eks: A clear distinction has to made in regards to what code is run inside Cleipnir(the persistent part) and what is not(ephemeral)

As for our experience with using Cleipnir during this thesis, it has been a mixed experience. The most challenging part of Cleipnir was learning about its functionality when we only had access to its source code and a few practical examples. Bakkevig previously also seemed to struggle in this department in his thesis~\cite[p.~43-44]{PAPER:EivindPaper}. This is somewhat our fault as we were not accustomed to learning about frameworks by reading its source code. During our study, practically all tools and frameworks used had a form of written documentation. Although not all frameworks have well-written documentation. However, most commonly used frameworks usually have some form of community that uses which you can discuss unexpected issues when the need arises. As Cleipnir is still in development, it does not have a large community. We also understand that since Cleinir is constantly updated, writing a detailed documentation could be seen as wasteful because the functionality changes frequently. This means that the documentation would also need to be continuously updated, leading to a lot of extra work for each update. However, we do share Bakkevig’s opinion that if Cleipnir is to become well liked by developers, time must be invested into writing at least a small description for its unique components as well as a guideline for how each tool available in the framework should be used and what users should actively avoid. 

Cleipnir has many different practical examples that are very efficient in teaching users the basics of Cleipnir. We learned a lot by the available practical examples; however without a written documentation, the user likely needs to make assumptions about the tools used. If the assumption is wrong, it leads to the user experiencing both confusion and frustration when things are not working as intended. We were lucky enough that Stidsborg was available to answer any questions we had regarding Cleipnir, but we still made some misconceptions that lead to problems for the design of our application. The biggest misconception being the usage of \code{CTask}. Leading to our thesis getting delayed for about a month because of the race conditions caused by this issue. This is because we had absolutely no clues as to where and why the extra threads originated.   In short, we learned that making assumptions can be quite detrimental, and we should have perhaps queried Stidsborg earlier in the development about the difference between Cleipnir implementation of commonly used classes compared to their traditional use. 
\fi

\section{Future Work}
%1 mention fixing the broken public key system, and give examples(if you can think of any).
%2 potential things: 1. fix it so that the protocol workflow can handle any message being received in any order or 2. Implement a timeout process for the commit section so that the process can not become stuck in any scenario. Both are somewhat needed, but I have no idea how I can handle waiting for prepares messages before receiving the pre-prepare.
%3 Fix persistency. We believe we layed most of the foundation in regards to getting the system to be persistent. However, as mention in imp 2 issues are mainly present. 1. figure out a way to get rid of the original source referanse/access the original source reference so there is no longer duplicate requests. 2. Something is wrong with the synchronization. Not sure what is the cause, assume its the synchronization process is not fully finished before a new request is added.
% 4 Generally make the application and client more interesting. Currently the application state is a simple list of commands written to the console. Make the application actually perform a set of commands, and redesign the client to accomodate for this change.

As mentioned in \autoref{chapter:Design} our current cryptographic signature architecture is susceptible to impersonation and sybil attacks. Clearly keeping public keys ephemeral and generating them uniquely before start up is not a smart design when the system supports persistency. Creating static private and public keys is also not recommended as this would make the system less secure. The simplest solution would be to generate a couple unique key pairs for each replica and have these stored securely or given to the system by a separate trusted system. This system could for instance be a database where the cryptographic keys are stored encrypted. During system startup or during certain scenarios, such as view-changes and or system restarts, the replica reassigns its current cryptographic key pairs and re-establishes its secret key with the other replicas in the system. The other replicas only accept the renewed connection if the separate system acknowledges that the public key given matches one of the unique public keys that replica can have.

Currently we are using a digital signatures scheme for all message types, with the exception of the session messages, which is unnecessary and only slows down the system. The desired alternative is to follow the original \ac{pbft} system model and use \ac{mac} for authentication instead, as this would be more efficient. Although, we still recommend to continue to use the digital signature structure for view-change and new-view messages. Otherwise the view-change workflow would need to be redesigned to follow the more advanced workflow described in Castro's and Liskov's updated paper for \ac{pbft}~\cite[p.~410-414]{PAPER:PBFTRecovery}.

The protocol workflow currently suffers from the inability to handle pre-prepare and prepare being received out of order. In addition, prepare messages can also be lost if the message is received before the prepare listener is initialized. As described in \autoref{sec:protocolwork} this issue can cause the workflow to become stuck if too many prepare messages are lost while the workflow waits for a pre-prepare message. This is obviously something that should be corrected if the application is to be used in the future. One workaround to this problem would be to have a timeout functionality active during the period where the workflow waits for the desired number of prepare and commit messages. The timeout is stopped if both the reactive listeners have successfully created both protocol certificates. Otherwise the timeout expires and the reactive listeners are terminated using the same functionality used for the pre-prepare listener. In order for this functionality to be possible, another \code{Source} object would need to be added as the reactive stream used for reactive listeners for the prepare and commit message is of type \code{Stream<CList<PhaseMessage>~>} due to the stream being transformed by the \code{Scan} operator.

Solving the actual message order issue is a lot more difficult. It is not as simple as to initialize the prepare listener earlier as the listener needs to filter away any phase message that has a different sequence number than its current iteration. Unfortunately, non-primary sets the current sequence number based on the received pre-prepare, creating quite the conundrum. One solution to this problem would be for the server to store copies of the phase messages received in the network layer. By having this logger store the list of phase messages within a dictionary, it would be possible for the workflow to easily search for missing phase messages. Obviously the phase message records would be garbage collected once the protocol has successfully created the two desired protocol certificates for a given sequence number. This would however cause additional complexity to the protocol workflow as functionality for looking up and re-emitting lost phase messages would need to be added.

Currently our application does not fully support persistency. In the future it would be beneficial for both Cleipnir and our application if the issues described in \autoref{chapter:Imp} can be fixed to allow for our application to fully test Cleipnir's capability in regards to persistency. All of the groundwork has been laid for the application to work with persistency. This includes assigning all protocol object types their proper serialization and deserialization for Cleipnir to use, which have been tested on a smaller scale and works as intended. In addition, the network functionality for replicas to reconnect to the system has already been implemented and tested. The only thing left is for the system to successfully read the data stored by Cleipnirs storage engine and successfully restore its old state.
There are at least two notable issues that must be fixed in order for the application to become persistent. The first issue is that the original \code{Source} objects are duplicated by having Cleipnir somehow restore the original \code{Source} while also creating the desired new copy which was supposed to replace the old. Currently both \code{Source} objects react whenever new items are emitted to them by the network layer, meaning that for the protocol workflow, two iterations are created for a single sequence number. This in turn creates issues for the logger when multiple records for the same sequence number is stored. The second issue is that the logger synchronization isn't working properly and as a result records in the logger disappear after the replica restarts. This issue is most likely due to the synchronization either not being fully finished before moving with other operations or the synchronization is not done properly and as a result, some records are skipped. We assume this issue is caused by incorrect usage of \code{Sync} points set for Cleipnir, resulting in the state not being persisted correctly. As for the duplicate \code{Source} objects we are frankly not quite sure how this issue occurs. We theorize that it may occur due to some records being persisted in multiple objects, which in turn when persisted are not treated as the same \code{Source} object, leading to the duplicate issue. If this is the case, the issue would lie in the relationship between the server and the protocol workflow.

Generally the application functionality could be a lot more advanced than it is now. Currently the only operation the application performs after a request is processed successfully though the \ac{pbft} algorithm is simply printing the message attached to the request to the console window. The message is then added to a \code{CList} representing the state of the system. In the future it would be beneficial if the application functionality was changed to be a bit more practical. For instance, changing the message content in the request to instead be an operation which is performed by the application. The state list would then instead store a record of the operation performed as well as whether or not the application was able to perform the requested operation. In order to change the application functionality, the client functionality for creating requests must also be adjusted.

\section{Conclusion}
\iffalse
%clearly state that you accomplish the goal of the thesis.
%Clearly state the answer to the main research question
%Summarize and reflect on the research done for the thesis. In our case discoveries you've made based on usage of Cleipnir + async
%future work + what you have learned --> seperate segments
In conclusion we achieved our goal of creating a \ac{pbft} implementation using Cleipnir with the intended focus of making it faithful to the protocol description which also takes advantage asynchronous and reactive programming paradigms.
Original goal: Our goal for this thesis is to use the Cleipnir framework to implement the Practical Byzantine Fault Tolerance (PBFT) consensus algorithm using functionality from both asynchronous programming and reactive programming. The desired PBFT implementation
\fi
In conclusion, we achieved our goal of creating a simplistic \ac{pbft} implementation using Cleipnir with the intended focus of making it faithful to the protocol description, which also is designed to take advantage of asynchronous and reactive programming paradigms. The result is \ac{pbft} implementation that can perform the \ac{pbft} protocol over several multiple clients and has functional checkpoint and view-change functionality. We managed to design a normal workflow that fit our original criteria, but unfortunately, the protocol struggles with handling out-of-order protocol messages. The checkpoint and view-change workflow became too complex for the processes to be handled within a single function. Persistency functionality was sadly not successful for our \ac{pbft} implementation. Asynchronous programming is shown to be helpful when designing consensus algorithms. Asynchronous programming was notably useful in regards to networking functionality and for designing multi-client protocol workflows.
Similarly, reactive programming turns out to be fairly helpful for handling the operations regarding protocol messages and other event-based processes. Reactive programming, however, did appear to struggle with protocol message ordering when using a synchronous design. These two programming paradigms showed quite clearly that they work well together. We believe implementation consensus algorithms can be further simplified using these tools in the future, despite the problems addressed in this thesis. In regards to the Cleipnir framework, we acknowledge that the overall workflow of the Cleipnir reactive framework is user-friendly and has, for the most part, the functionality desired for designing a proper event handler for a consensus algorithm. We were unsuccessful in evaluating Cleipnir’s persistency functionality on our application. However, based on our experience with using the hybrid persistency functionality on our implementation. In addition to testing the persistency functionality for smaller parts of the program, we deem Cleipnir`s persistency functionality to be excellent.
To conclude this thesis, we do believe that the tools we have tested and evaluated during our \ac{pbft} implementation do make it easier to design consensus algorithms. In the future, we believe that consensus algorithms can be implemented simpler and more accurately to the protocol description. However, we acknowledge that due to the complex nature of distributed systems, it will be challenging to create accurate consensus algorithm implementations due to the numerous problems that can occur.
\fi

This chapter concludes the thesis by first listing the lessons we learned while working on the thesis. Then we list the potential future work which can be applied to the \ac{pbft} implementation.
Finally, a conclusion is drawn for the work performed for this thesis.

\iffalse
\section{Lessons Learned}
\iffalse
-PBFT
-Asynchronous programming with C#, Task architecture
-Reactive Programming basics
-Overall knowledge for Cleipnir
-Issues and advantages in regards to the topics listed over. For instance a lot of time was wasted due to not fully grasping how Cleipnir work internally when performing the reactive part and the CAwaitable emission --> resulting a month of frustration trying to figure out why collision errors occur.
-Lack of documentation can be quite fatal for continued support.
-The multitude of potential issues that could occur that aren't necessary dealt with in the theoretical consensus algorithm or pseudo code.
-Cleipnir and how it interacts with the other programming paradigms. Eks: A clear distinction has to made in regards to what code is run inside Cleipnir(the persistent part) and what is not called in Cleipnir (orthogonal part), mixing these will cause disastrous results, which we infact encountered several times during implementation.
%-Unit testing, simplicity of C# unit testing, issues in regards to unit testing networking as running tests in parallel causes inconsistent results and at worst case inf-loops(don't think this is really all that useful)
\fi
%first draft, probably be heavly changed after writing the other parts of the thesis
%REWRITE THIS SECTION!!!! MINDRE historie lesing, bryr oss ikke om det du har gjort noe særlig med mindre det er godt. Hvis du har noe viktig/revosulerende så beskriv det, men ellers ikke nødvendig.
\subsection{Consensus algorithm}
At the start of this thesis our knowledge in regards to consensus algorithms were limited to having previously implemented the Paxos algorithm using Golang language~\cite{WEB:golangmainpage}. We had never encountered any information in regards to the \ac{pbft} consensus algorithm, therefore some time needed to be spent on learning the inner workings of the \ac{pbft} algorithm. In addition, Cleipnir had already been used to implement the Paxos and Raft consensus algorithms. Therefore some time was also spent on understanding the basics of the Raft consensus algorithm to fully understand the source code used for the Raft implementation. The transition from one consensus algorithm when looking solely on the protocol descriptions is not all that complicated. This is mostly due to similarities found in their functionality. Components used to implement a functional consensus algorithm are shared by many consensus algorithms. This in turn makes it easier for someone familiar with one algorithm to understand another. An example of this being that all three previously mentioned algorithms use an election model in order to make a decision over the network. Furthermore one party in the election is given the leader role and is therefore responsible for governing the election process. Hence understanding the basic principles behind the \ac{pbft} algorithm through the project description was not challenging.

However, consensus algorithms are notoriously difficult to implement. This is because the protocol descriptions are by design written to be as simple as possible, otherwise developers would have issues fully understanding the basics on how they operate. This can unfortunately lead to some information being omitted, which can cause issues when designing an implementation for the algorithm. This was especially apparent for our implementation, since our goal was to make the protocol workflow as simplistic as possible using the tools at hand. Several times during development new issues became apparent in our design when certain scenarios or circumstances occurred during the protocol workflow. This was especially noticeable when thinking of all the different issues that could potentially occur when a restarted replica with an out of date persisted state attempted to collaborate with the other replicas.

In most of these cases we had to decide whether or not it was worth it to introduce additional complexity to the implementations in order to handle these issues, or to simply try to avoid them altogether. In most realistic scenarios the obvious choice would be to fix the issue, even if it adds more complexity to your system. Unfortunately, since our goal was to attempt to implement a very simple implementation, in addition to upholding a time constraint, we had to prioritize differently. Which in turn made our implementation less desirable compared to other more complex implementations. In short, our experience working on implementing the \ac{pbft} algorithm led us to believe that the largest difficulty in regards to the implementation of a consensus algorithm does not necessarily lie in lack of understanding the technicalities within the consensus algorithm. Instead we believe it lies in having to make an implementation that follows the simple protocol description, while still needing to make sure that the state of the system is not affected by any potential issue that can occur in any of the units in the systems. 

\subsection{Asynchronous programming}
Going into this thesis our experience using asynchronous programming were limited and were solely based on a few previous code projects. In addition the asynchronous programming used in these projects used the JavaScript asynchronous framework. Although the language barrier between the asynchronous tools were minimal, there were a few subtle differences. An example of this being how C\# \code{Task} objects function very similarly to \code{Promise} in JavaScript. Overall since both the asynchronous frameworks support the use of the async/await operators, programming asynchronous workflows were relatively similar.

On the other hand there were issues encountered in our application due to lack of understanding behind the details for the async and await operators early in development. Originally our application used asynchronous programming for a lot of tasks related to both networking and protocol handling which caused a lot of internal nested state machines to be created due to using async/await inside other async/await operators. Not only was it a pain to attempt to debug issues regarding nested state machines, but it further escalated when nested async/await operators were used inside \code{CTask}'s for normal \code{Task} functions, which created additional threads to appear as well. The result being a lot of race conditions, inconsistent states and just generally a nightmare to debug. The simple solution was to make any unnecessary asynchronous task become synchronous operations, which in turn removed a lot of the nested state machines. The second change being to separate \code{Task} functions and \code{CTask} functions as much as possible, which further helped since it removed the race conditions.

In short, due to our over usage of async/await workflow for tasks that didn't necessarily need to be asynchronous lead to a lot of issues for our application. Therefore, it is important when designing an asynchronous application to have a clear view over which computing tasks require asynchronous workflow and which can be satisfied by synchronous workflow. Using asynchronous programming for tasks where it is not needed only causes extra complexity to the code and as a result is not only harder to debug but also unnecessarily slows down the system.

\subsection{Reactive programming}
At the beginning of this project, we had very little to no previous experience in regards to reactive programming. Therefore it became quite the challenge learning the basics for reactive programming. Specifically the main challenge became using the basics for reactive programming in order to understand Cleipnir's reactive functionality. Majority of the documentation and tutorials around the web in regards to reactive programming focused mostly on the basics and the cornerstones used for implementing their own reactive operators. This did not quite translate well for our project as all of the reactive layers were already implemented in Cleipnir. Cleipnir reactive functionality in itself is very easy to use and is not all that hard to learn. However, making a direct comparison to the official reactive documentation~\cite{WEB:ReactiveXMainPage} and Cleipnir.Rx was not so simple. This mostly due to the cornerstones having different name schemes between the two.  (add more stuff here later)

\iffalse
In terms of our experience using the Cleipnir reactive layer it is exceptionally easy to use once the basics is learned. Although Cleipnir currently lacks support for the majority of the reactive operators listed in the documentation. The current support does however cover most used reactive operators. During development only a single instance did we encounter an issue in which we required the use of a reactive operator that the reactive layer did not support. Thankfully Thomas added that missing reactive operators within a few hours, essentially proving that Cleipnir's current design allows for developers to easily add missing reactive operators should the need ever arise. As for the usage of the reactive paradigm in the protocol workflow. The code operations performed over the reactive streams works well and are easy to keep track of due to how simple it is to chain reactive operators. On the other hand, chaining reactive operations can be somewhat restricting in some circumstances. The most troublesome issue encountered in regards to working with reactive operators was to handle exceptions to the protocol workflow. In our case stopping the reactive operators when a view-change occurred was quite troublesome to implement. When the program is required to wait for a reactive \code{Source} to finish its operations, the \code{Source} must receive an item in a stream which manages to pass an perform each and every reactive operator that is chained to the \code{Source} object. This functionality can get very easily stuck when the source doesn't get the desired items to the stream. There are two notable workarounds to this problem. The first is to simply ignore the problem since its \code{Source} objects should only be listened to in \code{CTask} asynchronous functions, therefore it won't block the main execution thread even if it never finishes all of the reactive operators. Meaning the program simply creates new iterations for the workflow whenever the protocol starts and never stops any existing asynchronous operations that are stuck. This can be achieved if the reactive stream has strict \code{Where} clauses as it allows for the old listeners to not be affected by any new items received in the reactive stream. The \code{Where} clause filters out the items long before it can affect the program in any way. This means the workaround is essentially just letting the listener run stuck until it is eventually garbage collected. This method does slow the system down somewhat since the old \code{Source} objects are still actively listening, receiving and filtering out items emitted to the stream even if it can never proceed past the \code{Where} clause. The second workaround uses the \code{Merge} operator to have the listener listen to changes on two different streams. By this method it is possible to effectively terminate the listener if it receives an item from the second source, as this is counted as irregular activity. This is the method used in our \ac{pbft} implementation to handle exiting active instances of the \ac{pbft} workflow in order to change view for the system. This workaround also has its fair share of issues. In order to use the \code{Merge} operator it requires both the \code{Source} objects to listen for the same type of object. This is not always easy to coordinate, especially when the other operators for the listener transform the stream to work on another object type. In addition, the \code{Merge} operator also works like any other operator. If the \code{Merge} is triggered by the other source object and the operator is called early in the stream, then the item received is still required to pass the other operators in order to terminate the listener. Which puts it back to the state of the original problem. The item received by the other source must also be unique so that the rest of the workflow can terminate the process when it receives the item from that other \code{Source} object.
\fi

To summarize, the use of reactive handlers works well for segmenting operations to perform for the consensus algorithm when a new event is received in the network layer. In addition it is relatively easy for developers to use and is a lot easier to read the workflow in comparison to traditional programming. However, reactive handlers can be tricky to deal with when used in protocol workflow that needs to handle exceptions to the normal workflow. As consensus algorithms must handle situations where parties on the network stop responding, this can become a rather frequent issue. It therefore would be most beneficial if additional workarounds were discovered for handling this issue.  

\subsection{Cleipnir}
%-Overall knowledge for Cleipnir
%-Issues and advantages in regards to the topics listed over. For instance a lot of time was wasted due to not fully grasping how Cleipnir work
%internally when performing the reactive part and the CAwaitable emission --> resulting a month of frustration trying to figure out why collision errors occur.
%-Lack of documentation can be quite fatal for continued support.
%-Cleipnir and how it interacts with the other programming paradigms. Eks: A clear distinction has to made in regards to what code is run inside Cleipnir(the persistent part) and what is not(ephemeral)

As for our experience with using Cleipnir during this thesis it has been a mixed experience. The most challenging part of Cleipnir was learning about its functionality when we only had access to its source code and a few practical examples. Bakkevig previously also seemed to struggle in this department in his thesis~\cite[p.~43-44]{PAPER:EivindPaper}. This is somewhat our fault as we were not accustomed to learning about frameworks by reading its source code. During our study practically all tools and frameworks used had a form of written documentation. Although not all frameworks have well written documentations, frameworks usually have some form of community that uses which you can discuss unexpected issues when the need arises. As Cleipnir is still in development it obviously does not have a community. We also understand that since Cleinir is constantly being updated, writing a detailed documentation could be seen as wasteful. This is because the functionality changes frequently which means that the documentation would also need to be constantly updated, leading to a lot of extra work for each update. However, we do share Bakkevig's opinion that if  Cleipnir is to become well liked by developers, time must be invested into writing at least a small description for its unique components as well as a guideline for how each tool available in the framework should be used and what users should actively avoid. 

Cleipnir does have a lot of different practical examples that are very efficient in teaching users the basics of Cleipnir. It is possible to learn a lot through the practical examples, however without a written documentation it is very likely that the user needs to make assumptions on the tools used. In the case where the assumption is wrong it leads to the user experiencing both confusion and frustration when things are not working as intended. We were lucky enough that Stidsborg was available to answer any questions we had in regards to Cleipnir, but we still made some misconceptions which lead to problems for the design of our application. The biggest misconception being the usage of \code{CTask} which lead to our project being delayed for about a month due to confusion as to where the additional threads came that lead to many race conditions.   In short we learned that making assumptions can be quite detrimental and we should have perhaps queried Stidsborg earlier in the development about the difference between Cleipnir implementation of commonly used classes compared to their traditional use. 
\fi

\section{Lessons Learned}
\subsection{Consensus algorithms}
At the start of this thesis, our knowledge in regards to consensus algorithms was limited to having previously implemented the Paxos algorithm using Golang language~\cite{WEB:golangmainpage}. We had never encountered any information in regards to the \ac{pbft} consensus algorithm; therefore, some time was spent on learning the inner workings of the \ac{pbft} algorithm. In addition, Cleipnir had already been used to implement the Paxos and Raft consensus algorithms. Therefore some time was also spent on understanding the basics of the Raft consensus algorithm to help understand the source code used for the Raft implementation. We realize that transition from one consensus algorithm to another when looking solely at the protocol descriptions is not all that complicated. Many components for dealing with specific issues regarding consensus algorithms are shared for many consensus algorithms. As a result, it became simpler for someone familiar with one consensus algorithm to understand another.

\subsection{Asynchronous Programming}
Asynchronous programming proved to be suitable for the network layer of the application. In addition, asynchronous programming became an excellent boon for designing a multi-client application when used on the protocol workflows. Since the \code{CTask} also took advantage of the async/await workflow, implementing \code{CTask} functions was just as simple as the .NET traditional asynchronous workflow.

However, we acknowledge that our inexperience with the background operations occurring in the async/await workflow hurt our initial design for our application. This, combined with our wrong assumption regarding \code{CTask}, delayed our thesis considerably. 
\subsection{Reactive Programming}
At the beginning of this project, we had very little to no previous experience in regards to reactive programming. Therefore it was challenging to learn the basics of reactive programming. The most significant complication became understanding Cleipnir’s reactive functionality knowing only the basics of reactive programming. Using the Cleipnir reactive functionality is straightforward once you learn the basics. However, making a direct comparison to the official reactive documentation~\cite{WEB:ReactiveXMainPage} to Cleipnir.Rx was not simple. This primarily due to the cornerstones having different name schemes between the two.

Regardless, we have demonstrated in our application that we ascertained the knowledge to use the reactive framework to handle the \ac{pbft} protocol messages and their resulting operation. It was showed that the framework was appropriate for handling event-driven scenarios in consensus algorithms. In addition, simply enough to have our reactive workflow be reused for several parts of our reactive implementations. 
\subsection{Cleipnir}
Starting the thesis, we had little to no prior experience with working with the Cleipnir framework. The most challenging part of learning how to use Cleipnir functionality was the lack of a detailed documentation. We only had access to its source code and a couple of well written practical examples. Bakkevig previously also seemed to struggle in this department in his thesis~\cite[p.~43-44]{PAPER:EivindPaper}. This is somewhat our fault as we were not accustomed to learning about frameworks by reading their source code. During our study, practically all tools and frameworks used had a form of written documentation. However, not all frameworks have well-written documentation. Although, most commonly used frameworks usually have some form of community that uses which you can discuss unexpected issues when the need arises. As Cleipnir is still in development, it does not have a large community. We also understand that since Cleinir is constantly updated, writing a detailed documentation could be seen as wasteful because the functionality changes frequently. This means that the documentation would also need to be continuously updated, leading to a lot of extra work for each update. However, we do share Bakkevig’s opinion that if Cleipnir is to become well liked by developers, time must be invested into writing at least a small description for its unique components as well as a guideline for how each tool available in the framework should be used and what users should actively avoid. Stidsborg was available to answer any questions we had regarding Cleipnir.  Despite this, flawed assumptions were made, leading to problems for the development of our implementation. Safe to say, we learned that making assumptions can be quite detrimental, and we should have perhaps queried Stidsborg earlier in the development about the difference between Cleipnir implementation of commonly used classes compared to their traditional counterparts.


\iffalse
%REWRITE THIS SECTION!!!! MINDRE historie lesing, bryr oss ikke om det du har gjort noe særlig med mindre det er godt. Hvis du har noe viktig/revosulerende så beskriv det, men ellers ikke nødvendig.
\subsection{Consensus algorithm}

At the start of this thesis, our knowledge in regards to consensus algorithms was limited to having previously implemented the Paxos algorithm using Golang language~\cite{WEB:golangmainpage}. We had never encountered any information in regards to the \ac{pbft} consensus algorithm; therefore, some time was spent on learning the inner workings of the \ac{pbft} algorithm. In addition, Cleipnir had already been used to implement the Paxos and Raft consensus algorithms. Therefore some time was also spent on understanding the basics of the Raft consensus algorithm to help understand the source code used for the Raft implementation. The transition from one consensus algorithm when looking solely at the protocol descriptions is not all that complicated. This is primarily due to similarities found in their functionality. Components used to implement a functional consensus algorithm are shared by many consensus algorithms. Which consequently made it easier for someone familiar with one algorithm to understand another.
An example of this is that two previously mentioned algorithms use an election model to make a decision over the network. Furthermore, one party in the election is given the leader role and is therefore responsible for governing the election process. Hence understanding the basic principles behind the \ac{pbft} algorithm through the project description was not challenging.

However, consensus algorithms are notoriously difficult to implement. This is because the protocol descriptions are by design written to be as simple as possible. Otherwise, developers would have issues fully understanding the basics of how they operate. This can, unfortunately, lead to some information being omitted, which can cause problems when designing an implementation for the algorithm. This was especially apparent for our implementation since our goal was to make the protocol workflow as simplistic as possible using the tools at hand. During development, new issues became apparent in our design when certain scenarios or circumstances occurred during the protocol workflow. This was especially noticeable when considering all the different issues that could occur when a restarted replica with an out-of-date persisted state attempted to collaborate with the other replicas.

In most cases, we had to decide whether or not it was worth introducing additional complexity to the implementations to handle these issues or try to avoid them altogether. The obvious choice would be to fix the issue in the most realistic scenarios, even if it adds more complexity to your system. Unfortunately, since our goal was to attempt to implement a very simple implementation, in addition to upholding a time constraint, we had to prioritize differently. Which, in turn, made our implementation less desirable compared to other more complex implementations. In short, our experience working on implementing the \ac{pbft} algorithm led us to believe that the largest difficulty in regards to the implementation of a consensus algorithm does not necessarily lie in a lack of understanding of the technicalities within the consensus algorithm. Instead, we believe it lies in having to make an implementation that follows the simple protocol description while still needing to make sure that the state of the system is not affected by any potential issue that can occur in any of the units in the distributed system. 

\subsection{Asynchronous programming}
Going into this thesis, our experience using asynchronous programming was limited and was solely based on a few previous code projects. In addition, the asynchronous programming used in these projects used the JavaScript asynchronous framework. Although the language barrier between the asynchronous tools was minimal, there were a few subtle differences. An example of this being how C\# \code{Task} objects function very similarly to \code{Promise} in JavaScript. Overall since both the asynchronous frameworks support the use of the async/await operators, asynchronous programming workflows were relatively similar.

On the other hand, our application had issues due to a lack of understanding behind the details for the async and await operators early in development. Originally, our application used asynchronous programming for many tasks related to networking and protocol handling, which caused many internal nested state machines to be created due to using async/await inside other async/await operators. Not only was it a pain to attempt to debug issues regarding nested state machines, but it further escalated when nested async/await operators were used inside \code{CTask} ’s for normal \code{Task} functions, which created additional threads to be added into the mix. The result was many race conditions, inconsistent states, and just generally a nightmare to debug. The simple solution was to make any unnecessary asynchronous task become synchronous operations, which removed a lot of the nested state machines. The second change was to separate \code{Task} functions and \code{CTask} functions as much as possible, which further helped since it removed the race conditions.

In short, due to our over usage of async/await workflow for tasks that didn’t necessarily need to be asynchronous led to a lot of issues for our application. Therefore, it is important when designing an asynchronous application to have a clear view of which computing tasks require asynchronous workflow and which can be satisfied by synchronous workflow. Using asynchronous programming for tasks where it is not needed only caused extra complexity to the code. As a result, it became harder to debug and unnecessarily slowed down our system.

\subsection{Reactive programming}
At the beginning of this project, we had very little to no previous experience in regards to reactive programming. Therefore it became quite a challenge to learn the basics of reactive programming. Specifically, the main challenge was using the basics for reactive programming to understand Cleipnir’s reactive functionality. The majority of the documentation and tutorials around the web regarding reactive programming focused mainly on the basics and the cornerstones used for implementing their own reactive operators. This did not quite translate well for our project as all of the reactive layers were already implemented in Cleipnir. Cleipnir reactive functionality in itself is very easy to use and is not all that hard to learn. However, making a direct comparison to the official reactive documentation~\cite{WEB:ReactiveXMainPage} and Cleipnir.Rx was not so simple. This mostly due to the cornerstones having different name schemes between the two.  (add more stuff here later)

Our experience using the Cleipnir reactive layer is that it is exceptionally easy to use once the basics are learned. Although Cleipnir currently lacks support for the majority of the reactive operators listed in the documentation. The current support does, however, cover the most used reactive operators. During development, only a single instance did we encounter an issue in which we required the use of a reactive operator that the reactive layer did not support. Thankfully Thomas added that missing reactive operators within a few hours, essentially proving that Cleipnir’s current design allows for developers to easily add missing reactive operators should the need ever arise. As for the usage of the reactive paradigm in the protocol workflow. The code operations performed over the reactive streams works well and are easy to keep track of due to how simple it is to chain reactive operators. On the other hand, chaining reactive operations can be somewhat restricted in some circumstances. The most troublesome issue encountered in working with reactive operators was handling exceptions to the protocol workflow. In our case, stopping the reactive operators when a view-change occurred was quite troublesome to implement. When the program is required to wait for a reactive \code{Source} to finish its operations, the \code{Source} must receive an item in a stream that manages to pass and perform each and every reactive operator that is chained to the \code{Source} object. This functionality can get very easily stuck when the \code{Source} object does not get the desired items to the stream. There are two notable workarounds to this problem. The first is to ignore the problem since its \code{Source} objects should only be listened to in \code{CTask} asynchronous functions; therefore it won’t block the main execution thread even if it never finishes all of the reactive operators. Meaning the program creates new iterations for the workflow whenever the protocol starts and never stops any existing asynchronous operations that are stuck. This can be achieved if the reactive stream has strict \code{Where} clauses as it allows for the old listeners not to be affected by any new items received in the reactive stream. The \code{Where} clause filters out the items long before they can affect the program in any way. This means the workaround is essentially just letting the listener run stuck until it is eventually garbage collected. This method does slow the system down somewhat since the old \code{Source} objects are still actively listening, receiving, and filtering out items emitted to the stream even if it can never proceed past the \code{Where} clause. The second workaround uses the \code{Merge} operator to have the listener listen to changes on two different streams. This method can effectively terminate the listener if it receives an item from the second code{Source} object, as this is counted as an irregular activity. This is the method used in our \ac{pbft} implementation to handle exiting active instances of the \ac{pbft} workflow to change view for the system. This workaround also has its fair share of issues. To use the \code{Merge} operator, it requires both the \code{Source} objects to listen for the same type of object, which is not always easy to coordinate. Especially when the other operators for the listener transform the stream to work on another object type. In addition, the \code{Merge} operator also works like any other operator. If the \code{Merge} is triggered by the other source object and the operator is called early in the stream, then the item received is still required to pass the other operators to terminate the listener. Which puts it back to the state of the original problem. The item received by the other \code{Source} object must also be unique so that the rest of the workflow can terminate the process when it receives the item from that other \code{Source} object.

To summarize, reactive handlers work well for segmenting operations to perform for the consensus algorithm when a new event is received in the network layer. In addition, it is relatively easy for developers to use and is a lot easier to read the workflow in comparison to traditional programming. However, reactive handlers can be tricky to deal with when used in protocol workflow that needs to handle exceptions to the normal workflow. As consensus algorithms must handle situations where parties on the network stops responding, this can become a rather frequent issue. Therefore would be most beneficial if additional workarounds were discovered for handling this issue.  

\subsection{Cleipnir}
%-Overall knowledge for Cleipnir
%-Issues and advantages in regards to the topics listed over. For instance a lot of time was wasted due to not fully grasping how Cleipnir work
%internally when performing the reactive part and the CAwaitable emission --> resulting a month of frustration trying to figure out why collision errors occur.
%-Lack of documentation can be quite fatal for continued support.
%-Cleipnir and how it interacts with the other programming paradigms. Eks: A clear distinction has to made in regards to what code is run inside Cleipnir(the persistent part) and what is not(ephemeral)

As for our experience with using Cleipnir during this thesis, it has been a mixed experience. The most challenging part of Cleipnir was learning about its functionality when we only had access to its source code and a few practical examples. Bakkevig previously also seemed to struggle in this department in his thesis~\cite[p.~43-44]{PAPER:EivindPaper}. This is somewhat our fault as we were not accustomed to learning about frameworks by reading its source code. During our study, practically all tools and frameworks used had a form of written documentation. Although not all frameworks have well-written documentation. However, most commonly used frameworks usually have some form of community that uses which you can discuss unexpected issues when the need arises. As Cleipnir is still in development, it does not have a large community. We also understand that since Cleinir is constantly updated, writing a detailed documentation could be seen as wasteful because the functionality changes frequently. This means that the documentation would also need to be continuously updated, leading to a lot of extra work for each update. However, we do share Bakkevig’s opinion that if Cleipnir is to become well liked by developers, time must be invested into writing at least a small description for its unique components as well as a guideline for how each tool available in the framework should be used and what users should actively avoid. 

Cleipnir has many different practical examples that are very efficient in teaching users the basics of Cleipnir. We learned a lot by the available practical examples; however without a written documentation, the user likely needs to make assumptions about the tools used. If the assumption is wrong, it leads to the user experiencing both confusion and frustration when things are not working as intended. We were lucky enough that Stidsborg was available to answer any questions we had regarding Cleipnir, but we still made some misconceptions that lead to problems for the design of our application. The biggest misconception being the usage of \code{CTask}. Leading to our thesis getting delayed for about a month because of the race conditions caused by this issue. This is because we had absolutely no clues as to where and why the extra threads originated.   In short, we learned that making assumptions can be quite detrimental, and we should have perhaps queried Stidsborg earlier in the development about the difference between Cleipnir implementation of commonly used classes compared to their traditional use. 
\fi

\section{Future Work}
As mentioned in \autoref{chapter:Design} our current cryptographic signature architecture is susceptible to impersonation and spoofing attacks. Clearly, keeping public keys ephemeral and generating them uniquely before start-up was not a smart design when the system supports persistency. Creating static private and public keys is also not recommended since this design would make the system less secure. One solution would be to generate a couple of unique key pairs for each replica and have these stored securely or given to the system by a separate trusted system. This system could, for instance, be a database where the cryptographic keys are stored encrypted. During system start-up or during certain scenarios, such as view-changes and or system restarts, the replica reassigns its current cryptographic key pairs and re-establishes its secret key with the other replicas in the system. The other replicas only accept the renewed connection if the separate system acknowledges that the public key given matches one of the unique public keys listed for that replica.

We are currently using a digital signatures scheme for all message types, except for the session messages. This is frankly unnecessary and only slows down the system. The desired alternative is to follow the original \ac{pbft} system model and use \ac{mac} for authentication instead, as this would be more efficient. Although, we still recommend continuing to use the digital signature structure for view-change and new-view messages. Otherwise, the view-change workflow would need to be redesigned to follow the more advanced workflow described in Castro’s and Liskov’s updated paper for \ac{pbft}~\cite[p.~410-414]{PAPER:PBFTRecovery}.

The protocol workflow currently suffers from the inability to handle pre-prepare and prepare being received out of order. In addition, prepare messages can also be lost if the message is received before the prepare listener is initialized. As described in \autoref{sec:protocolwork} this issue can cause the workflow to become stuck if too many prepare messages are lost while the workflow waits for a pre-prepare message. This is something that should be corrected if the application is to be used in the future. A workaround to this problem would be to have a timeout functionality active when the workflow waits for the desired number of prepare and commit messages. The timeout is stopped if both the reactive listeners have successfully created both protocol certificates. Otherwise, the timeout expires, and the reactive listeners are terminated using the same functionality used for the pre-prepare listener. For this functionality to be possible, another \code{Source} object would need to be added to the workflow to work with the \code{Merge} operator. This is because the reactive stream for reactive listeners to the prepare and commit message is of type \code{Stream<CList<PhaseMessage>~>} due to the stream being transformed by the \code{Scan} operator.

Solving the actual message ordering issue is a lot more complicated. It is not as simple as initializing the prepare listener earlier, as the listener needs to filter away any phase message with a different sequence number than its current iteration. Unfortunately, non-primary replicas set the current sequence number based on the received pre-prepare message, creating quite the conundrum. A solution to this problem is making the server store copies of the phase messages received in the network layer. By having this logger store a list of phase messages received for a sequence number within a dictionary, it would be possible for the workflow to easily search for missing phase messages. The phase message records stored in this logger would have to be garbage collected once the protocol has successfully created the two desired protocol certificates for the given sequence number. However, this would cause additional complexity to the protocol workflow as functionality for looking up, and re-emitting lost phase messages would need to be added.

Currently, our application does not fully support persistency. In the future, it would be favorable for both Cleipnir and our application if the issues described in \autoref{chapter:Imp} can be fixed to allow for our application to thoroughly test Cleipnir’s capability in regards to persistency. The groundwork has been laid for the application to work with persistency. This includes assigning all protocol object types their proper serialization and deserialization functions for Cleipnir to use, which have been tested on a smaller scale and works as intended. In addition, the network functionality for replicas to reconnect to the system has already been implemented and tested. The only thing left is for the system to successfully read the data stored by Cleipnirs storage engine and successfully restore its old state.
There are at least two notable issues that must be fixed for the application to become persistent. The first issue is that the original \code{Source} objects are duplicated by having Cleipnir somehow restore the original \code{Source} while also creating the desired new copy, which was supposed to replace the old. Currently, both \code{Source} objects react whenever new items are emitted to them by the network layer, meaning that for the protocol workflow, two iterations are created for a single sequence number. This, in turn, creates issues for the logger when multiple records for the same sequence number are stored. The second issue is that the logger synchronization isn’t working properly and as a result, records in the logger disappear after the replica restarts. This issue likely due to the synchronization not being fully finished before moving with other operations, or the synchronization is not done correctly, and as a result, some records are skipped. We assume this issue is caused by incorrect usage of \code{Sync} points set for Cleipnir, resulting in the state not being persisted correctly. As for the duplicate \code{Source} objects, we are frankly not quite sure how this issue occurs. We theorize that it may occur due to some records being persisted in multiple objects, and because of this, when the objects are persisted, the objects are not treated as the same \code{Source} object, leading to the duplicate \code{Source} object. If this is the case, the issue lies in the relationship between the server and the protocol workflow.

Generally, the application functionality could be a lot more advanced than it is now. Currently, the only operation the application performs after a request is processed successfully through the \ac{pbft} algorithm is simply printing the message attached to the request to the console window. The message is then added to a \code{CList} representing the state of the system. In the future, it would be beneficial if the application functionality was changed to be a bit more practical. For instance, changing the message content in the request to be an operation that is to be performed by the application instead of a string. The state list would then rather store a record of the operation performed and whether or not the application successfully performed the requested operation. In order to change the application functionality, the client functionality for creating requests must also be adjusted.

\section{Conclusion}
In conclusion, we achieved our goal of creating a simplistic \ac{pbft} implementation using Cleipnir with the intended focus of making it faithful to the protocol description, which also is designed to take advantage of asynchronous and reactive programming paradigms. The result is \ac{pbft} implementation that can perform the \ac{pbft} protocol over several multiple clients and has functional checkpoint and view-change functionality. We managed to design a normal workflow that fit our original criteria, but unfortunately, the protocol struggles with handling out-of-order protocol messages. The checkpoint and view-change workflow became too complex for the processes to be handled within a single function. Persistency functionality was sadly not successful for our \ac{pbft} implementation. Asynchronous programming is shown to be helpful when designing consensus algorithms. Asynchronous programming was notably useful in regards to networking functionality and for designing multi-client protocol workflows.
Similarly, reactive programming turns out to be fairly helpful for handling the operations regarding protocol messages and other event-based processes. Reactive programming, however, did appear to struggle with protocol message ordering when using synchronous design. These two programming paradigms showed quite clearly that they work well together. We believe implementation consensus algorithms can be further simplified using these tools in the future, despite the problems addressed in this thesis. In regards to the Cleipnir framework, we acknowledge that the overall workflow of the Cleipnir reactive framework is user-friendly and has, for the most part, the functionality desired for designing a proper event handler for a consensus algorithm. We were unsuccessful in evaluating Cleipnir’s persistency functionality on our application. However, based on our experience with using the hybrid persistency functionality on our implementation. In addition to testing the persistency functionality for smaller parts of the program, we deem Cleipnir’s persistency functionality to be excellent.
To conclude this thesis, we do believe that the tools we have tested and evaluated during our \ac{pbft} implementation do make it easier to design consensus algorithms. In the future, we believe that consensus algorithms can be implemented simpler and more accurately to the protocol description. However, we acknowledge that due to the complex nature of distributed systems, it will be challenging to create accurate consensus algorithm implementations due to the numerous problems that can occur.


	
\listoffigures
\lstlistoflistings

\newpage
\begin{appendices}
	\chapter{\ac{pbft} Implementation Source Code}
	\begin{itemize}
		\item[-] The \ac{pbft} implementation can be found on this \href{https://github.com/Lupu2/PBFT-Master}{Github repository}.
		\item[-] The source code used for the replica implementation is found in this \href{https://github.com/Lupu2/PBFT-Master/tree/main/PBFT}{sub directory}.
		\item[-] The source code used for our client implementation is found in this \href{https://github.com/Lupu2/PBFT-Master/tree/main/PBFTClient}{sub directory}. The unit tests can unfortunately not be all run concurrently, due to some issue with the network tests. We therefore recommend running each folder separately, and re-run the test that fail uniquely once they fail. 
		\item[-] The Github repository provides a detailed explanation in regards to how to run the \ac{pbft} implementation both locally and with docker containers. 
	\end{itemize} 
\end{appendices}
\printbibliography[heading=bibintoc]
\end{document}
